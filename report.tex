\documentclass{article}

\usepackage{blindtext}

\usepackage{own} % eigen package voor laden standaard packages en eventuele eigen code

\usepackage{circuitikz}

\usepackage{natbib}
\usepackage{url}

\begin{document}
\section{Plaatjes}
In tabel \ref{tab:python} staat een selectie van de meetdata. De vesnelling in verschillende richtingen is uitgezet in figuur \ref{fig:python}. Daarnaast is vergelijking \ref{eq:verplichte vergelijking} weergegeven. Dit voorbeeld document komt van \citet{land_latex_2018}.

\section{Bronnen en Bibtex}
Voor het makkelijk omzetten van boeken/websites naar Bibtex zijn de volgende websites beschikbaar:
\begin{itemize}
	\item \url{https://zbib.org/}
	\item \url{https://manas.tungare.name/software/isbn-to-bibtex}
	\item \url{https://www.zotero.org/}
\end{itemize}
Andere handige websites zijn:
\begin{itemize}
	\item \url{http://detexify.kirelabs.org/classify.html}
	\item \url{https://www.overleaf.com/learn/latex/Main_Page}
	\item \url{https://tex.stackexchange.com/}
\end{itemize}

\section{Begin voorbeeld document}
\blindtext

\begin{figure}[htbp]
	\centering%% Creator: Matplotlib, PGF backend
%%
%% To include the figure in your LaTeX document, write
%%   \input{<filename>.pgf}
%%
%% Make sure the required packages are loaded in your preamble
%%   \usepackage{pgf}
%%
%% Figures using additional raster images can only be included by \input if
%% they are in the same directory as the main LaTeX file. For loading figures
%% from other directories you can use the `import` package
%%   \usepackage{import}
%% and then include the figures with
%%   \import{<path to file>}{<filename>.pgf}
%%
%% Matplotlib used the following preamble
%%
\begingroup%
\makeatletter%
\begin{pgfpicture}%
\pgfpathrectangle{\pgfpointorigin}{\pgfqpoint{4.800000in}{3.000000in}}%
\pgfusepath{use as bounding box, clip}%
\begin{pgfscope}%
\pgfsetbuttcap%
\pgfsetmiterjoin%
\definecolor{currentfill}{rgb}{1.000000,1.000000,1.000000}%
\pgfsetfillcolor{currentfill}%
\pgfsetlinewidth{0.000000pt}%
\definecolor{currentstroke}{rgb}{1.000000,1.000000,1.000000}%
\pgfsetstrokecolor{currentstroke}%
\pgfsetdash{}{0pt}%
\pgfpathmoveto{\pgfqpoint{0.000000in}{0.000000in}}%
\pgfpathlineto{\pgfqpoint{4.800000in}{0.000000in}}%
\pgfpathlineto{\pgfqpoint{4.800000in}{3.000000in}}%
\pgfpathlineto{\pgfqpoint{0.000000in}{3.000000in}}%
\pgfpathclose%
\pgfusepath{fill}%
\end{pgfscope}%
\begin{pgfscope}%
\pgfsetbuttcap%
\pgfsetmiterjoin%
\definecolor{currentfill}{rgb}{1.000000,1.000000,1.000000}%
\pgfsetfillcolor{currentfill}%
\pgfsetlinewidth{0.000000pt}%
\definecolor{currentstroke}{rgb}{0.000000,0.000000,0.000000}%
\pgfsetstrokecolor{currentstroke}%
\pgfsetstrokeopacity{0.000000}%
\pgfsetdash{}{0pt}%
\pgfpathmoveto{\pgfqpoint{0.812440in}{0.565123in}}%
\pgfpathlineto{\pgfqpoint{4.615000in}{0.565123in}}%
\pgfpathlineto{\pgfqpoint{4.615000in}{2.815000in}}%
\pgfpathlineto{\pgfqpoint{0.812440in}{2.815000in}}%
\pgfpathclose%
\pgfusepath{fill}%
\end{pgfscope}%
\begin{pgfscope}%
\pgfpathrectangle{\pgfqpoint{0.812440in}{0.565123in}}{\pgfqpoint{3.802560in}{2.249877in}} %
\pgfusepath{clip}%
\pgfsetrectcap%
\pgfsetroundjoin%
\pgfsetlinewidth{0.803000pt}%
\definecolor{currentstroke}{rgb}{0.690196,0.690196,0.690196}%
\pgfsetstrokecolor{currentstroke}%
\pgfsetdash{}{0pt}%
\pgfpathmoveto{\pgfqpoint{0.985283in}{0.565123in}}%
\pgfpathlineto{\pgfqpoint{0.985283in}{2.815000in}}%
\pgfusepath{stroke}%
\end{pgfscope}%
\begin{pgfscope}%
\pgfsetbuttcap%
\pgfsetroundjoin%
\definecolor{currentfill}{rgb}{0.000000,0.000000,0.000000}%
\pgfsetfillcolor{currentfill}%
\pgfsetlinewidth{0.803000pt}%
\definecolor{currentstroke}{rgb}{0.000000,0.000000,0.000000}%
\pgfsetstrokecolor{currentstroke}%
\pgfsetdash{}{0pt}%
\pgfsys@defobject{currentmarker}{\pgfqpoint{0.000000in}{-0.048611in}}{\pgfqpoint{0.000000in}{0.000000in}}{%
\pgfpathmoveto{\pgfqpoint{0.000000in}{0.000000in}}%
\pgfpathlineto{\pgfqpoint{0.000000in}{-0.048611in}}%
\pgfusepath{stroke,fill}%
}%
\begin{pgfscope}%
\pgfsys@transformshift{0.985283in}{0.565123in}%
\pgfsys@useobject{currentmarker}{}%
\end{pgfscope}%
\end{pgfscope}%
\begin{pgfscope}%
\pgftext[x=0.985283in,y=0.467901in,,top]{\sffamily\fontsize{10.000000}{12.000000}\selectfont \(\displaystyle 0{,}00\)}%
\end{pgfscope}%
\begin{pgfscope}%
\pgfpathrectangle{\pgfqpoint{0.812440in}{0.565123in}}{\pgfqpoint{3.802560in}{2.249877in}} %
\pgfusepath{clip}%
\pgfsetrectcap%
\pgfsetroundjoin%
\pgfsetlinewidth{0.803000pt}%
\definecolor{currentstroke}{rgb}{0.690196,0.690196,0.690196}%
\pgfsetstrokecolor{currentstroke}%
\pgfsetdash{}{0pt}%
\pgfpathmoveto{\pgfqpoint{1.432495in}{0.565123in}}%
\pgfpathlineto{\pgfqpoint{1.432495in}{2.815000in}}%
\pgfusepath{stroke}%
\end{pgfscope}%
\begin{pgfscope}%
\pgfsetbuttcap%
\pgfsetroundjoin%
\definecolor{currentfill}{rgb}{0.000000,0.000000,0.000000}%
\pgfsetfillcolor{currentfill}%
\pgfsetlinewidth{0.803000pt}%
\definecolor{currentstroke}{rgb}{0.000000,0.000000,0.000000}%
\pgfsetstrokecolor{currentstroke}%
\pgfsetdash{}{0pt}%
\pgfsys@defobject{currentmarker}{\pgfqpoint{0.000000in}{-0.048611in}}{\pgfqpoint{0.000000in}{0.000000in}}{%
\pgfpathmoveto{\pgfqpoint{0.000000in}{0.000000in}}%
\pgfpathlineto{\pgfqpoint{0.000000in}{-0.048611in}}%
\pgfusepath{stroke,fill}%
}%
\begin{pgfscope}%
\pgfsys@transformshift{1.432495in}{0.565123in}%
\pgfsys@useobject{currentmarker}{}%
\end{pgfscope}%
\end{pgfscope}%
\begin{pgfscope}%
\pgftext[x=1.432495in,y=0.467901in,,top]{\sffamily\fontsize{10.000000}{12.000000}\selectfont \(\displaystyle 20{,}0\)}%
\end{pgfscope}%
\begin{pgfscope}%
\pgfpathrectangle{\pgfqpoint{0.812440in}{0.565123in}}{\pgfqpoint{3.802560in}{2.249877in}} %
\pgfusepath{clip}%
\pgfsetrectcap%
\pgfsetroundjoin%
\pgfsetlinewidth{0.803000pt}%
\definecolor{currentstroke}{rgb}{0.690196,0.690196,0.690196}%
\pgfsetstrokecolor{currentstroke}%
\pgfsetdash{}{0pt}%
\pgfpathmoveto{\pgfqpoint{1.879706in}{0.565123in}}%
\pgfpathlineto{\pgfqpoint{1.879706in}{2.815000in}}%
\pgfusepath{stroke}%
\end{pgfscope}%
\begin{pgfscope}%
\pgfsetbuttcap%
\pgfsetroundjoin%
\definecolor{currentfill}{rgb}{0.000000,0.000000,0.000000}%
\pgfsetfillcolor{currentfill}%
\pgfsetlinewidth{0.803000pt}%
\definecolor{currentstroke}{rgb}{0.000000,0.000000,0.000000}%
\pgfsetstrokecolor{currentstroke}%
\pgfsetdash{}{0pt}%
\pgfsys@defobject{currentmarker}{\pgfqpoint{0.000000in}{-0.048611in}}{\pgfqpoint{0.000000in}{0.000000in}}{%
\pgfpathmoveto{\pgfqpoint{0.000000in}{0.000000in}}%
\pgfpathlineto{\pgfqpoint{0.000000in}{-0.048611in}}%
\pgfusepath{stroke,fill}%
}%
\begin{pgfscope}%
\pgfsys@transformshift{1.879706in}{0.565123in}%
\pgfsys@useobject{currentmarker}{}%
\end{pgfscope}%
\end{pgfscope}%
\begin{pgfscope}%
\pgftext[x=1.879706in,y=0.467901in,,top]{\sffamily\fontsize{10.000000}{12.000000}\selectfont \(\displaystyle 40{,}0\)}%
\end{pgfscope}%
\begin{pgfscope}%
\pgfpathrectangle{\pgfqpoint{0.812440in}{0.565123in}}{\pgfqpoint{3.802560in}{2.249877in}} %
\pgfusepath{clip}%
\pgfsetrectcap%
\pgfsetroundjoin%
\pgfsetlinewidth{0.803000pt}%
\definecolor{currentstroke}{rgb}{0.690196,0.690196,0.690196}%
\pgfsetstrokecolor{currentstroke}%
\pgfsetdash{}{0pt}%
\pgfpathmoveto{\pgfqpoint{2.326917in}{0.565123in}}%
\pgfpathlineto{\pgfqpoint{2.326917in}{2.815000in}}%
\pgfusepath{stroke}%
\end{pgfscope}%
\begin{pgfscope}%
\pgfsetbuttcap%
\pgfsetroundjoin%
\definecolor{currentfill}{rgb}{0.000000,0.000000,0.000000}%
\pgfsetfillcolor{currentfill}%
\pgfsetlinewidth{0.803000pt}%
\definecolor{currentstroke}{rgb}{0.000000,0.000000,0.000000}%
\pgfsetstrokecolor{currentstroke}%
\pgfsetdash{}{0pt}%
\pgfsys@defobject{currentmarker}{\pgfqpoint{0.000000in}{-0.048611in}}{\pgfqpoint{0.000000in}{0.000000in}}{%
\pgfpathmoveto{\pgfqpoint{0.000000in}{0.000000in}}%
\pgfpathlineto{\pgfqpoint{0.000000in}{-0.048611in}}%
\pgfusepath{stroke,fill}%
}%
\begin{pgfscope}%
\pgfsys@transformshift{2.326917in}{0.565123in}%
\pgfsys@useobject{currentmarker}{}%
\end{pgfscope}%
\end{pgfscope}%
\begin{pgfscope}%
\pgftext[x=2.326917in,y=0.467901in,,top]{\sffamily\fontsize{10.000000}{12.000000}\selectfont \(\displaystyle 60{,}0\)}%
\end{pgfscope}%
\begin{pgfscope}%
\pgfpathrectangle{\pgfqpoint{0.812440in}{0.565123in}}{\pgfqpoint{3.802560in}{2.249877in}} %
\pgfusepath{clip}%
\pgfsetrectcap%
\pgfsetroundjoin%
\pgfsetlinewidth{0.803000pt}%
\definecolor{currentstroke}{rgb}{0.690196,0.690196,0.690196}%
\pgfsetstrokecolor{currentstroke}%
\pgfsetdash{}{0pt}%
\pgfpathmoveto{\pgfqpoint{2.774128in}{0.565123in}}%
\pgfpathlineto{\pgfqpoint{2.774128in}{2.815000in}}%
\pgfusepath{stroke}%
\end{pgfscope}%
\begin{pgfscope}%
\pgfsetbuttcap%
\pgfsetroundjoin%
\definecolor{currentfill}{rgb}{0.000000,0.000000,0.000000}%
\pgfsetfillcolor{currentfill}%
\pgfsetlinewidth{0.803000pt}%
\definecolor{currentstroke}{rgb}{0.000000,0.000000,0.000000}%
\pgfsetstrokecolor{currentstroke}%
\pgfsetdash{}{0pt}%
\pgfsys@defobject{currentmarker}{\pgfqpoint{0.000000in}{-0.048611in}}{\pgfqpoint{0.000000in}{0.000000in}}{%
\pgfpathmoveto{\pgfqpoint{0.000000in}{0.000000in}}%
\pgfpathlineto{\pgfqpoint{0.000000in}{-0.048611in}}%
\pgfusepath{stroke,fill}%
}%
\begin{pgfscope}%
\pgfsys@transformshift{2.774128in}{0.565123in}%
\pgfsys@useobject{currentmarker}{}%
\end{pgfscope}%
\end{pgfscope}%
\begin{pgfscope}%
\pgftext[x=2.774128in,y=0.467901in,,top]{\sffamily\fontsize{10.000000}{12.000000}\selectfont \(\displaystyle 80{,}0\)}%
\end{pgfscope}%
\begin{pgfscope}%
\pgfpathrectangle{\pgfqpoint{0.812440in}{0.565123in}}{\pgfqpoint{3.802560in}{2.249877in}} %
\pgfusepath{clip}%
\pgfsetrectcap%
\pgfsetroundjoin%
\pgfsetlinewidth{0.803000pt}%
\definecolor{currentstroke}{rgb}{0.690196,0.690196,0.690196}%
\pgfsetstrokecolor{currentstroke}%
\pgfsetdash{}{0pt}%
\pgfpathmoveto{\pgfqpoint{3.221340in}{0.565123in}}%
\pgfpathlineto{\pgfqpoint{3.221340in}{2.815000in}}%
\pgfusepath{stroke}%
\end{pgfscope}%
\begin{pgfscope}%
\pgfsetbuttcap%
\pgfsetroundjoin%
\definecolor{currentfill}{rgb}{0.000000,0.000000,0.000000}%
\pgfsetfillcolor{currentfill}%
\pgfsetlinewidth{0.803000pt}%
\definecolor{currentstroke}{rgb}{0.000000,0.000000,0.000000}%
\pgfsetstrokecolor{currentstroke}%
\pgfsetdash{}{0pt}%
\pgfsys@defobject{currentmarker}{\pgfqpoint{0.000000in}{-0.048611in}}{\pgfqpoint{0.000000in}{0.000000in}}{%
\pgfpathmoveto{\pgfqpoint{0.000000in}{0.000000in}}%
\pgfpathlineto{\pgfqpoint{0.000000in}{-0.048611in}}%
\pgfusepath{stroke,fill}%
}%
\begin{pgfscope}%
\pgfsys@transformshift{3.221340in}{0.565123in}%
\pgfsys@useobject{currentmarker}{}%
\end{pgfscope}%
\end{pgfscope}%
\begin{pgfscope}%
\pgftext[x=3.221340in,y=0.467901in,,top]{\sffamily\fontsize{10.000000}{12.000000}\selectfont \(\displaystyle 100\)}%
\end{pgfscope}%
\begin{pgfscope}%
\pgfpathrectangle{\pgfqpoint{0.812440in}{0.565123in}}{\pgfqpoint{3.802560in}{2.249877in}} %
\pgfusepath{clip}%
\pgfsetrectcap%
\pgfsetroundjoin%
\pgfsetlinewidth{0.803000pt}%
\definecolor{currentstroke}{rgb}{0.690196,0.690196,0.690196}%
\pgfsetstrokecolor{currentstroke}%
\pgfsetdash{}{0pt}%
\pgfpathmoveto{\pgfqpoint{3.668551in}{0.565123in}}%
\pgfpathlineto{\pgfqpoint{3.668551in}{2.815000in}}%
\pgfusepath{stroke}%
\end{pgfscope}%
\begin{pgfscope}%
\pgfsetbuttcap%
\pgfsetroundjoin%
\definecolor{currentfill}{rgb}{0.000000,0.000000,0.000000}%
\pgfsetfillcolor{currentfill}%
\pgfsetlinewidth{0.803000pt}%
\definecolor{currentstroke}{rgb}{0.000000,0.000000,0.000000}%
\pgfsetstrokecolor{currentstroke}%
\pgfsetdash{}{0pt}%
\pgfsys@defobject{currentmarker}{\pgfqpoint{0.000000in}{-0.048611in}}{\pgfqpoint{0.000000in}{0.000000in}}{%
\pgfpathmoveto{\pgfqpoint{0.000000in}{0.000000in}}%
\pgfpathlineto{\pgfqpoint{0.000000in}{-0.048611in}}%
\pgfusepath{stroke,fill}%
}%
\begin{pgfscope}%
\pgfsys@transformshift{3.668551in}{0.565123in}%
\pgfsys@useobject{currentmarker}{}%
\end{pgfscope}%
\end{pgfscope}%
\begin{pgfscope}%
\pgftext[x=3.668551in,y=0.467901in,,top]{\sffamily\fontsize{10.000000}{12.000000}\selectfont \(\displaystyle 120\)}%
\end{pgfscope}%
\begin{pgfscope}%
\pgfpathrectangle{\pgfqpoint{0.812440in}{0.565123in}}{\pgfqpoint{3.802560in}{2.249877in}} %
\pgfusepath{clip}%
\pgfsetrectcap%
\pgfsetroundjoin%
\pgfsetlinewidth{0.803000pt}%
\definecolor{currentstroke}{rgb}{0.690196,0.690196,0.690196}%
\pgfsetstrokecolor{currentstroke}%
\pgfsetdash{}{0pt}%
\pgfpathmoveto{\pgfqpoint{4.115762in}{0.565123in}}%
\pgfpathlineto{\pgfqpoint{4.115762in}{2.815000in}}%
\pgfusepath{stroke}%
\end{pgfscope}%
\begin{pgfscope}%
\pgfsetbuttcap%
\pgfsetroundjoin%
\definecolor{currentfill}{rgb}{0.000000,0.000000,0.000000}%
\pgfsetfillcolor{currentfill}%
\pgfsetlinewidth{0.803000pt}%
\definecolor{currentstroke}{rgb}{0.000000,0.000000,0.000000}%
\pgfsetstrokecolor{currentstroke}%
\pgfsetdash{}{0pt}%
\pgfsys@defobject{currentmarker}{\pgfqpoint{0.000000in}{-0.048611in}}{\pgfqpoint{0.000000in}{0.000000in}}{%
\pgfpathmoveto{\pgfqpoint{0.000000in}{0.000000in}}%
\pgfpathlineto{\pgfqpoint{0.000000in}{-0.048611in}}%
\pgfusepath{stroke,fill}%
}%
\begin{pgfscope}%
\pgfsys@transformshift{4.115762in}{0.565123in}%
\pgfsys@useobject{currentmarker}{}%
\end{pgfscope}%
\end{pgfscope}%
\begin{pgfscope}%
\pgftext[x=4.115762in,y=0.467901in,,top]{\sffamily\fontsize{10.000000}{12.000000}\selectfont \(\displaystyle 140\)}%
\end{pgfscope}%
\begin{pgfscope}%
\pgfpathrectangle{\pgfqpoint{0.812440in}{0.565123in}}{\pgfqpoint{3.802560in}{2.249877in}} %
\pgfusepath{clip}%
\pgfsetrectcap%
\pgfsetroundjoin%
\pgfsetlinewidth{0.803000pt}%
\definecolor{currentstroke}{rgb}{0.690196,0.690196,0.690196}%
\pgfsetstrokecolor{currentstroke}%
\pgfsetdash{}{0pt}%
\pgfpathmoveto{\pgfqpoint{4.562974in}{0.565123in}}%
\pgfpathlineto{\pgfqpoint{4.562974in}{2.815000in}}%
\pgfusepath{stroke}%
\end{pgfscope}%
\begin{pgfscope}%
\pgfsetbuttcap%
\pgfsetroundjoin%
\definecolor{currentfill}{rgb}{0.000000,0.000000,0.000000}%
\pgfsetfillcolor{currentfill}%
\pgfsetlinewidth{0.803000pt}%
\definecolor{currentstroke}{rgb}{0.000000,0.000000,0.000000}%
\pgfsetstrokecolor{currentstroke}%
\pgfsetdash{}{0pt}%
\pgfsys@defobject{currentmarker}{\pgfqpoint{0.000000in}{-0.048611in}}{\pgfqpoint{0.000000in}{0.000000in}}{%
\pgfpathmoveto{\pgfqpoint{0.000000in}{0.000000in}}%
\pgfpathlineto{\pgfqpoint{0.000000in}{-0.048611in}}%
\pgfusepath{stroke,fill}%
}%
\begin{pgfscope}%
\pgfsys@transformshift{4.562974in}{0.565123in}%
\pgfsys@useobject{currentmarker}{}%
\end{pgfscope}%
\end{pgfscope}%
\begin{pgfscope}%
\pgftext[x=4.562974in,y=0.467901in,,top]{\sffamily\fontsize{10.000000}{12.000000}\selectfont \(\displaystyle 160\)}%
\end{pgfscope}%
\begin{pgfscope}%
\pgftext[x=2.713720in,y=0.288889in,,top]{\sffamily\fontsize{10.000000}{12.000000}\selectfont  \(\displaystyle \,t \, [\mathrm{s}]\)}%
\end{pgfscope}%
\begin{pgfscope}%
\pgfpathrectangle{\pgfqpoint{0.812440in}{0.565123in}}{\pgfqpoint{3.802560in}{2.249877in}} %
\pgfusepath{clip}%
\pgfsetrectcap%
\pgfsetroundjoin%
\pgfsetlinewidth{0.803000pt}%
\definecolor{currentstroke}{rgb}{0.690196,0.690196,0.690196}%
\pgfsetstrokecolor{currentstroke}%
\pgfsetdash{}{0pt}%
\pgfpathmoveto{\pgfqpoint{0.812440in}{0.763535in}}%
\pgfpathlineto{\pgfqpoint{4.615000in}{0.763535in}}%
\pgfusepath{stroke}%
\end{pgfscope}%
\begin{pgfscope}%
\pgfsetbuttcap%
\pgfsetroundjoin%
\definecolor{currentfill}{rgb}{0.000000,0.000000,0.000000}%
\pgfsetfillcolor{currentfill}%
\pgfsetlinewidth{0.803000pt}%
\definecolor{currentstroke}{rgb}{0.000000,0.000000,0.000000}%
\pgfsetstrokecolor{currentstroke}%
\pgfsetdash{}{0pt}%
\pgfsys@defobject{currentmarker}{\pgfqpoint{-0.048611in}{0.000000in}}{\pgfqpoint{0.000000in}{0.000000in}}{%
\pgfpathmoveto{\pgfqpoint{0.000000in}{0.000000in}}%
\pgfpathlineto{\pgfqpoint{-0.048611in}{0.000000in}}%
\pgfusepath{stroke,fill}%
}%
\begin{pgfscope}%
\pgfsys@transformshift{0.812440in}{0.763535in}%
\pgfsys@useobject{currentmarker}{}%
\end{pgfscope}%
\end{pgfscope}%
\begin{pgfscope}%
\pgftext[x=0.429723in,y=0.715310in,left,base]{\sffamily\fontsize{10.000000}{12.000000}\selectfont \(\displaystyle -3{,}5\)}%
\end{pgfscope}%
\begin{pgfscope}%
\pgfpathrectangle{\pgfqpoint{0.812440in}{0.565123in}}{\pgfqpoint{3.802560in}{2.249877in}} %
\pgfusepath{clip}%
\pgfsetrectcap%
\pgfsetroundjoin%
\pgfsetlinewidth{0.803000pt}%
\definecolor{currentstroke}{rgb}{0.690196,0.690196,0.690196}%
\pgfsetstrokecolor{currentstroke}%
\pgfsetdash{}{0pt}%
\pgfpathmoveto{\pgfqpoint{0.812440in}{1.047230in}}%
\pgfpathlineto{\pgfqpoint{4.615000in}{1.047230in}}%
\pgfusepath{stroke}%
\end{pgfscope}%
\begin{pgfscope}%
\pgfsetbuttcap%
\pgfsetroundjoin%
\definecolor{currentfill}{rgb}{0.000000,0.000000,0.000000}%
\pgfsetfillcolor{currentfill}%
\pgfsetlinewidth{0.803000pt}%
\definecolor{currentstroke}{rgb}{0.000000,0.000000,0.000000}%
\pgfsetstrokecolor{currentstroke}%
\pgfsetdash{}{0pt}%
\pgfsys@defobject{currentmarker}{\pgfqpoint{-0.048611in}{0.000000in}}{\pgfqpoint{0.000000in}{0.000000in}}{%
\pgfpathmoveto{\pgfqpoint{0.000000in}{0.000000in}}%
\pgfpathlineto{\pgfqpoint{-0.048611in}{0.000000in}}%
\pgfusepath{stroke,fill}%
}%
\begin{pgfscope}%
\pgfsys@transformshift{0.812440in}{1.047230in}%
\pgfsys@useobject{currentmarker}{}%
\end{pgfscope}%
\end{pgfscope}%
\begin{pgfscope}%
\pgftext[x=0.429723in,y=0.999005in,left,base]{\sffamily\fontsize{10.000000}{12.000000}\selectfont \(\displaystyle -3{,}0\)}%
\end{pgfscope}%
\begin{pgfscope}%
\pgfpathrectangle{\pgfqpoint{0.812440in}{0.565123in}}{\pgfqpoint{3.802560in}{2.249877in}} %
\pgfusepath{clip}%
\pgfsetrectcap%
\pgfsetroundjoin%
\pgfsetlinewidth{0.803000pt}%
\definecolor{currentstroke}{rgb}{0.690196,0.690196,0.690196}%
\pgfsetstrokecolor{currentstroke}%
\pgfsetdash{}{0pt}%
\pgfpathmoveto{\pgfqpoint{0.812440in}{1.330926in}}%
\pgfpathlineto{\pgfqpoint{4.615000in}{1.330926in}}%
\pgfusepath{stroke}%
\end{pgfscope}%
\begin{pgfscope}%
\pgfsetbuttcap%
\pgfsetroundjoin%
\definecolor{currentfill}{rgb}{0.000000,0.000000,0.000000}%
\pgfsetfillcolor{currentfill}%
\pgfsetlinewidth{0.803000pt}%
\definecolor{currentstroke}{rgb}{0.000000,0.000000,0.000000}%
\pgfsetstrokecolor{currentstroke}%
\pgfsetdash{}{0pt}%
\pgfsys@defobject{currentmarker}{\pgfqpoint{-0.048611in}{0.000000in}}{\pgfqpoint{0.000000in}{0.000000in}}{%
\pgfpathmoveto{\pgfqpoint{0.000000in}{0.000000in}}%
\pgfpathlineto{\pgfqpoint{-0.048611in}{0.000000in}}%
\pgfusepath{stroke,fill}%
}%
\begin{pgfscope}%
\pgfsys@transformshift{0.812440in}{1.330926in}%
\pgfsys@useobject{currentmarker}{}%
\end{pgfscope}%
\end{pgfscope}%
\begin{pgfscope}%
\pgftext[x=0.429723in,y=1.282700in,left,base]{\sffamily\fontsize{10.000000}{12.000000}\selectfont \(\displaystyle -2{,}5\)}%
\end{pgfscope}%
\begin{pgfscope}%
\pgfpathrectangle{\pgfqpoint{0.812440in}{0.565123in}}{\pgfqpoint{3.802560in}{2.249877in}} %
\pgfusepath{clip}%
\pgfsetrectcap%
\pgfsetroundjoin%
\pgfsetlinewidth{0.803000pt}%
\definecolor{currentstroke}{rgb}{0.690196,0.690196,0.690196}%
\pgfsetstrokecolor{currentstroke}%
\pgfsetdash{}{0pt}%
\pgfpathmoveto{\pgfqpoint{0.812440in}{1.614621in}}%
\pgfpathlineto{\pgfqpoint{4.615000in}{1.614621in}}%
\pgfusepath{stroke}%
\end{pgfscope}%
\begin{pgfscope}%
\pgfsetbuttcap%
\pgfsetroundjoin%
\definecolor{currentfill}{rgb}{0.000000,0.000000,0.000000}%
\pgfsetfillcolor{currentfill}%
\pgfsetlinewidth{0.803000pt}%
\definecolor{currentstroke}{rgb}{0.000000,0.000000,0.000000}%
\pgfsetstrokecolor{currentstroke}%
\pgfsetdash{}{0pt}%
\pgfsys@defobject{currentmarker}{\pgfqpoint{-0.048611in}{0.000000in}}{\pgfqpoint{0.000000in}{0.000000in}}{%
\pgfpathmoveto{\pgfqpoint{0.000000in}{0.000000in}}%
\pgfpathlineto{\pgfqpoint{-0.048611in}{0.000000in}}%
\pgfusepath{stroke,fill}%
}%
\begin{pgfscope}%
\pgfsys@transformshift{0.812440in}{1.614621in}%
\pgfsys@useobject{currentmarker}{}%
\end{pgfscope}%
\end{pgfscope}%
\begin{pgfscope}%
\pgftext[x=0.429723in,y=1.566396in,left,base]{\sffamily\fontsize{10.000000}{12.000000}\selectfont \(\displaystyle -2{,}0\)}%
\end{pgfscope}%
\begin{pgfscope}%
\pgfpathrectangle{\pgfqpoint{0.812440in}{0.565123in}}{\pgfqpoint{3.802560in}{2.249877in}} %
\pgfusepath{clip}%
\pgfsetrectcap%
\pgfsetroundjoin%
\pgfsetlinewidth{0.803000pt}%
\definecolor{currentstroke}{rgb}{0.690196,0.690196,0.690196}%
\pgfsetstrokecolor{currentstroke}%
\pgfsetdash{}{0pt}%
\pgfpathmoveto{\pgfqpoint{0.812440in}{1.898316in}}%
\pgfpathlineto{\pgfqpoint{4.615000in}{1.898316in}}%
\pgfusepath{stroke}%
\end{pgfscope}%
\begin{pgfscope}%
\pgfsetbuttcap%
\pgfsetroundjoin%
\definecolor{currentfill}{rgb}{0.000000,0.000000,0.000000}%
\pgfsetfillcolor{currentfill}%
\pgfsetlinewidth{0.803000pt}%
\definecolor{currentstroke}{rgb}{0.000000,0.000000,0.000000}%
\pgfsetstrokecolor{currentstroke}%
\pgfsetdash{}{0pt}%
\pgfsys@defobject{currentmarker}{\pgfqpoint{-0.048611in}{0.000000in}}{\pgfqpoint{0.000000in}{0.000000in}}{%
\pgfpathmoveto{\pgfqpoint{0.000000in}{0.000000in}}%
\pgfpathlineto{\pgfqpoint{-0.048611in}{0.000000in}}%
\pgfusepath{stroke,fill}%
}%
\begin{pgfscope}%
\pgfsys@transformshift{0.812440in}{1.898316in}%
\pgfsys@useobject{currentmarker}{}%
\end{pgfscope}%
\end{pgfscope}%
\begin{pgfscope}%
\pgftext[x=0.429723in,y=1.850091in,left,base]{\sffamily\fontsize{10.000000}{12.000000}\selectfont \(\displaystyle -1{,}5\)}%
\end{pgfscope}%
\begin{pgfscope}%
\pgfpathrectangle{\pgfqpoint{0.812440in}{0.565123in}}{\pgfqpoint{3.802560in}{2.249877in}} %
\pgfusepath{clip}%
\pgfsetrectcap%
\pgfsetroundjoin%
\pgfsetlinewidth{0.803000pt}%
\definecolor{currentstroke}{rgb}{0.690196,0.690196,0.690196}%
\pgfsetstrokecolor{currentstroke}%
\pgfsetdash{}{0pt}%
\pgfpathmoveto{\pgfqpoint{0.812440in}{2.182012in}}%
\pgfpathlineto{\pgfqpoint{4.615000in}{2.182012in}}%
\pgfusepath{stroke}%
\end{pgfscope}%
\begin{pgfscope}%
\pgfsetbuttcap%
\pgfsetroundjoin%
\definecolor{currentfill}{rgb}{0.000000,0.000000,0.000000}%
\pgfsetfillcolor{currentfill}%
\pgfsetlinewidth{0.803000pt}%
\definecolor{currentstroke}{rgb}{0.000000,0.000000,0.000000}%
\pgfsetstrokecolor{currentstroke}%
\pgfsetdash{}{0pt}%
\pgfsys@defobject{currentmarker}{\pgfqpoint{-0.048611in}{0.000000in}}{\pgfqpoint{0.000000in}{0.000000in}}{%
\pgfpathmoveto{\pgfqpoint{0.000000in}{0.000000in}}%
\pgfpathlineto{\pgfqpoint{-0.048611in}{0.000000in}}%
\pgfusepath{stroke,fill}%
}%
\begin{pgfscope}%
\pgfsys@transformshift{0.812440in}{2.182012in}%
\pgfsys@useobject{currentmarker}{}%
\end{pgfscope}%
\end{pgfscope}%
\begin{pgfscope}%
\pgftext[x=0.429723in,y=2.133786in,left,base]{\sffamily\fontsize{10.000000}{12.000000}\selectfont \(\displaystyle -1{,}0\)}%
\end{pgfscope}%
\begin{pgfscope}%
\pgfpathrectangle{\pgfqpoint{0.812440in}{0.565123in}}{\pgfqpoint{3.802560in}{2.249877in}} %
\pgfusepath{clip}%
\pgfsetrectcap%
\pgfsetroundjoin%
\pgfsetlinewidth{0.803000pt}%
\definecolor{currentstroke}{rgb}{0.690196,0.690196,0.690196}%
\pgfsetstrokecolor{currentstroke}%
\pgfsetdash{}{0pt}%
\pgfpathmoveto{\pgfqpoint{0.812440in}{2.465707in}}%
\pgfpathlineto{\pgfqpoint{4.615000in}{2.465707in}}%
\pgfusepath{stroke}%
\end{pgfscope}%
\begin{pgfscope}%
\pgfsetbuttcap%
\pgfsetroundjoin%
\definecolor{currentfill}{rgb}{0.000000,0.000000,0.000000}%
\pgfsetfillcolor{currentfill}%
\pgfsetlinewidth{0.803000pt}%
\definecolor{currentstroke}{rgb}{0.000000,0.000000,0.000000}%
\pgfsetstrokecolor{currentstroke}%
\pgfsetdash{}{0pt}%
\pgfsys@defobject{currentmarker}{\pgfqpoint{-0.048611in}{0.000000in}}{\pgfqpoint{0.000000in}{0.000000in}}{%
\pgfpathmoveto{\pgfqpoint{0.000000in}{0.000000in}}%
\pgfpathlineto{\pgfqpoint{-0.048611in}{0.000000in}}%
\pgfusepath{stroke,fill}%
}%
\begin{pgfscope}%
\pgfsys@transformshift{0.812440in}{2.465707in}%
\pgfsys@useobject{currentmarker}{}%
\end{pgfscope}%
\end{pgfscope}%
\begin{pgfscope}%
\pgftext[x=0.360278in,y=2.417482in,left,base]{\sffamily\fontsize{10.000000}{12.000000}\selectfont \(\displaystyle -0{,}50\)}%
\end{pgfscope}%
\begin{pgfscope}%
\pgfpathrectangle{\pgfqpoint{0.812440in}{0.565123in}}{\pgfqpoint{3.802560in}{2.249877in}} %
\pgfusepath{clip}%
\pgfsetrectcap%
\pgfsetroundjoin%
\pgfsetlinewidth{0.803000pt}%
\definecolor{currentstroke}{rgb}{0.690196,0.690196,0.690196}%
\pgfsetstrokecolor{currentstroke}%
\pgfsetdash{}{0pt}%
\pgfpathmoveto{\pgfqpoint{0.812440in}{2.749402in}}%
\pgfpathlineto{\pgfqpoint{4.615000in}{2.749402in}}%
\pgfusepath{stroke}%
\end{pgfscope}%
\begin{pgfscope}%
\pgfsetbuttcap%
\pgfsetroundjoin%
\definecolor{currentfill}{rgb}{0.000000,0.000000,0.000000}%
\pgfsetfillcolor{currentfill}%
\pgfsetlinewidth{0.803000pt}%
\definecolor{currentstroke}{rgb}{0.000000,0.000000,0.000000}%
\pgfsetstrokecolor{currentstroke}%
\pgfsetdash{}{0pt}%
\pgfsys@defobject{currentmarker}{\pgfqpoint{-0.048611in}{0.000000in}}{\pgfqpoint{0.000000in}{0.000000in}}{%
\pgfpathmoveto{\pgfqpoint{0.000000in}{0.000000in}}%
\pgfpathlineto{\pgfqpoint{-0.048611in}{0.000000in}}%
\pgfusepath{stroke,fill}%
}%
\begin{pgfscope}%
\pgfsys@transformshift{0.812440in}{2.749402in}%
\pgfsys@useobject{currentmarker}{}%
\end{pgfscope}%
\end{pgfscope}%
\begin{pgfscope}%
\pgftext[x=0.537748in,y=2.701177in,left,base]{\sffamily\fontsize{10.000000}{12.000000}\selectfont \(\displaystyle 0{,}0\)}%
\end{pgfscope}%
\begin{pgfscope}%
\pgftext[x=0.304723in,y=1.690062in,,bottom,rotate=90.000000]{\sffamily\fontsize{10.000000}{12.000000}\selectfont  \(\displaystyle \,a \, [\mathrm{m/s^2}]\)}%
\end{pgfscope}%
\begin{pgfscope}%
\pgfpathrectangle{\pgfqpoint{0.812440in}{0.565123in}}{\pgfqpoint{3.802560in}{2.249877in}} %
\pgfusepath{clip}%
\pgfsetrectcap%
\pgfsetroundjoin%
\pgfsetlinewidth{1.505625pt}%
\definecolor{currentstroke}{rgb}{0.121569,0.466667,0.705882}%
\pgfsetstrokecolor{currentstroke}%
\pgfsetdash{}{0pt}%
\pgfpathmoveto{\pgfqpoint{0.985283in}{1.555607in}}%
\pgfpathlineto{\pgfqpoint{0.985666in}{1.535235in}}%
\pgfpathlineto{\pgfqpoint{0.985666in}{1.535235in}}%
\pgfpathlineto{\pgfqpoint{0.985666in}{1.535235in}}%
\pgfpathlineto{\pgfqpoint{0.986435in}{1.559681in}}%
\pgfpathlineto{\pgfqpoint{0.986818in}{1.555607in}}%
\pgfpathlineto{\pgfqpoint{0.987201in}{1.555607in}}%
\pgfpathlineto{\pgfqpoint{0.988351in}{1.539309in}}%
\pgfpathlineto{\pgfqpoint{0.988734in}{1.539309in}}%
\pgfpathlineto{\pgfqpoint{0.990650in}{1.551532in}}%
\pgfpathlineto{\pgfqpoint{0.991415in}{1.535235in}}%
\pgfpathlineto{\pgfqpoint{0.991799in}{1.559681in}}%
\pgfpathlineto{\pgfqpoint{0.992567in}{1.551532in}}%
\pgfpathlineto{\pgfqpoint{0.992948in}{1.559681in}}%
\pgfpathlineto{\pgfqpoint{0.994096in}{1.535235in}}%
\pgfpathlineto{\pgfqpoint{0.994479in}{1.551532in}}%
\pgfpathlineto{\pgfqpoint{0.994863in}{1.543384in}}%
\pgfpathlineto{\pgfqpoint{0.995246in}{1.531161in}}%
\pgfpathlineto{\pgfqpoint{0.996012in}{1.539309in}}%
\pgfpathlineto{\pgfqpoint{0.996395in}{1.547458in}}%
\pgfpathlineto{\pgfqpoint{0.996395in}{1.547458in}}%
\pgfpathlineto{\pgfqpoint{0.996395in}{1.547458in}}%
\pgfpathlineto{\pgfqpoint{0.996778in}{1.535235in}}%
\pgfpathlineto{\pgfqpoint{0.997162in}{1.539309in}}%
\pgfpathlineto{\pgfqpoint{0.997929in}{1.567830in}}%
\pgfpathlineto{\pgfqpoint{0.998312in}{1.547458in}}%
\pgfpathlineto{\pgfqpoint{0.999078in}{1.535235in}}%
\pgfpathlineto{\pgfqpoint{0.999462in}{1.539309in}}%
\pgfpathlineto{\pgfqpoint{1.001378in}{1.559681in}}%
\pgfpathlineto{\pgfqpoint{1.001762in}{1.555607in}}%
\pgfpathlineto{\pgfqpoint{1.002527in}{1.523012in}}%
\pgfpathlineto{\pgfqpoint{1.002911in}{1.555607in}}%
\pgfpathlineto{\pgfqpoint{1.003677in}{1.531161in}}%
\pgfpathlineto{\pgfqpoint{1.004835in}{1.551532in}}%
\pgfpathlineto{\pgfqpoint{1.005218in}{1.539309in}}%
\pgfpathlineto{\pgfqpoint{1.005985in}{1.547458in}}%
\pgfpathlineto{\pgfqpoint{1.006751in}{1.535235in}}%
\pgfpathlineto{\pgfqpoint{1.007134in}{1.559681in}}%
\pgfpathlineto{\pgfqpoint{1.007517in}{1.543384in}}%
\pgfpathlineto{\pgfqpoint{1.007901in}{1.535235in}}%
\pgfpathlineto{\pgfqpoint{1.008283in}{1.543384in}}%
\pgfpathlineto{\pgfqpoint{1.009050in}{1.563756in}}%
\pgfpathlineto{\pgfqpoint{1.010583in}{1.535235in}}%
\pgfpathlineto{\pgfqpoint{1.012114in}{1.563756in}}%
\pgfpathlineto{\pgfqpoint{1.014116in}{1.543384in}}%
\pgfpathlineto{\pgfqpoint{1.014499in}{1.539309in}}%
\pgfpathlineto{\pgfqpoint{1.015649in}{1.551532in}}%
\pgfpathlineto{\pgfqpoint{1.016415in}{1.535235in}}%
\pgfpathlineto{\pgfqpoint{1.017182in}{1.563756in}}%
\pgfpathlineto{\pgfqpoint{1.017565in}{1.555607in}}%
\pgfpathlineto{\pgfqpoint{1.019098in}{1.535235in}}%
\pgfpathlineto{\pgfqpoint{1.019865in}{1.555607in}}%
\pgfpathlineto{\pgfqpoint{1.020249in}{1.539309in}}%
\pgfpathlineto{\pgfqpoint{1.020632in}{1.531161in}}%
\pgfpathlineto{\pgfqpoint{1.021014in}{1.563756in}}%
\pgfpathlineto{\pgfqpoint{1.021780in}{1.547458in}}%
\pgfpathlineto{\pgfqpoint{1.022931in}{1.563756in}}%
\pgfpathlineto{\pgfqpoint{1.024080in}{1.539309in}}%
\pgfpathlineto{\pgfqpoint{1.025995in}{1.567830in}}%
\pgfpathlineto{\pgfqpoint{1.027146in}{1.539309in}}%
\pgfpathlineto{\pgfqpoint{1.027530in}{1.555607in}}%
\pgfpathlineto{\pgfqpoint{1.028296in}{1.551532in}}%
\pgfpathlineto{\pgfqpoint{1.029445in}{1.555607in}}%
\pgfpathlineto{\pgfqpoint{1.029828in}{1.551532in}}%
\pgfpathlineto{\pgfqpoint{1.030978in}{1.559681in}}%
\pgfpathlineto{\pgfqpoint{1.032129in}{1.531161in}}%
\pgfpathlineto{\pgfqpoint{1.032511in}{1.563756in}}%
\pgfpathlineto{\pgfqpoint{1.033279in}{1.547458in}}%
\pgfpathlineto{\pgfqpoint{1.033663in}{1.547458in}}%
\pgfpathlineto{\pgfqpoint{1.034428in}{1.539309in}}%
\pgfpathlineto{\pgfqpoint{1.034811in}{1.543384in}}%
\pgfpathlineto{\pgfqpoint{1.035577in}{1.547458in}}%
\pgfpathlineto{\pgfqpoint{1.036345in}{1.535235in}}%
\pgfpathlineto{\pgfqpoint{1.037111in}{1.563756in}}%
\pgfpathlineto{\pgfqpoint{1.037494in}{1.555607in}}%
\pgfpathlineto{\pgfqpoint{1.037877in}{1.543384in}}%
\pgfpathlineto{\pgfqpoint{1.038261in}{1.547458in}}%
\pgfpathlineto{\pgfqpoint{1.039028in}{1.571904in}}%
\pgfpathlineto{\pgfqpoint{1.040561in}{1.527086in}}%
\pgfpathlineto{\pgfqpoint{1.042094in}{1.567830in}}%
\pgfpathlineto{\pgfqpoint{1.043627in}{1.531161in}}%
\pgfpathlineto{\pgfqpoint{1.045166in}{1.551532in}}%
\pgfpathlineto{\pgfqpoint{1.045550in}{1.531161in}}%
\pgfpathlineto{\pgfqpoint{1.045550in}{1.531161in}}%
\pgfpathlineto{\pgfqpoint{1.045550in}{1.531161in}}%
\pgfpathlineto{\pgfqpoint{1.045933in}{1.555607in}}%
\pgfpathlineto{\pgfqpoint{1.045933in}{1.555607in}}%
\pgfpathlineto{\pgfqpoint{1.045933in}{1.555607in}}%
\pgfpathlineto{\pgfqpoint{1.046317in}{1.523012in}}%
\pgfpathlineto{\pgfqpoint{1.047084in}{1.543384in}}%
\pgfpathlineto{\pgfqpoint{1.048232in}{1.551532in}}%
\pgfpathlineto{\pgfqpoint{1.048614in}{1.535235in}}%
\pgfpathlineto{\pgfqpoint{1.048997in}{1.539309in}}%
\pgfpathlineto{\pgfqpoint{1.049763in}{1.555607in}}%
\pgfpathlineto{\pgfqpoint{1.050145in}{1.535235in}}%
\pgfpathlineto{\pgfqpoint{1.050527in}{1.547458in}}%
\pgfpathlineto{\pgfqpoint{1.051376in}{1.563756in}}%
\pgfpathlineto{\pgfqpoint{1.051758in}{1.543384in}}%
\pgfpathlineto{\pgfqpoint{1.052141in}{1.555607in}}%
\pgfpathlineto{\pgfqpoint{1.052523in}{1.563756in}}%
\pgfpathlineto{\pgfqpoint{1.052906in}{1.555607in}}%
\pgfpathlineto{\pgfqpoint{1.053288in}{1.555607in}}%
\pgfpathlineto{\pgfqpoint{1.054438in}{1.531161in}}%
\pgfpathlineto{\pgfqpoint{1.055585in}{1.567830in}}%
\pgfpathlineto{\pgfqpoint{1.055967in}{1.551532in}}%
\pgfpathlineto{\pgfqpoint{1.057499in}{1.567830in}}%
\pgfpathlineto{\pgfqpoint{1.059031in}{1.539309in}}%
\pgfpathlineto{\pgfqpoint{1.059414in}{1.551532in}}%
\pgfpathlineto{\pgfqpoint{1.060179in}{1.543384in}}%
\pgfpathlineto{\pgfqpoint{1.060562in}{1.531161in}}%
\pgfpathlineto{\pgfqpoint{1.061327in}{1.539309in}}%
\pgfpathlineto{\pgfqpoint{1.061709in}{1.539309in}}%
\pgfpathlineto{\pgfqpoint{1.062092in}{1.551532in}}%
\pgfpathlineto{\pgfqpoint{1.062092in}{1.551532in}}%
\pgfpathlineto{\pgfqpoint{1.062092in}{1.551532in}}%
\pgfpathlineto{\pgfqpoint{1.062858in}{1.535235in}}%
\pgfpathlineto{\pgfqpoint{1.064008in}{1.559681in}}%
\pgfpathlineto{\pgfqpoint{1.064390in}{1.539309in}}%
\pgfpathlineto{\pgfqpoint{1.065155in}{1.543384in}}%
\pgfpathlineto{\pgfqpoint{1.065537in}{1.559681in}}%
\pgfpathlineto{\pgfqpoint{1.065920in}{1.543384in}}%
\pgfpathlineto{\pgfqpoint{1.066686in}{1.543384in}}%
\pgfpathlineto{\pgfqpoint{1.068217in}{1.559681in}}%
\pgfpathlineto{\pgfqpoint{1.068597in}{1.518937in}}%
\pgfpathlineto{\pgfqpoint{1.069362in}{1.543384in}}%
\pgfpathlineto{\pgfqpoint{1.070897in}{1.571904in}}%
\pgfpathlineto{\pgfqpoint{1.071280in}{1.567830in}}%
\pgfpathlineto{\pgfqpoint{1.071663in}{1.543384in}}%
\pgfpathlineto{\pgfqpoint{1.072430in}{1.559681in}}%
\pgfpathlineto{\pgfqpoint{1.073198in}{1.539309in}}%
\pgfpathlineto{\pgfqpoint{1.073580in}{1.555607in}}%
\pgfpathlineto{\pgfqpoint{1.073963in}{1.555607in}}%
\pgfpathlineto{\pgfqpoint{1.075113in}{1.543384in}}%
\pgfpathlineto{\pgfqpoint{1.075496in}{1.555607in}}%
\pgfpathlineto{\pgfqpoint{1.075965in}{1.543384in}}%
\pgfpathlineto{\pgfqpoint{1.076348in}{1.527086in}}%
\pgfpathlineto{\pgfqpoint{1.076731in}{1.543384in}}%
\pgfpathlineto{\pgfqpoint{1.077115in}{1.547458in}}%
\pgfpathlineto{\pgfqpoint{1.077498in}{1.527086in}}%
\pgfpathlineto{\pgfqpoint{1.077498in}{1.527086in}}%
\pgfpathlineto{\pgfqpoint{1.077498in}{1.527086in}}%
\pgfpathlineto{\pgfqpoint{1.077881in}{1.559681in}}%
\pgfpathlineto{\pgfqpoint{1.078647in}{1.551532in}}%
\pgfpathlineto{\pgfqpoint{1.079030in}{1.547458in}}%
\pgfpathlineto{\pgfqpoint{1.079414in}{1.555607in}}%
\pgfpathlineto{\pgfqpoint{1.079414in}{1.555607in}}%
\pgfpathlineto{\pgfqpoint{1.079414in}{1.555607in}}%
\pgfpathlineto{\pgfqpoint{1.080562in}{1.543384in}}%
\pgfpathlineto{\pgfqpoint{1.082095in}{1.559681in}}%
\pgfpathlineto{\pgfqpoint{1.082478in}{1.547458in}}%
\pgfpathlineto{\pgfqpoint{1.082478in}{1.547458in}}%
\pgfpathlineto{\pgfqpoint{1.082478in}{1.547458in}}%
\pgfpathlineto{\pgfqpoint{1.082861in}{1.563756in}}%
\pgfpathlineto{\pgfqpoint{1.083243in}{1.551532in}}%
\pgfpathlineto{\pgfqpoint{1.084018in}{1.535235in}}%
\pgfpathlineto{\pgfqpoint{1.084784in}{1.539309in}}%
\pgfpathlineto{\pgfqpoint{1.085166in}{1.559681in}}%
\pgfpathlineto{\pgfqpoint{1.085549in}{1.551532in}}%
\pgfpathlineto{\pgfqpoint{1.085931in}{1.523012in}}%
\pgfpathlineto{\pgfqpoint{1.086698in}{1.547458in}}%
\pgfpathlineto{\pgfqpoint{1.087847in}{1.539309in}}%
\pgfpathlineto{\pgfqpoint{1.089380in}{1.563756in}}%
\pgfpathlineto{\pgfqpoint{1.090529in}{1.535235in}}%
\pgfpathlineto{\pgfqpoint{1.090913in}{1.551532in}}%
\pgfpathlineto{\pgfqpoint{1.091295in}{1.539309in}}%
\pgfpathlineto{\pgfqpoint{1.091678in}{1.535235in}}%
\pgfpathlineto{\pgfqpoint{1.093213in}{1.563756in}}%
\pgfpathlineto{\pgfqpoint{1.093979in}{1.543384in}}%
\pgfpathlineto{\pgfqpoint{1.094361in}{1.580053in}}%
\pgfpathlineto{\pgfqpoint{1.094744in}{1.543384in}}%
\pgfpathlineto{\pgfqpoint{1.095129in}{1.535235in}}%
\pgfpathlineto{\pgfqpoint{1.095895in}{1.539309in}}%
\pgfpathlineto{\pgfqpoint{1.096278in}{1.563756in}}%
\pgfpathlineto{\pgfqpoint{1.097045in}{1.555607in}}%
\pgfpathlineto{\pgfqpoint{1.097812in}{1.539309in}}%
\pgfpathlineto{\pgfqpoint{1.098195in}{1.555607in}}%
\pgfpathlineto{\pgfqpoint{1.098960in}{1.551532in}}%
\pgfpathlineto{\pgfqpoint{1.099726in}{1.539309in}}%
\pgfpathlineto{\pgfqpoint{1.100109in}{1.563756in}}%
\pgfpathlineto{\pgfqpoint{1.100876in}{1.547458in}}%
\pgfpathlineto{\pgfqpoint{1.101259in}{1.559681in}}%
\pgfpathlineto{\pgfqpoint{1.101643in}{1.547458in}}%
\pgfpathlineto{\pgfqpoint{1.102026in}{1.547458in}}%
\pgfpathlineto{\pgfqpoint{1.102409in}{1.559681in}}%
\pgfpathlineto{\pgfqpoint{1.102792in}{1.547458in}}%
\pgfpathlineto{\pgfqpoint{1.103175in}{1.543384in}}%
\pgfpathlineto{\pgfqpoint{1.103559in}{1.547458in}}%
\pgfpathlineto{\pgfqpoint{1.103942in}{1.559681in}}%
\pgfpathlineto{\pgfqpoint{1.104325in}{1.547458in}}%
\pgfpathlineto{\pgfqpoint{1.105090in}{1.535235in}}%
\pgfpathlineto{\pgfqpoint{1.105856in}{1.551532in}}%
\pgfpathlineto{\pgfqpoint{1.106240in}{1.531161in}}%
\pgfpathlineto{\pgfqpoint{1.107007in}{1.547458in}}%
\pgfpathlineto{\pgfqpoint{1.107391in}{1.547458in}}%
\pgfpathlineto{\pgfqpoint{1.107774in}{1.527086in}}%
\pgfpathlineto{\pgfqpoint{1.107774in}{1.527086in}}%
\pgfpathlineto{\pgfqpoint{1.107774in}{1.527086in}}%
\pgfpathlineto{\pgfqpoint{1.108925in}{1.551532in}}%
\pgfpathlineto{\pgfqpoint{1.109307in}{1.543384in}}%
\pgfpathlineto{\pgfqpoint{1.109307in}{1.543384in}}%
\pgfpathlineto{\pgfqpoint{1.109307in}{1.543384in}}%
\pgfpathlineto{\pgfqpoint{1.109690in}{1.555607in}}%
\pgfpathlineto{\pgfqpoint{1.109690in}{1.555607in}}%
\pgfpathlineto{\pgfqpoint{1.109690in}{1.555607in}}%
\pgfpathlineto{\pgfqpoint{1.110073in}{1.539309in}}%
\pgfpathlineto{\pgfqpoint{1.110839in}{1.551532in}}%
\pgfpathlineto{\pgfqpoint{1.111222in}{1.543384in}}%
\pgfpathlineto{\pgfqpoint{1.111690in}{1.551532in}}%
\pgfpathlineto{\pgfqpoint{1.112073in}{1.555607in}}%
\pgfpathlineto{\pgfqpoint{1.112456in}{1.551532in}}%
\pgfpathlineto{\pgfqpoint{1.112840in}{1.543384in}}%
\pgfpathlineto{\pgfqpoint{1.113224in}{1.551532in}}%
\pgfpathlineto{\pgfqpoint{1.113607in}{1.559681in}}%
\pgfpathlineto{\pgfqpoint{1.113990in}{1.527086in}}%
\pgfpathlineto{\pgfqpoint{1.114756in}{1.535235in}}%
\pgfpathlineto{\pgfqpoint{1.115139in}{1.551532in}}%
\pgfpathlineto{\pgfqpoint{1.115139in}{1.551532in}}%
\pgfpathlineto{\pgfqpoint{1.115139in}{1.551532in}}%
\pgfpathlineto{\pgfqpoint{1.115524in}{1.531161in}}%
\pgfpathlineto{\pgfqpoint{1.116289in}{1.547458in}}%
\pgfpathlineto{\pgfqpoint{1.117437in}{1.555607in}}%
\pgfpathlineto{\pgfqpoint{1.118587in}{1.523012in}}%
\pgfpathlineto{\pgfqpoint{1.118970in}{1.535235in}}%
\pgfpathlineto{\pgfqpoint{1.119353in}{1.563756in}}%
\pgfpathlineto{\pgfqpoint{1.119353in}{1.563756in}}%
\pgfpathlineto{\pgfqpoint{1.119353in}{1.563756in}}%
\pgfpathlineto{\pgfqpoint{1.119736in}{1.531161in}}%
\pgfpathlineto{\pgfqpoint{1.120502in}{1.547458in}}%
\pgfpathlineto{\pgfqpoint{1.120885in}{1.543384in}}%
\pgfpathlineto{\pgfqpoint{1.121268in}{1.559681in}}%
\pgfpathlineto{\pgfqpoint{1.121651in}{1.551532in}}%
\pgfpathlineto{\pgfqpoint{1.122034in}{1.543384in}}%
\pgfpathlineto{\pgfqpoint{1.122417in}{1.547458in}}%
\pgfpathlineto{\pgfqpoint{1.122801in}{1.555607in}}%
\pgfpathlineto{\pgfqpoint{1.122801in}{1.555607in}}%
\pgfpathlineto{\pgfqpoint{1.122801in}{1.555607in}}%
\pgfpathlineto{\pgfqpoint{1.123184in}{1.543384in}}%
\pgfpathlineto{\pgfqpoint{1.123184in}{1.543384in}}%
\pgfpathlineto{\pgfqpoint{1.123184in}{1.543384in}}%
\pgfpathlineto{\pgfqpoint{1.123568in}{1.563756in}}%
\pgfpathlineto{\pgfqpoint{1.124342in}{1.547458in}}%
\pgfpathlineto{\pgfqpoint{1.125108in}{1.543384in}}%
\pgfpathlineto{\pgfqpoint{1.125492in}{1.555607in}}%
\pgfpathlineto{\pgfqpoint{1.125492in}{1.555607in}}%
\pgfpathlineto{\pgfqpoint{1.125492in}{1.555607in}}%
\pgfpathlineto{\pgfqpoint{1.126643in}{1.527086in}}%
\pgfpathlineto{\pgfqpoint{1.127794in}{1.547458in}}%
\pgfpathlineto{\pgfqpoint{1.128178in}{1.547458in}}%
\pgfpathlineto{\pgfqpoint{1.128560in}{1.543384in}}%
\pgfpathlineto{\pgfqpoint{1.128944in}{1.547458in}}%
\pgfpathlineto{\pgfqpoint{1.129710in}{1.559681in}}%
\pgfpathlineto{\pgfqpoint{1.130092in}{1.551532in}}%
\pgfpathlineto{\pgfqpoint{1.131244in}{1.535235in}}%
\pgfpathlineto{\pgfqpoint{1.132391in}{1.555607in}}%
\pgfpathlineto{\pgfqpoint{1.132773in}{1.535235in}}%
\pgfpathlineto{\pgfqpoint{1.133540in}{1.543384in}}%
\pgfpathlineto{\pgfqpoint{1.134689in}{1.555607in}}%
\pgfpathlineto{\pgfqpoint{1.135072in}{1.531161in}}%
\pgfpathlineto{\pgfqpoint{1.135838in}{1.551532in}}%
\pgfpathlineto{\pgfqpoint{1.136605in}{1.543384in}}%
\pgfpathlineto{\pgfqpoint{1.136989in}{1.547458in}}%
\pgfpathlineto{\pgfqpoint{1.137372in}{1.547458in}}%
\pgfpathlineto{\pgfqpoint{1.137755in}{1.543384in}}%
\pgfpathlineto{\pgfqpoint{1.138904in}{1.555607in}}%
\pgfpathlineto{\pgfqpoint{1.139671in}{1.535235in}}%
\pgfpathlineto{\pgfqpoint{1.140054in}{1.555607in}}%
\pgfpathlineto{\pgfqpoint{1.140820in}{1.547458in}}%
\pgfpathlineto{\pgfqpoint{1.141586in}{1.551532in}}%
\pgfpathlineto{\pgfqpoint{1.141970in}{1.531161in}}%
\pgfpathlineto{\pgfqpoint{1.142354in}{1.543384in}}%
\pgfpathlineto{\pgfqpoint{1.142737in}{1.575979in}}%
\pgfpathlineto{\pgfqpoint{1.143503in}{1.547458in}}%
\pgfpathlineto{\pgfqpoint{1.143885in}{1.535235in}}%
\pgfpathlineto{\pgfqpoint{1.144268in}{1.539309in}}%
\pgfpathlineto{\pgfqpoint{1.145419in}{1.551532in}}%
\pgfpathlineto{\pgfqpoint{1.145802in}{1.547458in}}%
\pgfpathlineto{\pgfqpoint{1.146185in}{1.563756in}}%
\pgfpathlineto{\pgfqpoint{1.146568in}{1.551532in}}%
\pgfpathlineto{\pgfqpoint{1.146951in}{1.543384in}}%
\pgfpathlineto{\pgfqpoint{1.147335in}{1.559681in}}%
\pgfpathlineto{\pgfqpoint{1.147335in}{1.559681in}}%
\pgfpathlineto{\pgfqpoint{1.147335in}{1.559681in}}%
\pgfpathlineto{\pgfqpoint{1.147718in}{1.531161in}}%
\pgfpathlineto{\pgfqpoint{1.148101in}{1.547458in}}%
\pgfpathlineto{\pgfqpoint{1.148485in}{1.563756in}}%
\pgfpathlineto{\pgfqpoint{1.148868in}{1.551532in}}%
\pgfpathlineto{\pgfqpoint{1.149336in}{1.543384in}}%
\pgfpathlineto{\pgfqpoint{1.149719in}{1.551532in}}%
\pgfpathlineto{\pgfqpoint{1.150104in}{1.551532in}}%
\pgfpathlineto{\pgfqpoint{1.150486in}{1.567830in}}%
\pgfpathlineto{\pgfqpoint{1.150486in}{1.567830in}}%
\pgfpathlineto{\pgfqpoint{1.150486in}{1.567830in}}%
\pgfpathlineto{\pgfqpoint{1.152022in}{1.527086in}}%
\pgfpathlineto{\pgfqpoint{1.153555in}{1.563756in}}%
\pgfpathlineto{\pgfqpoint{1.155088in}{1.543384in}}%
\pgfpathlineto{\pgfqpoint{1.155471in}{1.551532in}}%
\pgfpathlineto{\pgfqpoint{1.155471in}{1.551532in}}%
\pgfpathlineto{\pgfqpoint{1.155471in}{1.551532in}}%
\pgfpathlineto{\pgfqpoint{1.155853in}{1.539309in}}%
\pgfpathlineto{\pgfqpoint{1.155853in}{1.539309in}}%
\pgfpathlineto{\pgfqpoint{1.155853in}{1.539309in}}%
\pgfpathlineto{\pgfqpoint{1.156239in}{1.555607in}}%
\pgfpathlineto{\pgfqpoint{1.156623in}{1.539309in}}%
\pgfpathlineto{\pgfqpoint{1.157005in}{1.531161in}}%
\pgfpathlineto{\pgfqpoint{1.158154in}{1.559681in}}%
\pgfpathlineto{\pgfqpoint{1.158537in}{1.531161in}}%
\pgfpathlineto{\pgfqpoint{1.158921in}{1.551532in}}%
\pgfpathlineto{\pgfqpoint{1.159304in}{1.563756in}}%
\pgfpathlineto{\pgfqpoint{1.159304in}{1.563756in}}%
\pgfpathlineto{\pgfqpoint{1.159304in}{1.563756in}}%
\pgfpathlineto{\pgfqpoint{1.159688in}{1.543384in}}%
\pgfpathlineto{\pgfqpoint{1.160454in}{1.547458in}}%
\pgfpathlineto{\pgfqpoint{1.161221in}{1.575979in}}%
\pgfpathlineto{\pgfqpoint{1.161605in}{1.555607in}}%
\pgfpathlineto{\pgfqpoint{1.163138in}{1.539309in}}%
\pgfpathlineto{\pgfqpoint{1.163520in}{1.571904in}}%
\pgfpathlineto{\pgfqpoint{1.163912in}{1.539309in}}%
\pgfpathlineto{\pgfqpoint{1.164295in}{1.539309in}}%
\pgfpathlineto{\pgfqpoint{1.165829in}{1.551532in}}%
\pgfpathlineto{\pgfqpoint{1.166213in}{1.535235in}}%
\pgfpathlineto{\pgfqpoint{1.166596in}{1.551532in}}%
\pgfpathlineto{\pgfqpoint{1.166979in}{1.555607in}}%
\pgfpathlineto{\pgfqpoint{1.167361in}{1.531161in}}%
\pgfpathlineto{\pgfqpoint{1.168129in}{1.547458in}}%
\pgfpathlineto{\pgfqpoint{1.168512in}{1.543384in}}%
\pgfpathlineto{\pgfqpoint{1.168896in}{1.547458in}}%
\pgfpathlineto{\pgfqpoint{1.169279in}{1.547458in}}%
\pgfpathlineto{\pgfqpoint{1.169662in}{1.535235in}}%
\pgfpathlineto{\pgfqpoint{1.170045in}{1.547458in}}%
\pgfpathlineto{\pgfqpoint{1.170428in}{1.551532in}}%
\pgfpathlineto{\pgfqpoint{1.171195in}{1.535235in}}%
\pgfpathlineto{\pgfqpoint{1.171579in}{1.567830in}}%
\pgfpathlineto{\pgfqpoint{1.172345in}{1.539309in}}%
\pgfpathlineto{\pgfqpoint{1.172728in}{1.559681in}}%
\pgfpathlineto{\pgfqpoint{1.172728in}{1.559681in}}%
\pgfpathlineto{\pgfqpoint{1.172728in}{1.559681in}}%
\pgfpathlineto{\pgfqpoint{1.173110in}{1.531161in}}%
\pgfpathlineto{\pgfqpoint{1.173110in}{1.531161in}}%
\pgfpathlineto{\pgfqpoint{1.173110in}{1.531161in}}%
\pgfpathlineto{\pgfqpoint{1.173494in}{1.571904in}}%
\pgfpathlineto{\pgfqpoint{1.174260in}{1.547458in}}%
\pgfpathlineto{\pgfqpoint{1.175027in}{1.535235in}}%
\pgfpathlineto{\pgfqpoint{1.175410in}{1.543384in}}%
\pgfpathlineto{\pgfqpoint{1.175793in}{1.539309in}}%
\pgfpathlineto{\pgfqpoint{1.176562in}{1.559681in}}%
\pgfpathlineto{\pgfqpoint{1.176947in}{1.543384in}}%
\pgfpathlineto{\pgfqpoint{1.177330in}{1.539309in}}%
\pgfpathlineto{\pgfqpoint{1.177713in}{1.523012in}}%
\pgfpathlineto{\pgfqpoint{1.178096in}{1.563756in}}%
\pgfpathlineto{\pgfqpoint{1.178861in}{1.539309in}}%
\pgfpathlineto{\pgfqpoint{1.180013in}{1.559681in}}%
\pgfpathlineto{\pgfqpoint{1.180396in}{1.535235in}}%
\pgfpathlineto{\pgfqpoint{1.181162in}{1.543384in}}%
\pgfpathlineto{\pgfqpoint{1.181545in}{1.551532in}}%
\pgfpathlineto{\pgfqpoint{1.181928in}{1.547458in}}%
\pgfpathlineto{\pgfqpoint{1.182313in}{1.527086in}}%
\pgfpathlineto{\pgfqpoint{1.182313in}{1.527086in}}%
\pgfpathlineto{\pgfqpoint{1.182313in}{1.527086in}}%
\pgfpathlineto{\pgfqpoint{1.182695in}{1.555607in}}%
\pgfpathlineto{\pgfqpoint{1.183461in}{1.539309in}}%
\pgfpathlineto{\pgfqpoint{1.184996in}{1.551532in}}%
\pgfpathlineto{\pgfqpoint{1.186145in}{1.535235in}}%
\pgfpathlineto{\pgfqpoint{1.187679in}{1.563756in}}%
\pgfpathlineto{\pgfqpoint{1.188062in}{1.559681in}}%
\pgfpathlineto{\pgfqpoint{1.188445in}{1.547458in}}%
\pgfpathlineto{\pgfqpoint{1.189211in}{1.551532in}}%
\pgfpathlineto{\pgfqpoint{1.189595in}{1.555607in}}%
\pgfpathlineto{\pgfqpoint{1.189978in}{1.547458in}}%
\pgfpathlineto{\pgfqpoint{1.190746in}{1.551532in}}%
\pgfpathlineto{\pgfqpoint{1.191129in}{1.563756in}}%
\pgfpathlineto{\pgfqpoint{1.191513in}{1.559681in}}%
\pgfpathlineto{\pgfqpoint{1.192278in}{1.543384in}}%
\pgfpathlineto{\pgfqpoint{1.192662in}{1.559681in}}%
\pgfpathlineto{\pgfqpoint{1.193046in}{1.551532in}}%
\pgfpathlineto{\pgfqpoint{1.193428in}{1.543384in}}%
\pgfpathlineto{\pgfqpoint{1.194279in}{1.547458in}}%
\pgfpathlineto{\pgfqpoint{1.194663in}{1.547458in}}%
\pgfpathlineto{\pgfqpoint{1.195046in}{1.539309in}}%
\pgfpathlineto{\pgfqpoint{1.195046in}{1.539309in}}%
\pgfpathlineto{\pgfqpoint{1.195046in}{1.539309in}}%
\pgfpathlineto{\pgfqpoint{1.195812in}{1.551532in}}%
\pgfpathlineto{\pgfqpoint{1.196195in}{1.547458in}}%
\pgfpathlineto{\pgfqpoint{1.197344in}{1.535235in}}%
\pgfpathlineto{\pgfqpoint{1.197727in}{1.567830in}}%
\pgfpathlineto{\pgfqpoint{1.198492in}{1.547458in}}%
\pgfpathlineto{\pgfqpoint{1.199643in}{1.531161in}}%
\pgfpathlineto{\pgfqpoint{1.200027in}{1.531161in}}%
\pgfpathlineto{\pgfqpoint{1.201560in}{1.559681in}}%
\pgfpathlineto{\pgfqpoint{1.202710in}{1.539309in}}%
\pgfpathlineto{\pgfqpoint{1.203092in}{1.539309in}}%
\pgfpathlineto{\pgfqpoint{1.203475in}{1.555607in}}%
\pgfpathlineto{\pgfqpoint{1.204251in}{1.547458in}}%
\pgfpathlineto{\pgfqpoint{1.204634in}{1.539309in}}%
\pgfpathlineto{\pgfqpoint{1.205400in}{1.571904in}}%
\pgfpathlineto{\pgfqpoint{1.205783in}{1.563756in}}%
\pgfpathlineto{\pgfqpoint{1.207699in}{1.535235in}}%
\pgfpathlineto{\pgfqpoint{1.208848in}{1.547458in}}%
\pgfpathlineto{\pgfqpoint{1.209617in}{1.539309in}}%
\pgfpathlineto{\pgfqpoint{1.210765in}{1.559681in}}%
\pgfpathlineto{\pgfqpoint{1.211148in}{1.539309in}}%
\pgfpathlineto{\pgfqpoint{1.211148in}{1.539309in}}%
\pgfpathlineto{\pgfqpoint{1.211148in}{1.539309in}}%
\pgfpathlineto{\pgfqpoint{1.211531in}{1.575979in}}%
\pgfpathlineto{\pgfqpoint{1.212298in}{1.547458in}}%
\pgfpathlineto{\pgfqpoint{1.212682in}{1.527086in}}%
\pgfpathlineto{\pgfqpoint{1.213448in}{1.539309in}}%
\pgfpathlineto{\pgfqpoint{1.213832in}{1.531161in}}%
\pgfpathlineto{\pgfqpoint{1.215366in}{1.551532in}}%
\pgfpathlineto{\pgfqpoint{1.216515in}{1.539309in}}%
\pgfpathlineto{\pgfqpoint{1.216898in}{1.555607in}}%
\pgfpathlineto{\pgfqpoint{1.217282in}{1.543384in}}%
\pgfpathlineto{\pgfqpoint{1.218049in}{1.535235in}}%
\pgfpathlineto{\pgfqpoint{1.219581in}{1.559681in}}%
\pgfpathlineto{\pgfqpoint{1.219966in}{1.559681in}}%
\pgfpathlineto{\pgfqpoint{1.221116in}{1.543384in}}%
\pgfpathlineto{\pgfqpoint{1.221499in}{1.547458in}}%
\pgfpathlineto{\pgfqpoint{1.221882in}{1.543384in}}%
\pgfpathlineto{\pgfqpoint{1.222265in}{1.531161in}}%
\pgfpathlineto{\pgfqpoint{1.222649in}{1.559681in}}%
\pgfpathlineto{\pgfqpoint{1.223416in}{1.539309in}}%
\pgfpathlineto{\pgfqpoint{1.224565in}{1.559681in}}%
\pgfpathlineto{\pgfqpoint{1.224949in}{1.551532in}}%
\pgfpathlineto{\pgfqpoint{1.224949in}{1.551532in}}%
\pgfpathlineto{\pgfqpoint{1.224949in}{1.551532in}}%
\pgfpathlineto{\pgfqpoint{1.225332in}{1.563756in}}%
\pgfpathlineto{\pgfqpoint{1.225332in}{1.563756in}}%
\pgfpathlineto{\pgfqpoint{1.225332in}{1.563756in}}%
\pgfpathlineto{\pgfqpoint{1.225716in}{1.543384in}}%
\pgfpathlineto{\pgfqpoint{1.226482in}{1.555607in}}%
\pgfpathlineto{\pgfqpoint{1.227632in}{1.531161in}}%
\pgfpathlineto{\pgfqpoint{1.228782in}{1.559681in}}%
\pgfpathlineto{\pgfqpoint{1.229166in}{1.527086in}}%
\pgfpathlineto{\pgfqpoint{1.229549in}{1.547458in}}%
\pgfpathlineto{\pgfqpoint{1.229932in}{1.559681in}}%
\pgfpathlineto{\pgfqpoint{1.230315in}{1.551532in}}%
\pgfpathlineto{\pgfqpoint{1.230699in}{1.543384in}}%
\pgfpathlineto{\pgfqpoint{1.231083in}{1.559681in}}%
\pgfpathlineto{\pgfqpoint{1.231466in}{1.551532in}}%
\pgfpathlineto{\pgfqpoint{1.232998in}{1.527086in}}%
\pgfpathlineto{\pgfqpoint{1.233382in}{1.527086in}}%
\pgfpathlineto{\pgfqpoint{1.234149in}{1.555607in}}%
\pgfpathlineto{\pgfqpoint{1.234532in}{1.547458in}}%
\pgfpathlineto{\pgfqpoint{1.234915in}{1.539309in}}%
\pgfpathlineto{\pgfqpoint{1.235384in}{1.563756in}}%
\pgfpathlineto{\pgfqpoint{1.235767in}{1.543384in}}%
\pgfpathlineto{\pgfqpoint{1.236151in}{1.543384in}}%
\pgfpathlineto{\pgfqpoint{1.237302in}{1.559681in}}%
\pgfpathlineto{\pgfqpoint{1.238836in}{1.535235in}}%
\pgfpathlineto{\pgfqpoint{1.240370in}{1.559681in}}%
\pgfpathlineto{\pgfqpoint{1.240753in}{1.555607in}}%
\pgfpathlineto{\pgfqpoint{1.241136in}{1.539309in}}%
\pgfpathlineto{\pgfqpoint{1.241519in}{1.543384in}}%
\pgfpathlineto{\pgfqpoint{1.241903in}{1.555607in}}%
\pgfpathlineto{\pgfqpoint{1.242670in}{1.547458in}}%
\pgfpathlineto{\pgfqpoint{1.243054in}{1.539309in}}%
\pgfpathlineto{\pgfqpoint{1.243437in}{1.547458in}}%
\pgfpathlineto{\pgfqpoint{1.243829in}{1.551532in}}%
\pgfpathlineto{\pgfqpoint{1.244978in}{1.543384in}}%
\pgfpathlineto{\pgfqpoint{1.245362in}{1.547458in}}%
\pgfpathlineto{\pgfqpoint{1.245746in}{1.563756in}}%
\pgfpathlineto{\pgfqpoint{1.245746in}{1.563756in}}%
\pgfpathlineto{\pgfqpoint{1.245746in}{1.563756in}}%
\pgfpathlineto{\pgfqpoint{1.246129in}{1.535235in}}%
\pgfpathlineto{\pgfqpoint{1.246895in}{1.543384in}}%
\pgfpathlineto{\pgfqpoint{1.247279in}{1.567830in}}%
\pgfpathlineto{\pgfqpoint{1.247662in}{1.543384in}}%
\pgfpathlineto{\pgfqpoint{1.248045in}{1.531161in}}%
\pgfpathlineto{\pgfqpoint{1.248812in}{1.563756in}}%
\pgfpathlineto{\pgfqpoint{1.249195in}{1.551532in}}%
\pgfpathlineto{\pgfqpoint{1.250343in}{1.527086in}}%
\pgfpathlineto{\pgfqpoint{1.250726in}{1.535235in}}%
\pgfpathlineto{\pgfqpoint{1.251108in}{1.539309in}}%
\pgfpathlineto{\pgfqpoint{1.252643in}{1.567830in}}%
\pgfpathlineto{\pgfqpoint{1.253026in}{1.563756in}}%
\pgfpathlineto{\pgfqpoint{1.254559in}{1.543384in}}%
\pgfpathlineto{\pgfqpoint{1.254942in}{1.547458in}}%
\pgfpathlineto{\pgfqpoint{1.255325in}{1.539309in}}%
\pgfpathlineto{\pgfqpoint{1.255325in}{1.539309in}}%
\pgfpathlineto{\pgfqpoint{1.255325in}{1.539309in}}%
\pgfpathlineto{\pgfqpoint{1.255708in}{1.551532in}}%
\pgfpathlineto{\pgfqpoint{1.256474in}{1.543384in}}%
\pgfpathlineto{\pgfqpoint{1.258007in}{1.559681in}}%
\pgfpathlineto{\pgfqpoint{1.259156in}{1.539309in}}%
\pgfpathlineto{\pgfqpoint{1.259539in}{1.555607in}}%
\pgfpathlineto{\pgfqpoint{1.259539in}{1.555607in}}%
\pgfpathlineto{\pgfqpoint{1.259539in}{1.555607in}}%
\pgfpathlineto{\pgfqpoint{1.259922in}{1.531161in}}%
\pgfpathlineto{\pgfqpoint{1.259922in}{1.531161in}}%
\pgfpathlineto{\pgfqpoint{1.259922in}{1.531161in}}%
\pgfpathlineto{\pgfqpoint{1.261072in}{1.567830in}}%
\pgfpathlineto{\pgfqpoint{1.261838in}{1.539309in}}%
\pgfpathlineto{\pgfqpoint{1.262221in}{1.547458in}}%
\pgfpathlineto{\pgfqpoint{1.262604in}{1.559681in}}%
\pgfpathlineto{\pgfqpoint{1.262987in}{1.551532in}}%
\pgfpathlineto{\pgfqpoint{1.264137in}{1.527086in}}%
\pgfpathlineto{\pgfqpoint{1.264902in}{1.559681in}}%
\pgfpathlineto{\pgfqpoint{1.265670in}{1.551532in}}%
\pgfpathlineto{\pgfqpoint{1.266053in}{1.539309in}}%
\pgfpathlineto{\pgfqpoint{1.266435in}{1.543384in}}%
\pgfpathlineto{\pgfqpoint{1.266818in}{1.575979in}}%
\pgfpathlineto{\pgfqpoint{1.267583in}{1.559681in}}%
\pgfpathlineto{\pgfqpoint{1.269501in}{1.531161in}}%
\pgfpathlineto{\pgfqpoint{1.271034in}{1.559681in}}%
\pgfpathlineto{\pgfqpoint{1.272183in}{1.539309in}}%
\pgfpathlineto{\pgfqpoint{1.272567in}{1.551532in}}%
\pgfpathlineto{\pgfqpoint{1.272949in}{1.563756in}}%
\pgfpathlineto{\pgfqpoint{1.273332in}{1.527086in}}%
\pgfpathlineto{\pgfqpoint{1.274097in}{1.555607in}}%
\pgfpathlineto{\pgfqpoint{1.276099in}{1.535235in}}%
\pgfpathlineto{\pgfqpoint{1.276865in}{1.551532in}}%
\pgfpathlineto{\pgfqpoint{1.277249in}{1.527086in}}%
\pgfpathlineto{\pgfqpoint{1.277632in}{1.535235in}}%
\pgfpathlineto{\pgfqpoint{1.278015in}{1.559681in}}%
\pgfpathlineto{\pgfqpoint{1.278015in}{1.559681in}}%
\pgfpathlineto{\pgfqpoint{1.278015in}{1.559681in}}%
\pgfpathlineto{\pgfqpoint{1.278398in}{1.527086in}}%
\pgfpathlineto{\pgfqpoint{1.279164in}{1.543384in}}%
\pgfpathlineto{\pgfqpoint{1.279547in}{1.539309in}}%
\pgfpathlineto{\pgfqpoint{1.279930in}{1.563756in}}%
\pgfpathlineto{\pgfqpoint{1.280698in}{1.547458in}}%
\pgfpathlineto{\pgfqpoint{1.281081in}{1.551532in}}%
\pgfpathlineto{\pgfqpoint{1.281850in}{1.543384in}}%
\pgfpathlineto{\pgfqpoint{1.282233in}{1.559681in}}%
\pgfpathlineto{\pgfqpoint{1.282999in}{1.555607in}}%
\pgfpathlineto{\pgfqpoint{1.283383in}{1.535235in}}%
\pgfpathlineto{\pgfqpoint{1.284157in}{1.547458in}}%
\pgfpathlineto{\pgfqpoint{1.284541in}{1.543384in}}%
\pgfpathlineto{\pgfqpoint{1.284924in}{1.563756in}}%
\pgfpathlineto{\pgfqpoint{1.285690in}{1.551532in}}%
\pgfpathlineto{\pgfqpoint{1.286457in}{1.535235in}}%
\pgfpathlineto{\pgfqpoint{1.286840in}{1.559681in}}%
\pgfpathlineto{\pgfqpoint{1.287224in}{1.535235in}}%
\pgfpathlineto{\pgfqpoint{1.287607in}{1.535235in}}%
\pgfpathlineto{\pgfqpoint{1.287992in}{1.527086in}}%
\pgfpathlineto{\pgfqpoint{1.289141in}{1.575979in}}%
\pgfpathlineto{\pgfqpoint{1.289524in}{1.563756in}}%
\pgfpathlineto{\pgfqpoint{1.290291in}{1.531161in}}%
\pgfpathlineto{\pgfqpoint{1.291059in}{1.539309in}}%
\pgfpathlineto{\pgfqpoint{1.291442in}{1.551532in}}%
\pgfpathlineto{\pgfqpoint{1.292207in}{1.543384in}}%
\pgfpathlineto{\pgfqpoint{1.292591in}{1.543384in}}%
\pgfpathlineto{\pgfqpoint{1.292976in}{1.555607in}}%
\pgfpathlineto{\pgfqpoint{1.293742in}{1.547458in}}%
\pgfpathlineto{\pgfqpoint{1.294125in}{1.547458in}}%
\pgfpathlineto{\pgfqpoint{1.294508in}{1.535235in}}%
\pgfpathlineto{\pgfqpoint{1.294508in}{1.535235in}}%
\pgfpathlineto{\pgfqpoint{1.294508in}{1.535235in}}%
\pgfpathlineto{\pgfqpoint{1.295659in}{1.551532in}}%
\pgfpathlineto{\pgfqpoint{1.296042in}{1.547458in}}%
\pgfpathlineto{\pgfqpoint{1.296425in}{1.555607in}}%
\pgfpathlineto{\pgfqpoint{1.296808in}{1.531161in}}%
\pgfpathlineto{\pgfqpoint{1.297191in}{1.539309in}}%
\pgfpathlineto{\pgfqpoint{1.297574in}{1.555607in}}%
\pgfpathlineto{\pgfqpoint{1.298341in}{1.547458in}}%
\pgfpathlineto{\pgfqpoint{1.298725in}{1.555607in}}%
\pgfpathlineto{\pgfqpoint{1.299490in}{1.531161in}}%
\pgfpathlineto{\pgfqpoint{1.299874in}{1.535235in}}%
\pgfpathlineto{\pgfqpoint{1.301407in}{1.559681in}}%
\pgfpathlineto{\pgfqpoint{1.302173in}{1.539309in}}%
\pgfpathlineto{\pgfqpoint{1.302556in}{1.551532in}}%
\pgfpathlineto{\pgfqpoint{1.302939in}{1.551532in}}%
\pgfpathlineto{\pgfqpoint{1.303322in}{1.563756in}}%
\pgfpathlineto{\pgfqpoint{1.303706in}{1.518937in}}%
\pgfpathlineto{\pgfqpoint{1.303706in}{1.518937in}}%
\pgfpathlineto{\pgfqpoint{1.303706in}{1.518937in}}%
\pgfpathlineto{\pgfqpoint{1.304090in}{1.571904in}}%
\pgfpathlineto{\pgfqpoint{1.304857in}{1.555607in}}%
\pgfpathlineto{\pgfqpoint{1.305623in}{1.539309in}}%
\pgfpathlineto{\pgfqpoint{1.306005in}{1.551532in}}%
\pgfpathlineto{\pgfqpoint{1.307156in}{1.555607in}}%
\pgfpathlineto{\pgfqpoint{1.308688in}{1.531161in}}%
\pgfpathlineto{\pgfqpoint{1.309072in}{1.547458in}}%
\pgfpathlineto{\pgfqpoint{1.309454in}{1.531161in}}%
\pgfpathlineto{\pgfqpoint{1.310220in}{1.531161in}}%
\pgfpathlineto{\pgfqpoint{1.311753in}{1.559681in}}%
\pgfpathlineto{\pgfqpoint{1.312136in}{1.543384in}}%
\pgfpathlineto{\pgfqpoint{1.312520in}{1.551532in}}%
\pgfpathlineto{\pgfqpoint{1.312903in}{1.567830in}}%
\pgfpathlineto{\pgfqpoint{1.313285in}{1.559681in}}%
\pgfpathlineto{\pgfqpoint{1.313668in}{1.551532in}}%
\pgfpathlineto{\pgfqpoint{1.314051in}{1.567830in}}%
\pgfpathlineto{\pgfqpoint{1.314051in}{1.567830in}}%
\pgfpathlineto{\pgfqpoint{1.314051in}{1.567830in}}%
\pgfpathlineto{\pgfqpoint{1.315201in}{1.539309in}}%
\pgfpathlineto{\pgfqpoint{1.315584in}{1.559681in}}%
\pgfpathlineto{\pgfqpoint{1.315584in}{1.559681in}}%
\pgfpathlineto{\pgfqpoint{1.315584in}{1.559681in}}%
\pgfpathlineto{\pgfqpoint{1.316052in}{1.531161in}}%
\pgfpathlineto{\pgfqpoint{1.316435in}{1.551532in}}%
\pgfpathlineto{\pgfqpoint{1.316818in}{1.559681in}}%
\pgfpathlineto{\pgfqpoint{1.317584in}{1.539309in}}%
\pgfpathlineto{\pgfqpoint{1.317966in}{1.543384in}}%
\pgfpathlineto{\pgfqpoint{1.318733in}{1.559681in}}%
\pgfpathlineto{\pgfqpoint{1.319116in}{1.547458in}}%
\pgfpathlineto{\pgfqpoint{1.319499in}{1.531161in}}%
\pgfpathlineto{\pgfqpoint{1.319499in}{1.531161in}}%
\pgfpathlineto{\pgfqpoint{1.319499in}{1.531161in}}%
\pgfpathlineto{\pgfqpoint{1.319882in}{1.559681in}}%
\pgfpathlineto{\pgfqpoint{1.320647in}{1.555607in}}%
\pgfpathlineto{\pgfqpoint{1.321797in}{1.539309in}}%
\pgfpathlineto{\pgfqpoint{1.322180in}{1.547458in}}%
\pgfpathlineto{\pgfqpoint{1.322180in}{1.547458in}}%
\pgfpathlineto{\pgfqpoint{1.322180in}{1.547458in}}%
\pgfpathlineto{\pgfqpoint{1.323719in}{1.527086in}}%
\pgfpathlineto{\pgfqpoint{1.324486in}{1.571904in}}%
\pgfpathlineto{\pgfqpoint{1.324869in}{1.551532in}}%
\pgfpathlineto{\pgfqpoint{1.325635in}{1.527086in}}%
\pgfpathlineto{\pgfqpoint{1.326018in}{1.559681in}}%
\pgfpathlineto{\pgfqpoint{1.326785in}{1.551532in}}%
\pgfpathlineto{\pgfqpoint{1.327551in}{1.547458in}}%
\pgfpathlineto{\pgfqpoint{1.328317in}{1.563756in}}%
\pgfpathlineto{\pgfqpoint{1.329850in}{1.527086in}}%
\pgfpathlineto{\pgfqpoint{1.330232in}{1.563756in}}%
\pgfpathlineto{\pgfqpoint{1.330998in}{1.539309in}}%
\pgfpathlineto{\pgfqpoint{1.331381in}{1.543384in}}%
\pgfpathlineto{\pgfqpoint{1.331764in}{1.559681in}}%
\pgfpathlineto{\pgfqpoint{1.332148in}{1.547458in}}%
\pgfpathlineto{\pgfqpoint{1.332914in}{1.543384in}}%
\pgfpathlineto{\pgfqpoint{1.333297in}{1.563756in}}%
\pgfpathlineto{\pgfqpoint{1.333681in}{1.543384in}}%
\pgfpathlineto{\pgfqpoint{1.334064in}{1.543384in}}%
\pgfpathlineto{\pgfqpoint{1.334447in}{1.527086in}}%
\pgfpathlineto{\pgfqpoint{1.335213in}{1.539309in}}%
\pgfpathlineto{\pgfqpoint{1.336363in}{1.551532in}}%
\pgfpathlineto{\pgfqpoint{1.337512in}{1.543384in}}%
\pgfpathlineto{\pgfqpoint{1.337896in}{1.555607in}}%
\pgfpathlineto{\pgfqpoint{1.338663in}{1.551532in}}%
\pgfpathlineto{\pgfqpoint{1.339045in}{1.543384in}}%
\pgfpathlineto{\pgfqpoint{1.339810in}{1.547458in}}%
\pgfpathlineto{\pgfqpoint{1.340576in}{1.539309in}}%
\pgfpathlineto{\pgfqpoint{1.341728in}{1.567830in}}%
\pgfpathlineto{\pgfqpoint{1.342111in}{1.531161in}}%
\pgfpathlineto{\pgfqpoint{1.342876in}{1.535235in}}%
\pgfpathlineto{\pgfqpoint{1.344410in}{1.555607in}}%
\pgfpathlineto{\pgfqpoint{1.344793in}{1.543384in}}%
\pgfpathlineto{\pgfqpoint{1.345559in}{1.551532in}}%
\pgfpathlineto{\pgfqpoint{1.345943in}{1.551532in}}%
\pgfpathlineto{\pgfqpoint{1.346326in}{1.547458in}}%
\pgfpathlineto{\pgfqpoint{1.346710in}{1.527086in}}%
\pgfpathlineto{\pgfqpoint{1.346710in}{1.527086in}}%
\pgfpathlineto{\pgfqpoint{1.346710in}{1.527086in}}%
\pgfpathlineto{\pgfqpoint{1.347860in}{1.555607in}}%
\pgfpathlineto{\pgfqpoint{1.349393in}{1.543384in}}%
\pgfpathlineto{\pgfqpoint{1.349776in}{1.547458in}}%
\pgfpathlineto{\pgfqpoint{1.350160in}{1.539309in}}%
\pgfpathlineto{\pgfqpoint{1.350160in}{1.539309in}}%
\pgfpathlineto{\pgfqpoint{1.350160in}{1.539309in}}%
\pgfpathlineto{\pgfqpoint{1.350925in}{1.555607in}}%
\pgfpathlineto{\pgfqpoint{1.351393in}{1.543384in}}%
\pgfpathlineto{\pgfqpoint{1.352159in}{1.551532in}}%
\pgfpathlineto{\pgfqpoint{1.352543in}{1.543384in}}%
\pgfpathlineto{\pgfqpoint{1.352927in}{1.563756in}}%
\pgfpathlineto{\pgfqpoint{1.352927in}{1.563756in}}%
\pgfpathlineto{\pgfqpoint{1.352927in}{1.563756in}}%
\pgfpathlineto{\pgfqpoint{1.354076in}{1.539309in}}%
\pgfpathlineto{\pgfqpoint{1.355226in}{1.551532in}}%
\pgfpathlineto{\pgfqpoint{1.355609in}{1.523012in}}%
\pgfpathlineto{\pgfqpoint{1.355992in}{1.543384in}}%
\pgfpathlineto{\pgfqpoint{1.356375in}{1.559681in}}%
\pgfpathlineto{\pgfqpoint{1.357140in}{1.555607in}}%
\pgfpathlineto{\pgfqpoint{1.357523in}{1.539309in}}%
\pgfpathlineto{\pgfqpoint{1.358289in}{1.551532in}}%
\pgfpathlineto{\pgfqpoint{1.358672in}{1.563756in}}%
\pgfpathlineto{\pgfqpoint{1.359056in}{1.551532in}}%
\pgfpathlineto{\pgfqpoint{1.360206in}{1.539309in}}%
\pgfpathlineto{\pgfqpoint{1.360589in}{1.543384in}}%
\pgfpathlineto{\pgfqpoint{1.360973in}{1.559681in}}%
\pgfpathlineto{\pgfqpoint{1.361355in}{1.523012in}}%
\pgfpathlineto{\pgfqpoint{1.362122in}{1.551532in}}%
\pgfpathlineto{\pgfqpoint{1.362505in}{1.555607in}}%
\pgfpathlineto{\pgfqpoint{1.363271in}{1.523012in}}%
\pgfpathlineto{\pgfqpoint{1.364428in}{1.535235in}}%
\pgfpathlineto{\pgfqpoint{1.364811in}{1.551532in}}%
\pgfpathlineto{\pgfqpoint{1.364811in}{1.551532in}}%
\pgfpathlineto{\pgfqpoint{1.364811in}{1.551532in}}%
\pgfpathlineto{\pgfqpoint{1.365192in}{1.531161in}}%
\pgfpathlineto{\pgfqpoint{1.365575in}{1.539309in}}%
\pgfpathlineto{\pgfqpoint{1.365960in}{1.563756in}}%
\pgfpathlineto{\pgfqpoint{1.366726in}{1.551532in}}%
\pgfpathlineto{\pgfqpoint{1.367492in}{1.551532in}}%
\pgfpathlineto{\pgfqpoint{1.368259in}{1.563756in}}%
\pgfpathlineto{\pgfqpoint{1.368643in}{1.535235in}}%
\pgfpathlineto{\pgfqpoint{1.369408in}{1.547458in}}%
\pgfpathlineto{\pgfqpoint{1.369791in}{1.555607in}}%
\pgfpathlineto{\pgfqpoint{1.370173in}{1.523012in}}%
\pgfpathlineto{\pgfqpoint{1.370940in}{1.547458in}}%
\pgfpathlineto{\pgfqpoint{1.371324in}{1.551532in}}%
\pgfpathlineto{\pgfqpoint{1.371707in}{1.543384in}}%
\pgfpathlineto{\pgfqpoint{1.372473in}{1.559681in}}%
\pgfpathlineto{\pgfqpoint{1.372857in}{1.535235in}}%
\pgfpathlineto{\pgfqpoint{1.373623in}{1.543384in}}%
\pgfpathlineto{\pgfqpoint{1.374007in}{1.539309in}}%
\pgfpathlineto{\pgfqpoint{1.374391in}{1.551532in}}%
\pgfpathlineto{\pgfqpoint{1.375157in}{1.547458in}}%
\pgfpathlineto{\pgfqpoint{1.375540in}{1.547458in}}%
\pgfpathlineto{\pgfqpoint{1.375922in}{1.559681in}}%
\pgfpathlineto{\pgfqpoint{1.376305in}{1.531161in}}%
\pgfpathlineto{\pgfqpoint{1.377073in}{1.555607in}}%
\pgfpathlineto{\pgfqpoint{1.377457in}{1.531161in}}%
\pgfpathlineto{\pgfqpoint{1.377457in}{1.531161in}}%
\pgfpathlineto{\pgfqpoint{1.377457in}{1.531161in}}%
\pgfpathlineto{\pgfqpoint{1.377840in}{1.559681in}}%
\pgfpathlineto{\pgfqpoint{1.378606in}{1.539309in}}%
\pgfpathlineto{\pgfqpoint{1.378989in}{1.527086in}}%
\pgfpathlineto{\pgfqpoint{1.378989in}{1.527086in}}%
\pgfpathlineto{\pgfqpoint{1.378989in}{1.527086in}}%
\pgfpathlineto{\pgfqpoint{1.379756in}{1.555607in}}%
\pgfpathlineto{\pgfqpoint{1.380140in}{1.547458in}}%
\pgfpathlineto{\pgfqpoint{1.380523in}{1.531161in}}%
\pgfpathlineto{\pgfqpoint{1.380523in}{1.531161in}}%
\pgfpathlineto{\pgfqpoint{1.380523in}{1.531161in}}%
\pgfpathlineto{\pgfqpoint{1.381672in}{1.563756in}}%
\pgfpathlineto{\pgfqpoint{1.383972in}{1.531161in}}%
\pgfpathlineto{\pgfqpoint{1.385122in}{1.559681in}}%
\pgfpathlineto{\pgfqpoint{1.385888in}{1.523012in}}%
\pgfpathlineto{\pgfqpoint{1.386271in}{1.555607in}}%
\pgfpathlineto{\pgfqpoint{1.387036in}{1.551532in}}%
\pgfpathlineto{\pgfqpoint{1.387418in}{1.567830in}}%
\pgfpathlineto{\pgfqpoint{1.388270in}{1.559681in}}%
\pgfpathlineto{\pgfqpoint{1.389420in}{1.535235in}}%
\pgfpathlineto{\pgfqpoint{1.389803in}{1.547458in}}%
\pgfpathlineto{\pgfqpoint{1.390186in}{1.543384in}}%
\pgfpathlineto{\pgfqpoint{1.390568in}{1.551532in}}%
\pgfpathlineto{\pgfqpoint{1.390951in}{1.535235in}}%
\pgfpathlineto{\pgfqpoint{1.390951in}{1.535235in}}%
\pgfpathlineto{\pgfqpoint{1.390951in}{1.535235in}}%
\pgfpathlineto{\pgfqpoint{1.391334in}{1.559681in}}%
\pgfpathlineto{\pgfqpoint{1.391334in}{1.559681in}}%
\pgfpathlineto{\pgfqpoint{1.391334in}{1.559681in}}%
\pgfpathlineto{\pgfqpoint{1.391718in}{1.531161in}}%
\pgfpathlineto{\pgfqpoint{1.392485in}{1.551532in}}%
\pgfpathlineto{\pgfqpoint{1.392868in}{1.559681in}}%
\pgfpathlineto{\pgfqpoint{1.393251in}{1.531161in}}%
\pgfpathlineto{\pgfqpoint{1.394018in}{1.539309in}}%
\pgfpathlineto{\pgfqpoint{1.395168in}{1.547458in}}%
\pgfpathlineto{\pgfqpoint{1.395935in}{1.539309in}}%
\pgfpathlineto{\pgfqpoint{1.396318in}{1.563756in}}%
\pgfpathlineto{\pgfqpoint{1.396701in}{1.551532in}}%
\pgfpathlineto{\pgfqpoint{1.397085in}{1.539309in}}%
\pgfpathlineto{\pgfqpoint{1.397468in}{1.547458in}}%
\pgfpathlineto{\pgfqpoint{1.398235in}{1.559681in}}%
\pgfpathlineto{\pgfqpoint{1.398618in}{1.555607in}}%
\pgfpathlineto{\pgfqpoint{1.400151in}{1.543384in}}%
\pgfpathlineto{\pgfqpoint{1.400535in}{1.567830in}}%
\pgfpathlineto{\pgfqpoint{1.401301in}{1.559681in}}%
\pgfpathlineto{\pgfqpoint{1.402066in}{1.535235in}}%
\pgfpathlineto{\pgfqpoint{1.402450in}{1.539309in}}%
\pgfpathlineto{\pgfqpoint{1.402833in}{1.551532in}}%
\pgfpathlineto{\pgfqpoint{1.403218in}{1.539309in}}%
\pgfpathlineto{\pgfqpoint{1.403608in}{1.539309in}}%
\pgfpathlineto{\pgfqpoint{1.404759in}{1.547458in}}%
\pgfpathlineto{\pgfqpoint{1.405525in}{1.531161in}}%
\pgfpathlineto{\pgfqpoint{1.406674in}{1.555607in}}%
\pgfpathlineto{\pgfqpoint{1.408206in}{1.547458in}}%
\pgfpathlineto{\pgfqpoint{1.408971in}{1.563756in}}%
\pgfpathlineto{\pgfqpoint{1.409355in}{1.551532in}}%
\pgfpathlineto{\pgfqpoint{1.409738in}{1.527086in}}%
\pgfpathlineto{\pgfqpoint{1.410505in}{1.543384in}}%
\pgfpathlineto{\pgfqpoint{1.410887in}{1.531161in}}%
\pgfpathlineto{\pgfqpoint{1.411270in}{1.543384in}}%
\pgfpathlineto{\pgfqpoint{1.412416in}{1.555607in}}%
\pgfpathlineto{\pgfqpoint{1.413183in}{1.531161in}}%
\pgfpathlineto{\pgfqpoint{1.413951in}{1.535235in}}%
\pgfpathlineto{\pgfqpoint{1.414334in}{1.555607in}}%
\pgfpathlineto{\pgfqpoint{1.415100in}{1.551532in}}%
\pgfpathlineto{\pgfqpoint{1.415868in}{1.527086in}}%
\pgfpathlineto{\pgfqpoint{1.416252in}{1.531161in}}%
\pgfpathlineto{\pgfqpoint{1.416635in}{1.559681in}}%
\pgfpathlineto{\pgfqpoint{1.417401in}{1.543384in}}%
\pgfpathlineto{\pgfqpoint{1.417784in}{1.539309in}}%
\pgfpathlineto{\pgfqpoint{1.418167in}{1.547458in}}%
\pgfpathlineto{\pgfqpoint{1.418935in}{1.543384in}}%
\pgfpathlineto{\pgfqpoint{1.419319in}{1.539309in}}%
\pgfpathlineto{\pgfqpoint{1.419701in}{1.547458in}}%
\pgfpathlineto{\pgfqpoint{1.420468in}{1.543384in}}%
\pgfpathlineto{\pgfqpoint{1.420851in}{1.527086in}}%
\pgfpathlineto{\pgfqpoint{1.421617in}{1.539309in}}%
\pgfpathlineto{\pgfqpoint{1.422767in}{1.555607in}}%
\pgfpathlineto{\pgfqpoint{1.423918in}{1.527086in}}%
\pgfpathlineto{\pgfqpoint{1.425451in}{1.555607in}}%
\pgfpathlineto{\pgfqpoint{1.425834in}{1.567830in}}%
\pgfpathlineto{\pgfqpoint{1.425834in}{1.567830in}}%
\pgfpathlineto{\pgfqpoint{1.425834in}{1.567830in}}%
\pgfpathlineto{\pgfqpoint{1.426601in}{1.543384in}}%
\pgfpathlineto{\pgfqpoint{1.427070in}{1.555607in}}%
\pgfpathlineto{\pgfqpoint{1.427836in}{1.543384in}}%
\pgfpathlineto{\pgfqpoint{1.428221in}{1.551532in}}%
\pgfpathlineto{\pgfqpoint{1.429372in}{1.563756in}}%
\pgfpathlineto{\pgfqpoint{1.430139in}{1.531161in}}%
\pgfpathlineto{\pgfqpoint{1.430522in}{1.539309in}}%
\pgfpathlineto{\pgfqpoint{1.430905in}{1.547458in}}%
\pgfpathlineto{\pgfqpoint{1.431289in}{1.531161in}}%
\pgfpathlineto{\pgfqpoint{1.432057in}{1.535235in}}%
\pgfpathlineto{\pgfqpoint{1.433974in}{1.555607in}}%
\pgfpathlineto{\pgfqpoint{1.434358in}{1.547458in}}%
\pgfpathlineto{\pgfqpoint{1.434358in}{1.547458in}}%
\pgfpathlineto{\pgfqpoint{1.434358in}{1.547458in}}%
\pgfpathlineto{\pgfqpoint{1.434742in}{1.559681in}}%
\pgfpathlineto{\pgfqpoint{1.435125in}{1.551532in}}%
\pgfpathlineto{\pgfqpoint{1.435508in}{1.531161in}}%
\pgfpathlineto{\pgfqpoint{1.435508in}{1.531161in}}%
\pgfpathlineto{\pgfqpoint{1.435508in}{1.531161in}}%
\pgfpathlineto{\pgfqpoint{1.435891in}{1.555607in}}%
\pgfpathlineto{\pgfqpoint{1.436657in}{1.539309in}}%
\pgfpathlineto{\pgfqpoint{1.437041in}{1.547458in}}%
\pgfpathlineto{\pgfqpoint{1.437041in}{1.547458in}}%
\pgfpathlineto{\pgfqpoint{1.437041in}{1.547458in}}%
\pgfpathlineto{\pgfqpoint{1.438191in}{1.527086in}}%
\pgfpathlineto{\pgfqpoint{1.439340in}{1.563756in}}%
\pgfpathlineto{\pgfqpoint{1.439724in}{1.535235in}}%
\pgfpathlineto{\pgfqpoint{1.440491in}{1.543384in}}%
\pgfpathlineto{\pgfqpoint{1.441257in}{1.575979in}}%
\pgfpathlineto{\pgfqpoint{1.441640in}{1.551532in}}%
\pgfpathlineto{\pgfqpoint{1.442023in}{1.535235in}}%
\pgfpathlineto{\pgfqpoint{1.442790in}{1.547458in}}%
\pgfpathlineto{\pgfqpoint{1.443175in}{1.563756in}}%
\pgfpathlineto{\pgfqpoint{1.443565in}{1.527086in}}%
\pgfpathlineto{\pgfqpoint{1.444333in}{1.547458in}}%
\pgfpathlineto{\pgfqpoint{1.444716in}{1.535235in}}%
\pgfpathlineto{\pgfqpoint{1.445099in}{1.539309in}}%
\pgfpathlineto{\pgfqpoint{1.445483in}{1.551532in}}%
\pgfpathlineto{\pgfqpoint{1.445483in}{1.551532in}}%
\pgfpathlineto{\pgfqpoint{1.445483in}{1.551532in}}%
\pgfpathlineto{\pgfqpoint{1.445866in}{1.531161in}}%
\pgfpathlineto{\pgfqpoint{1.446249in}{1.543384in}}%
\pgfpathlineto{\pgfqpoint{1.447398in}{1.551532in}}%
\pgfpathlineto{\pgfqpoint{1.447782in}{1.547458in}}%
\pgfpathlineto{\pgfqpoint{1.448165in}{1.527086in}}%
\pgfpathlineto{\pgfqpoint{1.448165in}{1.527086in}}%
\pgfpathlineto{\pgfqpoint{1.448165in}{1.527086in}}%
\pgfpathlineto{\pgfqpoint{1.448547in}{1.559681in}}%
\pgfpathlineto{\pgfqpoint{1.449312in}{1.551532in}}%
\pgfpathlineto{\pgfqpoint{1.449695in}{1.551532in}}%
\pgfpathlineto{\pgfqpoint{1.450081in}{1.531161in}}%
\pgfpathlineto{\pgfqpoint{1.450464in}{1.535235in}}%
\pgfpathlineto{\pgfqpoint{1.451230in}{1.559681in}}%
\pgfpathlineto{\pgfqpoint{1.451613in}{1.547458in}}%
\pgfpathlineto{\pgfqpoint{1.452763in}{1.555607in}}%
\pgfpathlineto{\pgfqpoint{1.454296in}{1.539309in}}%
\pgfpathlineto{\pgfqpoint{1.454679in}{1.555607in}}%
\pgfpathlineto{\pgfqpoint{1.455446in}{1.543384in}}%
\pgfpathlineto{\pgfqpoint{1.455829in}{1.527086in}}%
\pgfpathlineto{\pgfqpoint{1.455829in}{1.527086in}}%
\pgfpathlineto{\pgfqpoint{1.455829in}{1.527086in}}%
\pgfpathlineto{\pgfqpoint{1.456595in}{1.559681in}}%
\pgfpathlineto{\pgfqpoint{1.457362in}{1.555607in}}%
\pgfpathlineto{\pgfqpoint{1.457745in}{1.551532in}}%
\pgfpathlineto{\pgfqpoint{1.458129in}{1.523012in}}%
\pgfpathlineto{\pgfqpoint{1.458896in}{1.543384in}}%
\pgfpathlineto{\pgfqpoint{1.459279in}{1.535235in}}%
\pgfpathlineto{\pgfqpoint{1.460044in}{1.563756in}}%
\pgfpathlineto{\pgfqpoint{1.460427in}{1.535235in}}%
\pgfpathlineto{\pgfqpoint{1.461195in}{1.539309in}}%
\pgfpathlineto{\pgfqpoint{1.461579in}{1.551532in}}%
\pgfpathlineto{\pgfqpoint{1.461962in}{1.547458in}}%
\pgfpathlineto{\pgfqpoint{1.462346in}{1.535235in}}%
\pgfpathlineto{\pgfqpoint{1.463112in}{1.539309in}}%
\pgfpathlineto{\pgfqpoint{1.463496in}{1.563756in}}%
\pgfpathlineto{\pgfqpoint{1.463880in}{1.551532in}}%
\pgfpathlineto{\pgfqpoint{1.464263in}{1.527086in}}%
\pgfpathlineto{\pgfqpoint{1.464263in}{1.527086in}}%
\pgfpathlineto{\pgfqpoint{1.464263in}{1.527086in}}%
\pgfpathlineto{\pgfqpoint{1.464645in}{1.563756in}}%
\pgfpathlineto{\pgfqpoint{1.465411in}{1.531161in}}%
\pgfpathlineto{\pgfqpoint{1.466945in}{1.555607in}}%
\pgfpathlineto{\pgfqpoint{1.468094in}{1.543384in}}%
\pgfpathlineto{\pgfqpoint{1.468861in}{1.555607in}}%
\pgfpathlineto{\pgfqpoint{1.470011in}{1.531161in}}%
\pgfpathlineto{\pgfqpoint{1.470395in}{1.551532in}}%
\pgfpathlineto{\pgfqpoint{1.471161in}{1.543384in}}%
\pgfpathlineto{\pgfqpoint{1.471546in}{1.543384in}}%
\pgfpathlineto{\pgfqpoint{1.471929in}{1.539309in}}%
\pgfpathlineto{\pgfqpoint{1.472398in}{1.567830in}}%
\pgfpathlineto{\pgfqpoint{1.472781in}{1.559681in}}%
\pgfpathlineto{\pgfqpoint{1.473164in}{1.531161in}}%
\pgfpathlineto{\pgfqpoint{1.473930in}{1.551532in}}%
\pgfpathlineto{\pgfqpoint{1.474313in}{1.559681in}}%
\pgfpathlineto{\pgfqpoint{1.474313in}{1.559681in}}%
\pgfpathlineto{\pgfqpoint{1.474313in}{1.559681in}}%
\pgfpathlineto{\pgfqpoint{1.475080in}{1.547458in}}%
\pgfpathlineto{\pgfqpoint{1.475463in}{1.555607in}}%
\pgfpathlineto{\pgfqpoint{1.475846in}{1.555607in}}%
\pgfpathlineto{\pgfqpoint{1.476612in}{1.535235in}}%
\pgfpathlineto{\pgfqpoint{1.477378in}{1.555607in}}%
\pgfpathlineto{\pgfqpoint{1.477762in}{1.551532in}}%
\pgfpathlineto{\pgfqpoint{1.478910in}{1.535235in}}%
\pgfpathlineto{\pgfqpoint{1.479293in}{1.539309in}}%
\pgfpathlineto{\pgfqpoint{1.480442in}{1.575979in}}%
\pgfpathlineto{\pgfqpoint{1.481208in}{1.559681in}}%
\pgfpathlineto{\pgfqpoint{1.481974in}{1.547458in}}%
\pgfpathlineto{\pgfqpoint{1.482357in}{1.535235in}}%
\pgfpathlineto{\pgfqpoint{1.482740in}{1.563756in}}%
\pgfpathlineto{\pgfqpoint{1.483123in}{1.551532in}}%
\pgfpathlineto{\pgfqpoint{1.483897in}{1.535235in}}%
\pgfpathlineto{\pgfqpoint{1.484280in}{1.551532in}}%
\pgfpathlineto{\pgfqpoint{1.484280in}{1.551532in}}%
\pgfpathlineto{\pgfqpoint{1.484280in}{1.551532in}}%
\pgfpathlineto{\pgfqpoint{1.484663in}{1.531161in}}%
\pgfpathlineto{\pgfqpoint{1.485429in}{1.539309in}}%
\pgfpathlineto{\pgfqpoint{1.485813in}{1.531161in}}%
\pgfpathlineto{\pgfqpoint{1.486579in}{1.559681in}}%
\pgfpathlineto{\pgfqpoint{1.486962in}{1.555607in}}%
\pgfpathlineto{\pgfqpoint{1.487345in}{1.535235in}}%
\pgfpathlineto{\pgfqpoint{1.488111in}{1.551532in}}%
\pgfpathlineto{\pgfqpoint{1.488495in}{1.551532in}}%
\pgfpathlineto{\pgfqpoint{1.489261in}{1.535235in}}%
\pgfpathlineto{\pgfqpoint{1.490409in}{1.551532in}}%
\pgfpathlineto{\pgfqpoint{1.491558in}{1.531161in}}%
\pgfpathlineto{\pgfqpoint{1.492326in}{1.555607in}}%
\pgfpathlineto{\pgfqpoint{1.492709in}{1.547458in}}%
\pgfpathlineto{\pgfqpoint{1.493091in}{1.523012in}}%
\pgfpathlineto{\pgfqpoint{1.493858in}{1.543384in}}%
\pgfpathlineto{\pgfqpoint{1.494242in}{1.547458in}}%
\pgfpathlineto{\pgfqpoint{1.495391in}{1.518937in}}%
\pgfpathlineto{\pgfqpoint{1.496158in}{1.563756in}}%
\pgfpathlineto{\pgfqpoint{1.496925in}{1.551532in}}%
\pgfpathlineto{\pgfqpoint{1.497308in}{1.551532in}}%
\pgfpathlineto{\pgfqpoint{1.497691in}{1.535235in}}%
\pgfpathlineto{\pgfqpoint{1.497691in}{1.535235in}}%
\pgfpathlineto{\pgfqpoint{1.497691in}{1.535235in}}%
\pgfpathlineto{\pgfqpoint{1.498074in}{1.563756in}}%
\pgfpathlineto{\pgfqpoint{1.498840in}{1.543384in}}%
\pgfpathlineto{\pgfqpoint{1.499615in}{1.539309in}}%
\pgfpathlineto{\pgfqpoint{1.500000in}{1.543384in}}%
\pgfpathlineto{\pgfqpoint{1.501153in}{1.559681in}}%
\pgfpathlineto{\pgfqpoint{1.501536in}{1.551532in}}%
\pgfpathlineto{\pgfqpoint{1.501919in}{1.555607in}}%
\pgfpathlineto{\pgfqpoint{1.502302in}{1.559681in}}%
\pgfpathlineto{\pgfqpoint{1.503069in}{1.543384in}}%
\pgfpathlineto{\pgfqpoint{1.503453in}{1.547458in}}%
\pgfpathlineto{\pgfqpoint{1.504984in}{1.563756in}}%
\pgfpathlineto{\pgfqpoint{1.505366in}{1.531161in}}%
\pgfpathlineto{\pgfqpoint{1.506133in}{1.555607in}}%
\pgfpathlineto{\pgfqpoint{1.507285in}{1.539309in}}%
\pgfpathlineto{\pgfqpoint{1.507668in}{1.563756in}}%
\pgfpathlineto{\pgfqpoint{1.508051in}{1.555607in}}%
\pgfpathlineto{\pgfqpoint{1.508817in}{1.539309in}}%
\pgfpathlineto{\pgfqpoint{1.509201in}{1.559681in}}%
\pgfpathlineto{\pgfqpoint{1.509968in}{1.551532in}}%
\pgfpathlineto{\pgfqpoint{1.510734in}{1.531161in}}%
\pgfpathlineto{\pgfqpoint{1.511117in}{1.539309in}}%
\pgfpathlineto{\pgfqpoint{1.511500in}{1.543384in}}%
\pgfpathlineto{\pgfqpoint{1.511886in}{1.531161in}}%
\pgfpathlineto{\pgfqpoint{1.511886in}{1.531161in}}%
\pgfpathlineto{\pgfqpoint{1.511886in}{1.531161in}}%
\pgfpathlineto{\pgfqpoint{1.512652in}{1.563756in}}%
\pgfpathlineto{\pgfqpoint{1.513035in}{1.543384in}}%
\pgfpathlineto{\pgfqpoint{1.513418in}{1.527086in}}%
\pgfpathlineto{\pgfqpoint{1.514185in}{1.535235in}}%
\pgfpathlineto{\pgfqpoint{1.514570in}{1.563756in}}%
\pgfpathlineto{\pgfqpoint{1.515421in}{1.547458in}}%
\pgfpathlineto{\pgfqpoint{1.516186in}{1.559681in}}%
\pgfpathlineto{\pgfqpoint{1.516570in}{1.551532in}}%
\pgfpathlineto{\pgfqpoint{1.516953in}{1.547458in}}%
\pgfpathlineto{\pgfqpoint{1.517337in}{1.567830in}}%
\pgfpathlineto{\pgfqpoint{1.518104in}{1.551532in}}%
\pgfpathlineto{\pgfqpoint{1.518487in}{1.563756in}}%
\pgfpathlineto{\pgfqpoint{1.518487in}{1.563756in}}%
\pgfpathlineto{\pgfqpoint{1.518487in}{1.563756in}}%
\pgfpathlineto{\pgfqpoint{1.519636in}{1.539309in}}%
\pgfpathlineto{\pgfqpoint{1.520019in}{1.567830in}}%
\pgfpathlineto{\pgfqpoint{1.520786in}{1.547458in}}%
\pgfpathlineto{\pgfqpoint{1.521171in}{1.559681in}}%
\pgfpathlineto{\pgfqpoint{1.521937in}{1.551532in}}%
\pgfpathlineto{\pgfqpoint{1.522319in}{1.527086in}}%
\pgfpathlineto{\pgfqpoint{1.523087in}{1.547458in}}%
\pgfpathlineto{\pgfqpoint{1.524244in}{1.539309in}}%
\pgfpathlineto{\pgfqpoint{1.525395in}{1.571904in}}%
\pgfpathlineto{\pgfqpoint{1.525778in}{1.535235in}}%
\pgfpathlineto{\pgfqpoint{1.526544in}{1.551532in}}%
\pgfpathlineto{\pgfqpoint{1.526927in}{1.563756in}}%
\pgfpathlineto{\pgfqpoint{1.527695in}{1.555607in}}%
\pgfpathlineto{\pgfqpoint{1.528460in}{1.567830in}}%
\pgfpathlineto{\pgfqpoint{1.529226in}{1.535235in}}%
\pgfpathlineto{\pgfqpoint{1.529609in}{1.539309in}}%
\pgfpathlineto{\pgfqpoint{1.530759in}{1.563756in}}%
\pgfpathlineto{\pgfqpoint{1.531525in}{1.539309in}}%
\pgfpathlineto{\pgfqpoint{1.531907in}{1.547458in}}%
\pgfpathlineto{\pgfqpoint{1.532673in}{1.555607in}}%
\pgfpathlineto{\pgfqpoint{1.533057in}{1.539309in}}%
\pgfpathlineto{\pgfqpoint{1.533441in}{1.547458in}}%
\pgfpathlineto{\pgfqpoint{1.533824in}{1.555607in}}%
\pgfpathlineto{\pgfqpoint{1.533824in}{1.555607in}}%
\pgfpathlineto{\pgfqpoint{1.533824in}{1.555607in}}%
\pgfpathlineto{\pgfqpoint{1.534207in}{1.543384in}}%
\pgfpathlineto{\pgfqpoint{1.534973in}{1.547458in}}%
\pgfpathlineto{\pgfqpoint{1.535740in}{1.555607in}}%
\pgfpathlineto{\pgfqpoint{1.536123in}{1.539309in}}%
\pgfpathlineto{\pgfqpoint{1.536889in}{1.551532in}}%
\pgfpathlineto{\pgfqpoint{1.538037in}{1.543384in}}%
\pgfpathlineto{\pgfqpoint{1.538421in}{1.559681in}}%
\pgfpathlineto{\pgfqpoint{1.538421in}{1.559681in}}%
\pgfpathlineto{\pgfqpoint{1.538421in}{1.559681in}}%
\pgfpathlineto{\pgfqpoint{1.539954in}{1.531161in}}%
\pgfpathlineto{\pgfqpoint{1.540337in}{1.563756in}}%
\pgfpathlineto{\pgfqpoint{1.541103in}{1.539309in}}%
\pgfpathlineto{\pgfqpoint{1.541487in}{1.543384in}}%
\pgfpathlineto{\pgfqpoint{1.541871in}{1.535235in}}%
\pgfpathlineto{\pgfqpoint{1.541871in}{1.535235in}}%
\pgfpathlineto{\pgfqpoint{1.541871in}{1.535235in}}%
\pgfpathlineto{\pgfqpoint{1.543020in}{1.551532in}}%
\pgfpathlineto{\pgfqpoint{1.544172in}{1.531161in}}%
\pgfpathlineto{\pgfqpoint{1.545321in}{1.555607in}}%
\pgfpathlineto{\pgfqpoint{1.545703in}{1.535235in}}%
\pgfpathlineto{\pgfqpoint{1.545703in}{1.535235in}}%
\pgfpathlineto{\pgfqpoint{1.545703in}{1.535235in}}%
\pgfpathlineto{\pgfqpoint{1.546854in}{1.559681in}}%
\pgfpathlineto{\pgfqpoint{1.548387in}{1.535235in}}%
\pgfpathlineto{\pgfqpoint{1.549153in}{1.551532in}}%
\pgfpathlineto{\pgfqpoint{1.549537in}{1.543384in}}%
\pgfpathlineto{\pgfqpoint{1.550303in}{1.539309in}}%
\pgfpathlineto{\pgfqpoint{1.550687in}{1.559681in}}%
\pgfpathlineto{\pgfqpoint{1.551450in}{1.543384in}}%
\pgfpathlineto{\pgfqpoint{1.552601in}{1.551532in}}%
\pgfpathlineto{\pgfqpoint{1.552985in}{1.547458in}}%
\pgfpathlineto{\pgfqpoint{1.553750in}{1.575979in}}%
\pgfpathlineto{\pgfqpoint{1.555367in}{1.543384in}}%
\pgfpathlineto{\pgfqpoint{1.555751in}{1.543384in}}%
\pgfpathlineto{\pgfqpoint{1.556518in}{1.539309in}}%
\pgfpathlineto{\pgfqpoint{1.556901in}{1.555607in}}%
\pgfpathlineto{\pgfqpoint{1.557666in}{1.551532in}}%
\pgfpathlineto{\pgfqpoint{1.558431in}{1.547458in}}%
\pgfpathlineto{\pgfqpoint{1.558815in}{1.551532in}}%
\pgfpathlineto{\pgfqpoint{1.559199in}{1.535235in}}%
\pgfpathlineto{\pgfqpoint{1.559199in}{1.535235in}}%
\pgfpathlineto{\pgfqpoint{1.559199in}{1.535235in}}%
\pgfpathlineto{\pgfqpoint{1.559965in}{1.559681in}}%
\pgfpathlineto{\pgfqpoint{1.560348in}{1.531161in}}%
\pgfpathlineto{\pgfqpoint{1.560732in}{1.535235in}}%
\pgfpathlineto{\pgfqpoint{1.561115in}{1.559681in}}%
\pgfpathlineto{\pgfqpoint{1.561882in}{1.539309in}}%
\pgfpathlineto{\pgfqpoint{1.562265in}{1.539309in}}%
\pgfpathlineto{\pgfqpoint{1.562649in}{1.535235in}}%
\pgfpathlineto{\pgfqpoint{1.563806in}{1.551532in}}%
\pgfpathlineto{\pgfqpoint{1.564573in}{1.531161in}}%
\pgfpathlineto{\pgfqpoint{1.564956in}{1.539309in}}%
\pgfpathlineto{\pgfqpoint{1.565722in}{1.563756in}}%
\pgfpathlineto{\pgfqpoint{1.566105in}{1.547458in}}%
\pgfpathlineto{\pgfqpoint{1.566489in}{1.535235in}}%
\pgfpathlineto{\pgfqpoint{1.566872in}{1.539309in}}%
\pgfpathlineto{\pgfqpoint{1.567639in}{1.559681in}}%
\pgfpathlineto{\pgfqpoint{1.568405in}{1.555607in}}%
\pgfpathlineto{\pgfqpoint{1.570323in}{1.527086in}}%
\pgfpathlineto{\pgfqpoint{1.571089in}{1.555607in}}%
\pgfpathlineto{\pgfqpoint{1.571474in}{1.547458in}}%
\pgfpathlineto{\pgfqpoint{1.572240in}{1.543384in}}%
\pgfpathlineto{\pgfqpoint{1.573390in}{1.555607in}}%
\pgfpathlineto{\pgfqpoint{1.573773in}{1.535235in}}%
\pgfpathlineto{\pgfqpoint{1.574156in}{1.551532in}}%
\pgfpathlineto{\pgfqpoint{1.574922in}{1.563756in}}%
\pgfpathlineto{\pgfqpoint{1.576072in}{1.531161in}}%
\pgfpathlineto{\pgfqpoint{1.576455in}{1.547458in}}%
\pgfpathlineto{\pgfqpoint{1.576838in}{1.563756in}}%
\pgfpathlineto{\pgfqpoint{1.576838in}{1.563756in}}%
\pgfpathlineto{\pgfqpoint{1.576838in}{1.563756in}}%
\pgfpathlineto{\pgfqpoint{1.577222in}{1.539309in}}%
\pgfpathlineto{\pgfqpoint{1.577989in}{1.547458in}}%
\pgfpathlineto{\pgfqpoint{1.578373in}{1.567830in}}%
\pgfpathlineto{\pgfqpoint{1.579138in}{1.555607in}}%
\pgfpathlineto{\pgfqpoint{1.580673in}{1.527086in}}%
\pgfpathlineto{\pgfqpoint{1.581821in}{1.555607in}}%
\pgfpathlineto{\pgfqpoint{1.582205in}{1.543384in}}%
\pgfpathlineto{\pgfqpoint{1.582589in}{1.547458in}}%
\pgfpathlineto{\pgfqpoint{1.583355in}{1.555607in}}%
\pgfpathlineto{\pgfqpoint{1.583739in}{1.535235in}}%
\pgfpathlineto{\pgfqpoint{1.584504in}{1.547458in}}%
\pgfpathlineto{\pgfqpoint{1.584889in}{1.547458in}}%
\pgfpathlineto{\pgfqpoint{1.585655in}{1.539309in}}%
\pgfpathlineto{\pgfqpoint{1.586804in}{1.555607in}}%
\pgfpathlineto{\pgfqpoint{1.587572in}{1.539309in}}%
\pgfpathlineto{\pgfqpoint{1.588339in}{1.559681in}}%
\pgfpathlineto{\pgfqpoint{1.588722in}{1.543384in}}%
\pgfpathlineto{\pgfqpoint{1.589105in}{1.535235in}}%
\pgfpathlineto{\pgfqpoint{1.589488in}{1.539309in}}%
\pgfpathlineto{\pgfqpoint{1.590254in}{1.551532in}}%
\pgfpathlineto{\pgfqpoint{1.590637in}{1.547458in}}%
\pgfpathlineto{\pgfqpoint{1.591020in}{1.531161in}}%
\pgfpathlineto{\pgfqpoint{1.591403in}{1.563756in}}%
\pgfpathlineto{\pgfqpoint{1.592170in}{1.539309in}}%
\pgfpathlineto{\pgfqpoint{1.592553in}{1.539309in}}%
\pgfpathlineto{\pgfqpoint{1.592936in}{1.527086in}}%
\pgfpathlineto{\pgfqpoint{1.593319in}{1.559681in}}%
\pgfpathlineto{\pgfqpoint{1.594086in}{1.555607in}}%
\pgfpathlineto{\pgfqpoint{1.594937in}{1.539309in}}%
\pgfpathlineto{\pgfqpoint{1.595319in}{1.543384in}}%
\pgfpathlineto{\pgfqpoint{1.596085in}{1.555607in}}%
\pgfpathlineto{\pgfqpoint{1.596469in}{1.567830in}}%
\pgfpathlineto{\pgfqpoint{1.597619in}{1.543384in}}%
\pgfpathlineto{\pgfqpoint{1.598002in}{1.547458in}}%
\pgfpathlineto{\pgfqpoint{1.598384in}{1.539309in}}%
\pgfpathlineto{\pgfqpoint{1.598384in}{1.539309in}}%
\pgfpathlineto{\pgfqpoint{1.598384in}{1.539309in}}%
\pgfpathlineto{\pgfqpoint{1.599535in}{1.555607in}}%
\pgfpathlineto{\pgfqpoint{1.600685in}{1.543384in}}%
\pgfpathlineto{\pgfqpoint{1.601451in}{1.567830in}}%
\pgfpathlineto{\pgfqpoint{1.601833in}{1.551532in}}%
\pgfpathlineto{\pgfqpoint{1.602601in}{1.543384in}}%
\pgfpathlineto{\pgfqpoint{1.602984in}{1.555607in}}%
\pgfpathlineto{\pgfqpoint{1.602984in}{1.555607in}}%
\pgfpathlineto{\pgfqpoint{1.602984in}{1.555607in}}%
\pgfpathlineto{\pgfqpoint{1.603758in}{1.535235in}}%
\pgfpathlineto{\pgfqpoint{1.604525in}{1.567830in}}%
\pgfpathlineto{\pgfqpoint{1.605292in}{1.559681in}}%
\pgfpathlineto{\pgfqpoint{1.605675in}{1.563756in}}%
\pgfpathlineto{\pgfqpoint{1.606443in}{1.539309in}}%
\pgfpathlineto{\pgfqpoint{1.607210in}{1.547458in}}%
\pgfpathlineto{\pgfqpoint{1.607977in}{1.555607in}}%
\pgfpathlineto{\pgfqpoint{1.608743in}{1.539309in}}%
\pgfpathlineto{\pgfqpoint{1.609126in}{1.547458in}}%
\pgfpathlineto{\pgfqpoint{1.609510in}{1.547458in}}%
\pgfpathlineto{\pgfqpoint{1.609894in}{1.535235in}}%
\pgfpathlineto{\pgfqpoint{1.609894in}{1.535235in}}%
\pgfpathlineto{\pgfqpoint{1.609894in}{1.535235in}}%
\pgfpathlineto{\pgfqpoint{1.611043in}{1.559681in}}%
\pgfpathlineto{\pgfqpoint{1.612192in}{1.531161in}}%
\pgfpathlineto{\pgfqpoint{1.613341in}{1.563756in}}%
\pgfpathlineto{\pgfqpoint{1.614875in}{1.539309in}}%
\pgfpathlineto{\pgfqpoint{1.616409in}{1.555607in}}%
\pgfpathlineto{\pgfqpoint{1.616793in}{1.559681in}}%
\pgfpathlineto{\pgfqpoint{1.618326in}{1.531161in}}%
\pgfpathlineto{\pgfqpoint{1.619093in}{1.559681in}}%
\pgfpathlineto{\pgfqpoint{1.619476in}{1.547458in}}%
\pgfpathlineto{\pgfqpoint{1.619859in}{1.527086in}}%
\pgfpathlineto{\pgfqpoint{1.620243in}{1.539309in}}%
\pgfpathlineto{\pgfqpoint{1.621009in}{1.567830in}}%
\pgfpathlineto{\pgfqpoint{1.621393in}{1.535235in}}%
\pgfpathlineto{\pgfqpoint{1.622159in}{1.551532in}}%
\pgfpathlineto{\pgfqpoint{1.622542in}{1.555607in}}%
\pgfpathlineto{\pgfqpoint{1.623311in}{1.535235in}}%
\pgfpathlineto{\pgfqpoint{1.623695in}{1.543384in}}%
\pgfpathlineto{\pgfqpoint{1.624077in}{1.535235in}}%
\pgfpathlineto{\pgfqpoint{1.624843in}{1.555607in}}%
\pgfpathlineto{\pgfqpoint{1.625227in}{1.535235in}}%
\pgfpathlineto{\pgfqpoint{1.625227in}{1.535235in}}%
\pgfpathlineto{\pgfqpoint{1.625227in}{1.535235in}}%
\pgfpathlineto{\pgfqpoint{1.625610in}{1.559681in}}%
\pgfpathlineto{\pgfqpoint{1.625993in}{1.551532in}}%
\pgfpathlineto{\pgfqpoint{1.626759in}{1.535235in}}%
\pgfpathlineto{\pgfqpoint{1.627143in}{1.559681in}}%
\pgfpathlineto{\pgfqpoint{1.627143in}{1.559681in}}%
\pgfpathlineto{\pgfqpoint{1.627143in}{1.559681in}}%
\pgfpathlineto{\pgfqpoint{1.627526in}{1.531161in}}%
\pgfpathlineto{\pgfqpoint{1.628294in}{1.543384in}}%
\pgfpathlineto{\pgfqpoint{1.628677in}{1.543384in}}%
\pgfpathlineto{\pgfqpoint{1.629826in}{1.555607in}}%
\pgfpathlineto{\pgfqpoint{1.630593in}{1.535235in}}%
\pgfpathlineto{\pgfqpoint{1.630976in}{1.543384in}}%
\pgfpathlineto{\pgfqpoint{1.631359in}{1.555607in}}%
\pgfpathlineto{\pgfqpoint{1.631743in}{1.543384in}}%
\pgfpathlineto{\pgfqpoint{1.632126in}{1.539309in}}%
\pgfpathlineto{\pgfqpoint{1.632509in}{1.543384in}}%
\pgfpathlineto{\pgfqpoint{1.633660in}{1.559681in}}%
\pgfpathlineto{\pgfqpoint{1.634042in}{1.527086in}}%
\pgfpathlineto{\pgfqpoint{1.634809in}{1.547458in}}%
\pgfpathlineto{\pgfqpoint{1.635192in}{1.539309in}}%
\pgfpathlineto{\pgfqpoint{1.635192in}{1.539309in}}%
\pgfpathlineto{\pgfqpoint{1.635192in}{1.539309in}}%
\pgfpathlineto{\pgfqpoint{1.635575in}{1.551532in}}%
\pgfpathlineto{\pgfqpoint{1.635575in}{1.551532in}}%
\pgfpathlineto{\pgfqpoint{1.635575in}{1.551532in}}%
\pgfpathlineto{\pgfqpoint{1.635959in}{1.535235in}}%
\pgfpathlineto{\pgfqpoint{1.636725in}{1.547458in}}%
\pgfpathlineto{\pgfqpoint{1.637578in}{1.523012in}}%
\pgfpathlineto{\pgfqpoint{1.637961in}{1.539309in}}%
\pgfpathlineto{\pgfqpoint{1.638343in}{1.555607in}}%
\pgfpathlineto{\pgfqpoint{1.638726in}{1.539309in}}%
\pgfpathlineto{\pgfqpoint{1.639109in}{1.539309in}}%
\pgfpathlineto{\pgfqpoint{1.639493in}{1.551532in}}%
\pgfpathlineto{\pgfqpoint{1.640260in}{1.543384in}}%
\pgfpathlineto{\pgfqpoint{1.640643in}{1.543384in}}%
\pgfpathlineto{\pgfqpoint{1.641026in}{1.547458in}}%
\pgfpathlineto{\pgfqpoint{1.641409in}{1.531161in}}%
\pgfpathlineto{\pgfqpoint{1.641792in}{1.535235in}}%
\pgfpathlineto{\pgfqpoint{1.642559in}{1.551532in}}%
\pgfpathlineto{\pgfqpoint{1.642942in}{1.547458in}}%
\pgfpathlineto{\pgfqpoint{1.643332in}{1.547458in}}%
\pgfpathlineto{\pgfqpoint{1.643715in}{1.543384in}}%
\pgfpathlineto{\pgfqpoint{1.644481in}{1.555607in}}%
\pgfpathlineto{\pgfqpoint{1.644864in}{1.551532in}}%
\pgfpathlineto{\pgfqpoint{1.646011in}{1.539309in}}%
\pgfpathlineto{\pgfqpoint{1.646393in}{1.559681in}}%
\pgfpathlineto{\pgfqpoint{1.646776in}{1.551532in}}%
\pgfpathlineto{\pgfqpoint{1.647158in}{1.531161in}}%
\pgfpathlineto{\pgfqpoint{1.647158in}{1.531161in}}%
\pgfpathlineto{\pgfqpoint{1.647158in}{1.531161in}}%
\pgfpathlineto{\pgfqpoint{1.647923in}{1.559681in}}%
\pgfpathlineto{\pgfqpoint{1.648307in}{1.547458in}}%
\pgfpathlineto{\pgfqpoint{1.648689in}{1.555607in}}%
\pgfpathlineto{\pgfqpoint{1.649072in}{1.551532in}}%
\pgfpathlineto{\pgfqpoint{1.650219in}{1.539309in}}%
\pgfpathlineto{\pgfqpoint{1.651751in}{1.559681in}}%
\pgfpathlineto{\pgfqpoint{1.652515in}{1.527086in}}%
\pgfpathlineto{\pgfqpoint{1.652898in}{1.539309in}}%
\pgfpathlineto{\pgfqpoint{1.654046in}{1.567830in}}%
\pgfpathlineto{\pgfqpoint{1.654812in}{1.535235in}}%
\pgfpathlineto{\pgfqpoint{1.655577in}{1.547458in}}%
\pgfpathlineto{\pgfqpoint{1.655960in}{1.547458in}}%
\pgfpathlineto{\pgfqpoint{1.656342in}{1.555607in}}%
\pgfpathlineto{\pgfqpoint{1.656724in}{1.527086in}}%
\pgfpathlineto{\pgfqpoint{1.656724in}{1.527086in}}%
\pgfpathlineto{\pgfqpoint{1.656724in}{1.527086in}}%
\pgfpathlineto{\pgfqpoint{1.657106in}{1.563756in}}%
\pgfpathlineto{\pgfqpoint{1.657872in}{1.539309in}}%
\pgfpathlineto{\pgfqpoint{1.658256in}{1.531161in}}%
\pgfpathlineto{\pgfqpoint{1.659022in}{1.571904in}}%
\pgfpathlineto{\pgfqpoint{1.659786in}{1.555607in}}%
\pgfpathlineto{\pgfqpoint{1.660551in}{1.535235in}}%
\pgfpathlineto{\pgfqpoint{1.660933in}{1.547458in}}%
\pgfpathlineto{\pgfqpoint{1.661317in}{1.555607in}}%
\pgfpathlineto{\pgfqpoint{1.661700in}{1.535235in}}%
\pgfpathlineto{\pgfqpoint{1.662466in}{1.547458in}}%
\pgfpathlineto{\pgfqpoint{1.662848in}{1.547458in}}%
\pgfpathlineto{\pgfqpoint{1.663698in}{1.543384in}}%
\pgfpathlineto{\pgfqpoint{1.664462in}{1.559681in}}%
\pgfpathlineto{\pgfqpoint{1.664846in}{1.551532in}}%
\pgfpathlineto{\pgfqpoint{1.665228in}{1.551532in}}%
\pgfpathlineto{\pgfqpoint{1.665613in}{1.535235in}}%
\pgfpathlineto{\pgfqpoint{1.665995in}{1.543384in}}%
\pgfpathlineto{\pgfqpoint{1.667144in}{1.559681in}}%
\pgfpathlineto{\pgfqpoint{1.668293in}{1.527086in}}%
\pgfpathlineto{\pgfqpoint{1.668676in}{1.543384in}}%
\pgfpathlineto{\pgfqpoint{1.669059in}{1.559681in}}%
\pgfpathlineto{\pgfqpoint{1.669442in}{1.527086in}}%
\pgfpathlineto{\pgfqpoint{1.670208in}{1.555607in}}%
\pgfpathlineto{\pgfqpoint{1.670591in}{1.527086in}}%
\pgfpathlineto{\pgfqpoint{1.670974in}{1.551532in}}%
\pgfpathlineto{\pgfqpoint{1.671357in}{1.563756in}}%
\pgfpathlineto{\pgfqpoint{1.671742in}{1.531161in}}%
\pgfpathlineto{\pgfqpoint{1.672508in}{1.543384in}}%
\pgfpathlineto{\pgfqpoint{1.673274in}{1.555607in}}%
\pgfpathlineto{\pgfqpoint{1.673657in}{1.535235in}}%
\pgfpathlineto{\pgfqpoint{1.674425in}{1.539309in}}%
\pgfpathlineto{\pgfqpoint{1.674808in}{1.563756in}}%
\pgfpathlineto{\pgfqpoint{1.675575in}{1.555607in}}%
\pgfpathlineto{\pgfqpoint{1.675957in}{1.555607in}}%
\pgfpathlineto{\pgfqpoint{1.676342in}{1.559681in}}%
\pgfpathlineto{\pgfqpoint{1.676725in}{1.575979in}}%
\pgfpathlineto{\pgfqpoint{1.678258in}{1.535235in}}%
\pgfpathlineto{\pgfqpoint{1.679025in}{1.551532in}}%
\pgfpathlineto{\pgfqpoint{1.679409in}{1.535235in}}%
\pgfpathlineto{\pgfqpoint{1.679409in}{1.535235in}}%
\pgfpathlineto{\pgfqpoint{1.679409in}{1.535235in}}%
\pgfpathlineto{\pgfqpoint{1.680941in}{1.571904in}}%
\pgfpathlineto{\pgfqpoint{1.682091in}{1.539309in}}%
\pgfpathlineto{\pgfqpoint{1.682475in}{1.539309in}}%
\pgfpathlineto{\pgfqpoint{1.684015in}{1.555607in}}%
\pgfpathlineto{\pgfqpoint{1.684781in}{1.535235in}}%
\pgfpathlineto{\pgfqpoint{1.685164in}{1.543384in}}%
\pgfpathlineto{\pgfqpoint{1.685930in}{1.535235in}}%
\pgfpathlineto{\pgfqpoint{1.686697in}{1.563756in}}%
\pgfpathlineto{\pgfqpoint{1.687846in}{1.531161in}}%
\pgfpathlineto{\pgfqpoint{1.688612in}{1.551532in}}%
\pgfpathlineto{\pgfqpoint{1.688995in}{1.539309in}}%
\pgfpathlineto{\pgfqpoint{1.689762in}{1.543384in}}%
\pgfpathlineto{\pgfqpoint{1.690527in}{1.535235in}}%
\pgfpathlineto{\pgfqpoint{1.690909in}{1.543384in}}%
\pgfpathlineto{\pgfqpoint{1.691293in}{1.527086in}}%
\pgfpathlineto{\pgfqpoint{1.692060in}{1.531161in}}%
\pgfpathlineto{\pgfqpoint{1.692827in}{1.559681in}}%
\pgfpathlineto{\pgfqpoint{1.693211in}{1.547458in}}%
\pgfpathlineto{\pgfqpoint{1.693593in}{1.547458in}}%
\pgfpathlineto{\pgfqpoint{1.694360in}{1.527086in}}%
\pgfpathlineto{\pgfqpoint{1.694744in}{1.551532in}}%
\pgfpathlineto{\pgfqpoint{1.695511in}{1.547458in}}%
\pgfpathlineto{\pgfqpoint{1.695894in}{1.543384in}}%
\pgfpathlineto{\pgfqpoint{1.696277in}{1.559681in}}%
\pgfpathlineto{\pgfqpoint{1.696277in}{1.559681in}}%
\pgfpathlineto{\pgfqpoint{1.696277in}{1.559681in}}%
\pgfpathlineto{\pgfqpoint{1.696660in}{1.539309in}}%
\pgfpathlineto{\pgfqpoint{1.697426in}{1.547458in}}%
\pgfpathlineto{\pgfqpoint{1.697809in}{1.547458in}}%
\pgfpathlineto{\pgfqpoint{1.698192in}{1.531161in}}%
\pgfpathlineto{\pgfqpoint{1.698959in}{1.543384in}}%
\pgfpathlineto{\pgfqpoint{1.699341in}{1.547458in}}%
\pgfpathlineto{\pgfqpoint{1.700192in}{1.527086in}}%
\pgfpathlineto{\pgfqpoint{1.700958in}{1.567830in}}%
\pgfpathlineto{\pgfqpoint{1.701341in}{1.547458in}}%
\pgfpathlineto{\pgfqpoint{1.701725in}{1.555607in}}%
\pgfpathlineto{\pgfqpoint{1.702109in}{1.535235in}}%
\pgfpathlineto{\pgfqpoint{1.702875in}{1.547458in}}%
\pgfpathlineto{\pgfqpoint{1.703259in}{1.551532in}}%
\pgfpathlineto{\pgfqpoint{1.704025in}{1.531161in}}%
\pgfpathlineto{\pgfqpoint{1.704408in}{1.547458in}}%
\pgfpathlineto{\pgfqpoint{1.705175in}{1.539309in}}%
\pgfpathlineto{\pgfqpoint{1.705559in}{1.567830in}}%
\pgfpathlineto{\pgfqpoint{1.705559in}{1.567830in}}%
\pgfpathlineto{\pgfqpoint{1.705559in}{1.567830in}}%
\pgfpathlineto{\pgfqpoint{1.705942in}{1.535235in}}%
\pgfpathlineto{\pgfqpoint{1.706708in}{1.551532in}}%
\pgfpathlineto{\pgfqpoint{1.707859in}{1.535235in}}%
\pgfpathlineto{\pgfqpoint{1.709390in}{1.555607in}}%
\pgfpathlineto{\pgfqpoint{1.709774in}{1.551532in}}%
\pgfpathlineto{\pgfqpoint{1.710159in}{1.559681in}}%
\pgfpathlineto{\pgfqpoint{1.710159in}{1.559681in}}%
\pgfpathlineto{\pgfqpoint{1.710159in}{1.559681in}}%
\pgfpathlineto{\pgfqpoint{1.710542in}{1.547458in}}%
\pgfpathlineto{\pgfqpoint{1.711308in}{1.551532in}}%
\pgfpathlineto{\pgfqpoint{1.711690in}{1.563756in}}%
\pgfpathlineto{\pgfqpoint{1.711690in}{1.563756in}}%
\pgfpathlineto{\pgfqpoint{1.711690in}{1.563756in}}%
\pgfpathlineto{\pgfqpoint{1.712458in}{1.543384in}}%
\pgfpathlineto{\pgfqpoint{1.712843in}{1.547458in}}%
\pgfpathlineto{\pgfqpoint{1.713226in}{1.559681in}}%
\pgfpathlineto{\pgfqpoint{1.713609in}{1.551532in}}%
\pgfpathlineto{\pgfqpoint{1.713991in}{1.543384in}}%
\pgfpathlineto{\pgfqpoint{1.714374in}{1.559681in}}%
\pgfpathlineto{\pgfqpoint{1.714757in}{1.555607in}}%
\pgfpathlineto{\pgfqpoint{1.715908in}{1.539309in}}%
\pgfpathlineto{\pgfqpoint{1.716291in}{1.563756in}}%
\pgfpathlineto{\pgfqpoint{1.717058in}{1.543384in}}%
\pgfpathlineto{\pgfqpoint{1.717441in}{1.539309in}}%
\pgfpathlineto{\pgfqpoint{1.717824in}{1.543384in}}%
\pgfpathlineto{\pgfqpoint{1.718208in}{1.563756in}}%
\pgfpathlineto{\pgfqpoint{1.718591in}{1.551532in}}%
\pgfpathlineto{\pgfqpoint{1.718974in}{1.539309in}}%
\pgfpathlineto{\pgfqpoint{1.719357in}{1.543384in}}%
\pgfpathlineto{\pgfqpoint{1.719740in}{1.551532in}}%
\pgfpathlineto{\pgfqpoint{1.720124in}{1.547458in}}%
\pgfpathlineto{\pgfqpoint{1.720507in}{1.535235in}}%
\pgfpathlineto{\pgfqpoint{1.720892in}{1.543384in}}%
\pgfpathlineto{\pgfqpoint{1.721658in}{1.555607in}}%
\pgfpathlineto{\pgfqpoint{1.722807in}{1.543384in}}%
\pgfpathlineto{\pgfqpoint{1.723198in}{1.555607in}}%
\pgfpathlineto{\pgfqpoint{1.723198in}{1.555607in}}%
\pgfpathlineto{\pgfqpoint{1.723198in}{1.555607in}}%
\pgfpathlineto{\pgfqpoint{1.723965in}{1.535235in}}%
\pgfpathlineto{\pgfqpoint{1.724348in}{1.551532in}}%
\pgfpathlineto{\pgfqpoint{1.724732in}{1.535235in}}%
\pgfpathlineto{\pgfqpoint{1.725498in}{1.539309in}}%
\pgfpathlineto{\pgfqpoint{1.727031in}{1.551532in}}%
\pgfpathlineto{\pgfqpoint{1.728565in}{1.543384in}}%
\pgfpathlineto{\pgfqpoint{1.728947in}{1.543384in}}%
\pgfpathlineto{\pgfqpoint{1.729713in}{1.563756in}}%
\pgfpathlineto{\pgfqpoint{1.730096in}{1.547458in}}%
\pgfpathlineto{\pgfqpoint{1.730481in}{1.527086in}}%
\pgfpathlineto{\pgfqpoint{1.730481in}{1.527086in}}%
\pgfpathlineto{\pgfqpoint{1.730481in}{1.527086in}}%
\pgfpathlineto{\pgfqpoint{1.730863in}{1.551532in}}%
\pgfpathlineto{\pgfqpoint{1.731630in}{1.539309in}}%
\pgfpathlineto{\pgfqpoint{1.732013in}{1.539309in}}%
\pgfpathlineto{\pgfqpoint{1.733164in}{1.551532in}}%
\pgfpathlineto{\pgfqpoint{1.733548in}{1.535235in}}%
\pgfpathlineto{\pgfqpoint{1.733548in}{1.535235in}}%
\pgfpathlineto{\pgfqpoint{1.733548in}{1.535235in}}%
\pgfpathlineto{\pgfqpoint{1.733930in}{1.555607in}}%
\pgfpathlineto{\pgfqpoint{1.734313in}{1.535235in}}%
\pgfpathlineto{\pgfqpoint{1.734696in}{1.535235in}}%
\pgfpathlineto{\pgfqpoint{1.735080in}{1.567830in}}%
\pgfpathlineto{\pgfqpoint{1.735463in}{1.551532in}}%
\pgfpathlineto{\pgfqpoint{1.735847in}{1.531161in}}%
\pgfpathlineto{\pgfqpoint{1.736613in}{1.547458in}}%
\pgfpathlineto{\pgfqpoint{1.737378in}{1.555607in}}%
\pgfpathlineto{\pgfqpoint{1.737761in}{1.543384in}}%
\pgfpathlineto{\pgfqpoint{1.738527in}{1.551532in}}%
\pgfpathlineto{\pgfqpoint{1.738910in}{1.547458in}}%
\pgfpathlineto{\pgfqpoint{1.739294in}{1.551532in}}%
\pgfpathlineto{\pgfqpoint{1.739678in}{1.551532in}}%
\pgfpathlineto{\pgfqpoint{1.740826in}{1.547458in}}%
\pgfpathlineto{\pgfqpoint{1.741295in}{1.555607in}}%
\pgfpathlineto{\pgfqpoint{1.741295in}{1.555607in}}%
\pgfpathlineto{\pgfqpoint{1.741295in}{1.555607in}}%
\pgfpathlineto{\pgfqpoint{1.741678in}{1.543384in}}%
\pgfpathlineto{\pgfqpoint{1.741678in}{1.543384in}}%
\pgfpathlineto{\pgfqpoint{1.741678in}{1.543384in}}%
\pgfpathlineto{\pgfqpoint{1.742063in}{1.559681in}}%
\pgfpathlineto{\pgfqpoint{1.742828in}{1.547458in}}%
\pgfpathlineto{\pgfqpoint{1.743211in}{1.559681in}}%
\pgfpathlineto{\pgfqpoint{1.743594in}{1.555607in}}%
\pgfpathlineto{\pgfqpoint{1.743977in}{1.543384in}}%
\pgfpathlineto{\pgfqpoint{1.743977in}{1.543384in}}%
\pgfpathlineto{\pgfqpoint{1.743977in}{1.543384in}}%
\pgfpathlineto{\pgfqpoint{1.744361in}{1.559681in}}%
\pgfpathlineto{\pgfqpoint{1.744745in}{1.543384in}}%
\pgfpathlineto{\pgfqpoint{1.745511in}{1.523012in}}%
\pgfpathlineto{\pgfqpoint{1.747043in}{1.555607in}}%
\pgfpathlineto{\pgfqpoint{1.747427in}{1.555607in}}%
\pgfpathlineto{\pgfqpoint{1.747810in}{1.543384in}}%
\pgfpathlineto{\pgfqpoint{1.747810in}{1.543384in}}%
\pgfpathlineto{\pgfqpoint{1.747810in}{1.543384in}}%
\pgfpathlineto{\pgfqpoint{1.748193in}{1.559681in}}%
\pgfpathlineto{\pgfqpoint{1.748575in}{1.547458in}}%
\pgfpathlineto{\pgfqpoint{1.748959in}{1.543384in}}%
\pgfpathlineto{\pgfqpoint{1.749342in}{1.563756in}}%
\pgfpathlineto{\pgfqpoint{1.749725in}{1.555607in}}%
\pgfpathlineto{\pgfqpoint{1.750876in}{1.539309in}}%
\pgfpathlineto{\pgfqpoint{1.752024in}{1.559681in}}%
\pgfpathlineto{\pgfqpoint{1.753557in}{1.539309in}}%
\pgfpathlineto{\pgfqpoint{1.753941in}{1.551532in}}%
\pgfpathlineto{\pgfqpoint{1.753941in}{1.551532in}}%
\pgfpathlineto{\pgfqpoint{1.753941in}{1.551532in}}%
\pgfpathlineto{\pgfqpoint{1.755473in}{1.527086in}}%
\pgfpathlineto{\pgfqpoint{1.755856in}{1.559681in}}%
\pgfpathlineto{\pgfqpoint{1.756239in}{1.543384in}}%
\pgfpathlineto{\pgfqpoint{1.756622in}{1.523012in}}%
\pgfpathlineto{\pgfqpoint{1.757005in}{1.571904in}}%
\pgfpathlineto{\pgfqpoint{1.757771in}{1.539309in}}%
\pgfpathlineto{\pgfqpoint{1.758537in}{1.559681in}}%
\pgfpathlineto{\pgfqpoint{1.758920in}{1.555607in}}%
\pgfpathlineto{\pgfqpoint{1.759303in}{1.547458in}}%
\pgfpathlineto{\pgfqpoint{1.759686in}{1.567830in}}%
\pgfpathlineto{\pgfqpoint{1.760069in}{1.547458in}}%
\pgfpathlineto{\pgfqpoint{1.760453in}{1.539309in}}%
\pgfpathlineto{\pgfqpoint{1.761601in}{1.563756in}}%
\pgfpathlineto{\pgfqpoint{1.762753in}{1.535235in}}%
\pgfpathlineto{\pgfqpoint{1.763144in}{1.539309in}}%
\pgfpathlineto{\pgfqpoint{1.763528in}{1.539309in}}%
\pgfpathlineto{\pgfqpoint{1.764678in}{1.555607in}}%
\pgfpathlineto{\pgfqpoint{1.765061in}{1.547458in}}%
\pgfpathlineto{\pgfqpoint{1.765061in}{1.547458in}}%
\pgfpathlineto{\pgfqpoint{1.765061in}{1.547458in}}%
\pgfpathlineto{\pgfqpoint{1.765443in}{1.559681in}}%
\pgfpathlineto{\pgfqpoint{1.765827in}{1.523012in}}%
\pgfpathlineto{\pgfqpoint{1.766594in}{1.539309in}}%
\pgfpathlineto{\pgfqpoint{1.768126in}{1.559681in}}%
\pgfpathlineto{\pgfqpoint{1.769277in}{1.543384in}}%
\pgfpathlineto{\pgfqpoint{1.769660in}{1.551532in}}%
\pgfpathlineto{\pgfqpoint{1.770043in}{1.523012in}}%
\pgfpathlineto{\pgfqpoint{1.770426in}{1.531161in}}%
\pgfpathlineto{\pgfqpoint{1.771960in}{1.555607in}}%
\pgfpathlineto{\pgfqpoint{1.773109in}{1.535235in}}%
\pgfpathlineto{\pgfqpoint{1.774258in}{1.559681in}}%
\pgfpathlineto{\pgfqpoint{1.774642in}{1.535235in}}%
\pgfpathlineto{\pgfqpoint{1.775025in}{1.547458in}}%
\pgfpathlineto{\pgfqpoint{1.775407in}{1.563756in}}%
\pgfpathlineto{\pgfqpoint{1.775407in}{1.563756in}}%
\pgfpathlineto{\pgfqpoint{1.775407in}{1.563756in}}%
\pgfpathlineto{\pgfqpoint{1.775790in}{1.535235in}}%
\pgfpathlineto{\pgfqpoint{1.776173in}{1.547458in}}%
\pgfpathlineto{\pgfqpoint{1.776556in}{1.563756in}}%
\pgfpathlineto{\pgfqpoint{1.776939in}{1.551532in}}%
\pgfpathlineto{\pgfqpoint{1.777322in}{1.547458in}}%
\pgfpathlineto{\pgfqpoint{1.777705in}{1.555607in}}%
\pgfpathlineto{\pgfqpoint{1.778088in}{1.535235in}}%
\pgfpathlineto{\pgfqpoint{1.778940in}{1.543384in}}%
\pgfpathlineto{\pgfqpoint{1.779706in}{1.527086in}}%
\pgfpathlineto{\pgfqpoint{1.780472in}{1.555607in}}%
\pgfpathlineto{\pgfqpoint{1.780855in}{1.543384in}}%
\pgfpathlineto{\pgfqpoint{1.781239in}{1.531161in}}%
\pgfpathlineto{\pgfqpoint{1.781622in}{1.571904in}}%
\pgfpathlineto{\pgfqpoint{1.782388in}{1.551532in}}%
\pgfpathlineto{\pgfqpoint{1.782771in}{1.531161in}}%
\pgfpathlineto{\pgfqpoint{1.783155in}{1.547458in}}%
\pgfpathlineto{\pgfqpoint{1.783537in}{1.559681in}}%
\pgfpathlineto{\pgfqpoint{1.783537in}{1.559681in}}%
\pgfpathlineto{\pgfqpoint{1.783537in}{1.559681in}}%
\pgfpathlineto{\pgfqpoint{1.783920in}{1.539309in}}%
\pgfpathlineto{\pgfqpoint{1.784687in}{1.555607in}}%
\pgfpathlineto{\pgfqpoint{1.785071in}{1.551532in}}%
\pgfpathlineto{\pgfqpoint{1.785454in}{1.563756in}}%
\pgfpathlineto{\pgfqpoint{1.786219in}{1.555607in}}%
\pgfpathlineto{\pgfqpoint{1.787370in}{1.527086in}}%
\pgfpathlineto{\pgfqpoint{1.788520in}{1.563756in}}%
\pgfpathlineto{\pgfqpoint{1.788902in}{1.551532in}}%
\pgfpathlineto{\pgfqpoint{1.789670in}{1.535235in}}%
\pgfpathlineto{\pgfqpoint{1.790054in}{1.543384in}}%
\pgfpathlineto{\pgfqpoint{1.790437in}{1.547458in}}%
\pgfpathlineto{\pgfqpoint{1.790820in}{1.535235in}}%
\pgfpathlineto{\pgfqpoint{1.791203in}{1.547458in}}%
\pgfpathlineto{\pgfqpoint{1.791587in}{1.551532in}}%
\pgfpathlineto{\pgfqpoint{1.791971in}{1.531161in}}%
\pgfpathlineto{\pgfqpoint{1.792354in}{1.547458in}}%
\pgfpathlineto{\pgfqpoint{1.792738in}{1.559681in}}%
\pgfpathlineto{\pgfqpoint{1.793121in}{1.555607in}}%
\pgfpathlineto{\pgfqpoint{1.793504in}{1.531161in}}%
\pgfpathlineto{\pgfqpoint{1.794271in}{1.551532in}}%
\pgfpathlineto{\pgfqpoint{1.795038in}{1.539309in}}%
\pgfpathlineto{\pgfqpoint{1.795421in}{1.547458in}}%
\pgfpathlineto{\pgfqpoint{1.795804in}{1.559681in}}%
\pgfpathlineto{\pgfqpoint{1.796187in}{1.551532in}}%
\pgfpathlineto{\pgfqpoint{1.796571in}{1.535235in}}%
\pgfpathlineto{\pgfqpoint{1.796571in}{1.535235in}}%
\pgfpathlineto{\pgfqpoint{1.796571in}{1.535235in}}%
\pgfpathlineto{\pgfqpoint{1.796954in}{1.563756in}}%
\pgfpathlineto{\pgfqpoint{1.796954in}{1.563756in}}%
\pgfpathlineto{\pgfqpoint{1.796954in}{1.563756in}}%
\pgfpathlineto{\pgfqpoint{1.797338in}{1.531161in}}%
\pgfpathlineto{\pgfqpoint{1.798104in}{1.543384in}}%
\pgfpathlineto{\pgfqpoint{1.798487in}{1.547458in}}%
\pgfpathlineto{\pgfqpoint{1.798870in}{1.535235in}}%
\pgfpathlineto{\pgfqpoint{1.799253in}{1.559681in}}%
\pgfpathlineto{\pgfqpoint{1.799637in}{1.551532in}}%
\pgfpathlineto{\pgfqpoint{1.800021in}{1.535235in}}%
\pgfpathlineto{\pgfqpoint{1.800021in}{1.535235in}}%
\pgfpathlineto{\pgfqpoint{1.800021in}{1.535235in}}%
\pgfpathlineto{\pgfqpoint{1.800787in}{1.559681in}}%
\pgfpathlineto{\pgfqpoint{1.801171in}{1.555607in}}%
\pgfpathlineto{\pgfqpoint{1.801554in}{1.531161in}}%
\pgfpathlineto{\pgfqpoint{1.801554in}{1.531161in}}%
\pgfpathlineto{\pgfqpoint{1.801554in}{1.531161in}}%
\pgfpathlineto{\pgfqpoint{1.801937in}{1.571904in}}%
\pgfpathlineto{\pgfqpoint{1.802320in}{1.555607in}}%
\pgfpathlineto{\pgfqpoint{1.802704in}{1.527086in}}%
\pgfpathlineto{\pgfqpoint{1.802704in}{1.527086in}}%
\pgfpathlineto{\pgfqpoint{1.802704in}{1.527086in}}%
\pgfpathlineto{\pgfqpoint{1.803095in}{1.563756in}}%
\pgfpathlineto{\pgfqpoint{1.803864in}{1.539309in}}%
\pgfpathlineto{\pgfqpoint{1.804248in}{1.535235in}}%
\pgfpathlineto{\pgfqpoint{1.804631in}{1.539309in}}%
\pgfpathlineto{\pgfqpoint{1.805014in}{1.563756in}}%
\pgfpathlineto{\pgfqpoint{1.805014in}{1.563756in}}%
\pgfpathlineto{\pgfqpoint{1.805014in}{1.563756in}}%
\pgfpathlineto{\pgfqpoint{1.806163in}{1.531161in}}%
\pgfpathlineto{\pgfqpoint{1.806930in}{1.547458in}}%
\pgfpathlineto{\pgfqpoint{1.807313in}{1.535235in}}%
\pgfpathlineto{\pgfqpoint{1.808463in}{1.551532in}}%
\pgfpathlineto{\pgfqpoint{1.808846in}{1.539309in}}%
\pgfpathlineto{\pgfqpoint{1.808846in}{1.539309in}}%
\pgfpathlineto{\pgfqpoint{1.808846in}{1.539309in}}%
\pgfpathlineto{\pgfqpoint{1.809230in}{1.559681in}}%
\pgfpathlineto{\pgfqpoint{1.809997in}{1.551532in}}%
\pgfpathlineto{\pgfqpoint{1.810380in}{1.551532in}}%
\pgfpathlineto{\pgfqpoint{1.811145in}{1.539309in}}%
\pgfpathlineto{\pgfqpoint{1.811528in}{1.551532in}}%
\pgfpathlineto{\pgfqpoint{1.811528in}{1.551532in}}%
\pgfpathlineto{\pgfqpoint{1.811528in}{1.551532in}}%
\pgfpathlineto{\pgfqpoint{1.812295in}{1.531161in}}%
\pgfpathlineto{\pgfqpoint{1.812679in}{1.555607in}}%
\pgfpathlineto{\pgfqpoint{1.813062in}{1.543384in}}%
\pgfpathlineto{\pgfqpoint{1.813445in}{1.531161in}}%
\pgfpathlineto{\pgfqpoint{1.813445in}{1.531161in}}%
\pgfpathlineto{\pgfqpoint{1.813445in}{1.531161in}}%
\pgfpathlineto{\pgfqpoint{1.814977in}{1.551532in}}%
\pgfpathlineto{\pgfqpoint{1.815361in}{1.543384in}}%
\pgfpathlineto{\pgfqpoint{1.815745in}{1.575979in}}%
\pgfpathlineto{\pgfqpoint{1.816128in}{1.547458in}}%
\pgfpathlineto{\pgfqpoint{1.817277in}{1.535235in}}%
\pgfpathlineto{\pgfqpoint{1.818044in}{1.547458in}}%
\pgfpathlineto{\pgfqpoint{1.818427in}{1.531161in}}%
\pgfpathlineto{\pgfqpoint{1.818810in}{1.535235in}}%
\pgfpathlineto{\pgfqpoint{1.820428in}{1.567830in}}%
\pgfpathlineto{\pgfqpoint{1.821194in}{1.535235in}}%
\pgfpathlineto{\pgfqpoint{1.821962in}{1.539309in}}%
\pgfpathlineto{\pgfqpoint{1.822728in}{1.567830in}}%
\pgfpathlineto{\pgfqpoint{1.823110in}{1.543384in}}%
\pgfpathlineto{\pgfqpoint{1.823493in}{1.535235in}}%
\pgfpathlineto{\pgfqpoint{1.823493in}{1.535235in}}%
\pgfpathlineto{\pgfqpoint{1.823493in}{1.535235in}}%
\pgfpathlineto{\pgfqpoint{1.824645in}{1.551532in}}%
\pgfpathlineto{\pgfqpoint{1.825028in}{1.547458in}}%
\pgfpathlineto{\pgfqpoint{1.825411in}{1.551532in}}%
\pgfpathlineto{\pgfqpoint{1.825794in}{1.563756in}}%
\pgfpathlineto{\pgfqpoint{1.825794in}{1.563756in}}%
\pgfpathlineto{\pgfqpoint{1.825794in}{1.563756in}}%
\pgfpathlineto{\pgfqpoint{1.826560in}{1.531161in}}%
\pgfpathlineto{\pgfqpoint{1.826943in}{1.543384in}}%
\pgfpathlineto{\pgfqpoint{1.827328in}{1.563756in}}%
\pgfpathlineto{\pgfqpoint{1.827711in}{1.547458in}}%
\pgfpathlineto{\pgfqpoint{1.828860in}{1.527086in}}%
\pgfpathlineto{\pgfqpoint{1.829244in}{1.543384in}}%
\pgfpathlineto{\pgfqpoint{1.830010in}{1.539309in}}%
\pgfpathlineto{\pgfqpoint{1.830777in}{1.547458in}}%
\pgfpathlineto{\pgfqpoint{1.831161in}{1.539309in}}%
\pgfpathlineto{\pgfqpoint{1.831161in}{1.539309in}}%
\pgfpathlineto{\pgfqpoint{1.831161in}{1.539309in}}%
\pgfpathlineto{\pgfqpoint{1.831927in}{1.567830in}}%
\pgfpathlineto{\pgfqpoint{1.832311in}{1.551532in}}%
\pgfpathlineto{\pgfqpoint{1.834226in}{1.535235in}}%
\pgfpathlineto{\pgfqpoint{1.835761in}{1.559681in}}%
\pgfpathlineto{\pgfqpoint{1.836909in}{1.547458in}}%
\pgfpathlineto{\pgfqpoint{1.837290in}{1.559681in}}%
\pgfpathlineto{\pgfqpoint{1.837674in}{1.531161in}}%
\pgfpathlineto{\pgfqpoint{1.838441in}{1.543384in}}%
\pgfpathlineto{\pgfqpoint{1.838824in}{1.535235in}}%
\pgfpathlineto{\pgfqpoint{1.839207in}{1.555607in}}%
\pgfpathlineto{\pgfqpoint{1.839977in}{1.543384in}}%
\pgfpathlineto{\pgfqpoint{1.840360in}{1.531161in}}%
\pgfpathlineto{\pgfqpoint{1.841127in}{1.535235in}}%
\pgfpathlineto{\pgfqpoint{1.841893in}{1.555607in}}%
\pgfpathlineto{\pgfqpoint{1.842277in}{1.535235in}}%
\pgfpathlineto{\pgfqpoint{1.843050in}{1.551532in}}%
\pgfpathlineto{\pgfqpoint{1.843816in}{1.551532in}}%
\pgfpathlineto{\pgfqpoint{1.845351in}{1.543384in}}%
\pgfpathlineto{\pgfqpoint{1.846883in}{1.555607in}}%
\pgfpathlineto{\pgfqpoint{1.847266in}{1.551532in}}%
\pgfpathlineto{\pgfqpoint{1.847649in}{1.531161in}}%
\pgfpathlineto{\pgfqpoint{1.848416in}{1.543384in}}%
\pgfpathlineto{\pgfqpoint{1.848799in}{1.559681in}}%
\pgfpathlineto{\pgfqpoint{1.848799in}{1.559681in}}%
\pgfpathlineto{\pgfqpoint{1.848799in}{1.559681in}}%
\pgfpathlineto{\pgfqpoint{1.849182in}{1.535235in}}%
\pgfpathlineto{\pgfqpoint{1.849949in}{1.539309in}}%
\pgfpathlineto{\pgfqpoint{1.850332in}{1.567830in}}%
\pgfpathlineto{\pgfqpoint{1.850715in}{1.547458in}}%
\pgfpathlineto{\pgfqpoint{1.851098in}{1.539309in}}%
\pgfpathlineto{\pgfqpoint{1.852632in}{1.563756in}}%
\pgfpathlineto{\pgfqpoint{1.853782in}{1.531161in}}%
\pgfpathlineto{\pgfqpoint{1.854165in}{1.535235in}}%
\pgfpathlineto{\pgfqpoint{1.854931in}{1.555607in}}%
\pgfpathlineto{\pgfqpoint{1.855312in}{1.547458in}}%
\pgfpathlineto{\pgfqpoint{1.856463in}{1.531161in}}%
\pgfpathlineto{\pgfqpoint{1.857228in}{1.567830in}}%
\pgfpathlineto{\pgfqpoint{1.857993in}{1.551532in}}%
\pgfpathlineto{\pgfqpoint{1.858377in}{1.543384in}}%
\pgfpathlineto{\pgfqpoint{1.859145in}{1.563756in}}%
\pgfpathlineto{\pgfqpoint{1.859528in}{1.559681in}}%
\pgfpathlineto{\pgfqpoint{1.860762in}{1.543384in}}%
\pgfpathlineto{\pgfqpoint{1.861145in}{1.539309in}}%
\pgfpathlineto{\pgfqpoint{1.861912in}{1.555607in}}%
\pgfpathlineto{\pgfqpoint{1.862296in}{1.543384in}}%
\pgfpathlineto{\pgfqpoint{1.862679in}{1.535235in}}%
\pgfpathlineto{\pgfqpoint{1.862679in}{1.535235in}}%
\pgfpathlineto{\pgfqpoint{1.862679in}{1.535235in}}%
\pgfpathlineto{\pgfqpoint{1.863827in}{1.567830in}}%
\pgfpathlineto{\pgfqpoint{1.864210in}{1.555607in}}%
\pgfpathlineto{\pgfqpoint{1.864592in}{1.555607in}}%
\pgfpathlineto{\pgfqpoint{1.866126in}{1.535235in}}%
\pgfpathlineto{\pgfqpoint{1.867658in}{1.571904in}}%
\pgfpathlineto{\pgfqpoint{1.868807in}{1.535235in}}%
\pgfpathlineto{\pgfqpoint{1.869190in}{1.563756in}}%
\pgfpathlineto{\pgfqpoint{1.869573in}{1.539309in}}%
\pgfpathlineto{\pgfqpoint{1.870723in}{1.535235in}}%
\pgfpathlineto{\pgfqpoint{1.871106in}{1.559681in}}%
\pgfpathlineto{\pgfqpoint{1.871872in}{1.547458in}}%
\pgfpathlineto{\pgfqpoint{1.872255in}{1.555607in}}%
\pgfpathlineto{\pgfqpoint{1.872638in}{1.547458in}}%
\pgfpathlineto{\pgfqpoint{1.873021in}{1.535235in}}%
\pgfpathlineto{\pgfqpoint{1.873021in}{1.535235in}}%
\pgfpathlineto{\pgfqpoint{1.873021in}{1.535235in}}%
\pgfpathlineto{\pgfqpoint{1.874172in}{1.555607in}}%
\pgfpathlineto{\pgfqpoint{1.874555in}{1.535235in}}%
\pgfpathlineto{\pgfqpoint{1.875322in}{1.547458in}}%
\pgfpathlineto{\pgfqpoint{1.876088in}{1.547458in}}%
\pgfpathlineto{\pgfqpoint{1.877239in}{1.535235in}}%
\pgfpathlineto{\pgfqpoint{1.877622in}{1.559681in}}%
\pgfpathlineto{\pgfqpoint{1.878387in}{1.551532in}}%
\pgfpathlineto{\pgfqpoint{1.879538in}{1.531161in}}%
\pgfpathlineto{\pgfqpoint{1.879921in}{1.535235in}}%
\pgfpathlineto{\pgfqpoint{1.880303in}{1.555607in}}%
\pgfpathlineto{\pgfqpoint{1.880686in}{1.551532in}}%
\pgfpathlineto{\pgfqpoint{1.881453in}{1.531161in}}%
\pgfpathlineto{\pgfqpoint{1.881836in}{1.543384in}}%
\pgfpathlineto{\pgfqpoint{1.882219in}{1.539309in}}%
\pgfpathlineto{\pgfqpoint{1.882602in}{1.543384in}}%
\pgfpathlineto{\pgfqpoint{1.882992in}{1.567830in}}%
\pgfpathlineto{\pgfqpoint{1.883376in}{1.559681in}}%
\pgfpathlineto{\pgfqpoint{1.883760in}{1.539309in}}%
\pgfpathlineto{\pgfqpoint{1.884526in}{1.547458in}}%
\pgfpathlineto{\pgfqpoint{1.885676in}{1.531161in}}%
\pgfpathlineto{\pgfqpoint{1.886058in}{1.563756in}}%
\pgfpathlineto{\pgfqpoint{1.886825in}{1.539309in}}%
\pgfpathlineto{\pgfqpoint{1.887208in}{1.563756in}}%
\pgfpathlineto{\pgfqpoint{1.887973in}{1.555607in}}%
\pgfpathlineto{\pgfqpoint{1.888356in}{1.559681in}}%
\pgfpathlineto{\pgfqpoint{1.889888in}{1.543384in}}%
\pgfpathlineto{\pgfqpoint{1.890655in}{1.555607in}}%
\pgfpathlineto{\pgfqpoint{1.892188in}{1.535235in}}%
\pgfpathlineto{\pgfqpoint{1.893339in}{1.555607in}}%
\pgfpathlineto{\pgfqpoint{1.893722in}{1.527086in}}%
\pgfpathlineto{\pgfqpoint{1.894105in}{1.543384in}}%
\pgfpathlineto{\pgfqpoint{1.895255in}{1.567830in}}%
\pgfpathlineto{\pgfqpoint{1.896022in}{1.543384in}}%
\pgfpathlineto{\pgfqpoint{1.896405in}{1.547458in}}%
\pgfpathlineto{\pgfqpoint{1.896787in}{1.547458in}}%
\pgfpathlineto{\pgfqpoint{1.897255in}{1.551532in}}%
\pgfpathlineto{\pgfqpoint{1.897638in}{1.539309in}}%
\pgfpathlineto{\pgfqpoint{1.898406in}{1.543384in}}%
\pgfpathlineto{\pgfqpoint{1.899556in}{1.559681in}}%
\pgfpathlineto{\pgfqpoint{1.899938in}{1.555607in}}%
\pgfpathlineto{\pgfqpoint{1.900321in}{1.555607in}}%
\pgfpathlineto{\pgfqpoint{1.900704in}{1.531161in}}%
\pgfpathlineto{\pgfqpoint{1.901087in}{1.539309in}}%
\pgfpathlineto{\pgfqpoint{1.901472in}{1.563756in}}%
\pgfpathlineto{\pgfqpoint{1.902238in}{1.559681in}}%
\pgfpathlineto{\pgfqpoint{1.903003in}{1.535235in}}%
\pgfpathlineto{\pgfqpoint{1.903387in}{1.551532in}}%
\pgfpathlineto{\pgfqpoint{1.903771in}{1.555607in}}%
\pgfpathlineto{\pgfqpoint{1.904154in}{1.531161in}}%
\pgfpathlineto{\pgfqpoint{1.904538in}{1.547458in}}%
\pgfpathlineto{\pgfqpoint{1.905687in}{1.567830in}}%
\pgfpathlineto{\pgfqpoint{1.906454in}{1.535235in}}%
\pgfpathlineto{\pgfqpoint{1.906837in}{1.539309in}}%
\pgfpathlineto{\pgfqpoint{1.907221in}{1.563756in}}%
\pgfpathlineto{\pgfqpoint{1.907604in}{1.547458in}}%
\pgfpathlineto{\pgfqpoint{1.907987in}{1.539309in}}%
\pgfpathlineto{\pgfqpoint{1.907987in}{1.539309in}}%
\pgfpathlineto{\pgfqpoint{1.907987in}{1.539309in}}%
\pgfpathlineto{\pgfqpoint{1.908754in}{1.559681in}}%
\pgfpathlineto{\pgfqpoint{1.909137in}{1.551532in}}%
\pgfpathlineto{\pgfqpoint{1.910287in}{1.527086in}}%
\pgfpathlineto{\pgfqpoint{1.911053in}{1.555607in}}%
\pgfpathlineto{\pgfqpoint{1.911820in}{1.551532in}}%
\pgfpathlineto{\pgfqpoint{1.912204in}{1.559681in}}%
\pgfpathlineto{\pgfqpoint{1.912589in}{1.555607in}}%
\pgfpathlineto{\pgfqpoint{1.912972in}{1.527086in}}%
\pgfpathlineto{\pgfqpoint{1.913355in}{1.543384in}}%
\pgfpathlineto{\pgfqpoint{1.914504in}{1.555607in}}%
\pgfpathlineto{\pgfqpoint{1.914888in}{1.539309in}}%
\pgfpathlineto{\pgfqpoint{1.915272in}{1.547458in}}%
\pgfpathlineto{\pgfqpoint{1.915655in}{1.555607in}}%
\pgfpathlineto{\pgfqpoint{1.915655in}{1.555607in}}%
\pgfpathlineto{\pgfqpoint{1.915655in}{1.555607in}}%
\pgfpathlineto{\pgfqpoint{1.917188in}{1.539309in}}%
\pgfpathlineto{\pgfqpoint{1.917571in}{1.543384in}}%
\pgfpathlineto{\pgfqpoint{1.917954in}{1.559681in}}%
\pgfpathlineto{\pgfqpoint{1.918337in}{1.555607in}}%
\pgfpathlineto{\pgfqpoint{1.919487in}{1.543384in}}%
\pgfpathlineto{\pgfqpoint{1.920637in}{1.547458in}}%
\pgfpathlineto{\pgfqpoint{1.922168in}{1.535235in}}%
\pgfpathlineto{\pgfqpoint{1.922551in}{1.539309in}}%
\pgfpathlineto{\pgfqpoint{1.923708in}{1.555607in}}%
\pgfpathlineto{\pgfqpoint{1.924091in}{1.535235in}}%
\pgfpathlineto{\pgfqpoint{1.924091in}{1.535235in}}%
\pgfpathlineto{\pgfqpoint{1.924091in}{1.535235in}}%
\pgfpathlineto{\pgfqpoint{1.924474in}{1.559681in}}%
\pgfpathlineto{\pgfqpoint{1.924858in}{1.539309in}}%
\pgfpathlineto{\pgfqpoint{1.925241in}{1.531161in}}%
\pgfpathlineto{\pgfqpoint{1.925625in}{1.563756in}}%
\pgfpathlineto{\pgfqpoint{1.926391in}{1.543384in}}%
\pgfpathlineto{\pgfqpoint{1.926774in}{1.523012in}}%
\pgfpathlineto{\pgfqpoint{1.927157in}{1.543384in}}%
\pgfpathlineto{\pgfqpoint{1.927540in}{1.543384in}}%
\pgfpathlineto{\pgfqpoint{1.927924in}{1.539309in}}%
\pgfpathlineto{\pgfqpoint{1.928307in}{1.555607in}}%
\pgfpathlineto{\pgfqpoint{1.928691in}{1.543384in}}%
\pgfpathlineto{\pgfqpoint{1.929074in}{1.535235in}}%
\pgfpathlineto{\pgfqpoint{1.930223in}{1.559681in}}%
\pgfpathlineto{\pgfqpoint{1.931374in}{1.539309in}}%
\pgfpathlineto{\pgfqpoint{1.931757in}{1.543384in}}%
\pgfpathlineto{\pgfqpoint{1.932523in}{1.555607in}}%
\pgfpathlineto{\pgfqpoint{1.933291in}{1.551532in}}%
\pgfpathlineto{\pgfqpoint{1.933673in}{1.539309in}}%
\pgfpathlineto{\pgfqpoint{1.934440in}{1.543384in}}%
\pgfpathlineto{\pgfqpoint{1.934823in}{1.551532in}}%
\pgfpathlineto{\pgfqpoint{1.935206in}{1.531161in}}%
\pgfpathlineto{\pgfqpoint{1.935591in}{1.543384in}}%
\pgfpathlineto{\pgfqpoint{1.935975in}{1.559681in}}%
\pgfpathlineto{\pgfqpoint{1.936741in}{1.547458in}}%
\pgfpathlineto{\pgfqpoint{1.937124in}{1.547458in}}%
\pgfpathlineto{\pgfqpoint{1.937507in}{1.551532in}}%
\pgfpathlineto{\pgfqpoint{1.938741in}{1.543384in}}%
\pgfpathlineto{\pgfqpoint{1.939891in}{1.555607in}}%
\pgfpathlineto{\pgfqpoint{1.940274in}{1.527086in}}%
\pgfpathlineto{\pgfqpoint{1.940274in}{1.527086in}}%
\pgfpathlineto{\pgfqpoint{1.940274in}{1.527086in}}%
\pgfpathlineto{\pgfqpoint{1.940657in}{1.559681in}}%
\pgfpathlineto{\pgfqpoint{1.941423in}{1.555607in}}%
\pgfpathlineto{\pgfqpoint{1.941806in}{1.559681in}}%
\pgfpathlineto{\pgfqpoint{1.942190in}{1.547458in}}%
\pgfpathlineto{\pgfqpoint{1.942190in}{1.547458in}}%
\pgfpathlineto{\pgfqpoint{1.942190in}{1.547458in}}%
\pgfpathlineto{\pgfqpoint{1.942573in}{1.563756in}}%
\pgfpathlineto{\pgfqpoint{1.942956in}{1.547458in}}%
\pgfpathlineto{\pgfqpoint{1.943338in}{1.543384in}}%
\pgfpathlineto{\pgfqpoint{1.943721in}{1.555607in}}%
\pgfpathlineto{\pgfqpoint{1.943721in}{1.555607in}}%
\pgfpathlineto{\pgfqpoint{1.943721in}{1.555607in}}%
\pgfpathlineto{\pgfqpoint{1.944486in}{1.527086in}}%
\pgfpathlineto{\pgfqpoint{1.944869in}{1.555607in}}%
\pgfpathlineto{\pgfqpoint{1.945637in}{1.543384in}}%
\pgfpathlineto{\pgfqpoint{1.946020in}{1.547458in}}%
\pgfpathlineto{\pgfqpoint{1.946404in}{1.539309in}}%
\pgfpathlineto{\pgfqpoint{1.946404in}{1.539309in}}%
\pgfpathlineto{\pgfqpoint{1.946404in}{1.539309in}}%
\pgfpathlineto{\pgfqpoint{1.947552in}{1.567830in}}%
\pgfpathlineto{\pgfqpoint{1.947937in}{1.539309in}}%
\pgfpathlineto{\pgfqpoint{1.948703in}{1.551532in}}%
\pgfpathlineto{\pgfqpoint{1.949087in}{1.567830in}}%
\pgfpathlineto{\pgfqpoint{1.949087in}{1.567830in}}%
\pgfpathlineto{\pgfqpoint{1.949087in}{1.567830in}}%
\pgfpathlineto{\pgfqpoint{1.950234in}{1.539309in}}%
\pgfpathlineto{\pgfqpoint{1.950618in}{1.559681in}}%
\pgfpathlineto{\pgfqpoint{1.951002in}{1.551532in}}%
\pgfpathlineto{\pgfqpoint{1.951385in}{1.539309in}}%
\pgfpathlineto{\pgfqpoint{1.952151in}{1.547458in}}%
\pgfpathlineto{\pgfqpoint{1.952533in}{1.539309in}}%
\pgfpathlineto{\pgfqpoint{1.952533in}{1.539309in}}%
\pgfpathlineto{\pgfqpoint{1.952533in}{1.539309in}}%
\pgfpathlineto{\pgfqpoint{1.953300in}{1.551532in}}%
\pgfpathlineto{\pgfqpoint{1.953683in}{1.547458in}}%
\pgfpathlineto{\pgfqpoint{1.954066in}{1.543384in}}%
\pgfpathlineto{\pgfqpoint{1.955598in}{1.567830in}}%
\pgfpathlineto{\pgfqpoint{1.956749in}{1.543384in}}%
\pgfpathlineto{\pgfqpoint{1.957131in}{1.555607in}}%
\pgfpathlineto{\pgfqpoint{1.957514in}{1.531161in}}%
\pgfpathlineto{\pgfqpoint{1.957897in}{1.539309in}}%
\pgfpathlineto{\pgfqpoint{1.958280in}{1.559681in}}%
\pgfpathlineto{\pgfqpoint{1.958663in}{1.543384in}}%
\pgfpathlineto{\pgfqpoint{1.959047in}{1.531161in}}%
\pgfpathlineto{\pgfqpoint{1.959047in}{1.531161in}}%
\pgfpathlineto{\pgfqpoint{1.959047in}{1.531161in}}%
\pgfpathlineto{\pgfqpoint{1.960197in}{1.563756in}}%
\pgfpathlineto{\pgfqpoint{1.960580in}{1.555607in}}%
\pgfpathlineto{\pgfqpoint{1.961346in}{1.535235in}}%
\pgfpathlineto{\pgfqpoint{1.961729in}{1.539309in}}%
\pgfpathlineto{\pgfqpoint{1.962113in}{1.551532in}}%
\pgfpathlineto{\pgfqpoint{1.962886in}{1.547458in}}%
\pgfpathlineto{\pgfqpoint{1.963269in}{1.551532in}}%
\pgfpathlineto{\pgfqpoint{1.963652in}{1.567830in}}%
\pgfpathlineto{\pgfqpoint{1.963652in}{1.567830in}}%
\pgfpathlineto{\pgfqpoint{1.963652in}{1.567830in}}%
\pgfpathlineto{\pgfqpoint{1.965186in}{1.531161in}}%
\pgfpathlineto{\pgfqpoint{1.966720in}{1.551532in}}%
\pgfpathlineto{\pgfqpoint{1.967487in}{1.539309in}}%
\pgfpathlineto{\pgfqpoint{1.967870in}{1.551532in}}%
\pgfpathlineto{\pgfqpoint{1.968252in}{1.539309in}}%
\pgfpathlineto{\pgfqpoint{1.968635in}{1.531161in}}%
\pgfpathlineto{\pgfqpoint{1.969018in}{1.559681in}}%
\pgfpathlineto{\pgfqpoint{1.969404in}{1.535235in}}%
\pgfpathlineto{\pgfqpoint{1.970170in}{1.527086in}}%
\pgfpathlineto{\pgfqpoint{1.970553in}{1.563756in}}%
\pgfpathlineto{\pgfqpoint{1.971318in}{1.551532in}}%
\pgfpathlineto{\pgfqpoint{1.972087in}{1.543384in}}%
\pgfpathlineto{\pgfqpoint{1.972470in}{1.551532in}}%
\pgfpathlineto{\pgfqpoint{1.972853in}{1.547458in}}%
\pgfpathlineto{\pgfqpoint{1.973619in}{1.531161in}}%
\pgfpathlineto{\pgfqpoint{1.974770in}{1.547458in}}%
\pgfpathlineto{\pgfqpoint{1.975536in}{1.527086in}}%
\pgfpathlineto{\pgfqpoint{1.976004in}{1.547458in}}%
\pgfpathlineto{\pgfqpoint{1.976771in}{1.539309in}}%
\pgfpathlineto{\pgfqpoint{1.977537in}{1.531161in}}%
\pgfpathlineto{\pgfqpoint{1.979069in}{1.559681in}}%
\pgfpathlineto{\pgfqpoint{1.979453in}{1.555607in}}%
\pgfpathlineto{\pgfqpoint{1.979835in}{1.563756in}}%
\pgfpathlineto{\pgfqpoint{1.980985in}{1.543384in}}%
\pgfpathlineto{\pgfqpoint{1.981368in}{1.571904in}}%
\pgfpathlineto{\pgfqpoint{1.981751in}{1.555607in}}%
\pgfpathlineto{\pgfqpoint{1.982134in}{1.535235in}}%
\pgfpathlineto{\pgfqpoint{1.982134in}{1.535235in}}%
\pgfpathlineto{\pgfqpoint{1.982134in}{1.535235in}}%
\pgfpathlineto{\pgfqpoint{1.983668in}{1.567830in}}%
\pgfpathlineto{\pgfqpoint{1.984434in}{1.555607in}}%
\pgfpathlineto{\pgfqpoint{1.984816in}{1.531161in}}%
\pgfpathlineto{\pgfqpoint{1.985583in}{1.547458in}}%
\pgfpathlineto{\pgfqpoint{1.987117in}{1.563756in}}%
\pgfpathlineto{\pgfqpoint{1.987500in}{1.531161in}}%
\pgfpathlineto{\pgfqpoint{1.987500in}{1.531161in}}%
\pgfpathlineto{\pgfqpoint{1.987500in}{1.531161in}}%
\pgfpathlineto{\pgfqpoint{1.987882in}{1.567830in}}%
\pgfpathlineto{\pgfqpoint{1.988648in}{1.559681in}}%
\pgfpathlineto{\pgfqpoint{1.989031in}{1.547458in}}%
\pgfpathlineto{\pgfqpoint{1.989415in}{1.551532in}}%
\pgfpathlineto{\pgfqpoint{1.989799in}{1.563756in}}%
\pgfpathlineto{\pgfqpoint{1.990948in}{1.539309in}}%
\pgfpathlineto{\pgfqpoint{1.991331in}{1.543384in}}%
\pgfpathlineto{\pgfqpoint{1.991715in}{1.527086in}}%
\pgfpathlineto{\pgfqpoint{1.992483in}{1.539309in}}%
\pgfpathlineto{\pgfqpoint{1.993249in}{1.555607in}}%
\pgfpathlineto{\pgfqpoint{1.993632in}{1.539309in}}%
\pgfpathlineto{\pgfqpoint{1.994400in}{1.551532in}}%
\pgfpathlineto{\pgfqpoint{1.994783in}{1.551532in}}%
\pgfpathlineto{\pgfqpoint{1.995166in}{1.539309in}}%
\pgfpathlineto{\pgfqpoint{1.995549in}{1.551532in}}%
\pgfpathlineto{\pgfqpoint{1.995932in}{1.567830in}}%
\pgfpathlineto{\pgfqpoint{1.995932in}{1.567830in}}%
\pgfpathlineto{\pgfqpoint{1.995932in}{1.567830in}}%
\pgfpathlineto{\pgfqpoint{1.996315in}{1.543384in}}%
\pgfpathlineto{\pgfqpoint{1.997082in}{1.559681in}}%
\pgfpathlineto{\pgfqpoint{1.997465in}{1.551532in}}%
\pgfpathlineto{\pgfqpoint{1.997849in}{1.567830in}}%
\pgfpathlineto{\pgfqpoint{1.998615in}{1.563756in}}%
\pgfpathlineto{\pgfqpoint{1.999766in}{1.531161in}}%
\pgfpathlineto{\pgfqpoint{2.000150in}{1.567830in}}%
\pgfpathlineto{\pgfqpoint{2.000916in}{1.547458in}}%
\pgfpathlineto{\pgfqpoint{2.001299in}{1.539309in}}%
\pgfpathlineto{\pgfqpoint{2.001682in}{1.547458in}}%
\pgfpathlineto{\pgfqpoint{2.002839in}{1.575979in}}%
\pgfpathlineto{\pgfqpoint{2.003990in}{1.523012in}}%
\pgfpathlineto{\pgfqpoint{2.004757in}{1.559681in}}%
\pgfpathlineto{\pgfqpoint{2.005140in}{1.551532in}}%
\pgfpathlineto{\pgfqpoint{2.006289in}{1.535235in}}%
\pgfpathlineto{\pgfqpoint{2.007822in}{1.563756in}}%
\pgfpathlineto{\pgfqpoint{2.008588in}{1.543384in}}%
\pgfpathlineto{\pgfqpoint{2.008971in}{1.559681in}}%
\pgfpathlineto{\pgfqpoint{2.009354in}{1.559681in}}%
\pgfpathlineto{\pgfqpoint{2.010503in}{1.535235in}}%
\pgfpathlineto{\pgfqpoint{2.010887in}{1.539309in}}%
\pgfpathlineto{\pgfqpoint{2.011270in}{1.555607in}}%
\pgfpathlineto{\pgfqpoint{2.012035in}{1.543384in}}%
\pgfpathlineto{\pgfqpoint{2.012803in}{1.535235in}}%
\pgfpathlineto{\pgfqpoint{2.013953in}{1.551532in}}%
\pgfpathlineto{\pgfqpoint{2.014335in}{1.527086in}}%
\pgfpathlineto{\pgfqpoint{2.015187in}{1.539309in}}%
\pgfpathlineto{\pgfqpoint{2.016337in}{1.559681in}}%
\pgfpathlineto{\pgfqpoint{2.016721in}{1.551532in}}%
\pgfpathlineto{\pgfqpoint{2.017104in}{1.527086in}}%
\pgfpathlineto{\pgfqpoint{2.017870in}{1.535235in}}%
\pgfpathlineto{\pgfqpoint{2.018253in}{1.567830in}}%
\pgfpathlineto{\pgfqpoint{2.019020in}{1.547458in}}%
\pgfpathlineto{\pgfqpoint{2.019405in}{1.555607in}}%
\pgfpathlineto{\pgfqpoint{2.020171in}{1.531161in}}%
\pgfpathlineto{\pgfqpoint{2.020554in}{1.547458in}}%
\pgfpathlineto{\pgfqpoint{2.020936in}{1.535235in}}%
\pgfpathlineto{\pgfqpoint{2.021704in}{1.543384in}}%
\pgfpathlineto{\pgfqpoint{2.022088in}{1.563756in}}%
\pgfpathlineto{\pgfqpoint{2.022853in}{1.547458in}}%
\pgfpathlineto{\pgfqpoint{2.023236in}{1.563756in}}%
\pgfpathlineto{\pgfqpoint{2.023236in}{1.563756in}}%
\pgfpathlineto{\pgfqpoint{2.023236in}{1.563756in}}%
\pgfpathlineto{\pgfqpoint{2.024387in}{1.539309in}}%
\pgfpathlineto{\pgfqpoint{2.024770in}{1.567830in}}%
\pgfpathlineto{\pgfqpoint{2.025536in}{1.551532in}}%
\pgfpathlineto{\pgfqpoint{2.026685in}{1.551532in}}%
\pgfpathlineto{\pgfqpoint{2.027836in}{1.539309in}}%
\pgfpathlineto{\pgfqpoint{2.028602in}{1.559681in}}%
\pgfpathlineto{\pgfqpoint{2.028985in}{1.547458in}}%
\pgfpathlineto{\pgfqpoint{2.030519in}{1.527086in}}%
\pgfpathlineto{\pgfqpoint{2.030903in}{1.547458in}}%
\pgfpathlineto{\pgfqpoint{2.031669in}{1.535235in}}%
\pgfpathlineto{\pgfqpoint{2.032821in}{1.563756in}}%
\pgfpathlineto{\pgfqpoint{2.033204in}{1.539309in}}%
\pgfpathlineto{\pgfqpoint{2.033970in}{1.543384in}}%
\pgfpathlineto{\pgfqpoint{2.034736in}{1.551532in}}%
\pgfpathlineto{\pgfqpoint{2.035120in}{1.539309in}}%
\pgfpathlineto{\pgfqpoint{2.035120in}{1.539309in}}%
\pgfpathlineto{\pgfqpoint{2.035120in}{1.539309in}}%
\pgfpathlineto{\pgfqpoint{2.035505in}{1.555607in}}%
\pgfpathlineto{\pgfqpoint{2.035505in}{1.555607in}}%
\pgfpathlineto{\pgfqpoint{2.035505in}{1.555607in}}%
\pgfpathlineto{\pgfqpoint{2.035888in}{1.531161in}}%
\pgfpathlineto{\pgfqpoint{2.036271in}{1.547458in}}%
\pgfpathlineto{\pgfqpoint{2.036653in}{1.555607in}}%
\pgfpathlineto{\pgfqpoint{2.037419in}{1.551532in}}%
\pgfpathlineto{\pgfqpoint{2.037804in}{1.539309in}}%
\pgfpathlineto{\pgfqpoint{2.038188in}{1.551532in}}%
\pgfpathlineto{\pgfqpoint{2.038571in}{1.555607in}}%
\pgfpathlineto{\pgfqpoint{2.039719in}{1.531161in}}%
\pgfpathlineto{\pgfqpoint{2.040103in}{1.559681in}}%
\pgfpathlineto{\pgfqpoint{2.040869in}{1.547458in}}%
\pgfpathlineto{\pgfqpoint{2.041252in}{1.547458in}}%
\pgfpathlineto{\pgfqpoint{2.041635in}{1.555607in}}%
\pgfpathlineto{\pgfqpoint{2.041635in}{1.555607in}}%
\pgfpathlineto{\pgfqpoint{2.041635in}{1.555607in}}%
\pgfpathlineto{\pgfqpoint{2.042401in}{1.527086in}}%
\pgfpathlineto{\pgfqpoint{2.042792in}{1.539309in}}%
\pgfpathlineto{\pgfqpoint{2.043175in}{1.563756in}}%
\pgfpathlineto{\pgfqpoint{2.043558in}{1.559681in}}%
\pgfpathlineto{\pgfqpoint{2.044709in}{1.531161in}}%
\pgfpathlineto{\pgfqpoint{2.045475in}{1.563756in}}%
\pgfpathlineto{\pgfqpoint{2.045859in}{1.535235in}}%
\pgfpathlineto{\pgfqpoint{2.047390in}{1.563756in}}%
\pgfpathlineto{\pgfqpoint{2.047774in}{1.527086in}}%
\pgfpathlineto{\pgfqpoint{2.048541in}{1.551532in}}%
\pgfpathlineto{\pgfqpoint{2.050070in}{1.531161in}}%
\pgfpathlineto{\pgfqpoint{2.051600in}{1.559681in}}%
\pgfpathlineto{\pgfqpoint{2.053132in}{1.539309in}}%
\pgfpathlineto{\pgfqpoint{2.053600in}{1.551532in}}%
\pgfpathlineto{\pgfqpoint{2.054367in}{1.543384in}}%
\pgfpathlineto{\pgfqpoint{2.054751in}{1.535235in}}%
\pgfpathlineto{\pgfqpoint{2.055517in}{1.559681in}}%
\pgfpathlineto{\pgfqpoint{2.056666in}{1.531161in}}%
\pgfpathlineto{\pgfqpoint{2.057432in}{1.567830in}}%
\pgfpathlineto{\pgfqpoint{2.057816in}{1.555607in}}%
\pgfpathlineto{\pgfqpoint{2.058199in}{1.531161in}}%
\pgfpathlineto{\pgfqpoint{2.058964in}{1.539309in}}%
\pgfpathlineto{\pgfqpoint{2.059347in}{1.551532in}}%
\pgfpathlineto{\pgfqpoint{2.059730in}{1.543384in}}%
\pgfpathlineto{\pgfqpoint{2.060113in}{1.539309in}}%
\pgfpathlineto{\pgfqpoint{2.061264in}{1.555607in}}%
\pgfpathlineto{\pgfqpoint{2.061647in}{1.547458in}}%
\pgfpathlineto{\pgfqpoint{2.061647in}{1.547458in}}%
\pgfpathlineto{\pgfqpoint{2.061647in}{1.547458in}}%
\pgfpathlineto{\pgfqpoint{2.062030in}{1.559681in}}%
\pgfpathlineto{\pgfqpoint{2.062413in}{1.523012in}}%
\pgfpathlineto{\pgfqpoint{2.063191in}{1.551532in}}%
\pgfpathlineto{\pgfqpoint{2.063574in}{1.547458in}}%
\pgfpathlineto{\pgfqpoint{2.063957in}{1.531161in}}%
\pgfpathlineto{\pgfqpoint{2.064340in}{1.547458in}}%
\pgfpathlineto{\pgfqpoint{2.064723in}{1.547458in}}%
\pgfpathlineto{\pgfqpoint{2.065108in}{1.555607in}}%
\pgfpathlineto{\pgfqpoint{2.065874in}{1.527086in}}%
\pgfpathlineto{\pgfqpoint{2.066257in}{1.535235in}}%
\pgfpathlineto{\pgfqpoint{2.068174in}{1.571904in}}%
\pgfpathlineto{\pgfqpoint{2.068556in}{1.535235in}}%
\pgfpathlineto{\pgfqpoint{2.069322in}{1.547458in}}%
\pgfpathlineto{\pgfqpoint{2.070475in}{1.567830in}}%
\pgfpathlineto{\pgfqpoint{2.072006in}{1.543384in}}%
\pgfpathlineto{\pgfqpoint{2.072773in}{1.571904in}}%
\pgfpathlineto{\pgfqpoint{2.073157in}{1.551532in}}%
\pgfpathlineto{\pgfqpoint{2.073540in}{1.551532in}}%
\pgfpathlineto{\pgfqpoint{2.074689in}{1.543384in}}%
\pgfpathlineto{\pgfqpoint{2.075456in}{1.551532in}}%
\pgfpathlineto{\pgfqpoint{2.075840in}{1.523012in}}%
\pgfpathlineto{\pgfqpoint{2.076606in}{1.543384in}}%
\pgfpathlineto{\pgfqpoint{2.076989in}{1.543384in}}%
\pgfpathlineto{\pgfqpoint{2.077372in}{1.555607in}}%
\pgfpathlineto{\pgfqpoint{2.077755in}{1.543384in}}%
\pgfpathlineto{\pgfqpoint{2.078522in}{1.535235in}}%
\pgfpathlineto{\pgfqpoint{2.080054in}{1.555607in}}%
\pgfpathlineto{\pgfqpoint{2.080821in}{1.547458in}}%
\pgfpathlineto{\pgfqpoint{2.081206in}{1.531161in}}%
\pgfpathlineto{\pgfqpoint{2.081589in}{1.547458in}}%
\pgfpathlineto{\pgfqpoint{2.081972in}{1.551532in}}%
\pgfpathlineto{\pgfqpoint{2.082356in}{1.543384in}}%
\pgfpathlineto{\pgfqpoint{2.083131in}{1.567830in}}%
\pgfpathlineto{\pgfqpoint{2.083515in}{1.551532in}}%
\pgfpathlineto{\pgfqpoint{2.083898in}{1.531161in}}%
\pgfpathlineto{\pgfqpoint{2.084664in}{1.543384in}}%
\pgfpathlineto{\pgfqpoint{2.085047in}{1.543384in}}%
\pgfpathlineto{\pgfqpoint{2.085431in}{1.539309in}}%
\pgfpathlineto{\pgfqpoint{2.085815in}{1.555607in}}%
\pgfpathlineto{\pgfqpoint{2.085815in}{1.555607in}}%
\pgfpathlineto{\pgfqpoint{2.085815in}{1.555607in}}%
\pgfpathlineto{\pgfqpoint{2.086198in}{1.535235in}}%
\pgfpathlineto{\pgfqpoint{2.086963in}{1.539309in}}%
\pgfpathlineto{\pgfqpoint{2.087346in}{1.567830in}}%
\pgfpathlineto{\pgfqpoint{2.088114in}{1.555607in}}%
\pgfpathlineto{\pgfqpoint{2.088497in}{1.559681in}}%
\pgfpathlineto{\pgfqpoint{2.090029in}{1.539309in}}%
\pgfpathlineto{\pgfqpoint{2.090412in}{1.563756in}}%
\pgfpathlineto{\pgfqpoint{2.091179in}{1.543384in}}%
\pgfpathlineto{\pgfqpoint{2.092329in}{1.551532in}}%
\pgfpathlineto{\pgfqpoint{2.092712in}{1.539309in}}%
\pgfpathlineto{\pgfqpoint{2.092712in}{1.539309in}}%
\pgfpathlineto{\pgfqpoint{2.092712in}{1.539309in}}%
\pgfpathlineto{\pgfqpoint{2.093096in}{1.559681in}}%
\pgfpathlineto{\pgfqpoint{2.093096in}{1.559681in}}%
\pgfpathlineto{\pgfqpoint{2.093096in}{1.559681in}}%
\pgfpathlineto{\pgfqpoint{2.093479in}{1.535235in}}%
\pgfpathlineto{\pgfqpoint{2.093479in}{1.535235in}}%
\pgfpathlineto{\pgfqpoint{2.093479in}{1.535235in}}%
\pgfpathlineto{\pgfqpoint{2.093863in}{1.563756in}}%
\pgfpathlineto{\pgfqpoint{2.094628in}{1.551532in}}%
\pgfpathlineto{\pgfqpoint{2.095012in}{1.531161in}}%
\pgfpathlineto{\pgfqpoint{2.095778in}{1.543384in}}%
\pgfpathlineto{\pgfqpoint{2.096161in}{1.555607in}}%
\pgfpathlineto{\pgfqpoint{2.096545in}{1.551532in}}%
\pgfpathlineto{\pgfqpoint{2.096928in}{1.543384in}}%
\pgfpathlineto{\pgfqpoint{2.096928in}{1.543384in}}%
\pgfpathlineto{\pgfqpoint{2.096928in}{1.543384in}}%
\pgfpathlineto{\pgfqpoint{2.097311in}{1.555607in}}%
\pgfpathlineto{\pgfqpoint{2.097694in}{1.543384in}}%
\pgfpathlineto{\pgfqpoint{2.098077in}{1.535235in}}%
\pgfpathlineto{\pgfqpoint{2.098077in}{1.535235in}}%
\pgfpathlineto{\pgfqpoint{2.098077in}{1.535235in}}%
\pgfpathlineto{\pgfqpoint{2.099310in}{1.563756in}}%
\pgfpathlineto{\pgfqpoint{2.099694in}{1.559681in}}%
\pgfpathlineto{\pgfqpoint{2.100843in}{1.547458in}}%
\pgfpathlineto{\pgfqpoint{2.101226in}{1.551532in}}%
\pgfpathlineto{\pgfqpoint{2.101609in}{1.547458in}}%
\pgfpathlineto{\pgfqpoint{2.101992in}{1.543384in}}%
\pgfpathlineto{\pgfqpoint{2.102375in}{1.559681in}}%
\pgfpathlineto{\pgfqpoint{2.102375in}{1.559681in}}%
\pgfpathlineto{\pgfqpoint{2.102375in}{1.559681in}}%
\pgfpathlineto{\pgfqpoint{2.103908in}{1.535235in}}%
\pgfpathlineto{\pgfqpoint{2.104292in}{1.539309in}}%
\pgfpathlineto{\pgfqpoint{2.105825in}{1.559681in}}%
\pgfpathlineto{\pgfqpoint{2.106592in}{1.531161in}}%
\pgfpathlineto{\pgfqpoint{2.106975in}{1.543384in}}%
\pgfpathlineto{\pgfqpoint{2.107357in}{1.551532in}}%
\pgfpathlineto{\pgfqpoint{2.107740in}{1.543384in}}%
\pgfpathlineto{\pgfqpoint{2.108123in}{1.543384in}}%
\pgfpathlineto{\pgfqpoint{2.108890in}{1.559681in}}%
\pgfpathlineto{\pgfqpoint{2.110038in}{1.543384in}}%
\pgfpathlineto{\pgfqpoint{2.110421in}{1.543384in}}%
\pgfpathlineto{\pgfqpoint{2.110805in}{1.563756in}}%
\pgfpathlineto{\pgfqpoint{2.111570in}{1.555607in}}%
\pgfpathlineto{\pgfqpoint{2.113102in}{1.539309in}}%
\pgfpathlineto{\pgfqpoint{2.113868in}{1.559681in}}%
\pgfpathlineto{\pgfqpoint{2.114635in}{1.551532in}}%
\pgfpathlineto{\pgfqpoint{2.115402in}{1.559681in}}%
\pgfpathlineto{\pgfqpoint{2.115784in}{1.535235in}}%
\pgfpathlineto{\pgfqpoint{2.116551in}{1.555607in}}%
\pgfpathlineto{\pgfqpoint{2.116935in}{1.535235in}}%
\pgfpathlineto{\pgfqpoint{2.117702in}{1.543384in}}%
\pgfpathlineto{\pgfqpoint{2.118085in}{1.547458in}}%
\pgfpathlineto{\pgfqpoint{2.119233in}{1.535235in}}%
\pgfpathlineto{\pgfqpoint{2.119617in}{1.551532in}}%
\pgfpathlineto{\pgfqpoint{2.120385in}{1.539309in}}%
\pgfpathlineto{\pgfqpoint{2.121151in}{1.555607in}}%
\pgfpathlineto{\pgfqpoint{2.121534in}{1.543384in}}%
\pgfpathlineto{\pgfqpoint{2.123074in}{1.559681in}}%
\pgfpathlineto{\pgfqpoint{2.123840in}{1.543384in}}%
\pgfpathlineto{\pgfqpoint{2.124223in}{1.551532in}}%
\pgfpathlineto{\pgfqpoint{2.124607in}{1.551532in}}%
\pgfpathlineto{\pgfqpoint{2.125373in}{1.535235in}}%
\pgfpathlineto{\pgfqpoint{2.125756in}{1.555607in}}%
\pgfpathlineto{\pgfqpoint{2.126521in}{1.543384in}}%
\pgfpathlineto{\pgfqpoint{2.126905in}{1.543384in}}%
\pgfpathlineto{\pgfqpoint{2.127288in}{1.547458in}}%
\pgfpathlineto{\pgfqpoint{2.127672in}{1.535235in}}%
\pgfpathlineto{\pgfqpoint{2.127672in}{1.535235in}}%
\pgfpathlineto{\pgfqpoint{2.127672in}{1.535235in}}%
\pgfpathlineto{\pgfqpoint{2.128438in}{1.567830in}}%
\pgfpathlineto{\pgfqpoint{2.128821in}{1.563756in}}%
\pgfpathlineto{\pgfqpoint{2.129204in}{1.535235in}}%
\pgfpathlineto{\pgfqpoint{2.129970in}{1.547458in}}%
\pgfpathlineto{\pgfqpoint{2.130353in}{1.559681in}}%
\pgfpathlineto{\pgfqpoint{2.130353in}{1.559681in}}%
\pgfpathlineto{\pgfqpoint{2.130353in}{1.559681in}}%
\pgfpathlineto{\pgfqpoint{2.131503in}{1.535235in}}%
\pgfpathlineto{\pgfqpoint{2.131886in}{1.551532in}}%
\pgfpathlineto{\pgfqpoint{2.132652in}{1.543384in}}%
\pgfpathlineto{\pgfqpoint{2.133036in}{1.543384in}}%
\pgfpathlineto{\pgfqpoint{2.133419in}{1.551532in}}%
\pgfpathlineto{\pgfqpoint{2.133803in}{1.527086in}}%
\pgfpathlineto{\pgfqpoint{2.133803in}{1.527086in}}%
\pgfpathlineto{\pgfqpoint{2.133803in}{1.527086in}}%
\pgfpathlineto{\pgfqpoint{2.134271in}{1.559681in}}%
\pgfpathlineto{\pgfqpoint{2.135037in}{1.543384in}}%
\pgfpathlineto{\pgfqpoint{2.135420in}{1.543384in}}%
\pgfpathlineto{\pgfqpoint{2.135803in}{1.531161in}}%
\pgfpathlineto{\pgfqpoint{2.135803in}{1.531161in}}%
\pgfpathlineto{\pgfqpoint{2.135803in}{1.531161in}}%
\pgfpathlineto{\pgfqpoint{2.136187in}{1.551532in}}%
\pgfpathlineto{\pgfqpoint{2.136954in}{1.543384in}}%
\pgfpathlineto{\pgfqpoint{2.137719in}{1.547458in}}%
\pgfpathlineto{\pgfqpoint{2.138102in}{1.543384in}}%
\pgfpathlineto{\pgfqpoint{2.138485in}{1.547458in}}%
\pgfpathlineto{\pgfqpoint{2.139635in}{1.563756in}}%
\pgfpathlineto{\pgfqpoint{2.140018in}{1.559681in}}%
\pgfpathlineto{\pgfqpoint{2.140401in}{1.543384in}}%
\pgfpathlineto{\pgfqpoint{2.141167in}{1.555607in}}%
\pgfpathlineto{\pgfqpoint{2.141550in}{1.555607in}}%
\pgfpathlineto{\pgfqpoint{2.143083in}{1.535235in}}%
\pgfpathlineto{\pgfqpoint{2.144998in}{1.555607in}}%
\pgfpathlineto{\pgfqpoint{2.146148in}{1.535235in}}%
\pgfpathlineto{\pgfqpoint{2.146530in}{1.543384in}}%
\pgfpathlineto{\pgfqpoint{2.146913in}{1.547458in}}%
\pgfpathlineto{\pgfqpoint{2.147297in}{1.535235in}}%
\pgfpathlineto{\pgfqpoint{2.147297in}{1.535235in}}%
\pgfpathlineto{\pgfqpoint{2.147297in}{1.535235in}}%
\pgfpathlineto{\pgfqpoint{2.147681in}{1.555607in}}%
\pgfpathlineto{\pgfqpoint{2.148447in}{1.551532in}}%
\pgfpathlineto{\pgfqpoint{2.148831in}{1.539309in}}%
\pgfpathlineto{\pgfqpoint{2.149214in}{1.547458in}}%
\pgfpathlineto{\pgfqpoint{2.149597in}{1.551532in}}%
\pgfpathlineto{\pgfqpoint{2.149980in}{1.543384in}}%
\pgfpathlineto{\pgfqpoint{2.150364in}{1.551532in}}%
\pgfpathlineto{\pgfqpoint{2.151131in}{1.563756in}}%
\pgfpathlineto{\pgfqpoint{2.151514in}{1.555607in}}%
\pgfpathlineto{\pgfqpoint{2.152663in}{1.531161in}}%
\pgfpathlineto{\pgfqpoint{2.153813in}{1.547458in}}%
\pgfpathlineto{\pgfqpoint{2.154196in}{1.543384in}}%
\pgfpathlineto{\pgfqpoint{2.154579in}{1.559681in}}%
\pgfpathlineto{\pgfqpoint{2.155345in}{1.547458in}}%
\pgfpathlineto{\pgfqpoint{2.156110in}{1.555607in}}%
\pgfpathlineto{\pgfqpoint{2.157643in}{1.535235in}}%
\pgfpathlineto{\pgfqpoint{2.158792in}{1.543384in}}%
\pgfpathlineto{\pgfqpoint{2.159176in}{1.531161in}}%
\pgfpathlineto{\pgfqpoint{2.159176in}{1.531161in}}%
\pgfpathlineto{\pgfqpoint{2.159176in}{1.531161in}}%
\pgfpathlineto{\pgfqpoint{2.160707in}{1.559681in}}%
\pgfpathlineto{\pgfqpoint{2.161091in}{1.531161in}}%
\pgfpathlineto{\pgfqpoint{2.161857in}{1.543384in}}%
\pgfpathlineto{\pgfqpoint{2.162241in}{1.543384in}}%
\pgfpathlineto{\pgfqpoint{2.162630in}{1.571904in}}%
\pgfpathlineto{\pgfqpoint{2.163397in}{1.547458in}}%
\pgfpathlineto{\pgfqpoint{2.163781in}{1.523012in}}%
\pgfpathlineto{\pgfqpoint{2.163781in}{1.523012in}}%
\pgfpathlineto{\pgfqpoint{2.163781in}{1.523012in}}%
\pgfpathlineto{\pgfqpoint{2.164547in}{1.555607in}}%
\pgfpathlineto{\pgfqpoint{2.164930in}{1.523012in}}%
\pgfpathlineto{\pgfqpoint{2.165696in}{1.543384in}}%
\pgfpathlineto{\pgfqpoint{2.166080in}{1.555607in}}%
\pgfpathlineto{\pgfqpoint{2.166463in}{1.531161in}}%
\pgfpathlineto{\pgfqpoint{2.167229in}{1.535235in}}%
\pgfpathlineto{\pgfqpoint{2.167613in}{1.539309in}}%
\pgfpathlineto{\pgfqpoint{2.167995in}{1.531161in}}%
\pgfpathlineto{\pgfqpoint{2.168379in}{1.539309in}}%
\pgfpathlineto{\pgfqpoint{2.169145in}{1.559681in}}%
\pgfpathlineto{\pgfqpoint{2.169529in}{1.543384in}}%
\pgfpathlineto{\pgfqpoint{2.169913in}{1.531161in}}%
\pgfpathlineto{\pgfqpoint{2.169913in}{1.531161in}}%
\pgfpathlineto{\pgfqpoint{2.169913in}{1.531161in}}%
\pgfpathlineto{\pgfqpoint{2.171146in}{1.551532in}}%
\pgfpathlineto{\pgfqpoint{2.171529in}{1.551532in}}%
\pgfpathlineto{\pgfqpoint{2.171912in}{1.531161in}}%
\pgfpathlineto{\pgfqpoint{2.171912in}{1.531161in}}%
\pgfpathlineto{\pgfqpoint{2.171912in}{1.531161in}}%
\pgfpathlineto{\pgfqpoint{2.172679in}{1.555607in}}%
\pgfpathlineto{\pgfqpoint{2.173062in}{1.539309in}}%
\pgfpathlineto{\pgfqpoint{2.174210in}{1.559681in}}%
\pgfpathlineto{\pgfqpoint{2.174593in}{1.543384in}}%
\pgfpathlineto{\pgfqpoint{2.174593in}{1.543384in}}%
\pgfpathlineto{\pgfqpoint{2.174593in}{1.543384in}}%
\pgfpathlineto{\pgfqpoint{2.174976in}{1.563756in}}%
\pgfpathlineto{\pgfqpoint{2.174976in}{1.563756in}}%
\pgfpathlineto{\pgfqpoint{2.174976in}{1.563756in}}%
\pgfpathlineto{\pgfqpoint{2.175357in}{1.535235in}}%
\pgfpathlineto{\pgfqpoint{2.176124in}{1.547458in}}%
\pgfpathlineto{\pgfqpoint{2.176507in}{1.559681in}}%
\pgfpathlineto{\pgfqpoint{2.176507in}{1.559681in}}%
\pgfpathlineto{\pgfqpoint{2.176507in}{1.559681in}}%
\pgfpathlineto{\pgfqpoint{2.177658in}{1.543384in}}%
\pgfpathlineto{\pgfqpoint{2.178425in}{1.551532in}}%
\pgfpathlineto{\pgfqpoint{2.178808in}{1.567830in}}%
\pgfpathlineto{\pgfqpoint{2.179192in}{1.527086in}}%
\pgfpathlineto{\pgfqpoint{2.179958in}{1.555607in}}%
\pgfpathlineto{\pgfqpoint{2.180341in}{1.551532in}}%
\pgfpathlineto{\pgfqpoint{2.180724in}{1.531161in}}%
\pgfpathlineto{\pgfqpoint{2.181108in}{1.547458in}}%
\pgfpathlineto{\pgfqpoint{2.181493in}{1.567830in}}%
\pgfpathlineto{\pgfqpoint{2.181876in}{1.547458in}}%
\pgfpathlineto{\pgfqpoint{2.182261in}{1.547458in}}%
\pgfpathlineto{\pgfqpoint{2.182643in}{1.551532in}}%
\pgfpathlineto{\pgfqpoint{2.183793in}{1.539309in}}%
\pgfpathlineto{\pgfqpoint{2.184558in}{1.555607in}}%
\pgfpathlineto{\pgfqpoint{2.184942in}{1.543384in}}%
\pgfpathlineto{\pgfqpoint{2.185325in}{1.531161in}}%
\pgfpathlineto{\pgfqpoint{2.185325in}{1.531161in}}%
\pgfpathlineto{\pgfqpoint{2.185325in}{1.531161in}}%
\pgfpathlineto{\pgfqpoint{2.186475in}{1.563756in}}%
\pgfpathlineto{\pgfqpoint{2.187625in}{1.531161in}}%
\pgfpathlineto{\pgfqpoint{2.188774in}{1.567830in}}%
\pgfpathlineto{\pgfqpoint{2.190309in}{1.531161in}}%
\pgfpathlineto{\pgfqpoint{2.190692in}{1.571904in}}%
\pgfpathlineto{\pgfqpoint{2.191458in}{1.559681in}}%
\pgfpathlineto{\pgfqpoint{2.192991in}{1.527086in}}%
\pgfpathlineto{\pgfqpoint{2.193757in}{1.555607in}}%
\pgfpathlineto{\pgfqpoint{2.194140in}{1.531161in}}%
\pgfpathlineto{\pgfqpoint{2.195291in}{1.559681in}}%
\pgfpathlineto{\pgfqpoint{2.195674in}{1.555607in}}%
\pgfpathlineto{\pgfqpoint{2.196823in}{1.531161in}}%
\pgfpathlineto{\pgfqpoint{2.197974in}{1.547458in}}%
\pgfpathlineto{\pgfqpoint{2.198358in}{1.559681in}}%
\pgfpathlineto{\pgfqpoint{2.198741in}{1.551532in}}%
\pgfpathlineto{\pgfqpoint{2.199124in}{1.535235in}}%
\pgfpathlineto{\pgfqpoint{2.199891in}{1.543384in}}%
\pgfpathlineto{\pgfqpoint{2.200657in}{1.543384in}}%
\pgfpathlineto{\pgfqpoint{2.201041in}{1.555607in}}%
\pgfpathlineto{\pgfqpoint{2.201424in}{1.543384in}}%
\pgfpathlineto{\pgfqpoint{2.201808in}{1.531161in}}%
\pgfpathlineto{\pgfqpoint{2.201808in}{1.531161in}}%
\pgfpathlineto{\pgfqpoint{2.201808in}{1.531161in}}%
\pgfpathlineto{\pgfqpoint{2.202580in}{1.563756in}}%
\pgfpathlineto{\pgfqpoint{2.202963in}{1.535235in}}%
\pgfpathlineto{\pgfqpoint{2.204878in}{1.559681in}}%
\pgfpathlineto{\pgfqpoint{2.205261in}{1.555607in}}%
\pgfpathlineto{\pgfqpoint{2.206026in}{1.535235in}}%
\pgfpathlineto{\pgfqpoint{2.206410in}{1.551532in}}%
\pgfpathlineto{\pgfqpoint{2.206792in}{1.555607in}}%
\pgfpathlineto{\pgfqpoint{2.207175in}{1.531161in}}%
\pgfpathlineto{\pgfqpoint{2.207559in}{1.551532in}}%
\pgfpathlineto{\pgfqpoint{2.207941in}{1.559681in}}%
\pgfpathlineto{\pgfqpoint{2.207941in}{1.559681in}}%
\pgfpathlineto{\pgfqpoint{2.207941in}{1.559681in}}%
\pgfpathlineto{\pgfqpoint{2.208795in}{1.518937in}}%
\pgfpathlineto{\pgfqpoint{2.209177in}{1.575979in}}%
\pgfpathlineto{\pgfqpoint{2.209943in}{1.531161in}}%
\pgfpathlineto{\pgfqpoint{2.211859in}{1.567830in}}%
\pgfpathlineto{\pgfqpoint{2.212625in}{1.527086in}}%
\pgfpathlineto{\pgfqpoint{2.213008in}{1.555607in}}%
\pgfpathlineto{\pgfqpoint{2.214159in}{1.523012in}}%
\pgfpathlineto{\pgfqpoint{2.214925in}{1.563756in}}%
\pgfpathlineto{\pgfqpoint{2.215308in}{1.527086in}}%
\pgfpathlineto{\pgfqpoint{2.216842in}{1.547458in}}%
\pgfpathlineto{\pgfqpoint{2.217226in}{1.551532in}}%
\pgfpathlineto{\pgfqpoint{2.217609in}{1.547458in}}%
\pgfpathlineto{\pgfqpoint{2.217991in}{1.535235in}}%
\pgfpathlineto{\pgfqpoint{2.218375in}{1.539309in}}%
\pgfpathlineto{\pgfqpoint{2.218757in}{1.547458in}}%
\pgfpathlineto{\pgfqpoint{2.218757in}{1.547458in}}%
\pgfpathlineto{\pgfqpoint{2.218757in}{1.547458in}}%
\pgfpathlineto{\pgfqpoint{2.219140in}{1.535235in}}%
\pgfpathlineto{\pgfqpoint{2.219523in}{1.543384in}}%
\pgfpathlineto{\pgfqpoint{2.219908in}{1.547458in}}%
\pgfpathlineto{\pgfqpoint{2.220290in}{1.543384in}}%
\pgfpathlineto{\pgfqpoint{2.220673in}{1.567830in}}%
\pgfpathlineto{\pgfqpoint{2.221438in}{1.555607in}}%
\pgfpathlineto{\pgfqpoint{2.222588in}{1.531161in}}%
\pgfpathlineto{\pgfqpoint{2.222971in}{1.555607in}}%
\pgfpathlineto{\pgfqpoint{2.223737in}{1.539309in}}%
\pgfpathlineto{\pgfqpoint{2.224120in}{1.543384in}}%
\pgfpathlineto{\pgfqpoint{2.224504in}{1.559681in}}%
\pgfpathlineto{\pgfqpoint{2.224504in}{1.559681in}}%
\pgfpathlineto{\pgfqpoint{2.224504in}{1.559681in}}%
\pgfpathlineto{\pgfqpoint{2.224887in}{1.535235in}}%
\pgfpathlineto{\pgfqpoint{2.224887in}{1.535235in}}%
\pgfpathlineto{\pgfqpoint{2.224887in}{1.535235in}}%
\pgfpathlineto{\pgfqpoint{2.225270in}{1.567830in}}%
\pgfpathlineto{\pgfqpoint{2.226037in}{1.547458in}}%
\pgfpathlineto{\pgfqpoint{2.227186in}{1.539309in}}%
\pgfpathlineto{\pgfqpoint{2.227569in}{1.543384in}}%
\pgfpathlineto{\pgfqpoint{2.227952in}{1.563756in}}%
\pgfpathlineto{\pgfqpoint{2.227952in}{1.563756in}}%
\pgfpathlineto{\pgfqpoint{2.227952in}{1.563756in}}%
\pgfpathlineto{\pgfqpoint{2.228335in}{1.535235in}}%
\pgfpathlineto{\pgfqpoint{2.228718in}{1.555607in}}%
\pgfpathlineto{\pgfqpoint{2.229102in}{1.571904in}}%
\pgfpathlineto{\pgfqpoint{2.229102in}{1.571904in}}%
\pgfpathlineto{\pgfqpoint{2.229102in}{1.571904in}}%
\pgfpathlineto{\pgfqpoint{2.230633in}{1.539309in}}%
\pgfpathlineto{\pgfqpoint{2.231401in}{1.547458in}}%
\pgfpathlineto{\pgfqpoint{2.231785in}{1.535235in}}%
\pgfpathlineto{\pgfqpoint{2.231785in}{1.535235in}}%
\pgfpathlineto{\pgfqpoint{2.231785in}{1.535235in}}%
\pgfpathlineto{\pgfqpoint{2.232551in}{1.555607in}}%
\pgfpathlineto{\pgfqpoint{2.232934in}{1.543384in}}%
\pgfpathlineto{\pgfqpoint{2.233317in}{1.531161in}}%
\pgfpathlineto{\pgfqpoint{2.233317in}{1.531161in}}%
\pgfpathlineto{\pgfqpoint{2.233317in}{1.531161in}}%
\pgfpathlineto{\pgfqpoint{2.233701in}{1.547458in}}%
\pgfpathlineto{\pgfqpoint{2.234467in}{1.539309in}}%
\pgfpathlineto{\pgfqpoint{2.235999in}{1.559681in}}%
\pgfpathlineto{\pgfqpoint{2.237532in}{1.535235in}}%
\pgfpathlineto{\pgfqpoint{2.238298in}{1.555607in}}%
\pgfpathlineto{\pgfqpoint{2.238681in}{1.539309in}}%
\pgfpathlineto{\pgfqpoint{2.239065in}{1.543384in}}%
\pgfpathlineto{\pgfqpoint{2.239449in}{1.539309in}}%
\pgfpathlineto{\pgfqpoint{2.240215in}{1.527086in}}%
\pgfpathlineto{\pgfqpoint{2.241364in}{1.543384in}}%
\pgfpathlineto{\pgfqpoint{2.241746in}{1.531161in}}%
\pgfpathlineto{\pgfqpoint{2.241746in}{1.531161in}}%
\pgfpathlineto{\pgfqpoint{2.241746in}{1.531161in}}%
\pgfpathlineto{\pgfqpoint{2.242520in}{1.551532in}}%
\pgfpathlineto{\pgfqpoint{2.242904in}{1.547458in}}%
\pgfpathlineto{\pgfqpoint{2.243287in}{1.531161in}}%
\pgfpathlineto{\pgfqpoint{2.243670in}{1.543384in}}%
\pgfpathlineto{\pgfqpoint{2.244437in}{1.551532in}}%
\pgfpathlineto{\pgfqpoint{2.244819in}{1.531161in}}%
\pgfpathlineto{\pgfqpoint{2.244819in}{1.531161in}}%
\pgfpathlineto{\pgfqpoint{2.244819in}{1.531161in}}%
\pgfpathlineto{\pgfqpoint{2.246055in}{1.559681in}}%
\pgfpathlineto{\pgfqpoint{2.247588in}{1.531161in}}%
\pgfpathlineto{\pgfqpoint{2.248735in}{1.555607in}}%
\pgfpathlineto{\pgfqpoint{2.249500in}{1.539309in}}%
\pgfpathlineto{\pgfqpoint{2.249885in}{1.551532in}}%
\pgfpathlineto{\pgfqpoint{2.250269in}{1.559681in}}%
\pgfpathlineto{\pgfqpoint{2.250653in}{1.551532in}}%
\pgfpathlineto{\pgfqpoint{2.251802in}{1.531161in}}%
\pgfpathlineto{\pgfqpoint{2.253336in}{1.555607in}}%
\pgfpathlineto{\pgfqpoint{2.254869in}{1.539309in}}%
\pgfpathlineto{\pgfqpoint{2.255252in}{1.555607in}}%
\pgfpathlineto{\pgfqpoint{2.255636in}{1.543384in}}%
\pgfpathlineto{\pgfqpoint{2.256019in}{1.539309in}}%
\pgfpathlineto{\pgfqpoint{2.256787in}{1.555607in}}%
\pgfpathlineto{\pgfqpoint{2.257170in}{1.547458in}}%
\pgfpathlineto{\pgfqpoint{2.257553in}{1.539309in}}%
\pgfpathlineto{\pgfqpoint{2.257936in}{1.543384in}}%
\pgfpathlineto{\pgfqpoint{2.259086in}{1.563756in}}%
\pgfpathlineto{\pgfqpoint{2.260234in}{1.527086in}}%
\pgfpathlineto{\pgfqpoint{2.260617in}{1.543384in}}%
\pgfpathlineto{\pgfqpoint{2.261001in}{1.539309in}}%
\pgfpathlineto{\pgfqpoint{2.261385in}{1.543384in}}%
\pgfpathlineto{\pgfqpoint{2.261769in}{1.551532in}}%
\pgfpathlineto{\pgfqpoint{2.262152in}{1.531161in}}%
\pgfpathlineto{\pgfqpoint{2.262152in}{1.531161in}}%
\pgfpathlineto{\pgfqpoint{2.262152in}{1.531161in}}%
\pgfpathlineto{\pgfqpoint{2.263684in}{1.567830in}}%
\pgfpathlineto{\pgfqpoint{2.264067in}{1.514863in}}%
\pgfpathlineto{\pgfqpoint{2.264834in}{1.555607in}}%
\pgfpathlineto{\pgfqpoint{2.265218in}{1.531161in}}%
\pgfpathlineto{\pgfqpoint{2.265984in}{1.551532in}}%
\pgfpathlineto{\pgfqpoint{2.266367in}{1.551532in}}%
\pgfpathlineto{\pgfqpoint{2.266751in}{1.539309in}}%
\pgfpathlineto{\pgfqpoint{2.266751in}{1.539309in}}%
\pgfpathlineto{\pgfqpoint{2.266751in}{1.539309in}}%
\pgfpathlineto{\pgfqpoint{2.268284in}{1.571904in}}%
\pgfpathlineto{\pgfqpoint{2.269818in}{1.547458in}}%
\pgfpathlineto{\pgfqpoint{2.270201in}{1.551532in}}%
\pgfpathlineto{\pgfqpoint{2.270585in}{1.527086in}}%
\pgfpathlineto{\pgfqpoint{2.271351in}{1.547458in}}%
\pgfpathlineto{\pgfqpoint{2.271735in}{1.551532in}}%
\pgfpathlineto{\pgfqpoint{2.272119in}{1.535235in}}%
\pgfpathlineto{\pgfqpoint{2.272502in}{1.547458in}}%
\pgfpathlineto{\pgfqpoint{2.272885in}{1.555607in}}%
\pgfpathlineto{\pgfqpoint{2.272885in}{1.555607in}}%
\pgfpathlineto{\pgfqpoint{2.272885in}{1.555607in}}%
\pgfpathlineto{\pgfqpoint{2.273651in}{1.539309in}}%
\pgfpathlineto{\pgfqpoint{2.274033in}{1.559681in}}%
\pgfpathlineto{\pgfqpoint{2.274417in}{1.543384in}}%
\pgfpathlineto{\pgfqpoint{2.275185in}{1.539309in}}%
\pgfpathlineto{\pgfqpoint{2.275951in}{1.559681in}}%
\pgfpathlineto{\pgfqpoint{2.276333in}{1.555607in}}%
\pgfpathlineto{\pgfqpoint{2.276716in}{1.531161in}}%
\pgfpathlineto{\pgfqpoint{2.277099in}{1.535235in}}%
\pgfpathlineto{\pgfqpoint{2.277482in}{1.555607in}}%
\pgfpathlineto{\pgfqpoint{2.277866in}{1.539309in}}%
\pgfpathlineto{\pgfqpoint{2.278248in}{1.531161in}}%
\pgfpathlineto{\pgfqpoint{2.278248in}{1.531161in}}%
\pgfpathlineto{\pgfqpoint{2.278248in}{1.531161in}}%
\pgfpathlineto{\pgfqpoint{2.279014in}{1.567830in}}%
\pgfpathlineto{\pgfqpoint{2.279397in}{1.547458in}}%
\pgfpathlineto{\pgfqpoint{2.279780in}{1.531161in}}%
\pgfpathlineto{\pgfqpoint{2.280163in}{1.539309in}}%
\pgfpathlineto{\pgfqpoint{2.280546in}{1.559681in}}%
\pgfpathlineto{\pgfqpoint{2.281312in}{1.555607in}}%
\pgfpathlineto{\pgfqpoint{2.282164in}{1.539309in}}%
\pgfpathlineto{\pgfqpoint{2.282937in}{1.555607in}}%
\pgfpathlineto{\pgfqpoint{2.283320in}{1.551532in}}%
\pgfpathlineto{\pgfqpoint{2.284470in}{1.531161in}}%
\pgfpathlineto{\pgfqpoint{2.286002in}{1.555607in}}%
\pgfpathlineto{\pgfqpoint{2.286386in}{1.523012in}}%
\pgfpathlineto{\pgfqpoint{2.287152in}{1.539309in}}%
\pgfpathlineto{\pgfqpoint{2.287917in}{1.547458in}}%
\pgfpathlineto{\pgfqpoint{2.288299in}{1.563756in}}%
\pgfpathlineto{\pgfqpoint{2.288682in}{1.547458in}}%
\pgfpathlineto{\pgfqpoint{2.289066in}{1.535235in}}%
\pgfpathlineto{\pgfqpoint{2.289449in}{1.547458in}}%
\pgfpathlineto{\pgfqpoint{2.289832in}{1.567830in}}%
\pgfpathlineto{\pgfqpoint{2.289832in}{1.567830in}}%
\pgfpathlineto{\pgfqpoint{2.289832in}{1.567830in}}%
\pgfpathlineto{\pgfqpoint{2.291364in}{1.539309in}}%
\pgfpathlineto{\pgfqpoint{2.292514in}{1.551532in}}%
\pgfpathlineto{\pgfqpoint{2.292897in}{1.547458in}}%
\pgfpathlineto{\pgfqpoint{2.293281in}{1.551532in}}%
\pgfpathlineto{\pgfqpoint{2.294429in}{1.567830in}}%
\pgfpathlineto{\pgfqpoint{2.295196in}{1.555607in}}%
\pgfpathlineto{\pgfqpoint{2.295579in}{1.539309in}}%
\pgfpathlineto{\pgfqpoint{2.296346in}{1.551532in}}%
\pgfpathlineto{\pgfqpoint{2.296729in}{1.547458in}}%
\pgfpathlineto{\pgfqpoint{2.297112in}{1.551532in}}%
\pgfpathlineto{\pgfqpoint{2.297879in}{1.551532in}}%
\pgfpathlineto{\pgfqpoint{2.298645in}{1.535235in}}%
\pgfpathlineto{\pgfqpoint{2.300179in}{1.563756in}}%
\pgfpathlineto{\pgfqpoint{2.300945in}{1.543384in}}%
\pgfpathlineto{\pgfqpoint{2.301328in}{1.571904in}}%
\pgfpathlineto{\pgfqpoint{2.301710in}{1.543384in}}%
\pgfpathlineto{\pgfqpoint{2.302094in}{1.527086in}}%
\pgfpathlineto{\pgfqpoint{2.302094in}{1.527086in}}%
\pgfpathlineto{\pgfqpoint{2.302094in}{1.527086in}}%
\pgfpathlineto{\pgfqpoint{2.302478in}{1.547458in}}%
\pgfpathlineto{\pgfqpoint{2.303245in}{1.543384in}}%
\pgfpathlineto{\pgfqpoint{2.303628in}{1.543384in}}%
\pgfpathlineto{\pgfqpoint{2.304010in}{1.539309in}}%
\pgfpathlineto{\pgfqpoint{2.304393in}{1.547458in}}%
\pgfpathlineto{\pgfqpoint{2.304776in}{1.539309in}}%
\pgfpathlineto{\pgfqpoint{2.305159in}{1.539309in}}%
\pgfpathlineto{\pgfqpoint{2.305541in}{1.535235in}}%
\pgfpathlineto{\pgfqpoint{2.306691in}{1.563756in}}%
\pgfpathlineto{\pgfqpoint{2.307074in}{1.555607in}}%
\pgfpathlineto{\pgfqpoint{2.307839in}{1.551532in}}%
\pgfpathlineto{\pgfqpoint{2.308605in}{1.567830in}}%
\pgfpathlineto{\pgfqpoint{2.310138in}{1.543384in}}%
\pgfpathlineto{\pgfqpoint{2.310520in}{1.547458in}}%
\pgfpathlineto{\pgfqpoint{2.310903in}{1.543384in}}%
\pgfpathlineto{\pgfqpoint{2.311286in}{1.543384in}}%
\pgfpathlineto{\pgfqpoint{2.311669in}{1.535235in}}%
\pgfpathlineto{\pgfqpoint{2.312052in}{1.559681in}}%
\pgfpathlineto{\pgfqpoint{2.312818in}{1.555607in}}%
\pgfpathlineto{\pgfqpoint{2.313969in}{1.539309in}}%
\pgfpathlineto{\pgfqpoint{2.314735in}{1.555607in}}%
\pgfpathlineto{\pgfqpoint{2.315118in}{1.539309in}}%
\pgfpathlineto{\pgfqpoint{2.315502in}{1.551532in}}%
\pgfpathlineto{\pgfqpoint{2.315885in}{1.563756in}}%
\pgfpathlineto{\pgfqpoint{2.316269in}{1.551532in}}%
\pgfpathlineto{\pgfqpoint{2.317504in}{1.543384in}}%
\pgfpathlineto{\pgfqpoint{2.317888in}{1.551532in}}%
\pgfpathlineto{\pgfqpoint{2.318271in}{1.531161in}}%
\pgfpathlineto{\pgfqpoint{2.318271in}{1.531161in}}%
\pgfpathlineto{\pgfqpoint{2.318271in}{1.531161in}}%
\pgfpathlineto{\pgfqpoint{2.319420in}{1.559681in}}%
\pgfpathlineto{\pgfqpoint{2.320185in}{1.543384in}}%
\pgfpathlineto{\pgfqpoint{2.320569in}{1.551532in}}%
\pgfpathlineto{\pgfqpoint{2.321336in}{1.559681in}}%
\pgfpathlineto{\pgfqpoint{2.322102in}{1.539309in}}%
\pgfpathlineto{\pgfqpoint{2.322878in}{1.543384in}}%
\pgfpathlineto{\pgfqpoint{2.323262in}{1.563756in}}%
\pgfpathlineto{\pgfqpoint{2.323262in}{1.563756in}}%
\pgfpathlineto{\pgfqpoint{2.323262in}{1.563756in}}%
\pgfpathlineto{\pgfqpoint{2.324027in}{1.539309in}}%
\pgfpathlineto{\pgfqpoint{2.324410in}{1.559681in}}%
\pgfpathlineto{\pgfqpoint{2.324794in}{1.559681in}}%
\pgfpathlineto{\pgfqpoint{2.325561in}{1.543384in}}%
\pgfpathlineto{\pgfqpoint{2.325944in}{1.551532in}}%
\pgfpathlineto{\pgfqpoint{2.326327in}{1.551532in}}%
\pgfpathlineto{\pgfqpoint{2.326709in}{1.535235in}}%
\pgfpathlineto{\pgfqpoint{2.327093in}{1.567830in}}%
\pgfpathlineto{\pgfqpoint{2.327859in}{1.559681in}}%
\pgfpathlineto{\pgfqpoint{2.329009in}{1.543384in}}%
\pgfpathlineto{\pgfqpoint{2.329775in}{1.563756in}}%
\pgfpathlineto{\pgfqpoint{2.330157in}{1.531161in}}%
\pgfpathlineto{\pgfqpoint{2.330924in}{1.555607in}}%
\pgfpathlineto{\pgfqpoint{2.331692in}{1.539309in}}%
\pgfpathlineto{\pgfqpoint{2.332458in}{1.555607in}}%
\pgfpathlineto{\pgfqpoint{2.332841in}{1.531161in}}%
\pgfpathlineto{\pgfqpoint{2.333608in}{1.543384in}}%
\pgfpathlineto{\pgfqpoint{2.333991in}{1.551532in}}%
\pgfpathlineto{\pgfqpoint{2.334374in}{1.543384in}}%
\pgfpathlineto{\pgfqpoint{2.334757in}{1.527086in}}%
\pgfpathlineto{\pgfqpoint{2.334757in}{1.527086in}}%
\pgfpathlineto{\pgfqpoint{2.334757in}{1.527086in}}%
\pgfpathlineto{\pgfqpoint{2.335523in}{1.559681in}}%
\pgfpathlineto{\pgfqpoint{2.336290in}{1.551532in}}%
\pgfpathlineto{\pgfqpoint{2.337057in}{1.547458in}}%
\pgfpathlineto{\pgfqpoint{2.338206in}{1.551532in}}%
\pgfpathlineto{\pgfqpoint{2.338588in}{1.547458in}}%
\pgfpathlineto{\pgfqpoint{2.338971in}{1.510789in}}%
\pgfpathlineto{\pgfqpoint{2.338971in}{1.510789in}}%
\pgfpathlineto{\pgfqpoint{2.338971in}{1.510789in}}%
\pgfpathlineto{\pgfqpoint{2.339355in}{1.567830in}}%
\pgfpathlineto{\pgfqpoint{2.340122in}{1.547458in}}%
\pgfpathlineto{\pgfqpoint{2.340506in}{1.543384in}}%
\pgfpathlineto{\pgfqpoint{2.340889in}{1.551532in}}%
\pgfpathlineto{\pgfqpoint{2.341272in}{1.531161in}}%
\pgfpathlineto{\pgfqpoint{2.341653in}{1.543384in}}%
\pgfpathlineto{\pgfqpoint{2.342421in}{1.555607in}}%
\pgfpathlineto{\pgfqpoint{2.343571in}{1.539309in}}%
\pgfpathlineto{\pgfqpoint{2.343954in}{1.551532in}}%
\pgfpathlineto{\pgfqpoint{2.344337in}{1.547458in}}%
\pgfpathlineto{\pgfqpoint{2.345489in}{1.539309in}}%
\pgfpathlineto{\pgfqpoint{2.345872in}{1.539309in}}%
\pgfpathlineto{\pgfqpoint{2.346255in}{1.543384in}}%
\pgfpathlineto{\pgfqpoint{2.346638in}{1.539309in}}%
\pgfpathlineto{\pgfqpoint{2.347021in}{1.535235in}}%
\pgfpathlineto{\pgfqpoint{2.348173in}{1.555607in}}%
\pgfpathlineto{\pgfqpoint{2.349321in}{1.543384in}}%
\pgfpathlineto{\pgfqpoint{2.349705in}{1.527086in}}%
\pgfpathlineto{\pgfqpoint{2.349705in}{1.527086in}}%
\pgfpathlineto{\pgfqpoint{2.349705in}{1.527086in}}%
\pgfpathlineto{\pgfqpoint{2.351239in}{1.563756in}}%
\pgfpathlineto{\pgfqpoint{2.352006in}{1.547458in}}%
\pgfpathlineto{\pgfqpoint{2.352390in}{1.523012in}}%
\pgfpathlineto{\pgfqpoint{2.352390in}{1.523012in}}%
\pgfpathlineto{\pgfqpoint{2.352390in}{1.523012in}}%
\pgfpathlineto{\pgfqpoint{2.352773in}{1.551532in}}%
\pgfpathlineto{\pgfqpoint{2.353540in}{1.535235in}}%
\pgfpathlineto{\pgfqpoint{2.355075in}{1.559681in}}%
\pgfpathlineto{\pgfqpoint{2.356225in}{1.539309in}}%
\pgfpathlineto{\pgfqpoint{2.356608in}{1.547458in}}%
\pgfpathlineto{\pgfqpoint{2.356991in}{1.559681in}}%
\pgfpathlineto{\pgfqpoint{2.357375in}{1.551532in}}%
\pgfpathlineto{\pgfqpoint{2.358142in}{1.531161in}}%
\pgfpathlineto{\pgfqpoint{2.358525in}{1.555607in}}%
\pgfpathlineto{\pgfqpoint{2.359377in}{1.551532in}}%
\pgfpathlineto{\pgfqpoint{2.359760in}{1.535235in}}%
\pgfpathlineto{\pgfqpoint{2.360526in}{1.543384in}}%
\pgfpathlineto{\pgfqpoint{2.360909in}{1.551532in}}%
\pgfpathlineto{\pgfqpoint{2.361293in}{1.543384in}}%
\pgfpathlineto{\pgfqpoint{2.362833in}{1.535235in}}%
\pgfpathlineto{\pgfqpoint{2.364367in}{1.563756in}}%
\pgfpathlineto{\pgfqpoint{2.364750in}{1.514863in}}%
\pgfpathlineto{\pgfqpoint{2.365515in}{1.539309in}}%
\pgfpathlineto{\pgfqpoint{2.365900in}{1.531161in}}%
\pgfpathlineto{\pgfqpoint{2.366284in}{1.535235in}}%
\pgfpathlineto{\pgfqpoint{2.366667in}{1.567830in}}%
\pgfpathlineto{\pgfqpoint{2.367433in}{1.555607in}}%
\pgfpathlineto{\pgfqpoint{2.368582in}{1.531161in}}%
\pgfpathlineto{\pgfqpoint{2.370114in}{1.555607in}}%
\pgfpathlineto{\pgfqpoint{2.370497in}{1.555607in}}%
\pgfpathlineto{\pgfqpoint{2.371646in}{1.527086in}}%
\pgfpathlineto{\pgfqpoint{2.372795in}{1.555607in}}%
\pgfpathlineto{\pgfqpoint{2.373178in}{1.543384in}}%
\pgfpathlineto{\pgfqpoint{2.373561in}{1.535235in}}%
\pgfpathlineto{\pgfqpoint{2.373944in}{1.539309in}}%
\pgfpathlineto{\pgfqpoint{2.374327in}{1.567830in}}%
\pgfpathlineto{\pgfqpoint{2.375095in}{1.547458in}}%
\pgfpathlineto{\pgfqpoint{2.376244in}{1.539309in}}%
\pgfpathlineto{\pgfqpoint{2.376626in}{1.551532in}}%
\pgfpathlineto{\pgfqpoint{2.377010in}{1.527086in}}%
\pgfpathlineto{\pgfqpoint{2.377010in}{1.527086in}}%
\pgfpathlineto{\pgfqpoint{2.377010in}{1.527086in}}%
\pgfpathlineto{\pgfqpoint{2.377393in}{1.559681in}}%
\pgfpathlineto{\pgfqpoint{2.378159in}{1.539309in}}%
\pgfpathlineto{\pgfqpoint{2.378924in}{1.543384in}}%
\pgfpathlineto{\pgfqpoint{2.379306in}{1.563756in}}%
\pgfpathlineto{\pgfqpoint{2.379306in}{1.563756in}}%
\pgfpathlineto{\pgfqpoint{2.379306in}{1.563756in}}%
\pgfpathlineto{\pgfqpoint{2.379689in}{1.527086in}}%
\pgfpathlineto{\pgfqpoint{2.380456in}{1.559681in}}%
\pgfpathlineto{\pgfqpoint{2.381988in}{1.539309in}}%
\pgfpathlineto{\pgfqpoint{2.382371in}{1.551532in}}%
\pgfpathlineto{\pgfqpoint{2.382754in}{1.539309in}}%
\pgfpathlineto{\pgfqpoint{2.383138in}{1.539309in}}%
\pgfpathlineto{\pgfqpoint{2.383521in}{1.543384in}}%
\pgfpathlineto{\pgfqpoint{2.384288in}{1.563756in}}%
\pgfpathlineto{\pgfqpoint{2.384670in}{1.555607in}}%
\pgfpathlineto{\pgfqpoint{2.386203in}{1.531161in}}%
\pgfpathlineto{\pgfqpoint{2.386969in}{1.555607in}}%
\pgfpathlineto{\pgfqpoint{2.387352in}{1.539309in}}%
\pgfpathlineto{\pgfqpoint{2.387735in}{1.531161in}}%
\pgfpathlineto{\pgfqpoint{2.387735in}{1.531161in}}%
\pgfpathlineto{\pgfqpoint{2.387735in}{1.531161in}}%
\pgfpathlineto{\pgfqpoint{2.388885in}{1.543384in}}%
\pgfpathlineto{\pgfqpoint{2.389653in}{1.527086in}}%
\pgfpathlineto{\pgfqpoint{2.390035in}{1.539309in}}%
\pgfpathlineto{\pgfqpoint{2.390418in}{1.563756in}}%
\pgfpathlineto{\pgfqpoint{2.391185in}{1.547458in}}%
\pgfpathlineto{\pgfqpoint{2.391570in}{1.535235in}}%
\pgfpathlineto{\pgfqpoint{2.391570in}{1.535235in}}%
\pgfpathlineto{\pgfqpoint{2.391570in}{1.535235in}}%
\pgfpathlineto{\pgfqpoint{2.391952in}{1.555607in}}%
\pgfpathlineto{\pgfqpoint{2.392718in}{1.539309in}}%
\pgfpathlineto{\pgfqpoint{2.393102in}{1.547458in}}%
\pgfpathlineto{\pgfqpoint{2.393486in}{1.543384in}}%
\pgfpathlineto{\pgfqpoint{2.394254in}{1.535235in}}%
\pgfpathlineto{\pgfqpoint{2.394637in}{1.555607in}}%
\pgfpathlineto{\pgfqpoint{2.394637in}{1.555607in}}%
\pgfpathlineto{\pgfqpoint{2.394637in}{1.555607in}}%
\pgfpathlineto{\pgfqpoint{2.395787in}{1.527086in}}%
\pgfpathlineto{\pgfqpoint{2.396170in}{1.531161in}}%
\pgfpathlineto{\pgfqpoint{2.397406in}{1.563756in}}%
\pgfpathlineto{\pgfqpoint{2.397790in}{1.539309in}}%
\pgfpathlineto{\pgfqpoint{2.398556in}{1.547458in}}%
\pgfpathlineto{\pgfqpoint{2.399705in}{1.563756in}}%
\pgfpathlineto{\pgfqpoint{2.400853in}{1.539309in}}%
\pgfpathlineto{\pgfqpoint{2.402003in}{1.547458in}}%
\pgfpathlineto{\pgfqpoint{2.402394in}{1.547458in}}%
\pgfpathlineto{\pgfqpoint{2.403159in}{1.539309in}}%
\pgfpathlineto{\pgfqpoint{2.403925in}{1.563756in}}%
\pgfpathlineto{\pgfqpoint{2.404308in}{1.543384in}}%
\pgfpathlineto{\pgfqpoint{2.404692in}{1.543384in}}%
\pgfpathlineto{\pgfqpoint{2.405075in}{1.539309in}}%
\pgfpathlineto{\pgfqpoint{2.405458in}{1.559681in}}%
\pgfpathlineto{\pgfqpoint{2.405841in}{1.555607in}}%
\pgfpathlineto{\pgfqpoint{2.406991in}{1.535235in}}%
\pgfpathlineto{\pgfqpoint{2.407374in}{1.551532in}}%
\pgfpathlineto{\pgfqpoint{2.408142in}{1.543384in}}%
\pgfpathlineto{\pgfqpoint{2.408908in}{1.531161in}}%
\pgfpathlineto{\pgfqpoint{2.409291in}{1.555607in}}%
\pgfpathlineto{\pgfqpoint{2.410058in}{1.547458in}}%
\pgfpathlineto{\pgfqpoint{2.410441in}{1.539309in}}%
\pgfpathlineto{\pgfqpoint{2.410825in}{1.543384in}}%
\pgfpathlineto{\pgfqpoint{2.411591in}{1.555607in}}%
\pgfpathlineto{\pgfqpoint{2.411974in}{1.543384in}}%
\pgfpathlineto{\pgfqpoint{2.412741in}{1.551532in}}%
\pgfpathlineto{\pgfqpoint{2.413125in}{1.559681in}}%
\pgfpathlineto{\pgfqpoint{2.414275in}{1.531161in}}%
\pgfpathlineto{\pgfqpoint{2.414658in}{1.539309in}}%
\pgfpathlineto{\pgfqpoint{2.415041in}{1.571904in}}%
\pgfpathlineto{\pgfqpoint{2.415425in}{1.539309in}}%
\pgfpathlineto{\pgfqpoint{2.415808in}{1.535235in}}%
\pgfpathlineto{\pgfqpoint{2.416191in}{1.539309in}}%
\pgfpathlineto{\pgfqpoint{2.416574in}{1.555607in}}%
\pgfpathlineto{\pgfqpoint{2.417340in}{1.547458in}}%
\pgfpathlineto{\pgfqpoint{2.418492in}{1.539309in}}%
\pgfpathlineto{\pgfqpoint{2.419258in}{1.567830in}}%
\pgfpathlineto{\pgfqpoint{2.419641in}{1.555607in}}%
\pgfpathlineto{\pgfqpoint{2.420024in}{1.551532in}}%
\pgfpathlineto{\pgfqpoint{2.420407in}{1.563756in}}%
\pgfpathlineto{\pgfqpoint{2.420790in}{1.555607in}}%
\pgfpathlineto{\pgfqpoint{2.422324in}{1.531161in}}%
\pgfpathlineto{\pgfqpoint{2.422708in}{1.535235in}}%
\pgfpathlineto{\pgfqpoint{2.423091in}{1.559681in}}%
\pgfpathlineto{\pgfqpoint{2.423858in}{1.543384in}}%
\pgfpathlineto{\pgfqpoint{2.424241in}{1.551532in}}%
\pgfpathlineto{\pgfqpoint{2.424624in}{1.531161in}}%
\pgfpathlineto{\pgfqpoint{2.425390in}{1.539309in}}%
\pgfpathlineto{\pgfqpoint{2.425773in}{1.539309in}}%
\pgfpathlineto{\pgfqpoint{2.426157in}{1.551532in}}%
\pgfpathlineto{\pgfqpoint{2.426541in}{1.543384in}}%
\pgfpathlineto{\pgfqpoint{2.426925in}{1.535235in}}%
\pgfpathlineto{\pgfqpoint{2.427308in}{1.543384in}}%
\pgfpathlineto{\pgfqpoint{2.427691in}{1.547458in}}%
\pgfpathlineto{\pgfqpoint{2.428073in}{1.531161in}}%
\pgfpathlineto{\pgfqpoint{2.428073in}{1.531161in}}%
\pgfpathlineto{\pgfqpoint{2.428073in}{1.531161in}}%
\pgfpathlineto{\pgfqpoint{2.428458in}{1.559681in}}%
\pgfpathlineto{\pgfqpoint{2.429225in}{1.543384in}}%
\pgfpathlineto{\pgfqpoint{2.429991in}{1.559681in}}%
\pgfpathlineto{\pgfqpoint{2.430374in}{1.535235in}}%
\pgfpathlineto{\pgfqpoint{2.431140in}{1.547458in}}%
\pgfpathlineto{\pgfqpoint{2.431524in}{1.551532in}}%
\pgfpathlineto{\pgfqpoint{2.431908in}{1.535235in}}%
\pgfpathlineto{\pgfqpoint{2.432290in}{1.539309in}}%
\pgfpathlineto{\pgfqpoint{2.433056in}{1.551532in}}%
\pgfpathlineto{\pgfqpoint{2.433822in}{1.535235in}}%
\pgfpathlineto{\pgfqpoint{2.434204in}{1.547458in}}%
\pgfpathlineto{\pgfqpoint{2.434970in}{1.555607in}}%
\pgfpathlineto{\pgfqpoint{2.435353in}{1.539309in}}%
\pgfpathlineto{\pgfqpoint{2.435353in}{1.539309in}}%
\pgfpathlineto{\pgfqpoint{2.435353in}{1.539309in}}%
\pgfpathlineto{\pgfqpoint{2.435737in}{1.559681in}}%
\pgfpathlineto{\pgfqpoint{2.436588in}{1.551532in}}%
\pgfpathlineto{\pgfqpoint{2.437737in}{1.543384in}}%
\pgfpathlineto{\pgfqpoint{2.438504in}{1.567830in}}%
\pgfpathlineto{\pgfqpoint{2.438888in}{1.551532in}}%
\pgfpathlineto{\pgfqpoint{2.439271in}{1.555607in}}%
\pgfpathlineto{\pgfqpoint{2.440419in}{1.535235in}}%
\pgfpathlineto{\pgfqpoint{2.440803in}{1.543384in}}%
\pgfpathlineto{\pgfqpoint{2.441187in}{1.543384in}}%
\pgfpathlineto{\pgfqpoint{2.441570in}{1.535235in}}%
\pgfpathlineto{\pgfqpoint{2.441570in}{1.535235in}}%
\pgfpathlineto{\pgfqpoint{2.441570in}{1.535235in}}%
\pgfpathlineto{\pgfqpoint{2.443112in}{1.559681in}}%
\pgfpathlineto{\pgfqpoint{2.444261in}{1.543384in}}%
\pgfpathlineto{\pgfqpoint{2.444644in}{1.551532in}}%
\pgfpathlineto{\pgfqpoint{2.445026in}{1.547458in}}%
\pgfpathlineto{\pgfqpoint{2.446561in}{1.535235in}}%
\pgfpathlineto{\pgfqpoint{2.447711in}{1.567830in}}%
\pgfpathlineto{\pgfqpoint{2.448477in}{1.539309in}}%
\pgfpathlineto{\pgfqpoint{2.448861in}{1.547458in}}%
\pgfpathlineto{\pgfqpoint{2.449246in}{1.547458in}}%
\pgfpathlineto{\pgfqpoint{2.449630in}{1.539309in}}%
\pgfpathlineto{\pgfqpoint{2.450013in}{1.547458in}}%
\pgfpathlineto{\pgfqpoint{2.450396in}{1.551532in}}%
\pgfpathlineto{\pgfqpoint{2.450780in}{1.543384in}}%
\pgfpathlineto{\pgfqpoint{2.450780in}{1.543384in}}%
\pgfpathlineto{\pgfqpoint{2.450780in}{1.543384in}}%
\pgfpathlineto{\pgfqpoint{2.451545in}{1.555607in}}%
\pgfpathlineto{\pgfqpoint{2.451929in}{1.535235in}}%
\pgfpathlineto{\pgfqpoint{2.452695in}{1.547458in}}%
\pgfpathlineto{\pgfqpoint{2.453461in}{1.551532in}}%
\pgfpathlineto{\pgfqpoint{2.453845in}{1.535235in}}%
\pgfpathlineto{\pgfqpoint{2.454228in}{1.543384in}}%
\pgfpathlineto{\pgfqpoint{2.455377in}{1.559681in}}%
\pgfpathlineto{\pgfqpoint{2.456529in}{1.531161in}}%
\pgfpathlineto{\pgfqpoint{2.456911in}{1.543384in}}%
\pgfpathlineto{\pgfqpoint{2.458443in}{1.571904in}}%
\pgfpathlineto{\pgfqpoint{2.459594in}{1.531161in}}%
\pgfpathlineto{\pgfqpoint{2.459978in}{1.555607in}}%
\pgfpathlineto{\pgfqpoint{2.459978in}{1.555607in}}%
\pgfpathlineto{\pgfqpoint{2.459978in}{1.555607in}}%
\pgfpathlineto{\pgfqpoint{2.460361in}{1.523012in}}%
\pgfpathlineto{\pgfqpoint{2.461128in}{1.551532in}}%
\pgfpathlineto{\pgfqpoint{2.461513in}{1.555607in}}%
\pgfpathlineto{\pgfqpoint{2.461896in}{1.551532in}}%
\pgfpathlineto{\pgfqpoint{2.462279in}{1.551532in}}%
\pgfpathlineto{\pgfqpoint{2.462662in}{1.535235in}}%
\pgfpathlineto{\pgfqpoint{2.462662in}{1.535235in}}%
\pgfpathlineto{\pgfqpoint{2.462662in}{1.535235in}}%
\pgfpathlineto{\pgfqpoint{2.463812in}{1.559681in}}%
\pgfpathlineto{\pgfqpoint{2.464579in}{1.523012in}}%
\pgfpathlineto{\pgfqpoint{2.465728in}{1.559681in}}%
\pgfpathlineto{\pgfqpoint{2.466493in}{1.559681in}}%
\pgfpathlineto{\pgfqpoint{2.468028in}{1.535235in}}%
\pgfpathlineto{\pgfqpoint{2.468794in}{1.551532in}}%
\pgfpathlineto{\pgfqpoint{2.469177in}{1.535235in}}%
\pgfpathlineto{\pgfqpoint{2.469561in}{1.551532in}}%
\pgfpathlineto{\pgfqpoint{2.469944in}{1.551532in}}%
\pgfpathlineto{\pgfqpoint{2.470711in}{1.531161in}}%
\pgfpathlineto{\pgfqpoint{2.472242in}{1.559681in}}%
\pgfpathlineto{\pgfqpoint{2.472627in}{1.535235in}}%
\pgfpathlineto{\pgfqpoint{2.473394in}{1.555607in}}%
\pgfpathlineto{\pgfqpoint{2.473777in}{1.543384in}}%
\pgfpathlineto{\pgfqpoint{2.474159in}{1.547458in}}%
\pgfpathlineto{\pgfqpoint{2.474542in}{1.555607in}}%
\pgfpathlineto{\pgfqpoint{2.474542in}{1.555607in}}%
\pgfpathlineto{\pgfqpoint{2.474542in}{1.555607in}}%
\pgfpathlineto{\pgfqpoint{2.475310in}{1.531161in}}%
\pgfpathlineto{\pgfqpoint{2.476078in}{1.539309in}}%
\pgfpathlineto{\pgfqpoint{2.477226in}{1.563756in}}%
\pgfpathlineto{\pgfqpoint{2.478375in}{1.535235in}}%
\pgfpathlineto{\pgfqpoint{2.479993in}{1.563756in}}%
\pgfpathlineto{\pgfqpoint{2.481141in}{1.531161in}}%
\pgfpathlineto{\pgfqpoint{2.482684in}{1.567830in}}%
\pgfpathlineto{\pgfqpoint{2.483068in}{1.551532in}}%
\pgfpathlineto{\pgfqpoint{2.483452in}{1.543384in}}%
\pgfpathlineto{\pgfqpoint{2.484218in}{1.547458in}}%
\pgfpathlineto{\pgfqpoint{2.484601in}{1.555607in}}%
\pgfpathlineto{\pgfqpoint{2.484984in}{1.535235in}}%
\pgfpathlineto{\pgfqpoint{2.484984in}{1.535235in}}%
\pgfpathlineto{\pgfqpoint{2.484984in}{1.535235in}}%
\pgfpathlineto{\pgfqpoint{2.485367in}{1.559681in}}%
\pgfpathlineto{\pgfqpoint{2.485751in}{1.539309in}}%
\pgfpathlineto{\pgfqpoint{2.486134in}{1.535235in}}%
\pgfpathlineto{\pgfqpoint{2.487666in}{1.551532in}}%
\pgfpathlineto{\pgfqpoint{2.488048in}{1.539309in}}%
\pgfpathlineto{\pgfqpoint{2.488815in}{1.543384in}}%
\pgfpathlineto{\pgfqpoint{2.489199in}{1.539309in}}%
\pgfpathlineto{\pgfqpoint{2.489964in}{1.563756in}}%
\pgfpathlineto{\pgfqpoint{2.490347in}{1.547458in}}%
\pgfpathlineto{\pgfqpoint{2.490731in}{1.555607in}}%
\pgfpathlineto{\pgfqpoint{2.490731in}{1.555607in}}%
\pgfpathlineto{\pgfqpoint{2.490731in}{1.555607in}}%
\pgfpathlineto{\pgfqpoint{2.491881in}{1.535235in}}%
\pgfpathlineto{\pgfqpoint{2.492265in}{1.563756in}}%
\pgfpathlineto{\pgfqpoint{2.492648in}{1.539309in}}%
\pgfpathlineto{\pgfqpoint{2.493031in}{1.535235in}}%
\pgfpathlineto{\pgfqpoint{2.493798in}{1.559681in}}%
\pgfpathlineto{\pgfqpoint{2.494181in}{1.543384in}}%
\pgfpathlineto{\pgfqpoint{2.494947in}{1.551532in}}%
\pgfpathlineto{\pgfqpoint{2.495331in}{1.543384in}}%
\pgfpathlineto{\pgfqpoint{2.495331in}{1.543384in}}%
\pgfpathlineto{\pgfqpoint{2.495331in}{1.543384in}}%
\pgfpathlineto{\pgfqpoint{2.495714in}{1.555607in}}%
\pgfpathlineto{\pgfqpoint{2.496097in}{1.547458in}}%
\pgfpathlineto{\pgfqpoint{2.496481in}{1.539309in}}%
\pgfpathlineto{\pgfqpoint{2.496481in}{1.539309in}}%
\pgfpathlineto{\pgfqpoint{2.496481in}{1.539309in}}%
\pgfpathlineto{\pgfqpoint{2.496865in}{1.551532in}}%
\pgfpathlineto{\pgfqpoint{2.497248in}{1.539309in}}%
\pgfpathlineto{\pgfqpoint{2.497632in}{1.539309in}}%
\pgfpathlineto{\pgfqpoint{2.498398in}{1.543384in}}%
\pgfpathlineto{\pgfqpoint{2.498781in}{1.527086in}}%
\pgfpathlineto{\pgfqpoint{2.498781in}{1.527086in}}%
\pgfpathlineto{\pgfqpoint{2.498781in}{1.527086in}}%
\pgfpathlineto{\pgfqpoint{2.500315in}{1.555607in}}%
\pgfpathlineto{\pgfqpoint{2.501848in}{1.535235in}}%
\pgfpathlineto{\pgfqpoint{2.502617in}{1.559681in}}%
\pgfpathlineto{\pgfqpoint{2.503383in}{1.535235in}}%
\pgfpathlineto{\pgfqpoint{2.503765in}{1.539309in}}%
\pgfpathlineto{\pgfqpoint{2.504149in}{1.539309in}}%
\pgfpathlineto{\pgfqpoint{2.505682in}{1.559681in}}%
\pgfpathlineto{\pgfqpoint{2.507216in}{1.543384in}}%
\pgfpathlineto{\pgfqpoint{2.507599in}{1.535235in}}%
\pgfpathlineto{\pgfqpoint{2.507983in}{1.539309in}}%
\pgfpathlineto{\pgfqpoint{2.508748in}{1.551532in}}%
\pgfpathlineto{\pgfqpoint{2.509132in}{1.543384in}}%
\pgfpathlineto{\pgfqpoint{2.509515in}{1.547458in}}%
\pgfpathlineto{\pgfqpoint{2.510665in}{1.571904in}}%
\pgfpathlineto{\pgfqpoint{2.512198in}{1.523012in}}%
\pgfpathlineto{\pgfqpoint{2.512964in}{1.527086in}}%
\pgfpathlineto{\pgfqpoint{2.514497in}{1.567830in}}%
\pgfpathlineto{\pgfqpoint{2.516031in}{1.535235in}}%
\pgfpathlineto{\pgfqpoint{2.517563in}{1.547458in}}%
\pgfpathlineto{\pgfqpoint{2.517948in}{1.535235in}}%
\pgfpathlineto{\pgfqpoint{2.518332in}{1.539309in}}%
\pgfpathlineto{\pgfqpoint{2.519098in}{1.555607in}}%
\pgfpathlineto{\pgfqpoint{2.519481in}{1.531161in}}%
\pgfpathlineto{\pgfqpoint{2.520247in}{1.551532in}}%
\pgfpathlineto{\pgfqpoint{2.520716in}{1.559681in}}%
\pgfpathlineto{\pgfqpoint{2.521100in}{1.535235in}}%
\pgfpathlineto{\pgfqpoint{2.521866in}{1.551532in}}%
\pgfpathlineto{\pgfqpoint{2.522257in}{1.547458in}}%
\pgfpathlineto{\pgfqpoint{2.522640in}{1.551532in}}%
\pgfpathlineto{\pgfqpoint{2.523410in}{1.514863in}}%
\pgfpathlineto{\pgfqpoint{2.523793in}{1.547458in}}%
\pgfpathlineto{\pgfqpoint{2.524176in}{1.559681in}}%
\pgfpathlineto{\pgfqpoint{2.524559in}{1.535235in}}%
\pgfpathlineto{\pgfqpoint{2.525326in}{1.539309in}}%
\pgfpathlineto{\pgfqpoint{2.526859in}{1.559681in}}%
\pgfpathlineto{\pgfqpoint{2.527242in}{1.559681in}}%
\pgfpathlineto{\pgfqpoint{2.528776in}{1.535235in}}%
\pgfpathlineto{\pgfqpoint{2.529160in}{1.543384in}}%
\pgfpathlineto{\pgfqpoint{2.529160in}{1.543384in}}%
\pgfpathlineto{\pgfqpoint{2.529160in}{1.543384in}}%
\pgfpathlineto{\pgfqpoint{2.529542in}{1.531161in}}%
\pgfpathlineto{\pgfqpoint{2.529926in}{1.555607in}}%
\pgfpathlineto{\pgfqpoint{2.530693in}{1.547458in}}%
\pgfpathlineto{\pgfqpoint{2.531077in}{1.551532in}}%
\pgfpathlineto{\pgfqpoint{2.531461in}{1.539309in}}%
\pgfpathlineto{\pgfqpoint{2.532228in}{1.547458in}}%
\pgfpathlineto{\pgfqpoint{2.532610in}{1.543384in}}%
\pgfpathlineto{\pgfqpoint{2.532993in}{1.547458in}}%
\pgfpathlineto{\pgfqpoint{2.533377in}{1.547458in}}%
\pgfpathlineto{\pgfqpoint{2.533760in}{1.559681in}}%
\pgfpathlineto{\pgfqpoint{2.534145in}{1.551532in}}%
\pgfpathlineto{\pgfqpoint{2.534527in}{1.523012in}}%
\pgfpathlineto{\pgfqpoint{2.534527in}{1.523012in}}%
\pgfpathlineto{\pgfqpoint{2.534527in}{1.523012in}}%
\pgfpathlineto{\pgfqpoint{2.534910in}{1.559681in}}%
\pgfpathlineto{\pgfqpoint{2.535677in}{1.539309in}}%
\pgfpathlineto{\pgfqpoint{2.536444in}{1.531161in}}%
\pgfpathlineto{\pgfqpoint{2.537210in}{1.563756in}}%
\pgfpathlineto{\pgfqpoint{2.537976in}{1.551532in}}%
\pgfpathlineto{\pgfqpoint{2.538744in}{1.543384in}}%
\pgfpathlineto{\pgfqpoint{2.539128in}{1.555607in}}%
\pgfpathlineto{\pgfqpoint{2.539513in}{1.543384in}}%
\pgfpathlineto{\pgfqpoint{2.539896in}{1.539309in}}%
\pgfpathlineto{\pgfqpoint{2.540279in}{1.551532in}}%
\pgfpathlineto{\pgfqpoint{2.540279in}{1.551532in}}%
\pgfpathlineto{\pgfqpoint{2.540279in}{1.551532in}}%
\pgfpathlineto{\pgfqpoint{2.540661in}{1.535235in}}%
\pgfpathlineto{\pgfqpoint{2.541429in}{1.543384in}}%
\pgfpathlineto{\pgfqpoint{2.542579in}{1.559681in}}%
\pgfpathlineto{\pgfqpoint{2.542962in}{1.531161in}}%
\pgfpathlineto{\pgfqpoint{2.543345in}{1.539309in}}%
\pgfpathlineto{\pgfqpoint{2.543729in}{1.571904in}}%
\pgfpathlineto{\pgfqpoint{2.544495in}{1.555607in}}%
\pgfpathlineto{\pgfqpoint{2.544879in}{1.539309in}}%
\pgfpathlineto{\pgfqpoint{2.545644in}{1.551532in}}%
\pgfpathlineto{\pgfqpoint{2.546027in}{1.543384in}}%
\pgfpathlineto{\pgfqpoint{2.546027in}{1.543384in}}%
\pgfpathlineto{\pgfqpoint{2.546027in}{1.543384in}}%
\pgfpathlineto{\pgfqpoint{2.546411in}{1.555607in}}%
\pgfpathlineto{\pgfqpoint{2.546794in}{1.551532in}}%
\pgfpathlineto{\pgfqpoint{2.547177in}{1.531161in}}%
\pgfpathlineto{\pgfqpoint{2.547561in}{1.547458in}}%
\pgfpathlineto{\pgfqpoint{2.547944in}{1.567830in}}%
\pgfpathlineto{\pgfqpoint{2.548326in}{1.547458in}}%
\pgfpathlineto{\pgfqpoint{2.548709in}{1.547458in}}%
\pgfpathlineto{\pgfqpoint{2.549860in}{1.539309in}}%
\pgfpathlineto{\pgfqpoint{2.550245in}{1.563756in}}%
\pgfpathlineto{\pgfqpoint{2.551010in}{1.543384in}}%
\pgfpathlineto{\pgfqpoint{2.551776in}{1.555607in}}%
\pgfpathlineto{\pgfqpoint{2.552543in}{1.539309in}}%
\pgfpathlineto{\pgfqpoint{2.552927in}{1.543384in}}%
\pgfpathlineto{\pgfqpoint{2.553310in}{1.547458in}}%
\pgfpathlineto{\pgfqpoint{2.553693in}{1.543384in}}%
\pgfpathlineto{\pgfqpoint{2.554076in}{1.543384in}}%
\pgfpathlineto{\pgfqpoint{2.554460in}{1.559681in}}%
\pgfpathlineto{\pgfqpoint{2.554460in}{1.559681in}}%
\pgfpathlineto{\pgfqpoint{2.554460in}{1.559681in}}%
\pgfpathlineto{\pgfqpoint{2.554843in}{1.539309in}}%
\pgfpathlineto{\pgfqpoint{2.555610in}{1.551532in}}%
\pgfpathlineto{\pgfqpoint{2.555993in}{1.543384in}}%
\pgfpathlineto{\pgfqpoint{2.556376in}{1.559681in}}%
\pgfpathlineto{\pgfqpoint{2.557143in}{1.547458in}}%
\pgfpathlineto{\pgfqpoint{2.557527in}{1.555607in}}%
\pgfpathlineto{\pgfqpoint{2.557527in}{1.555607in}}%
\pgfpathlineto{\pgfqpoint{2.557527in}{1.555607in}}%
\pgfpathlineto{\pgfqpoint{2.557910in}{1.543384in}}%
\pgfpathlineto{\pgfqpoint{2.557910in}{1.543384in}}%
\pgfpathlineto{\pgfqpoint{2.557910in}{1.543384in}}%
\pgfpathlineto{\pgfqpoint{2.558294in}{1.559681in}}%
\pgfpathlineto{\pgfqpoint{2.558294in}{1.559681in}}%
\pgfpathlineto{\pgfqpoint{2.558294in}{1.559681in}}%
\pgfpathlineto{\pgfqpoint{2.558677in}{1.531161in}}%
\pgfpathlineto{\pgfqpoint{2.558677in}{1.531161in}}%
\pgfpathlineto{\pgfqpoint{2.558677in}{1.531161in}}%
\pgfpathlineto{\pgfqpoint{2.559060in}{1.571904in}}%
\pgfpathlineto{\pgfqpoint{2.559826in}{1.555607in}}%
\pgfpathlineto{\pgfqpoint{2.561360in}{1.531161in}}%
\pgfpathlineto{\pgfqpoint{2.562518in}{1.563756in}}%
\pgfpathlineto{\pgfqpoint{2.563285in}{1.551532in}}%
\pgfpathlineto{\pgfqpoint{2.563668in}{1.551532in}}%
\pgfpathlineto{\pgfqpoint{2.564051in}{1.543384in}}%
\pgfpathlineto{\pgfqpoint{2.564434in}{1.547458in}}%
\pgfpathlineto{\pgfqpoint{2.564818in}{1.559681in}}%
\pgfpathlineto{\pgfqpoint{2.564818in}{1.559681in}}%
\pgfpathlineto{\pgfqpoint{2.564818in}{1.559681in}}%
\pgfpathlineto{\pgfqpoint{2.565201in}{1.539309in}}%
\pgfpathlineto{\pgfqpoint{2.565967in}{1.547458in}}%
\pgfpathlineto{\pgfqpoint{2.566350in}{1.571904in}}%
\pgfpathlineto{\pgfqpoint{2.567116in}{1.551532in}}%
\pgfpathlineto{\pgfqpoint{2.569501in}{1.535235in}}%
\pgfpathlineto{\pgfqpoint{2.569884in}{1.535235in}}%
\pgfpathlineto{\pgfqpoint{2.570651in}{1.555607in}}%
\pgfpathlineto{\pgfqpoint{2.571035in}{1.539309in}}%
\pgfpathlineto{\pgfqpoint{2.571418in}{1.535235in}}%
\pgfpathlineto{\pgfqpoint{2.572183in}{1.559681in}}%
\pgfpathlineto{\pgfqpoint{2.572949in}{1.547458in}}%
\pgfpathlineto{\pgfqpoint{2.573333in}{1.543384in}}%
\pgfpathlineto{\pgfqpoint{2.573716in}{1.563756in}}%
\pgfpathlineto{\pgfqpoint{2.574481in}{1.547458in}}%
\pgfpathlineto{\pgfqpoint{2.574864in}{1.547458in}}%
\pgfpathlineto{\pgfqpoint{2.575247in}{1.543384in}}%
\pgfpathlineto{\pgfqpoint{2.576397in}{1.551532in}}%
\pgfpathlineto{\pgfqpoint{2.576781in}{1.535235in}}%
\pgfpathlineto{\pgfqpoint{2.576781in}{1.535235in}}%
\pgfpathlineto{\pgfqpoint{2.576781in}{1.535235in}}%
\pgfpathlineto{\pgfqpoint{2.577931in}{1.555607in}}%
\pgfpathlineto{\pgfqpoint{2.579080in}{1.543384in}}%
\pgfpathlineto{\pgfqpoint{2.579464in}{1.543384in}}%
\pgfpathlineto{\pgfqpoint{2.580231in}{1.559681in}}%
\pgfpathlineto{\pgfqpoint{2.580996in}{1.523012in}}%
\pgfpathlineto{\pgfqpoint{2.581378in}{1.563756in}}%
\pgfpathlineto{\pgfqpoint{2.582145in}{1.555607in}}%
\pgfpathlineto{\pgfqpoint{2.583679in}{1.514863in}}%
\pgfpathlineto{\pgfqpoint{2.584062in}{1.531161in}}%
\pgfpathlineto{\pgfqpoint{2.584445in}{1.563756in}}%
\pgfpathlineto{\pgfqpoint{2.585214in}{1.535235in}}%
\pgfpathlineto{\pgfqpoint{2.585596in}{1.531161in}}%
\pgfpathlineto{\pgfqpoint{2.585980in}{1.547458in}}%
\pgfpathlineto{\pgfqpoint{2.586746in}{1.543384in}}%
\pgfpathlineto{\pgfqpoint{2.587514in}{1.535235in}}%
\pgfpathlineto{\pgfqpoint{2.587897in}{1.539309in}}%
\pgfpathlineto{\pgfqpoint{2.588663in}{1.563756in}}%
\pgfpathlineto{\pgfqpoint{2.589046in}{1.531161in}}%
\pgfpathlineto{\pgfqpoint{2.589811in}{1.543384in}}%
\pgfpathlineto{\pgfqpoint{2.590195in}{1.543384in}}%
\pgfpathlineto{\pgfqpoint{2.591346in}{1.551532in}}%
\pgfpathlineto{\pgfqpoint{2.592112in}{1.547458in}}%
\pgfpathlineto{\pgfqpoint{2.592495in}{1.555607in}}%
\pgfpathlineto{\pgfqpoint{2.592878in}{1.551532in}}%
\pgfpathlineto{\pgfqpoint{2.594412in}{1.539309in}}%
\pgfpathlineto{\pgfqpoint{2.594795in}{1.551532in}}%
\pgfpathlineto{\pgfqpoint{2.594795in}{1.551532in}}%
\pgfpathlineto{\pgfqpoint{2.594795in}{1.551532in}}%
\pgfpathlineto{\pgfqpoint{2.595947in}{1.527086in}}%
\pgfpathlineto{\pgfqpoint{2.596330in}{1.555607in}}%
\pgfpathlineto{\pgfqpoint{2.597095in}{1.539309in}}%
\pgfpathlineto{\pgfqpoint{2.598245in}{1.559681in}}%
\pgfpathlineto{\pgfqpoint{2.598629in}{1.518937in}}%
\pgfpathlineto{\pgfqpoint{2.599012in}{1.539309in}}%
\pgfpathlineto{\pgfqpoint{2.599779in}{1.575979in}}%
\pgfpathlineto{\pgfqpoint{2.600545in}{1.563756in}}%
\pgfpathlineto{\pgfqpoint{2.600930in}{1.563756in}}%
\pgfpathlineto{\pgfqpoint{2.601313in}{1.559681in}}%
\pgfpathlineto{\pgfqpoint{2.602471in}{1.527086in}}%
\pgfpathlineto{\pgfqpoint{2.602855in}{1.539309in}}%
\pgfpathlineto{\pgfqpoint{2.603238in}{1.543384in}}%
\pgfpathlineto{\pgfqpoint{2.603621in}{1.563756in}}%
\pgfpathlineto{\pgfqpoint{2.604387in}{1.555607in}}%
\pgfpathlineto{\pgfqpoint{2.604771in}{1.559681in}}%
\pgfpathlineto{\pgfqpoint{2.605538in}{1.531161in}}%
\pgfpathlineto{\pgfqpoint{2.605922in}{1.551532in}}%
\pgfpathlineto{\pgfqpoint{2.606305in}{1.535235in}}%
\pgfpathlineto{\pgfqpoint{2.606688in}{1.539309in}}%
\pgfpathlineto{\pgfqpoint{2.607071in}{1.575979in}}%
\pgfpathlineto{\pgfqpoint{2.607838in}{1.543384in}}%
\pgfpathlineto{\pgfqpoint{2.608989in}{1.555607in}}%
\pgfpathlineto{\pgfqpoint{2.609371in}{1.535235in}}%
\pgfpathlineto{\pgfqpoint{2.609755in}{1.551532in}}%
\pgfpathlineto{\pgfqpoint{2.610138in}{1.559681in}}%
\pgfpathlineto{\pgfqpoint{2.610138in}{1.559681in}}%
\pgfpathlineto{\pgfqpoint{2.610138in}{1.559681in}}%
\pgfpathlineto{\pgfqpoint{2.610905in}{1.547458in}}%
\pgfpathlineto{\pgfqpoint{2.611288in}{1.559681in}}%
\pgfpathlineto{\pgfqpoint{2.611757in}{1.523012in}}%
\pgfpathlineto{\pgfqpoint{2.612524in}{1.535235in}}%
\pgfpathlineto{\pgfqpoint{2.613673in}{1.551532in}}%
\pgfpathlineto{\pgfqpoint{2.614056in}{1.551532in}}%
\pgfpathlineto{\pgfqpoint{2.614440in}{1.523012in}}%
\pgfpathlineto{\pgfqpoint{2.615207in}{1.547458in}}%
\pgfpathlineto{\pgfqpoint{2.615974in}{1.543384in}}%
\pgfpathlineto{\pgfqpoint{2.616358in}{1.547458in}}%
\pgfpathlineto{\pgfqpoint{2.616741in}{1.539309in}}%
\pgfpathlineto{\pgfqpoint{2.616741in}{1.539309in}}%
\pgfpathlineto{\pgfqpoint{2.616741in}{1.539309in}}%
\pgfpathlineto{\pgfqpoint{2.617124in}{1.551532in}}%
\pgfpathlineto{\pgfqpoint{2.617507in}{1.539309in}}%
\pgfpathlineto{\pgfqpoint{2.617891in}{1.539309in}}%
\pgfpathlineto{\pgfqpoint{2.618273in}{1.527086in}}%
\pgfpathlineto{\pgfqpoint{2.618273in}{1.527086in}}%
\pgfpathlineto{\pgfqpoint{2.618273in}{1.527086in}}%
\pgfpathlineto{\pgfqpoint{2.619041in}{1.555607in}}%
\pgfpathlineto{\pgfqpoint{2.619807in}{1.551532in}}%
\pgfpathlineto{\pgfqpoint{2.620574in}{1.535235in}}%
\pgfpathlineto{\pgfqpoint{2.620954in}{1.539309in}}%
\pgfpathlineto{\pgfqpoint{2.621721in}{1.547458in}}%
\pgfpathlineto{\pgfqpoint{2.622105in}{1.563756in}}%
\pgfpathlineto{\pgfqpoint{2.622105in}{1.563756in}}%
\pgfpathlineto{\pgfqpoint{2.622105in}{1.563756in}}%
\pgfpathlineto{\pgfqpoint{2.622489in}{1.543384in}}%
\pgfpathlineto{\pgfqpoint{2.622489in}{1.543384in}}%
\pgfpathlineto{\pgfqpoint{2.622489in}{1.543384in}}%
\pgfpathlineto{\pgfqpoint{2.622872in}{1.567830in}}%
\pgfpathlineto{\pgfqpoint{2.622872in}{1.567830in}}%
\pgfpathlineto{\pgfqpoint{2.622872in}{1.567830in}}%
\pgfpathlineto{\pgfqpoint{2.623254in}{1.527086in}}%
\pgfpathlineto{\pgfqpoint{2.624020in}{1.543384in}}%
\pgfpathlineto{\pgfqpoint{2.624402in}{1.551532in}}%
\pgfpathlineto{\pgfqpoint{2.625935in}{1.527086in}}%
\pgfpathlineto{\pgfqpoint{2.626329in}{1.527086in}}%
\pgfpathlineto{\pgfqpoint{2.628250in}{1.555607in}}%
\pgfpathlineto{\pgfqpoint{2.629017in}{1.543384in}}%
\pgfpathlineto{\pgfqpoint{2.629402in}{1.563756in}}%
\pgfpathlineto{\pgfqpoint{2.630168in}{1.555607in}}%
\pgfpathlineto{\pgfqpoint{2.631318in}{1.531161in}}%
\pgfpathlineto{\pgfqpoint{2.632468in}{1.563756in}}%
\pgfpathlineto{\pgfqpoint{2.633234in}{1.535235in}}%
\pgfpathlineto{\pgfqpoint{2.633618in}{1.543384in}}%
\pgfpathlineto{\pgfqpoint{2.634768in}{1.559681in}}%
\pgfpathlineto{\pgfqpoint{2.635151in}{1.555607in}}%
\pgfpathlineto{\pgfqpoint{2.635534in}{1.555607in}}%
\pgfpathlineto{\pgfqpoint{2.636685in}{1.547458in}}%
\pgfpathlineto{\pgfqpoint{2.637069in}{1.551532in}}%
\pgfpathlineto{\pgfqpoint{2.637452in}{1.547458in}}%
\pgfpathlineto{\pgfqpoint{2.637835in}{1.543384in}}%
\pgfpathlineto{\pgfqpoint{2.639368in}{1.559681in}}%
\pgfpathlineto{\pgfqpoint{2.640517in}{1.535235in}}%
\pgfpathlineto{\pgfqpoint{2.642057in}{1.559681in}}%
\pgfpathlineto{\pgfqpoint{2.642441in}{1.543384in}}%
\pgfpathlineto{\pgfqpoint{2.642824in}{1.547458in}}%
\pgfpathlineto{\pgfqpoint{2.643206in}{1.559681in}}%
\pgfpathlineto{\pgfqpoint{2.643590in}{1.551532in}}%
\pgfpathlineto{\pgfqpoint{2.643975in}{1.527086in}}%
\pgfpathlineto{\pgfqpoint{2.643975in}{1.527086in}}%
\pgfpathlineto{\pgfqpoint{2.643975in}{1.527086in}}%
\pgfpathlineto{\pgfqpoint{2.644358in}{1.563756in}}%
\pgfpathlineto{\pgfqpoint{2.645123in}{1.543384in}}%
\pgfpathlineto{\pgfqpoint{2.645890in}{1.559681in}}%
\pgfpathlineto{\pgfqpoint{2.646273in}{1.539309in}}%
\pgfpathlineto{\pgfqpoint{2.647040in}{1.547458in}}%
\pgfpathlineto{\pgfqpoint{2.647423in}{1.547458in}}%
\pgfpathlineto{\pgfqpoint{2.648573in}{1.527086in}}%
\pgfpathlineto{\pgfqpoint{2.649724in}{1.559681in}}%
\pgfpathlineto{\pgfqpoint{2.650874in}{1.510789in}}%
\pgfpathlineto{\pgfqpoint{2.651256in}{1.535235in}}%
\pgfpathlineto{\pgfqpoint{2.652023in}{1.571904in}}%
\pgfpathlineto{\pgfqpoint{2.652407in}{1.547458in}}%
\pgfpathlineto{\pgfqpoint{2.652791in}{1.535235in}}%
\pgfpathlineto{\pgfqpoint{2.652791in}{1.535235in}}%
\pgfpathlineto{\pgfqpoint{2.652791in}{1.535235in}}%
\pgfpathlineto{\pgfqpoint{2.653940in}{1.551532in}}%
\pgfpathlineto{\pgfqpoint{2.654324in}{1.551532in}}%
\pgfpathlineto{\pgfqpoint{2.654708in}{1.555607in}}%
\pgfpathlineto{\pgfqpoint{2.655092in}{1.551532in}}%
\pgfpathlineto{\pgfqpoint{2.655475in}{1.539309in}}%
\pgfpathlineto{\pgfqpoint{2.656240in}{1.547458in}}%
\pgfpathlineto{\pgfqpoint{2.657006in}{1.539309in}}%
\pgfpathlineto{\pgfqpoint{2.658157in}{1.551532in}}%
\pgfpathlineto{\pgfqpoint{2.658540in}{1.539309in}}%
\pgfpathlineto{\pgfqpoint{2.658922in}{1.547458in}}%
\pgfpathlineto{\pgfqpoint{2.659391in}{1.563756in}}%
\pgfpathlineto{\pgfqpoint{2.659774in}{1.535235in}}%
\pgfpathlineto{\pgfqpoint{2.660540in}{1.555607in}}%
\pgfpathlineto{\pgfqpoint{2.660924in}{1.543384in}}%
\pgfpathlineto{\pgfqpoint{2.661691in}{1.547458in}}%
\pgfpathlineto{\pgfqpoint{2.662456in}{1.539309in}}%
\pgfpathlineto{\pgfqpoint{2.663223in}{1.563756in}}%
\pgfpathlineto{\pgfqpoint{2.663606in}{1.539309in}}%
\pgfpathlineto{\pgfqpoint{2.664373in}{1.551532in}}%
\pgfpathlineto{\pgfqpoint{2.664756in}{1.543384in}}%
\pgfpathlineto{\pgfqpoint{2.665139in}{1.551532in}}%
\pgfpathlineto{\pgfqpoint{2.665521in}{1.551532in}}%
\pgfpathlineto{\pgfqpoint{2.665904in}{1.535235in}}%
\pgfpathlineto{\pgfqpoint{2.665904in}{1.535235in}}%
\pgfpathlineto{\pgfqpoint{2.665904in}{1.535235in}}%
\pgfpathlineto{\pgfqpoint{2.666289in}{1.559681in}}%
\pgfpathlineto{\pgfqpoint{2.666672in}{1.555607in}}%
\pgfpathlineto{\pgfqpoint{2.667055in}{1.535235in}}%
\pgfpathlineto{\pgfqpoint{2.667819in}{1.551532in}}%
\pgfpathlineto{\pgfqpoint{2.668202in}{1.543384in}}%
\pgfpathlineto{\pgfqpoint{2.668202in}{1.543384in}}%
\pgfpathlineto{\pgfqpoint{2.668202in}{1.543384in}}%
\pgfpathlineto{\pgfqpoint{2.668585in}{1.555607in}}%
\pgfpathlineto{\pgfqpoint{2.669352in}{1.547458in}}%
\pgfpathlineto{\pgfqpoint{2.670119in}{1.547458in}}%
\pgfpathlineto{\pgfqpoint{2.671268in}{1.539309in}}%
\pgfpathlineto{\pgfqpoint{2.672035in}{1.555607in}}%
\pgfpathlineto{\pgfqpoint{2.672419in}{1.551532in}}%
\pgfpathlineto{\pgfqpoint{2.672803in}{1.527086in}}%
\pgfpathlineto{\pgfqpoint{2.673186in}{1.535235in}}%
\pgfpathlineto{\pgfqpoint{2.673568in}{1.555607in}}%
\pgfpathlineto{\pgfqpoint{2.674335in}{1.543384in}}%
\pgfpathlineto{\pgfqpoint{2.674718in}{1.527086in}}%
\pgfpathlineto{\pgfqpoint{2.674718in}{1.527086in}}%
\pgfpathlineto{\pgfqpoint{2.674718in}{1.527086in}}%
\pgfpathlineto{\pgfqpoint{2.676252in}{1.563756in}}%
\pgfpathlineto{\pgfqpoint{2.677018in}{1.535235in}}%
\pgfpathlineto{\pgfqpoint{2.677402in}{1.547458in}}%
\pgfpathlineto{\pgfqpoint{2.678551in}{1.559681in}}%
\pgfpathlineto{\pgfqpoint{2.679318in}{1.523012in}}%
\pgfpathlineto{\pgfqpoint{2.679701in}{1.551532in}}%
\pgfpathlineto{\pgfqpoint{2.680084in}{1.539309in}}%
\pgfpathlineto{\pgfqpoint{2.680084in}{1.539309in}}%
\pgfpathlineto{\pgfqpoint{2.680084in}{1.539309in}}%
\pgfpathlineto{\pgfqpoint{2.681234in}{1.555607in}}%
\pgfpathlineto{\pgfqpoint{2.682775in}{1.535235in}}%
\pgfpathlineto{\pgfqpoint{2.683158in}{1.531161in}}%
\pgfpathlineto{\pgfqpoint{2.683541in}{1.535235in}}%
\pgfpathlineto{\pgfqpoint{2.684307in}{1.551532in}}%
\pgfpathlineto{\pgfqpoint{2.684690in}{1.547458in}}%
\pgfpathlineto{\pgfqpoint{2.685073in}{1.535235in}}%
\pgfpathlineto{\pgfqpoint{2.685073in}{1.535235in}}%
\pgfpathlineto{\pgfqpoint{2.685073in}{1.535235in}}%
\pgfpathlineto{\pgfqpoint{2.685839in}{1.555607in}}%
\pgfpathlineto{\pgfqpoint{2.686222in}{1.543384in}}%
\pgfpathlineto{\pgfqpoint{2.686606in}{1.539309in}}%
\pgfpathlineto{\pgfqpoint{2.687756in}{1.547458in}}%
\pgfpathlineto{\pgfqpoint{2.688139in}{1.547458in}}%
\pgfpathlineto{\pgfqpoint{2.688905in}{1.535235in}}%
\pgfpathlineto{\pgfqpoint{2.690053in}{1.567830in}}%
\pgfpathlineto{\pgfqpoint{2.690820in}{1.531161in}}%
\pgfpathlineto{\pgfqpoint{2.691587in}{1.555607in}}%
\pgfpathlineto{\pgfqpoint{2.691969in}{1.563756in}}%
\pgfpathlineto{\pgfqpoint{2.693119in}{1.523012in}}%
\pgfpathlineto{\pgfqpoint{2.693504in}{1.563756in}}%
\pgfpathlineto{\pgfqpoint{2.694269in}{1.547458in}}%
\pgfpathlineto{\pgfqpoint{2.694652in}{1.555607in}}%
\pgfpathlineto{\pgfqpoint{2.695035in}{1.551532in}}%
\pgfpathlineto{\pgfqpoint{2.695418in}{1.535235in}}%
\pgfpathlineto{\pgfqpoint{2.696186in}{1.539309in}}%
\pgfpathlineto{\pgfqpoint{2.696952in}{1.539309in}}%
\pgfpathlineto{\pgfqpoint{2.697805in}{1.567830in}}%
\pgfpathlineto{\pgfqpoint{2.698570in}{1.559681in}}%
\pgfpathlineto{\pgfqpoint{2.698953in}{1.531161in}}%
\pgfpathlineto{\pgfqpoint{2.699336in}{1.547458in}}%
\pgfpathlineto{\pgfqpoint{2.700485in}{1.563756in}}%
\pgfpathlineto{\pgfqpoint{2.701633in}{1.527086in}}%
\pgfpathlineto{\pgfqpoint{2.702016in}{1.563756in}}%
\pgfpathlineto{\pgfqpoint{2.702783in}{1.531161in}}%
\pgfpathlineto{\pgfqpoint{2.703166in}{1.555607in}}%
\pgfpathlineto{\pgfqpoint{2.703549in}{1.551532in}}%
\pgfpathlineto{\pgfqpoint{2.703932in}{1.523012in}}%
\pgfpathlineto{\pgfqpoint{2.703932in}{1.523012in}}%
\pgfpathlineto{\pgfqpoint{2.703932in}{1.523012in}}%
\pgfpathlineto{\pgfqpoint{2.705082in}{1.563756in}}%
\pgfpathlineto{\pgfqpoint{2.705466in}{1.527086in}}%
\pgfpathlineto{\pgfqpoint{2.706232in}{1.551532in}}%
\pgfpathlineto{\pgfqpoint{2.707380in}{1.523012in}}%
\pgfpathlineto{\pgfqpoint{2.707763in}{1.535235in}}%
\pgfpathlineto{\pgfqpoint{2.708913in}{1.555607in}}%
\pgfpathlineto{\pgfqpoint{2.710063in}{1.535235in}}%
\pgfpathlineto{\pgfqpoint{2.711213in}{1.551532in}}%
\pgfpathlineto{\pgfqpoint{2.711597in}{1.547458in}}%
\pgfpathlineto{\pgfqpoint{2.711981in}{1.531161in}}%
\pgfpathlineto{\pgfqpoint{2.712746in}{1.539309in}}%
\pgfpathlineto{\pgfqpoint{2.713129in}{1.539309in}}%
\pgfpathlineto{\pgfqpoint{2.713894in}{1.547458in}}%
\pgfpathlineto{\pgfqpoint{2.714278in}{1.535235in}}%
\pgfpathlineto{\pgfqpoint{2.714278in}{1.535235in}}%
\pgfpathlineto{\pgfqpoint{2.714278in}{1.535235in}}%
\pgfpathlineto{\pgfqpoint{2.715810in}{1.555607in}}%
\pgfpathlineto{\pgfqpoint{2.716193in}{1.535235in}}%
\pgfpathlineto{\pgfqpoint{2.716193in}{1.535235in}}%
\pgfpathlineto{\pgfqpoint{2.716193in}{1.535235in}}%
\pgfpathlineto{\pgfqpoint{2.716576in}{1.567830in}}%
\pgfpathlineto{\pgfqpoint{2.716960in}{1.563756in}}%
\pgfpathlineto{\pgfqpoint{2.717344in}{1.535235in}}%
\pgfpathlineto{\pgfqpoint{2.718111in}{1.543384in}}%
\pgfpathlineto{\pgfqpoint{2.718494in}{1.555607in}}%
\pgfpathlineto{\pgfqpoint{2.718494in}{1.555607in}}%
\pgfpathlineto{\pgfqpoint{2.718494in}{1.555607in}}%
\pgfpathlineto{\pgfqpoint{2.719643in}{1.531161in}}%
\pgfpathlineto{\pgfqpoint{2.720793in}{1.563756in}}%
\pgfpathlineto{\pgfqpoint{2.721949in}{1.547458in}}%
\pgfpathlineto{\pgfqpoint{2.722332in}{1.551532in}}%
\pgfpathlineto{\pgfqpoint{2.722716in}{1.531161in}}%
\pgfpathlineto{\pgfqpoint{2.723098in}{1.543384in}}%
\pgfpathlineto{\pgfqpoint{2.723481in}{1.563756in}}%
\pgfpathlineto{\pgfqpoint{2.723865in}{1.543384in}}%
\pgfpathlineto{\pgfqpoint{2.724249in}{1.535235in}}%
\pgfpathlineto{\pgfqpoint{2.724249in}{1.535235in}}%
\pgfpathlineto{\pgfqpoint{2.724249in}{1.535235in}}%
\pgfpathlineto{\pgfqpoint{2.725398in}{1.551532in}}%
\pgfpathlineto{\pgfqpoint{2.726546in}{1.531161in}}%
\pgfpathlineto{\pgfqpoint{2.727313in}{1.563756in}}%
\pgfpathlineto{\pgfqpoint{2.727697in}{1.547458in}}%
\pgfpathlineto{\pgfqpoint{2.728080in}{1.547458in}}%
\pgfpathlineto{\pgfqpoint{2.728846in}{1.535235in}}%
\pgfpathlineto{\pgfqpoint{2.729228in}{1.551532in}}%
\pgfpathlineto{\pgfqpoint{2.729996in}{1.547458in}}%
\pgfpathlineto{\pgfqpoint{2.730380in}{1.539309in}}%
\pgfpathlineto{\pgfqpoint{2.731529in}{1.563756in}}%
\pgfpathlineto{\pgfqpoint{2.731912in}{1.527086in}}%
\pgfpathlineto{\pgfqpoint{2.732679in}{1.543384in}}%
\pgfpathlineto{\pgfqpoint{2.733062in}{1.539309in}}%
\pgfpathlineto{\pgfqpoint{2.733445in}{1.543384in}}%
\pgfpathlineto{\pgfqpoint{2.733914in}{1.543384in}}%
\pgfpathlineto{\pgfqpoint{2.734680in}{1.551532in}}%
\pgfpathlineto{\pgfqpoint{2.735063in}{1.547458in}}%
\pgfpathlineto{\pgfqpoint{2.735831in}{1.543384in}}%
\pgfpathlineto{\pgfqpoint{2.736214in}{1.547458in}}%
\pgfpathlineto{\pgfqpoint{2.736598in}{1.535235in}}%
\pgfpathlineto{\pgfqpoint{2.737746in}{1.559681in}}%
\pgfpathlineto{\pgfqpoint{2.739277in}{1.539309in}}%
\pgfpathlineto{\pgfqpoint{2.739660in}{1.551532in}}%
\pgfpathlineto{\pgfqpoint{2.740426in}{1.543384in}}%
\pgfpathlineto{\pgfqpoint{2.740809in}{1.527086in}}%
\pgfpathlineto{\pgfqpoint{2.740809in}{1.527086in}}%
\pgfpathlineto{\pgfqpoint{2.740809in}{1.527086in}}%
\pgfpathlineto{\pgfqpoint{2.741575in}{1.559681in}}%
\pgfpathlineto{\pgfqpoint{2.741959in}{1.555607in}}%
\pgfpathlineto{\pgfqpoint{2.742341in}{1.543384in}}%
\pgfpathlineto{\pgfqpoint{2.743107in}{1.551532in}}%
\pgfpathlineto{\pgfqpoint{2.743874in}{1.555607in}}%
\pgfpathlineto{\pgfqpoint{2.744257in}{1.531161in}}%
\pgfpathlineto{\pgfqpoint{2.745024in}{1.551532in}}%
\pgfpathlineto{\pgfqpoint{2.745407in}{1.567830in}}%
\pgfpathlineto{\pgfqpoint{2.745407in}{1.567830in}}%
\pgfpathlineto{\pgfqpoint{2.745407in}{1.567830in}}%
\pgfpathlineto{\pgfqpoint{2.746939in}{1.535235in}}%
\pgfpathlineto{\pgfqpoint{2.748472in}{1.547458in}}%
\pgfpathlineto{\pgfqpoint{2.748855in}{1.539309in}}%
\pgfpathlineto{\pgfqpoint{2.749237in}{1.571904in}}%
\pgfpathlineto{\pgfqpoint{2.749620in}{1.543384in}}%
\pgfpathlineto{\pgfqpoint{2.751154in}{1.531161in}}%
\pgfpathlineto{\pgfqpoint{2.751919in}{1.551532in}}%
\pgfpathlineto{\pgfqpoint{2.752303in}{1.543384in}}%
\pgfpathlineto{\pgfqpoint{2.752686in}{1.539309in}}%
\pgfpathlineto{\pgfqpoint{2.753454in}{1.567830in}}%
\pgfpathlineto{\pgfqpoint{2.753837in}{1.551532in}}%
\pgfpathlineto{\pgfqpoint{2.754219in}{1.543384in}}%
\pgfpathlineto{\pgfqpoint{2.754603in}{1.559681in}}%
\pgfpathlineto{\pgfqpoint{2.755368in}{1.555607in}}%
\pgfpathlineto{\pgfqpoint{2.756136in}{1.527086in}}%
\pgfpathlineto{\pgfqpoint{2.756519in}{1.547458in}}%
\pgfpathlineto{\pgfqpoint{2.756903in}{1.547458in}}%
\pgfpathlineto{\pgfqpoint{2.757286in}{1.559681in}}%
\pgfpathlineto{\pgfqpoint{2.757286in}{1.559681in}}%
\pgfpathlineto{\pgfqpoint{2.757286in}{1.559681in}}%
\pgfpathlineto{\pgfqpoint{2.758819in}{1.527086in}}%
\pgfpathlineto{\pgfqpoint{2.759968in}{1.551532in}}%
\pgfpathlineto{\pgfqpoint{2.760732in}{1.535235in}}%
\pgfpathlineto{\pgfqpoint{2.761115in}{1.559681in}}%
\pgfpathlineto{\pgfqpoint{2.761889in}{1.555607in}}%
\pgfpathlineto{\pgfqpoint{2.763037in}{1.539309in}}%
\pgfpathlineto{\pgfqpoint{2.764571in}{1.555607in}}%
\pgfpathlineto{\pgfqpoint{2.765336in}{1.531161in}}%
\pgfpathlineto{\pgfqpoint{2.765719in}{1.535235in}}%
\pgfpathlineto{\pgfqpoint{2.766103in}{1.571904in}}%
\pgfpathlineto{\pgfqpoint{2.766870in}{1.547458in}}%
\pgfpathlineto{\pgfqpoint{2.768019in}{1.535235in}}%
\pgfpathlineto{\pgfqpoint{2.768487in}{1.555607in}}%
\pgfpathlineto{\pgfqpoint{2.769253in}{1.543384in}}%
\pgfpathlineto{\pgfqpoint{2.769636in}{1.531161in}}%
\pgfpathlineto{\pgfqpoint{2.770019in}{1.555607in}}%
\pgfpathlineto{\pgfqpoint{2.770787in}{1.551532in}}%
\pgfpathlineto{\pgfqpoint{2.771170in}{1.551532in}}%
\pgfpathlineto{\pgfqpoint{2.771552in}{1.531161in}}%
\pgfpathlineto{\pgfqpoint{2.772318in}{1.535235in}}%
\pgfpathlineto{\pgfqpoint{2.773085in}{1.559681in}}%
\pgfpathlineto{\pgfqpoint{2.773851in}{1.555607in}}%
\pgfpathlineto{\pgfqpoint{2.775384in}{1.539309in}}%
\pgfpathlineto{\pgfqpoint{2.776150in}{1.543384in}}%
\pgfpathlineto{\pgfqpoint{2.777301in}{1.531161in}}%
\pgfpathlineto{\pgfqpoint{2.778834in}{1.547458in}}%
\pgfpathlineto{\pgfqpoint{2.779217in}{1.543384in}}%
\pgfpathlineto{\pgfqpoint{2.779600in}{1.563756in}}%
\pgfpathlineto{\pgfqpoint{2.779600in}{1.563756in}}%
\pgfpathlineto{\pgfqpoint{2.779600in}{1.563756in}}%
\pgfpathlineto{\pgfqpoint{2.779984in}{1.531161in}}%
\pgfpathlineto{\pgfqpoint{2.780750in}{1.555607in}}%
\pgfpathlineto{\pgfqpoint{2.781133in}{1.539309in}}%
\pgfpathlineto{\pgfqpoint{2.781901in}{1.543384in}}%
\pgfpathlineto{\pgfqpoint{2.782284in}{1.551532in}}%
\pgfpathlineto{\pgfqpoint{2.782667in}{1.535235in}}%
\pgfpathlineto{\pgfqpoint{2.783433in}{1.543384in}}%
\pgfpathlineto{\pgfqpoint{2.783816in}{1.551532in}}%
\pgfpathlineto{\pgfqpoint{2.783816in}{1.551532in}}%
\pgfpathlineto{\pgfqpoint{2.783816in}{1.551532in}}%
\pgfpathlineto{\pgfqpoint{2.784200in}{1.539309in}}%
\pgfpathlineto{\pgfqpoint{2.784200in}{1.539309in}}%
\pgfpathlineto{\pgfqpoint{2.784200in}{1.539309in}}%
\pgfpathlineto{\pgfqpoint{2.784584in}{1.555607in}}%
\pgfpathlineto{\pgfqpoint{2.785350in}{1.547458in}}%
\pgfpathlineto{\pgfqpoint{2.786500in}{1.535235in}}%
\pgfpathlineto{\pgfqpoint{2.788033in}{1.567830in}}%
\pgfpathlineto{\pgfqpoint{2.788416in}{1.523012in}}%
\pgfpathlineto{\pgfqpoint{2.789184in}{1.551532in}}%
\pgfpathlineto{\pgfqpoint{2.789950in}{1.535235in}}%
\pgfpathlineto{\pgfqpoint{2.791099in}{1.555607in}}%
\pgfpathlineto{\pgfqpoint{2.791482in}{1.527086in}}%
\pgfpathlineto{\pgfqpoint{2.792250in}{1.539309in}}%
\pgfpathlineto{\pgfqpoint{2.792633in}{1.567830in}}%
\pgfpathlineto{\pgfqpoint{2.793400in}{1.551532in}}%
\pgfpathlineto{\pgfqpoint{2.793783in}{1.551532in}}%
\pgfpathlineto{\pgfqpoint{2.794167in}{1.547458in}}%
\pgfpathlineto{\pgfqpoint{2.794550in}{1.555607in}}%
\pgfpathlineto{\pgfqpoint{2.794933in}{1.547458in}}%
\pgfpathlineto{\pgfqpoint{2.795317in}{1.539309in}}%
\pgfpathlineto{\pgfqpoint{2.795700in}{1.547458in}}%
\pgfpathlineto{\pgfqpoint{2.796084in}{1.547458in}}%
\pgfpathlineto{\pgfqpoint{2.796467in}{1.539309in}}%
\pgfpathlineto{\pgfqpoint{2.796850in}{1.543384in}}%
\pgfpathlineto{\pgfqpoint{2.797234in}{1.555607in}}%
\pgfpathlineto{\pgfqpoint{2.797234in}{1.555607in}}%
\pgfpathlineto{\pgfqpoint{2.797234in}{1.555607in}}%
\pgfpathlineto{\pgfqpoint{2.797616in}{1.535235in}}%
\pgfpathlineto{\pgfqpoint{2.798383in}{1.551532in}}%
\pgfpathlineto{\pgfqpoint{2.799149in}{1.535235in}}%
\pgfpathlineto{\pgfqpoint{2.799916in}{1.539309in}}%
\pgfpathlineto{\pgfqpoint{2.800684in}{1.571904in}}%
\pgfpathlineto{\pgfqpoint{2.801067in}{1.543384in}}%
\pgfpathlineto{\pgfqpoint{2.801840in}{1.543384in}}%
\pgfpathlineto{\pgfqpoint{2.802223in}{1.547458in}}%
\pgfpathlineto{\pgfqpoint{2.802607in}{1.563756in}}%
\pgfpathlineto{\pgfqpoint{2.802991in}{1.551532in}}%
\pgfpathlineto{\pgfqpoint{2.803373in}{1.535235in}}%
\pgfpathlineto{\pgfqpoint{2.804139in}{1.547458in}}%
\pgfpathlineto{\pgfqpoint{2.804906in}{1.531161in}}%
\pgfpathlineto{\pgfqpoint{2.805672in}{1.547458in}}%
\pgfpathlineto{\pgfqpoint{2.806056in}{1.535235in}}%
\pgfpathlineto{\pgfqpoint{2.807586in}{1.551532in}}%
\pgfpathlineto{\pgfqpoint{2.807969in}{1.535235in}}%
\pgfpathlineto{\pgfqpoint{2.807969in}{1.535235in}}%
\pgfpathlineto{\pgfqpoint{2.807969in}{1.535235in}}%
\pgfpathlineto{\pgfqpoint{2.808821in}{1.563756in}}%
\pgfpathlineto{\pgfqpoint{2.809203in}{1.559681in}}%
\pgfpathlineto{\pgfqpoint{2.809587in}{1.527086in}}%
\pgfpathlineto{\pgfqpoint{2.810352in}{1.535235in}}%
\pgfpathlineto{\pgfqpoint{2.810736in}{1.555607in}}%
\pgfpathlineto{\pgfqpoint{2.811119in}{1.543384in}}%
\pgfpathlineto{\pgfqpoint{2.811503in}{1.535235in}}%
\pgfpathlineto{\pgfqpoint{2.811503in}{1.535235in}}%
\pgfpathlineto{\pgfqpoint{2.811503in}{1.535235in}}%
\pgfpathlineto{\pgfqpoint{2.813418in}{1.571904in}}%
\pgfpathlineto{\pgfqpoint{2.814951in}{1.535235in}}%
\pgfpathlineto{\pgfqpoint{2.816100in}{1.547458in}}%
\pgfpathlineto{\pgfqpoint{2.816484in}{1.531161in}}%
\pgfpathlineto{\pgfqpoint{2.816484in}{1.531161in}}%
\pgfpathlineto{\pgfqpoint{2.816484in}{1.531161in}}%
\pgfpathlineto{\pgfqpoint{2.817249in}{1.555607in}}%
\pgfpathlineto{\pgfqpoint{2.817632in}{1.547458in}}%
\pgfpathlineto{\pgfqpoint{2.819162in}{1.539309in}}%
\pgfpathlineto{\pgfqpoint{2.819929in}{1.555607in}}%
\pgfpathlineto{\pgfqpoint{2.820312in}{1.563756in}}%
\pgfpathlineto{\pgfqpoint{2.820696in}{1.543384in}}%
\pgfpathlineto{\pgfqpoint{2.821462in}{1.559681in}}%
\pgfpathlineto{\pgfqpoint{2.822995in}{1.527086in}}%
\pgfpathlineto{\pgfqpoint{2.824527in}{1.555607in}}%
\pgfpathlineto{\pgfqpoint{2.825677in}{1.535235in}}%
\pgfpathlineto{\pgfqpoint{2.827210in}{1.567830in}}%
\pgfpathlineto{\pgfqpoint{2.827594in}{1.555607in}}%
\pgfpathlineto{\pgfqpoint{2.828360in}{1.559681in}}%
\pgfpathlineto{\pgfqpoint{2.828744in}{1.531161in}}%
\pgfpathlineto{\pgfqpoint{2.829509in}{1.535235in}}%
\pgfpathlineto{\pgfqpoint{2.829893in}{1.559681in}}%
\pgfpathlineto{\pgfqpoint{2.830660in}{1.547458in}}%
\pgfpathlineto{\pgfqpoint{2.831426in}{1.523012in}}%
\pgfpathlineto{\pgfqpoint{2.831809in}{1.543384in}}%
\pgfpathlineto{\pgfqpoint{2.832193in}{1.543384in}}%
\pgfpathlineto{\pgfqpoint{2.832575in}{1.555607in}}%
\pgfpathlineto{\pgfqpoint{2.832958in}{1.531161in}}%
\pgfpathlineto{\pgfqpoint{2.833342in}{1.543384in}}%
\pgfpathlineto{\pgfqpoint{2.833725in}{1.555607in}}%
\pgfpathlineto{\pgfqpoint{2.834493in}{1.551532in}}%
\pgfpathlineto{\pgfqpoint{2.834876in}{1.531161in}}%
\pgfpathlineto{\pgfqpoint{2.835641in}{1.547458in}}%
\pgfpathlineto{\pgfqpoint{2.836024in}{1.555607in}}%
\pgfpathlineto{\pgfqpoint{2.836408in}{1.551532in}}%
\pgfpathlineto{\pgfqpoint{2.837175in}{1.543384in}}%
\pgfpathlineto{\pgfqpoint{2.837558in}{1.559681in}}%
\pgfpathlineto{\pgfqpoint{2.837941in}{1.543384in}}%
\pgfpathlineto{\pgfqpoint{2.838323in}{1.523012in}}%
\pgfpathlineto{\pgfqpoint{2.838707in}{1.543384in}}%
\pgfpathlineto{\pgfqpoint{2.839090in}{1.567830in}}%
\pgfpathlineto{\pgfqpoint{2.839856in}{1.547458in}}%
\pgfpathlineto{\pgfqpoint{2.840621in}{1.555607in}}%
\pgfpathlineto{\pgfqpoint{2.841473in}{1.531161in}}%
\pgfpathlineto{\pgfqpoint{2.841864in}{1.539309in}}%
\pgfpathlineto{\pgfqpoint{2.843013in}{1.559681in}}%
\pgfpathlineto{\pgfqpoint{2.843780in}{1.547458in}}%
\pgfpathlineto{\pgfqpoint{2.845313in}{1.531161in}}%
\pgfpathlineto{\pgfqpoint{2.845697in}{1.523012in}}%
\pgfpathlineto{\pgfqpoint{2.846080in}{1.563756in}}%
\pgfpathlineto{\pgfqpoint{2.846846in}{1.543384in}}%
\pgfpathlineto{\pgfqpoint{2.847230in}{1.531161in}}%
\pgfpathlineto{\pgfqpoint{2.847230in}{1.531161in}}%
\pgfpathlineto{\pgfqpoint{2.847230in}{1.531161in}}%
\pgfpathlineto{\pgfqpoint{2.847613in}{1.551532in}}%
\pgfpathlineto{\pgfqpoint{2.847613in}{1.551532in}}%
\pgfpathlineto{\pgfqpoint{2.847613in}{1.551532in}}%
\pgfpathlineto{\pgfqpoint{2.847996in}{1.527086in}}%
\pgfpathlineto{\pgfqpoint{2.847996in}{1.527086in}}%
\pgfpathlineto{\pgfqpoint{2.847996in}{1.527086in}}%
\pgfpathlineto{\pgfqpoint{2.848380in}{1.559681in}}%
\pgfpathlineto{\pgfqpoint{2.848763in}{1.551532in}}%
\pgfpathlineto{\pgfqpoint{2.849148in}{1.527086in}}%
\pgfpathlineto{\pgfqpoint{2.849532in}{1.551532in}}%
\pgfpathlineto{\pgfqpoint{2.850298in}{1.559681in}}%
\pgfpathlineto{\pgfqpoint{2.851064in}{1.543384in}}%
\pgfpathlineto{\pgfqpoint{2.851446in}{1.551532in}}%
\pgfpathlineto{\pgfqpoint{2.851829in}{1.547458in}}%
\pgfpathlineto{\pgfqpoint{2.852211in}{1.555607in}}%
\pgfpathlineto{\pgfqpoint{2.852595in}{1.551532in}}%
\pgfpathlineto{\pgfqpoint{2.853359in}{1.539309in}}%
\pgfpathlineto{\pgfqpoint{2.854508in}{1.555607in}}%
\pgfpathlineto{\pgfqpoint{2.855658in}{1.518937in}}%
\pgfpathlineto{\pgfqpoint{2.856807in}{1.555607in}}%
\pgfpathlineto{\pgfqpoint{2.857574in}{1.535235in}}%
\pgfpathlineto{\pgfqpoint{2.859106in}{1.567830in}}%
\pgfpathlineto{\pgfqpoint{2.860256in}{1.535235in}}%
\pgfpathlineto{\pgfqpoint{2.860639in}{1.551532in}}%
\pgfpathlineto{\pgfqpoint{2.861405in}{1.543384in}}%
\pgfpathlineto{\pgfqpoint{2.861789in}{1.539309in}}%
\pgfpathlineto{\pgfqpoint{2.862556in}{1.547458in}}%
\pgfpathlineto{\pgfqpoint{2.862939in}{1.543384in}}%
\pgfpathlineto{\pgfqpoint{2.863322in}{1.543384in}}%
\pgfpathlineto{\pgfqpoint{2.864472in}{1.559681in}}%
\pgfpathlineto{\pgfqpoint{2.865621in}{1.543384in}}%
\pgfpathlineto{\pgfqpoint{2.866003in}{1.547458in}}%
\pgfpathlineto{\pgfqpoint{2.866387in}{1.567830in}}%
\pgfpathlineto{\pgfqpoint{2.866387in}{1.567830in}}%
\pgfpathlineto{\pgfqpoint{2.866387in}{1.567830in}}%
\pgfpathlineto{\pgfqpoint{2.866770in}{1.539309in}}%
\pgfpathlineto{\pgfqpoint{2.867537in}{1.555607in}}%
\pgfpathlineto{\pgfqpoint{2.868686in}{1.531161in}}%
\pgfpathlineto{\pgfqpoint{2.870220in}{1.559681in}}%
\pgfpathlineto{\pgfqpoint{2.870603in}{1.543384in}}%
\pgfpathlineto{\pgfqpoint{2.870603in}{1.543384in}}%
\pgfpathlineto{\pgfqpoint{2.870603in}{1.543384in}}%
\pgfpathlineto{\pgfqpoint{2.871369in}{1.563756in}}%
\pgfpathlineto{\pgfqpoint{2.871752in}{1.539309in}}%
\pgfpathlineto{\pgfqpoint{2.872518in}{1.551532in}}%
\pgfpathlineto{\pgfqpoint{2.873668in}{1.563756in}}%
\pgfpathlineto{\pgfqpoint{2.874051in}{1.518937in}}%
\pgfpathlineto{\pgfqpoint{2.874817in}{1.543384in}}%
\pgfpathlineto{\pgfqpoint{2.876735in}{1.563756in}}%
\pgfpathlineto{\pgfqpoint{2.877118in}{1.514863in}}%
\pgfpathlineto{\pgfqpoint{2.877885in}{1.547458in}}%
\pgfpathlineto{\pgfqpoint{2.879503in}{1.523012in}}%
\pgfpathlineto{\pgfqpoint{2.879886in}{1.547458in}}%
\pgfpathlineto{\pgfqpoint{2.879886in}{1.547458in}}%
\pgfpathlineto{\pgfqpoint{2.879886in}{1.547458in}}%
\pgfpathlineto{\pgfqpoint{2.880269in}{1.514863in}}%
\pgfpathlineto{\pgfqpoint{2.880652in}{1.531161in}}%
\pgfpathlineto{\pgfqpoint{2.881418in}{1.555607in}}%
\pgfpathlineto{\pgfqpoint{2.881808in}{1.539309in}}%
\pgfpathlineto{\pgfqpoint{2.882191in}{1.527086in}}%
\pgfpathlineto{\pgfqpoint{2.882575in}{1.555607in}}%
\pgfpathlineto{\pgfqpoint{2.883342in}{1.531161in}}%
\pgfpathlineto{\pgfqpoint{2.883727in}{1.551532in}}%
\pgfpathlineto{\pgfqpoint{2.884492in}{1.547458in}}%
\pgfpathlineto{\pgfqpoint{2.884876in}{1.547458in}}%
\pgfpathlineto{\pgfqpoint{2.885259in}{1.535235in}}%
\pgfpathlineto{\pgfqpoint{2.886027in}{1.543384in}}%
\pgfpathlineto{\pgfqpoint{2.886410in}{1.539309in}}%
\pgfpathlineto{\pgfqpoint{2.886792in}{1.563756in}}%
\pgfpathlineto{\pgfqpoint{2.887176in}{1.555607in}}%
\pgfpathlineto{\pgfqpoint{2.887941in}{1.539309in}}%
\pgfpathlineto{\pgfqpoint{2.888324in}{1.551532in}}%
\pgfpathlineto{\pgfqpoint{2.889090in}{1.543384in}}%
\pgfpathlineto{\pgfqpoint{2.889474in}{1.547458in}}%
\pgfpathlineto{\pgfqpoint{2.890239in}{1.535235in}}%
\pgfpathlineto{\pgfqpoint{2.890622in}{1.555607in}}%
\pgfpathlineto{\pgfqpoint{2.890622in}{1.555607in}}%
\pgfpathlineto{\pgfqpoint{2.890622in}{1.555607in}}%
\pgfpathlineto{\pgfqpoint{2.891005in}{1.531161in}}%
\pgfpathlineto{\pgfqpoint{2.891389in}{1.551532in}}%
\pgfpathlineto{\pgfqpoint{2.891772in}{1.555607in}}%
\pgfpathlineto{\pgfqpoint{2.892538in}{1.539309in}}%
\pgfpathlineto{\pgfqpoint{2.893687in}{1.567830in}}%
\pgfpathlineto{\pgfqpoint{2.894453in}{1.547458in}}%
\pgfpathlineto{\pgfqpoint{2.894837in}{1.567830in}}%
\pgfpathlineto{\pgfqpoint{2.895220in}{1.547458in}}%
\pgfpathlineto{\pgfqpoint{2.895602in}{1.539309in}}%
\pgfpathlineto{\pgfqpoint{2.895986in}{1.547458in}}%
\pgfpathlineto{\pgfqpoint{2.897134in}{1.559681in}}%
\pgfpathlineto{\pgfqpoint{2.897517in}{1.535235in}}%
\pgfpathlineto{\pgfqpoint{2.898284in}{1.539309in}}%
\pgfpathlineto{\pgfqpoint{2.899432in}{1.555607in}}%
\pgfpathlineto{\pgfqpoint{2.899813in}{1.555607in}}%
\pgfpathlineto{\pgfqpoint{2.900196in}{1.551532in}}%
\pgfpathlineto{\pgfqpoint{2.900581in}{1.531161in}}%
\pgfpathlineto{\pgfqpoint{2.901347in}{1.547458in}}%
\pgfpathlineto{\pgfqpoint{2.901730in}{1.551532in}}%
\pgfpathlineto{\pgfqpoint{2.902113in}{1.567830in}}%
\pgfpathlineto{\pgfqpoint{2.902113in}{1.567830in}}%
\pgfpathlineto{\pgfqpoint{2.902113in}{1.567830in}}%
\pgfpathlineto{\pgfqpoint{2.903645in}{1.539309in}}%
\pgfpathlineto{\pgfqpoint{2.904028in}{1.543384in}}%
\pgfpathlineto{\pgfqpoint{2.904410in}{1.535235in}}%
\pgfpathlineto{\pgfqpoint{2.904794in}{1.567830in}}%
\pgfpathlineto{\pgfqpoint{2.905560in}{1.543384in}}%
\pgfpathlineto{\pgfqpoint{2.907093in}{1.555607in}}%
\pgfpathlineto{\pgfqpoint{2.908626in}{1.535235in}}%
\pgfpathlineto{\pgfqpoint{2.909010in}{1.543384in}}%
\pgfpathlineto{\pgfqpoint{2.909394in}{1.555607in}}%
\pgfpathlineto{\pgfqpoint{2.909778in}{1.547458in}}%
\pgfpathlineto{\pgfqpoint{2.910544in}{1.531161in}}%
\pgfpathlineto{\pgfqpoint{2.910927in}{1.563756in}}%
\pgfpathlineto{\pgfqpoint{2.910927in}{1.563756in}}%
\pgfpathlineto{\pgfqpoint{2.910927in}{1.563756in}}%
\pgfpathlineto{\pgfqpoint{2.911309in}{1.527086in}}%
\pgfpathlineto{\pgfqpoint{2.912078in}{1.543384in}}%
\pgfpathlineto{\pgfqpoint{2.912460in}{1.555607in}}%
\pgfpathlineto{\pgfqpoint{2.912460in}{1.555607in}}%
\pgfpathlineto{\pgfqpoint{2.912460in}{1.555607in}}%
\pgfpathlineto{\pgfqpoint{2.912844in}{1.539309in}}%
\pgfpathlineto{\pgfqpoint{2.912844in}{1.539309in}}%
\pgfpathlineto{\pgfqpoint{2.912844in}{1.539309in}}%
\pgfpathlineto{\pgfqpoint{2.913227in}{1.559681in}}%
\pgfpathlineto{\pgfqpoint{2.913993in}{1.543384in}}%
\pgfpathlineto{\pgfqpoint{2.914378in}{1.551532in}}%
\pgfpathlineto{\pgfqpoint{2.914761in}{1.535235in}}%
\pgfpathlineto{\pgfqpoint{2.914761in}{1.535235in}}%
\pgfpathlineto{\pgfqpoint{2.914761in}{1.535235in}}%
\pgfpathlineto{\pgfqpoint{2.915144in}{1.555607in}}%
\pgfpathlineto{\pgfqpoint{2.915612in}{1.543384in}}%
\pgfpathlineto{\pgfqpoint{2.915996in}{1.535235in}}%
\pgfpathlineto{\pgfqpoint{2.916379in}{1.559681in}}%
\pgfpathlineto{\pgfqpoint{2.916762in}{1.535235in}}%
\pgfpathlineto{\pgfqpoint{2.917146in}{1.535235in}}%
\pgfpathlineto{\pgfqpoint{2.917913in}{1.567830in}}%
\pgfpathlineto{\pgfqpoint{2.918296in}{1.543384in}}%
\pgfpathlineto{\pgfqpoint{2.918679in}{1.559681in}}%
\pgfpathlineto{\pgfqpoint{2.919446in}{1.555607in}}%
\pgfpathlineto{\pgfqpoint{2.920213in}{1.535235in}}%
\pgfpathlineto{\pgfqpoint{2.920597in}{1.539309in}}%
\pgfpathlineto{\pgfqpoint{2.921364in}{1.559681in}}%
\pgfpathlineto{\pgfqpoint{2.921753in}{1.551532in}}%
\pgfpathlineto{\pgfqpoint{2.922137in}{1.535235in}}%
\pgfpathlineto{\pgfqpoint{2.922903in}{1.543384in}}%
\pgfpathlineto{\pgfqpoint{2.924053in}{1.559681in}}%
\pgfpathlineto{\pgfqpoint{2.925203in}{1.535235in}}%
\pgfpathlineto{\pgfqpoint{2.925970in}{1.559681in}}%
\pgfpathlineto{\pgfqpoint{2.926736in}{1.551532in}}%
\pgfpathlineto{\pgfqpoint{2.927119in}{1.547458in}}%
\pgfpathlineto{\pgfqpoint{2.927502in}{1.571904in}}%
\pgfpathlineto{\pgfqpoint{2.927887in}{1.559681in}}%
\pgfpathlineto{\pgfqpoint{2.929037in}{1.539309in}}%
\pgfpathlineto{\pgfqpoint{2.929420in}{1.539309in}}%
\pgfpathlineto{\pgfqpoint{2.930570in}{1.559681in}}%
\pgfpathlineto{\pgfqpoint{2.932103in}{1.543384in}}%
\pgfpathlineto{\pgfqpoint{2.932486in}{1.543384in}}%
\pgfpathlineto{\pgfqpoint{2.933253in}{1.551532in}}%
\pgfpathlineto{\pgfqpoint{2.933636in}{1.535235in}}%
\pgfpathlineto{\pgfqpoint{2.934019in}{1.551532in}}%
\pgfpathlineto{\pgfqpoint{2.934403in}{1.559681in}}%
\pgfpathlineto{\pgfqpoint{2.934786in}{1.551532in}}%
\pgfpathlineto{\pgfqpoint{2.935169in}{1.551532in}}%
\pgfpathlineto{\pgfqpoint{2.935552in}{1.539309in}}%
\pgfpathlineto{\pgfqpoint{2.936319in}{1.547458in}}%
\pgfpathlineto{\pgfqpoint{2.936702in}{1.543384in}}%
\pgfpathlineto{\pgfqpoint{2.937085in}{1.563756in}}%
\pgfpathlineto{\pgfqpoint{2.937085in}{1.563756in}}%
\pgfpathlineto{\pgfqpoint{2.937085in}{1.563756in}}%
\pgfpathlineto{\pgfqpoint{2.937468in}{1.535235in}}%
\pgfpathlineto{\pgfqpoint{2.938234in}{1.547458in}}%
\pgfpathlineto{\pgfqpoint{2.939769in}{1.555607in}}%
\pgfpathlineto{\pgfqpoint{2.940535in}{1.543384in}}%
\pgfpathlineto{\pgfqpoint{2.941303in}{1.559681in}}%
\pgfpathlineto{\pgfqpoint{2.942070in}{1.539309in}}%
\pgfpathlineto{\pgfqpoint{2.942453in}{1.559681in}}%
\pgfpathlineto{\pgfqpoint{2.943219in}{1.547458in}}%
\pgfpathlineto{\pgfqpoint{2.943602in}{1.535235in}}%
\pgfpathlineto{\pgfqpoint{2.944370in}{1.543384in}}%
\pgfpathlineto{\pgfqpoint{2.944753in}{1.547458in}}%
\pgfpathlineto{\pgfqpoint{2.945136in}{1.531161in}}%
\pgfpathlineto{\pgfqpoint{2.945902in}{1.539309in}}%
\pgfpathlineto{\pgfqpoint{2.946285in}{1.551532in}}%
\pgfpathlineto{\pgfqpoint{2.946669in}{1.547458in}}%
\pgfpathlineto{\pgfqpoint{2.947052in}{1.527086in}}%
\pgfpathlineto{\pgfqpoint{2.947052in}{1.527086in}}%
\pgfpathlineto{\pgfqpoint{2.947052in}{1.527086in}}%
\pgfpathlineto{\pgfqpoint{2.948202in}{1.555607in}}%
\pgfpathlineto{\pgfqpoint{2.948585in}{1.527086in}}%
\pgfpathlineto{\pgfqpoint{2.949352in}{1.543384in}}%
\pgfpathlineto{\pgfqpoint{2.949735in}{1.527086in}}%
\pgfpathlineto{\pgfqpoint{2.950119in}{1.535235in}}%
\pgfpathlineto{\pgfqpoint{2.950502in}{1.551532in}}%
\pgfpathlineto{\pgfqpoint{2.951269in}{1.547458in}}%
\pgfpathlineto{\pgfqpoint{2.951652in}{1.539309in}}%
\pgfpathlineto{\pgfqpoint{2.952036in}{1.543384in}}%
\pgfpathlineto{\pgfqpoint{2.952419in}{1.555607in}}%
\pgfpathlineto{\pgfqpoint{2.952802in}{1.551532in}}%
\pgfpathlineto{\pgfqpoint{2.953953in}{1.539309in}}%
\pgfpathlineto{\pgfqpoint{2.954336in}{1.563756in}}%
\pgfpathlineto{\pgfqpoint{2.954336in}{1.563756in}}%
\pgfpathlineto{\pgfqpoint{2.954336in}{1.563756in}}%
\pgfpathlineto{\pgfqpoint{2.954720in}{1.535235in}}%
\pgfpathlineto{\pgfqpoint{2.955488in}{1.555607in}}%
\pgfpathlineto{\pgfqpoint{2.955871in}{1.555607in}}%
\pgfpathlineto{\pgfqpoint{2.957022in}{1.559681in}}%
\pgfpathlineto{\pgfqpoint{2.957788in}{1.547458in}}%
\pgfpathlineto{\pgfqpoint{2.958172in}{1.551532in}}%
\pgfpathlineto{\pgfqpoint{2.958555in}{1.555607in}}%
\pgfpathlineto{\pgfqpoint{2.958937in}{1.527086in}}%
\pgfpathlineto{\pgfqpoint{2.959321in}{1.551532in}}%
\pgfpathlineto{\pgfqpoint{2.960088in}{1.555607in}}%
\pgfpathlineto{\pgfqpoint{2.960557in}{1.539309in}}%
\pgfpathlineto{\pgfqpoint{2.960557in}{1.539309in}}%
\pgfpathlineto{\pgfqpoint{2.960557in}{1.539309in}}%
\pgfpathlineto{\pgfqpoint{2.960940in}{1.563756in}}%
\pgfpathlineto{\pgfqpoint{2.961713in}{1.551532in}}%
\pgfpathlineto{\pgfqpoint{2.962865in}{1.543384in}}%
\pgfpathlineto{\pgfqpoint{2.963247in}{1.551532in}}%
\pgfpathlineto{\pgfqpoint{2.963631in}{1.543384in}}%
\pgfpathlineto{\pgfqpoint{2.964014in}{1.539309in}}%
\pgfpathlineto{\pgfqpoint{2.964396in}{1.551532in}}%
\pgfpathlineto{\pgfqpoint{2.964780in}{1.547458in}}%
\pgfpathlineto{\pgfqpoint{2.965164in}{1.539309in}}%
\pgfpathlineto{\pgfqpoint{2.965547in}{1.559681in}}%
\pgfpathlineto{\pgfqpoint{2.965930in}{1.539309in}}%
\pgfpathlineto{\pgfqpoint{2.966696in}{1.531161in}}%
\pgfpathlineto{\pgfqpoint{2.967080in}{1.547458in}}%
\pgfpathlineto{\pgfqpoint{2.967080in}{1.547458in}}%
\pgfpathlineto{\pgfqpoint{2.967080in}{1.547458in}}%
\pgfpathlineto{\pgfqpoint{2.967464in}{1.527086in}}%
\pgfpathlineto{\pgfqpoint{2.967847in}{1.535235in}}%
\pgfpathlineto{\pgfqpoint{2.968231in}{1.559681in}}%
\pgfpathlineto{\pgfqpoint{2.968614in}{1.551532in}}%
\pgfpathlineto{\pgfqpoint{2.968997in}{1.535235in}}%
\pgfpathlineto{\pgfqpoint{2.969764in}{1.539309in}}%
\pgfpathlineto{\pgfqpoint{2.970914in}{1.559681in}}%
\pgfpathlineto{\pgfqpoint{2.972063in}{1.531161in}}%
\pgfpathlineto{\pgfqpoint{2.972446in}{1.543384in}}%
\pgfpathlineto{\pgfqpoint{2.973213in}{1.563756in}}%
\pgfpathlineto{\pgfqpoint{2.973597in}{1.547458in}}%
\pgfpathlineto{\pgfqpoint{2.975129in}{1.535235in}}%
\pgfpathlineto{\pgfqpoint{2.977045in}{1.555607in}}%
\pgfpathlineto{\pgfqpoint{2.977428in}{1.535235in}}%
\pgfpathlineto{\pgfqpoint{2.977428in}{1.535235in}}%
\pgfpathlineto{\pgfqpoint{2.977428in}{1.535235in}}%
\pgfpathlineto{\pgfqpoint{2.977811in}{1.563756in}}%
\pgfpathlineto{\pgfqpoint{2.978578in}{1.559681in}}%
\pgfpathlineto{\pgfqpoint{2.979727in}{1.535235in}}%
\pgfpathlineto{\pgfqpoint{2.980493in}{1.559681in}}%
\pgfpathlineto{\pgfqpoint{2.980877in}{1.523012in}}%
\pgfpathlineto{\pgfqpoint{2.981643in}{1.547458in}}%
\pgfpathlineto{\pgfqpoint{2.982027in}{1.555607in}}%
\pgfpathlineto{\pgfqpoint{2.982409in}{1.535235in}}%
\pgfpathlineto{\pgfqpoint{2.982793in}{1.547458in}}%
\pgfpathlineto{\pgfqpoint{2.983559in}{1.563756in}}%
\pgfpathlineto{\pgfqpoint{2.983942in}{1.551532in}}%
\pgfpathlineto{\pgfqpoint{2.985093in}{1.535235in}}%
\pgfpathlineto{\pgfqpoint{2.985859in}{1.551532in}}%
\pgfpathlineto{\pgfqpoint{2.986242in}{1.547458in}}%
\pgfpathlineto{\pgfqpoint{2.986626in}{1.535235in}}%
\pgfpathlineto{\pgfqpoint{2.986626in}{1.535235in}}%
\pgfpathlineto{\pgfqpoint{2.986626in}{1.535235in}}%
\pgfpathlineto{\pgfqpoint{2.987394in}{1.559681in}}%
\pgfpathlineto{\pgfqpoint{2.987777in}{1.543384in}}%
\pgfpathlineto{\pgfqpoint{2.988160in}{1.543384in}}%
\pgfpathlineto{\pgfqpoint{2.989309in}{1.555607in}}%
\pgfpathlineto{\pgfqpoint{2.989693in}{1.547458in}}%
\pgfpathlineto{\pgfqpoint{2.990076in}{1.555607in}}%
\pgfpathlineto{\pgfqpoint{2.990459in}{1.555607in}}%
\pgfpathlineto{\pgfqpoint{2.990842in}{1.551532in}}%
\pgfpathlineto{\pgfqpoint{2.991225in}{1.531161in}}%
\pgfpathlineto{\pgfqpoint{2.991225in}{1.531161in}}%
\pgfpathlineto{\pgfqpoint{2.991225in}{1.531161in}}%
\pgfpathlineto{\pgfqpoint{2.991992in}{1.559681in}}%
\pgfpathlineto{\pgfqpoint{2.992375in}{1.543384in}}%
\pgfpathlineto{\pgfqpoint{2.992756in}{1.559681in}}%
\pgfpathlineto{\pgfqpoint{2.993138in}{1.543384in}}%
\pgfpathlineto{\pgfqpoint{2.993521in}{1.539309in}}%
\pgfpathlineto{\pgfqpoint{2.993903in}{1.543384in}}%
\pgfpathlineto{\pgfqpoint{2.994668in}{1.547458in}}%
\pgfpathlineto{\pgfqpoint{2.995817in}{1.531161in}}%
\pgfpathlineto{\pgfqpoint{2.996200in}{1.563756in}}%
\pgfpathlineto{\pgfqpoint{2.996965in}{1.543384in}}%
\pgfpathlineto{\pgfqpoint{2.997433in}{1.547458in}}%
\pgfpathlineto{\pgfqpoint{2.997815in}{1.563756in}}%
\pgfpathlineto{\pgfqpoint{2.998579in}{1.559681in}}%
\pgfpathlineto{\pgfqpoint{2.998962in}{1.555607in}}%
\pgfpathlineto{\pgfqpoint{2.999345in}{1.539309in}}%
\pgfpathlineto{\pgfqpoint{3.000110in}{1.543384in}}%
\pgfpathlineto{\pgfqpoint{3.000876in}{1.551532in}}%
\pgfpathlineto{\pgfqpoint{3.002033in}{1.527086in}}%
\pgfpathlineto{\pgfqpoint{3.003180in}{1.567830in}}%
\pgfpathlineto{\pgfqpoint{3.003563in}{1.559681in}}%
\pgfpathlineto{\pgfqpoint{3.003945in}{1.559681in}}%
\pgfpathlineto{\pgfqpoint{3.004709in}{1.547458in}}%
\pgfpathlineto{\pgfqpoint{3.005092in}{1.559681in}}%
\pgfpathlineto{\pgfqpoint{3.005474in}{1.551532in}}%
\pgfpathlineto{\pgfqpoint{3.005858in}{1.547458in}}%
\pgfpathlineto{\pgfqpoint{3.006240in}{1.563756in}}%
\pgfpathlineto{\pgfqpoint{3.006623in}{1.551532in}}%
\pgfpathlineto{\pgfqpoint{3.007388in}{1.535235in}}%
\pgfpathlineto{\pgfqpoint{3.008154in}{1.555607in}}%
\pgfpathlineto{\pgfqpoint{3.008920in}{1.510789in}}%
\pgfpathlineto{\pgfqpoint{3.009302in}{1.531161in}}%
\pgfpathlineto{\pgfqpoint{3.010450in}{1.555607in}}%
\pgfpathlineto{\pgfqpoint{3.010833in}{1.543384in}}%
\pgfpathlineto{\pgfqpoint{3.011216in}{1.543384in}}%
\pgfpathlineto{\pgfqpoint{3.011600in}{1.535235in}}%
\pgfpathlineto{\pgfqpoint{3.011984in}{1.543384in}}%
\pgfpathlineto{\pgfqpoint{3.012367in}{1.551532in}}%
\pgfpathlineto{\pgfqpoint{3.012367in}{1.551532in}}%
\pgfpathlineto{\pgfqpoint{3.012367in}{1.551532in}}%
\pgfpathlineto{\pgfqpoint{3.013133in}{1.531161in}}%
\pgfpathlineto{\pgfqpoint{3.013515in}{1.555607in}}%
\pgfpathlineto{\pgfqpoint{3.014281in}{1.547458in}}%
\pgfpathlineto{\pgfqpoint{3.015432in}{1.523012in}}%
\pgfpathlineto{\pgfqpoint{3.016197in}{1.563756in}}%
\pgfpathlineto{\pgfqpoint{3.016963in}{1.555607in}}%
\pgfpathlineto{\pgfqpoint{3.017347in}{1.518937in}}%
\pgfpathlineto{\pgfqpoint{3.018113in}{1.539309in}}%
\pgfpathlineto{\pgfqpoint{3.019262in}{1.555607in}}%
\pgfpathlineto{\pgfqpoint{3.020412in}{1.547458in}}%
\pgfpathlineto{\pgfqpoint{3.021179in}{1.551532in}}%
\pgfpathlineto{\pgfqpoint{3.021945in}{1.535235in}}%
\pgfpathlineto{\pgfqpoint{3.022328in}{1.555607in}}%
\pgfpathlineto{\pgfqpoint{3.023094in}{1.539309in}}%
\pgfpathlineto{\pgfqpoint{3.024626in}{1.563756in}}%
\pgfpathlineto{\pgfqpoint{3.025477in}{1.543384in}}%
\pgfpathlineto{\pgfqpoint{3.025860in}{1.551532in}}%
\pgfpathlineto{\pgfqpoint{3.026243in}{1.567830in}}%
\pgfpathlineto{\pgfqpoint{3.026627in}{1.551532in}}%
\pgfpathlineto{\pgfqpoint{3.027010in}{1.551532in}}%
\pgfpathlineto{\pgfqpoint{3.027393in}{1.531161in}}%
\pgfpathlineto{\pgfqpoint{3.027393in}{1.531161in}}%
\pgfpathlineto{\pgfqpoint{3.027393in}{1.531161in}}%
\pgfpathlineto{\pgfqpoint{3.028543in}{1.571904in}}%
\pgfpathlineto{\pgfqpoint{3.029694in}{1.535235in}}%
\pgfpathlineto{\pgfqpoint{3.030459in}{1.559681in}}%
\pgfpathlineto{\pgfqpoint{3.030843in}{1.535235in}}%
\pgfpathlineto{\pgfqpoint{3.031610in}{1.547458in}}%
\pgfpathlineto{\pgfqpoint{3.031993in}{1.547458in}}%
\pgfpathlineto{\pgfqpoint{3.032377in}{1.531161in}}%
\pgfpathlineto{\pgfqpoint{3.032377in}{1.531161in}}%
\pgfpathlineto{\pgfqpoint{3.032377in}{1.531161in}}%
\pgfpathlineto{\pgfqpoint{3.032760in}{1.555607in}}%
\pgfpathlineto{\pgfqpoint{3.032760in}{1.555607in}}%
\pgfpathlineto{\pgfqpoint{3.032760in}{1.555607in}}%
\pgfpathlineto{\pgfqpoint{3.033143in}{1.527086in}}%
\pgfpathlineto{\pgfqpoint{3.033909in}{1.535235in}}%
\pgfpathlineto{\pgfqpoint{3.035443in}{1.563756in}}%
\pgfpathlineto{\pgfqpoint{3.036593in}{1.527086in}}%
\pgfpathlineto{\pgfqpoint{3.037360in}{1.551532in}}%
\pgfpathlineto{\pgfqpoint{3.037743in}{1.535235in}}%
\pgfpathlineto{\pgfqpoint{3.038893in}{1.563756in}}%
\pgfpathlineto{\pgfqpoint{3.039658in}{1.531161in}}%
\pgfpathlineto{\pgfqpoint{3.040041in}{1.547458in}}%
\pgfpathlineto{\pgfqpoint{3.040425in}{1.567830in}}%
\pgfpathlineto{\pgfqpoint{3.041191in}{1.551532in}}%
\pgfpathlineto{\pgfqpoint{3.041581in}{1.551532in}}%
\pgfpathlineto{\pgfqpoint{3.041965in}{1.543384in}}%
\pgfpathlineto{\pgfqpoint{3.042350in}{1.547458in}}%
\pgfpathlineto{\pgfqpoint{3.042733in}{1.563756in}}%
\pgfpathlineto{\pgfqpoint{3.042733in}{1.563756in}}%
\pgfpathlineto{\pgfqpoint{3.042733in}{1.563756in}}%
\pgfpathlineto{\pgfqpoint{3.043499in}{1.535235in}}%
\pgfpathlineto{\pgfqpoint{3.043882in}{1.539309in}}%
\pgfpathlineto{\pgfqpoint{3.044266in}{1.563756in}}%
\pgfpathlineto{\pgfqpoint{3.045033in}{1.543384in}}%
\pgfpathlineto{\pgfqpoint{3.045417in}{1.555607in}}%
\pgfpathlineto{\pgfqpoint{3.045417in}{1.555607in}}%
\pgfpathlineto{\pgfqpoint{3.045417in}{1.555607in}}%
\pgfpathlineto{\pgfqpoint{3.046566in}{1.535235in}}%
\pgfpathlineto{\pgfqpoint{3.048099in}{1.555607in}}%
\pgfpathlineto{\pgfqpoint{3.048482in}{1.563756in}}%
\pgfpathlineto{\pgfqpoint{3.048482in}{1.563756in}}%
\pgfpathlineto{\pgfqpoint{3.048482in}{1.563756in}}%
\pgfpathlineto{\pgfqpoint{3.049249in}{1.543384in}}%
\pgfpathlineto{\pgfqpoint{3.049631in}{1.551532in}}%
\pgfpathlineto{\pgfqpoint{3.050014in}{1.555607in}}%
\pgfpathlineto{\pgfqpoint{3.050398in}{1.531161in}}%
\pgfpathlineto{\pgfqpoint{3.051165in}{1.539309in}}%
\pgfpathlineto{\pgfqpoint{3.052315in}{1.559681in}}%
\pgfpathlineto{\pgfqpoint{3.052698in}{1.555607in}}%
\pgfpathlineto{\pgfqpoint{3.053081in}{1.531161in}}%
\pgfpathlineto{\pgfqpoint{3.053847in}{1.539309in}}%
\pgfpathlineto{\pgfqpoint{3.054613in}{1.551532in}}%
\pgfpathlineto{\pgfqpoint{3.054996in}{1.547458in}}%
\pgfpathlineto{\pgfqpoint{3.055379in}{1.535235in}}%
\pgfpathlineto{\pgfqpoint{3.055763in}{1.547458in}}%
\pgfpathlineto{\pgfqpoint{3.056146in}{1.563756in}}%
\pgfpathlineto{\pgfqpoint{3.056911in}{1.551532in}}%
\pgfpathlineto{\pgfqpoint{3.058061in}{1.535235in}}%
\pgfpathlineto{\pgfqpoint{3.058827in}{1.547458in}}%
\pgfpathlineto{\pgfqpoint{3.059209in}{1.531161in}}%
\pgfpathlineto{\pgfqpoint{3.059209in}{1.531161in}}%
\pgfpathlineto{\pgfqpoint{3.059209in}{1.531161in}}%
\pgfpathlineto{\pgfqpoint{3.059592in}{1.559681in}}%
\pgfpathlineto{\pgfqpoint{3.060359in}{1.551532in}}%
\pgfpathlineto{\pgfqpoint{3.061509in}{1.531161in}}%
\pgfpathlineto{\pgfqpoint{3.063425in}{1.555607in}}%
\pgfpathlineto{\pgfqpoint{3.063808in}{1.555607in}}%
\pgfpathlineto{\pgfqpoint{3.064192in}{1.535235in}}%
\pgfpathlineto{\pgfqpoint{3.064959in}{1.547458in}}%
\pgfpathlineto{\pgfqpoint{3.065428in}{1.563756in}}%
\pgfpathlineto{\pgfqpoint{3.065428in}{1.563756in}}%
\pgfpathlineto{\pgfqpoint{3.065428in}{1.563756in}}%
\pgfpathlineto{\pgfqpoint{3.065811in}{1.539309in}}%
\pgfpathlineto{\pgfqpoint{3.066576in}{1.559681in}}%
\pgfpathlineto{\pgfqpoint{3.066960in}{1.559681in}}%
\pgfpathlineto{\pgfqpoint{3.067343in}{1.539309in}}%
\pgfpathlineto{\pgfqpoint{3.068110in}{1.551532in}}%
\pgfpathlineto{\pgfqpoint{3.069258in}{1.523012in}}%
\pgfpathlineto{\pgfqpoint{3.070791in}{1.567830in}}%
\pgfpathlineto{\pgfqpoint{3.071174in}{1.535235in}}%
\pgfpathlineto{\pgfqpoint{3.071940in}{1.551532in}}%
\pgfpathlineto{\pgfqpoint{3.072706in}{1.543384in}}%
\pgfpathlineto{\pgfqpoint{3.073088in}{1.551532in}}%
\pgfpathlineto{\pgfqpoint{3.073857in}{1.547458in}}%
\pgfpathlineto{\pgfqpoint{3.074623in}{1.535235in}}%
\pgfpathlineto{\pgfqpoint{3.075006in}{1.539309in}}%
\pgfpathlineto{\pgfqpoint{3.075389in}{1.563756in}}%
\pgfpathlineto{\pgfqpoint{3.076156in}{1.551532in}}%
\pgfpathlineto{\pgfqpoint{3.076540in}{1.543384in}}%
\pgfpathlineto{\pgfqpoint{3.076921in}{1.551532in}}%
\pgfpathlineto{\pgfqpoint{3.077687in}{1.559681in}}%
\pgfpathlineto{\pgfqpoint{3.078453in}{1.535235in}}%
\pgfpathlineto{\pgfqpoint{3.079219in}{1.543384in}}%
\pgfpathlineto{\pgfqpoint{3.079602in}{1.543384in}}%
\pgfpathlineto{\pgfqpoint{3.081133in}{1.559681in}}%
\pgfpathlineto{\pgfqpoint{3.082290in}{1.527086in}}%
\pgfpathlineto{\pgfqpoint{3.082673in}{1.539309in}}%
\pgfpathlineto{\pgfqpoint{3.084206in}{1.563756in}}%
\pgfpathlineto{\pgfqpoint{3.084972in}{1.543384in}}%
\pgfpathlineto{\pgfqpoint{3.085356in}{1.547458in}}%
\pgfpathlineto{\pgfqpoint{3.086503in}{1.563756in}}%
\pgfpathlineto{\pgfqpoint{3.088037in}{1.535235in}}%
\pgfpathlineto{\pgfqpoint{3.088803in}{1.559681in}}%
\pgfpathlineto{\pgfqpoint{3.089186in}{1.535235in}}%
\pgfpathlineto{\pgfqpoint{3.089953in}{1.543384in}}%
\pgfpathlineto{\pgfqpoint{3.090336in}{1.535235in}}%
\pgfpathlineto{\pgfqpoint{3.090719in}{1.539309in}}%
\pgfpathlineto{\pgfqpoint{3.091486in}{1.543384in}}%
\pgfpathlineto{\pgfqpoint{3.091869in}{1.531161in}}%
\pgfpathlineto{\pgfqpoint{3.091869in}{1.531161in}}%
\pgfpathlineto{\pgfqpoint{3.091869in}{1.531161in}}%
\pgfpathlineto{\pgfqpoint{3.092636in}{1.555607in}}%
\pgfpathlineto{\pgfqpoint{3.093019in}{1.531161in}}%
\pgfpathlineto{\pgfqpoint{3.093402in}{1.535235in}}%
\pgfpathlineto{\pgfqpoint{3.093784in}{1.555607in}}%
\pgfpathlineto{\pgfqpoint{3.094551in}{1.547458in}}%
\pgfpathlineto{\pgfqpoint{3.095702in}{1.531161in}}%
\pgfpathlineto{\pgfqpoint{3.096851in}{1.563756in}}%
\pgfpathlineto{\pgfqpoint{3.097999in}{1.551532in}}%
\pgfpathlineto{\pgfqpoint{3.098382in}{1.555607in}}%
\pgfpathlineto{\pgfqpoint{3.098766in}{1.527086in}}%
\pgfpathlineto{\pgfqpoint{3.099148in}{1.551532in}}%
\pgfpathlineto{\pgfqpoint{3.099617in}{1.547458in}}%
\pgfpathlineto{\pgfqpoint{3.100000in}{1.571904in}}%
\pgfpathlineto{\pgfqpoint{3.100384in}{1.547458in}}%
\pgfpathlineto{\pgfqpoint{3.100768in}{1.535235in}}%
\pgfpathlineto{\pgfqpoint{3.100768in}{1.535235in}}%
\pgfpathlineto{\pgfqpoint{3.100768in}{1.535235in}}%
\pgfpathlineto{\pgfqpoint{3.101152in}{1.551532in}}%
\pgfpathlineto{\pgfqpoint{3.101535in}{1.539309in}}%
\pgfpathlineto{\pgfqpoint{3.101919in}{1.531161in}}%
\pgfpathlineto{\pgfqpoint{3.102302in}{1.539309in}}%
\pgfpathlineto{\pgfqpoint{3.102685in}{1.539309in}}%
\pgfpathlineto{\pgfqpoint{3.103068in}{1.547458in}}%
\pgfpathlineto{\pgfqpoint{3.103834in}{1.543384in}}%
\pgfpathlineto{\pgfqpoint{3.104218in}{1.539309in}}%
\pgfpathlineto{\pgfqpoint{3.104602in}{1.547458in}}%
\pgfpathlineto{\pgfqpoint{3.104602in}{1.547458in}}%
\pgfpathlineto{\pgfqpoint{3.104602in}{1.547458in}}%
\pgfpathlineto{\pgfqpoint{3.105751in}{1.531161in}}%
\pgfpathlineto{\pgfqpoint{3.107668in}{1.551532in}}%
\pgfpathlineto{\pgfqpoint{3.108052in}{1.527086in}}%
\pgfpathlineto{\pgfqpoint{3.108817in}{1.543384in}}%
\pgfpathlineto{\pgfqpoint{3.109201in}{1.531161in}}%
\pgfpathlineto{\pgfqpoint{3.109584in}{1.543384in}}%
\pgfpathlineto{\pgfqpoint{3.110352in}{1.543384in}}%
\pgfpathlineto{\pgfqpoint{3.111119in}{1.563756in}}%
\pgfpathlineto{\pgfqpoint{3.111501in}{1.535235in}}%
\pgfpathlineto{\pgfqpoint{3.112268in}{1.551532in}}%
\pgfpathlineto{\pgfqpoint{3.113417in}{1.535235in}}%
\pgfpathlineto{\pgfqpoint{3.114184in}{1.559681in}}%
\pgfpathlineto{\pgfqpoint{3.114568in}{1.531161in}}%
\pgfpathlineto{\pgfqpoint{3.115335in}{1.555607in}}%
\pgfpathlineto{\pgfqpoint{3.116102in}{1.531161in}}%
\pgfpathlineto{\pgfqpoint{3.116485in}{1.551532in}}%
\pgfpathlineto{\pgfqpoint{3.116869in}{1.555607in}}%
\pgfpathlineto{\pgfqpoint{3.117252in}{1.535235in}}%
\pgfpathlineto{\pgfqpoint{3.118019in}{1.543384in}}%
\pgfpathlineto{\pgfqpoint{3.118402in}{1.543384in}}%
\pgfpathlineto{\pgfqpoint{3.118785in}{1.539309in}}%
\pgfpathlineto{\pgfqpoint{3.119934in}{1.551532in}}%
\pgfpathlineto{\pgfqpoint{3.120317in}{1.543384in}}%
\pgfpathlineto{\pgfqpoint{3.120701in}{1.547458in}}%
\pgfpathlineto{\pgfqpoint{3.121858in}{1.567830in}}%
\pgfpathlineto{\pgfqpoint{3.122624in}{1.531161in}}%
\pgfpathlineto{\pgfqpoint{3.123391in}{1.535235in}}%
\pgfpathlineto{\pgfqpoint{3.123775in}{1.567830in}}%
\pgfpathlineto{\pgfqpoint{3.123775in}{1.567830in}}%
\pgfpathlineto{\pgfqpoint{3.123775in}{1.567830in}}%
\pgfpathlineto{\pgfqpoint{3.124158in}{1.527086in}}%
\pgfpathlineto{\pgfqpoint{3.124924in}{1.547458in}}%
\pgfpathlineto{\pgfqpoint{3.125308in}{1.555607in}}%
\pgfpathlineto{\pgfqpoint{3.125690in}{1.551532in}}%
\pgfpathlineto{\pgfqpoint{3.126074in}{1.547458in}}%
\pgfpathlineto{\pgfqpoint{3.126457in}{1.571904in}}%
\pgfpathlineto{\pgfqpoint{3.127223in}{1.559681in}}%
\pgfpathlineto{\pgfqpoint{3.127990in}{1.539309in}}%
\pgfpathlineto{\pgfqpoint{3.128374in}{1.559681in}}%
\pgfpathlineto{\pgfqpoint{3.129140in}{1.543384in}}%
\pgfpathlineto{\pgfqpoint{3.129523in}{1.543384in}}%
\pgfpathlineto{\pgfqpoint{3.130673in}{1.563756in}}%
\pgfpathlineto{\pgfqpoint{3.131822in}{1.543384in}}%
\pgfpathlineto{\pgfqpoint{3.132205in}{1.559681in}}%
\pgfpathlineto{\pgfqpoint{3.132205in}{1.559681in}}%
\pgfpathlineto{\pgfqpoint{3.132205in}{1.559681in}}%
\pgfpathlineto{\pgfqpoint{3.132585in}{1.535235in}}%
\pgfpathlineto{\pgfqpoint{3.133352in}{1.539309in}}%
\pgfpathlineto{\pgfqpoint{3.134119in}{1.547458in}}%
\pgfpathlineto{\pgfqpoint{3.134502in}{1.531161in}}%
\pgfpathlineto{\pgfqpoint{3.134885in}{1.563756in}}%
\pgfpathlineto{\pgfqpoint{3.135652in}{1.547458in}}%
\pgfpathlineto{\pgfqpoint{3.136035in}{1.547458in}}%
\pgfpathlineto{\pgfqpoint{3.136419in}{1.543384in}}%
\pgfpathlineto{\pgfqpoint{3.136803in}{1.551532in}}%
\pgfpathlineto{\pgfqpoint{3.137569in}{1.547458in}}%
\pgfpathlineto{\pgfqpoint{3.138717in}{1.547458in}}%
\pgfpathlineto{\pgfqpoint{3.139100in}{1.543384in}}%
\pgfpathlineto{\pgfqpoint{3.140336in}{1.567830in}}%
\pgfpathlineto{\pgfqpoint{3.141485in}{1.543384in}}%
\pgfpathlineto{\pgfqpoint{3.141868in}{1.547458in}}%
\pgfpathlineto{\pgfqpoint{3.143019in}{1.563756in}}%
\pgfpathlineto{\pgfqpoint{3.144167in}{1.539309in}}%
\pgfpathlineto{\pgfqpoint{3.144934in}{1.543384in}}%
\pgfpathlineto{\pgfqpoint{3.145317in}{1.547458in}}%
\pgfpathlineto{\pgfqpoint{3.145701in}{1.543384in}}%
\pgfpathlineto{\pgfqpoint{3.146083in}{1.539309in}}%
\pgfpathlineto{\pgfqpoint{3.147615in}{1.563756in}}%
\pgfpathlineto{\pgfqpoint{3.148766in}{1.518937in}}%
\pgfpathlineto{\pgfqpoint{3.149533in}{1.555607in}}%
\pgfpathlineto{\pgfqpoint{3.149916in}{1.535235in}}%
\pgfpathlineto{\pgfqpoint{3.151451in}{1.555607in}}%
\pgfpathlineto{\pgfqpoint{3.151834in}{1.535235in}}%
\pgfpathlineto{\pgfqpoint{3.152601in}{1.539309in}}%
\pgfpathlineto{\pgfqpoint{3.152984in}{1.539309in}}%
\pgfpathlineto{\pgfqpoint{3.153368in}{1.523012in}}%
\pgfpathlineto{\pgfqpoint{3.153750in}{1.527086in}}%
\pgfpathlineto{\pgfqpoint{3.154518in}{1.559681in}}%
\pgfpathlineto{\pgfqpoint{3.154901in}{1.539309in}}%
\pgfpathlineto{\pgfqpoint{3.155284in}{1.535235in}}%
\pgfpathlineto{\pgfqpoint{3.155668in}{1.563756in}}%
\pgfpathlineto{\pgfqpoint{3.156435in}{1.543384in}}%
\pgfpathlineto{\pgfqpoint{3.157201in}{1.547458in}}%
\pgfpathlineto{\pgfqpoint{3.157584in}{1.539309in}}%
\pgfpathlineto{\pgfqpoint{3.157967in}{1.555607in}}%
\pgfpathlineto{\pgfqpoint{3.157967in}{1.555607in}}%
\pgfpathlineto{\pgfqpoint{3.157967in}{1.555607in}}%
\pgfpathlineto{\pgfqpoint{3.158350in}{1.535235in}}%
\pgfpathlineto{\pgfqpoint{3.159117in}{1.547458in}}%
\pgfpathlineto{\pgfqpoint{3.159883in}{1.555607in}}%
\pgfpathlineto{\pgfqpoint{3.160266in}{1.535235in}}%
\pgfpathlineto{\pgfqpoint{3.161032in}{1.543384in}}%
\pgfpathlineto{\pgfqpoint{3.161422in}{1.535235in}}%
\pgfpathlineto{\pgfqpoint{3.161807in}{1.539309in}}%
\pgfpathlineto{\pgfqpoint{3.162957in}{1.551532in}}%
\pgfpathlineto{\pgfqpoint{3.163341in}{1.535235in}}%
\pgfpathlineto{\pgfqpoint{3.163724in}{1.547458in}}%
\pgfpathlineto{\pgfqpoint{3.164108in}{1.551532in}}%
\pgfpathlineto{\pgfqpoint{3.165257in}{1.543384in}}%
\pgfpathlineto{\pgfqpoint{3.165640in}{1.551532in}}%
\pgfpathlineto{\pgfqpoint{3.166023in}{1.547458in}}%
\pgfpathlineto{\pgfqpoint{3.166789in}{1.539309in}}%
\pgfpathlineto{\pgfqpoint{3.167939in}{1.555607in}}%
\pgfpathlineto{\pgfqpoint{3.169473in}{1.539309in}}%
\pgfpathlineto{\pgfqpoint{3.169857in}{1.535235in}}%
\pgfpathlineto{\pgfqpoint{3.170623in}{1.555607in}}%
\pgfpathlineto{\pgfqpoint{3.171005in}{1.539309in}}%
\pgfpathlineto{\pgfqpoint{3.171389in}{1.539309in}}%
\pgfpathlineto{\pgfqpoint{3.172923in}{1.559681in}}%
\pgfpathlineto{\pgfqpoint{3.173306in}{1.535235in}}%
\pgfpathlineto{\pgfqpoint{3.174073in}{1.543384in}}%
\pgfpathlineto{\pgfqpoint{3.175223in}{1.555607in}}%
\pgfpathlineto{\pgfqpoint{3.176755in}{1.535235in}}%
\pgfpathlineto{\pgfqpoint{3.177522in}{1.559681in}}%
\pgfpathlineto{\pgfqpoint{3.177905in}{1.543384in}}%
\pgfpathlineto{\pgfqpoint{3.178289in}{1.551532in}}%
\pgfpathlineto{\pgfqpoint{3.179055in}{1.547458in}}%
\pgfpathlineto{\pgfqpoint{3.179822in}{1.535235in}}%
\pgfpathlineto{\pgfqpoint{3.180205in}{1.555607in}}%
\pgfpathlineto{\pgfqpoint{3.180972in}{1.539309in}}%
\pgfpathlineto{\pgfqpoint{3.182206in}{1.547458in}}%
\pgfpathlineto{\pgfqpoint{3.182589in}{1.535235in}}%
\pgfpathlineto{\pgfqpoint{3.182589in}{1.535235in}}%
\pgfpathlineto{\pgfqpoint{3.182589in}{1.535235in}}%
\pgfpathlineto{\pgfqpoint{3.182973in}{1.551532in}}%
\pgfpathlineto{\pgfqpoint{3.183741in}{1.543384in}}%
\pgfpathlineto{\pgfqpoint{3.184507in}{1.551532in}}%
\pgfpathlineto{\pgfqpoint{3.184890in}{1.547458in}}%
\pgfpathlineto{\pgfqpoint{3.185274in}{1.547458in}}%
\pgfpathlineto{\pgfqpoint{3.185658in}{1.551532in}}%
\pgfpathlineto{\pgfqpoint{3.186039in}{1.547458in}}%
\pgfpathlineto{\pgfqpoint{3.186422in}{1.547458in}}%
\pgfpathlineto{\pgfqpoint{3.186805in}{1.567830in}}%
\pgfpathlineto{\pgfqpoint{3.187188in}{1.523012in}}%
\pgfpathlineto{\pgfqpoint{3.187955in}{1.543384in}}%
\pgfpathlineto{\pgfqpoint{3.189106in}{1.559681in}}%
\pgfpathlineto{\pgfqpoint{3.189488in}{1.535235in}}%
\pgfpathlineto{\pgfqpoint{3.190265in}{1.555607in}}%
\pgfpathlineto{\pgfqpoint{3.191038in}{1.543384in}}%
\pgfpathlineto{\pgfqpoint{3.191422in}{1.547458in}}%
\pgfpathlineto{\pgfqpoint{3.192188in}{1.539309in}}%
\pgfpathlineto{\pgfqpoint{3.192572in}{1.543384in}}%
\pgfpathlineto{\pgfqpoint{3.192956in}{1.567830in}}%
\pgfpathlineto{\pgfqpoint{3.193722in}{1.555607in}}%
\pgfpathlineto{\pgfqpoint{3.194105in}{1.543384in}}%
\pgfpathlineto{\pgfqpoint{3.194105in}{1.543384in}}%
\pgfpathlineto{\pgfqpoint{3.194105in}{1.543384in}}%
\pgfpathlineto{\pgfqpoint{3.194489in}{1.559681in}}%
\pgfpathlineto{\pgfqpoint{3.194873in}{1.543384in}}%
\pgfpathlineto{\pgfqpoint{3.195640in}{1.543384in}}%
\pgfpathlineto{\pgfqpoint{3.196023in}{1.555607in}}%
\pgfpathlineto{\pgfqpoint{3.196791in}{1.551532in}}%
\pgfpathlineto{\pgfqpoint{3.197558in}{1.535235in}}%
\pgfpathlineto{\pgfqpoint{3.197941in}{1.559681in}}%
\pgfpathlineto{\pgfqpoint{3.198706in}{1.547458in}}%
\pgfpathlineto{\pgfqpoint{3.199090in}{1.543384in}}%
\pgfpathlineto{\pgfqpoint{3.199473in}{1.551532in}}%
\pgfpathlineto{\pgfqpoint{3.199857in}{1.547458in}}%
\pgfpathlineto{\pgfqpoint{3.200240in}{1.531161in}}%
\pgfpathlineto{\pgfqpoint{3.200240in}{1.531161in}}%
\pgfpathlineto{\pgfqpoint{3.200240in}{1.531161in}}%
\pgfpathlineto{\pgfqpoint{3.201008in}{1.559681in}}%
\pgfpathlineto{\pgfqpoint{3.201401in}{1.555607in}}%
\pgfpathlineto{\pgfqpoint{3.201785in}{1.535235in}}%
\pgfpathlineto{\pgfqpoint{3.202168in}{1.543384in}}%
\pgfpathlineto{\pgfqpoint{3.202551in}{1.563756in}}%
\pgfpathlineto{\pgfqpoint{3.203316in}{1.547458in}}%
\pgfpathlineto{\pgfqpoint{3.203700in}{1.539309in}}%
\pgfpathlineto{\pgfqpoint{3.204083in}{1.563756in}}%
\pgfpathlineto{\pgfqpoint{3.204850in}{1.555607in}}%
\pgfpathlineto{\pgfqpoint{3.205999in}{1.531161in}}%
\pgfpathlineto{\pgfqpoint{3.206383in}{1.559681in}}%
\pgfpathlineto{\pgfqpoint{3.207149in}{1.547458in}}%
\pgfpathlineto{\pgfqpoint{3.207532in}{1.555607in}}%
\pgfpathlineto{\pgfqpoint{3.208681in}{1.535235in}}%
\pgfpathlineto{\pgfqpoint{3.209832in}{1.551532in}}%
\pgfpathlineto{\pgfqpoint{3.210215in}{1.539309in}}%
\pgfpathlineto{\pgfqpoint{3.210981in}{1.543384in}}%
\pgfpathlineto{\pgfqpoint{3.211748in}{1.559681in}}%
\pgfpathlineto{\pgfqpoint{3.212131in}{1.531161in}}%
\pgfpathlineto{\pgfqpoint{3.212898in}{1.535235in}}%
\pgfpathlineto{\pgfqpoint{3.213281in}{1.539309in}}%
\pgfpathlineto{\pgfqpoint{3.213665in}{1.514863in}}%
\pgfpathlineto{\pgfqpoint{3.213665in}{1.514863in}}%
\pgfpathlineto{\pgfqpoint{3.213665in}{1.514863in}}%
\pgfpathlineto{\pgfqpoint{3.214431in}{1.563756in}}%
\pgfpathlineto{\pgfqpoint{3.215199in}{1.559681in}}%
\pgfpathlineto{\pgfqpoint{3.215582in}{1.555607in}}%
\pgfpathlineto{\pgfqpoint{3.215965in}{1.531161in}}%
\pgfpathlineto{\pgfqpoint{3.216731in}{1.535235in}}%
\pgfpathlineto{\pgfqpoint{3.217114in}{1.547458in}}%
\pgfpathlineto{\pgfqpoint{3.217882in}{1.539309in}}%
\pgfpathlineto{\pgfqpoint{3.218648in}{1.559681in}}%
\pgfpathlineto{\pgfqpoint{3.219032in}{1.551532in}}%
\pgfpathlineto{\pgfqpoint{3.219800in}{1.518937in}}%
\pgfpathlineto{\pgfqpoint{3.220567in}{1.527086in}}%
\pgfpathlineto{\pgfqpoint{3.221334in}{1.551532in}}%
\pgfpathlineto{\pgfqpoint{3.221716in}{1.539309in}}%
\pgfpathlineto{\pgfqpoint{3.223251in}{1.551532in}}%
\pgfpathlineto{\pgfqpoint{3.223634in}{1.539309in}}%
\pgfpathlineto{\pgfqpoint{3.223634in}{1.539309in}}%
\pgfpathlineto{\pgfqpoint{3.223634in}{1.539309in}}%
\pgfpathlineto{\pgfqpoint{3.224017in}{1.555607in}}%
\pgfpathlineto{\pgfqpoint{3.224017in}{1.555607in}}%
\pgfpathlineto{\pgfqpoint{3.224017in}{1.555607in}}%
\pgfpathlineto{\pgfqpoint{3.224400in}{1.527086in}}%
\pgfpathlineto{\pgfqpoint{3.225168in}{1.547458in}}%
\pgfpathlineto{\pgfqpoint{3.225933in}{1.559681in}}%
\pgfpathlineto{\pgfqpoint{3.226316in}{1.535235in}}%
\pgfpathlineto{\pgfqpoint{3.227082in}{1.547458in}}%
\pgfpathlineto{\pgfqpoint{3.227465in}{1.559681in}}%
\pgfpathlineto{\pgfqpoint{3.227849in}{1.547458in}}%
\pgfpathlineto{\pgfqpoint{3.228233in}{1.543384in}}%
\pgfpathlineto{\pgfqpoint{3.228616in}{1.563756in}}%
\pgfpathlineto{\pgfqpoint{3.228616in}{1.563756in}}%
\pgfpathlineto{\pgfqpoint{3.228616in}{1.563756in}}%
\pgfpathlineto{\pgfqpoint{3.229382in}{1.535235in}}%
\pgfpathlineto{\pgfqpoint{3.229765in}{1.547458in}}%
\pgfpathlineto{\pgfqpoint{3.230148in}{1.535235in}}%
\pgfpathlineto{\pgfqpoint{3.230148in}{1.535235in}}%
\pgfpathlineto{\pgfqpoint{3.230148in}{1.535235in}}%
\pgfpathlineto{\pgfqpoint{3.230531in}{1.555607in}}%
\pgfpathlineto{\pgfqpoint{3.231299in}{1.547458in}}%
\pgfpathlineto{\pgfqpoint{3.231682in}{1.547458in}}%
\pgfpathlineto{\pgfqpoint{3.232831in}{1.523012in}}%
\pgfpathlineto{\pgfqpoint{3.233982in}{1.567830in}}%
\pgfpathlineto{\pgfqpoint{3.234365in}{1.551532in}}%
\pgfpathlineto{\pgfqpoint{3.235217in}{1.539309in}}%
\pgfpathlineto{\pgfqpoint{3.235600in}{1.555607in}}%
\pgfpathlineto{\pgfqpoint{3.235983in}{1.543384in}}%
\pgfpathlineto{\pgfqpoint{3.236366in}{1.535235in}}%
\pgfpathlineto{\pgfqpoint{3.236366in}{1.535235in}}%
\pgfpathlineto{\pgfqpoint{3.236366in}{1.535235in}}%
\pgfpathlineto{\pgfqpoint{3.238283in}{1.575979in}}%
\pgfpathlineto{\pgfqpoint{3.239050in}{1.535235in}}%
\pgfpathlineto{\pgfqpoint{3.239816in}{1.555607in}}%
\pgfpathlineto{\pgfqpoint{3.241359in}{1.535235in}}%
\pgfpathlineto{\pgfqpoint{3.241743in}{1.543384in}}%
\pgfpathlineto{\pgfqpoint{3.242126in}{1.551532in}}%
\pgfpathlineto{\pgfqpoint{3.242893in}{1.531161in}}%
\pgfpathlineto{\pgfqpoint{3.243276in}{1.539309in}}%
\pgfpathlineto{\pgfqpoint{3.244044in}{1.539309in}}%
\pgfpathlineto{\pgfqpoint{3.245576in}{1.563756in}}%
\pgfpathlineto{\pgfqpoint{3.245959in}{1.559681in}}%
\pgfpathlineto{\pgfqpoint{3.246725in}{1.539309in}}%
\pgfpathlineto{\pgfqpoint{3.247109in}{1.547458in}}%
\pgfpathlineto{\pgfqpoint{3.248258in}{1.547458in}}%
\pgfpathlineto{\pgfqpoint{3.248641in}{1.539309in}}%
\pgfpathlineto{\pgfqpoint{3.248641in}{1.539309in}}%
\pgfpathlineto{\pgfqpoint{3.248641in}{1.539309in}}%
\pgfpathlineto{\pgfqpoint{3.249792in}{1.555607in}}%
\pgfpathlineto{\pgfqpoint{3.250175in}{1.551532in}}%
\pgfpathlineto{\pgfqpoint{3.251323in}{1.527086in}}%
\pgfpathlineto{\pgfqpoint{3.252092in}{1.555607in}}%
\pgfpathlineto{\pgfqpoint{3.252475in}{1.547458in}}%
\pgfpathlineto{\pgfqpoint{3.252857in}{1.543384in}}%
\pgfpathlineto{\pgfqpoint{3.253240in}{1.551532in}}%
\pgfpathlineto{\pgfqpoint{3.253240in}{1.551532in}}%
\pgfpathlineto{\pgfqpoint{3.253240in}{1.551532in}}%
\pgfpathlineto{\pgfqpoint{3.254007in}{1.531161in}}%
\pgfpathlineto{\pgfqpoint{3.254391in}{1.547458in}}%
\pgfpathlineto{\pgfqpoint{3.255158in}{1.531161in}}%
\pgfpathlineto{\pgfqpoint{3.255540in}{1.555607in}}%
\pgfpathlineto{\pgfqpoint{3.256307in}{1.539309in}}%
\pgfpathlineto{\pgfqpoint{3.256690in}{1.539309in}}%
\pgfpathlineto{\pgfqpoint{3.257074in}{1.563756in}}%
\pgfpathlineto{\pgfqpoint{3.257841in}{1.543384in}}%
\pgfpathlineto{\pgfqpoint{3.258224in}{1.535235in}}%
\pgfpathlineto{\pgfqpoint{3.258991in}{1.563756in}}%
\pgfpathlineto{\pgfqpoint{3.259373in}{1.559681in}}%
\pgfpathlineto{\pgfqpoint{3.260141in}{1.527086in}}%
\pgfpathlineto{\pgfqpoint{3.260524in}{1.539309in}}%
\pgfpathlineto{\pgfqpoint{3.260908in}{1.555607in}}%
\pgfpathlineto{\pgfqpoint{3.261291in}{1.551532in}}%
\pgfpathlineto{\pgfqpoint{3.262056in}{1.531161in}}%
\pgfpathlineto{\pgfqpoint{3.262440in}{1.547458in}}%
\pgfpathlineto{\pgfqpoint{3.262824in}{1.563756in}}%
\pgfpathlineto{\pgfqpoint{3.263590in}{1.551532in}}%
\pgfpathlineto{\pgfqpoint{3.263973in}{1.563756in}}%
\pgfpathlineto{\pgfqpoint{3.264357in}{1.527086in}}%
\pgfpathlineto{\pgfqpoint{3.265123in}{1.551532in}}%
\pgfpathlineto{\pgfqpoint{3.265507in}{1.555607in}}%
\pgfpathlineto{\pgfqpoint{3.265890in}{1.535235in}}%
\pgfpathlineto{\pgfqpoint{3.266273in}{1.555607in}}%
\pgfpathlineto{\pgfqpoint{3.266656in}{1.555607in}}%
\pgfpathlineto{\pgfqpoint{3.267038in}{1.563756in}}%
\pgfpathlineto{\pgfqpoint{3.267421in}{1.531161in}}%
\pgfpathlineto{\pgfqpoint{3.268188in}{1.539309in}}%
\pgfpathlineto{\pgfqpoint{3.268571in}{1.551532in}}%
\pgfpathlineto{\pgfqpoint{3.268954in}{1.547458in}}%
\pgfpathlineto{\pgfqpoint{3.269337in}{1.535235in}}%
\pgfpathlineto{\pgfqpoint{3.270104in}{1.543384in}}%
\pgfpathlineto{\pgfqpoint{3.270487in}{1.535235in}}%
\pgfpathlineto{\pgfqpoint{3.271637in}{1.551532in}}%
\pgfpathlineto{\pgfqpoint{3.272019in}{1.543384in}}%
\pgfpathlineto{\pgfqpoint{3.272786in}{1.547458in}}%
\pgfpathlineto{\pgfqpoint{3.273169in}{1.551532in}}%
\pgfpathlineto{\pgfqpoint{3.273552in}{1.547458in}}%
\pgfpathlineto{\pgfqpoint{3.274319in}{1.527086in}}%
\pgfpathlineto{\pgfqpoint{3.275085in}{1.567830in}}%
\pgfpathlineto{\pgfqpoint{3.275553in}{1.551532in}}%
\pgfpathlineto{\pgfqpoint{3.276702in}{1.535235in}}%
\pgfpathlineto{\pgfqpoint{3.278236in}{1.567830in}}%
\pgfpathlineto{\pgfqpoint{3.278618in}{1.539309in}}%
\pgfpathlineto{\pgfqpoint{3.279385in}{1.547458in}}%
\pgfpathlineto{\pgfqpoint{3.279768in}{1.555607in}}%
\pgfpathlineto{\pgfqpoint{3.280151in}{1.547458in}}%
\pgfpathlineto{\pgfqpoint{3.280917in}{1.535235in}}%
\pgfpathlineto{\pgfqpoint{3.282460in}{1.559681in}}%
\pgfpathlineto{\pgfqpoint{3.283228in}{1.559681in}}%
\pgfpathlineto{\pgfqpoint{3.283611in}{1.527086in}}%
\pgfpathlineto{\pgfqpoint{3.284379in}{1.543384in}}%
\pgfpathlineto{\pgfqpoint{3.284763in}{1.543384in}}%
\pgfpathlineto{\pgfqpoint{3.285913in}{1.555607in}}%
\pgfpathlineto{\pgfqpoint{3.286296in}{1.551532in}}%
\pgfpathlineto{\pgfqpoint{3.287064in}{1.563756in}}%
\pgfpathlineto{\pgfqpoint{3.287447in}{1.555607in}}%
\pgfpathlineto{\pgfqpoint{3.288596in}{1.535235in}}%
\pgfpathlineto{\pgfqpoint{3.288979in}{1.543384in}}%
\pgfpathlineto{\pgfqpoint{3.289747in}{1.555607in}}%
\pgfpathlineto{\pgfqpoint{3.291280in}{1.535235in}}%
\pgfpathlineto{\pgfqpoint{3.291663in}{1.539309in}}%
\pgfpathlineto{\pgfqpoint{3.293197in}{1.559681in}}%
\pgfpathlineto{\pgfqpoint{3.294346in}{1.527086in}}%
\pgfpathlineto{\pgfqpoint{3.294729in}{1.539309in}}%
\pgfpathlineto{\pgfqpoint{3.295497in}{1.559681in}}%
\pgfpathlineto{\pgfqpoint{3.295881in}{1.543384in}}%
\pgfpathlineto{\pgfqpoint{3.296647in}{1.551532in}}%
\pgfpathlineto{\pgfqpoint{3.297029in}{1.547458in}}%
\pgfpathlineto{\pgfqpoint{3.297413in}{1.535235in}}%
\pgfpathlineto{\pgfqpoint{3.297413in}{1.535235in}}%
\pgfpathlineto{\pgfqpoint{3.297413in}{1.535235in}}%
\pgfpathlineto{\pgfqpoint{3.297797in}{1.551532in}}%
\pgfpathlineto{\pgfqpoint{3.298564in}{1.547458in}}%
\pgfpathlineto{\pgfqpoint{3.299713in}{1.539309in}}%
\pgfpathlineto{\pgfqpoint{3.300097in}{1.551532in}}%
\pgfpathlineto{\pgfqpoint{3.300097in}{1.551532in}}%
\pgfpathlineto{\pgfqpoint{3.300097in}{1.551532in}}%
\pgfpathlineto{\pgfqpoint{3.300481in}{1.535235in}}%
\pgfpathlineto{\pgfqpoint{3.300864in}{1.551532in}}%
\pgfpathlineto{\pgfqpoint{3.301630in}{1.551532in}}%
\pgfpathlineto{\pgfqpoint{3.302014in}{1.539309in}}%
\pgfpathlineto{\pgfqpoint{3.302396in}{1.563756in}}%
\pgfpathlineto{\pgfqpoint{3.303164in}{1.547458in}}%
\pgfpathlineto{\pgfqpoint{3.303547in}{1.547458in}}%
\pgfpathlineto{\pgfqpoint{3.304314in}{1.539309in}}%
\pgfpathlineto{\pgfqpoint{3.304697in}{1.543384in}}%
\pgfpathlineto{\pgfqpoint{3.305464in}{1.555607in}}%
\pgfpathlineto{\pgfqpoint{3.305848in}{1.535235in}}%
\pgfpathlineto{\pgfqpoint{3.306614in}{1.543384in}}%
\pgfpathlineto{\pgfqpoint{3.306997in}{1.547458in}}%
\pgfpathlineto{\pgfqpoint{3.307762in}{1.531161in}}%
\pgfpathlineto{\pgfqpoint{3.308530in}{1.563756in}}%
\pgfpathlineto{\pgfqpoint{3.308913in}{1.543384in}}%
\pgfpathlineto{\pgfqpoint{3.309297in}{1.555607in}}%
\pgfpathlineto{\pgfqpoint{3.310063in}{1.551532in}}%
\pgfpathlineto{\pgfqpoint{3.310446in}{1.527086in}}%
\pgfpathlineto{\pgfqpoint{3.310829in}{1.539309in}}%
\pgfpathlineto{\pgfqpoint{3.311214in}{1.555607in}}%
\pgfpathlineto{\pgfqpoint{3.311980in}{1.551532in}}%
\pgfpathlineto{\pgfqpoint{3.312364in}{1.543384in}}%
\pgfpathlineto{\pgfqpoint{3.312746in}{1.559681in}}%
\pgfpathlineto{\pgfqpoint{3.313515in}{1.547458in}}%
\pgfpathlineto{\pgfqpoint{3.313898in}{1.531161in}}%
\pgfpathlineto{\pgfqpoint{3.313898in}{1.531161in}}%
\pgfpathlineto{\pgfqpoint{3.313898in}{1.531161in}}%
\pgfpathlineto{\pgfqpoint{3.315429in}{1.567830in}}%
\pgfpathlineto{\pgfqpoint{3.315813in}{1.527086in}}%
\pgfpathlineto{\pgfqpoint{3.316581in}{1.543384in}}%
\pgfpathlineto{\pgfqpoint{3.316964in}{1.551532in}}%
\pgfpathlineto{\pgfqpoint{3.317347in}{1.535235in}}%
\pgfpathlineto{\pgfqpoint{3.317347in}{1.535235in}}%
\pgfpathlineto{\pgfqpoint{3.317347in}{1.535235in}}%
\pgfpathlineto{\pgfqpoint{3.317730in}{1.555607in}}%
\pgfpathlineto{\pgfqpoint{3.317730in}{1.555607in}}%
\pgfpathlineto{\pgfqpoint{3.317730in}{1.555607in}}%
\pgfpathlineto{\pgfqpoint{3.318496in}{1.531161in}}%
\pgfpathlineto{\pgfqpoint{3.318880in}{1.563756in}}%
\pgfpathlineto{\pgfqpoint{3.319647in}{1.547458in}}%
\pgfpathlineto{\pgfqpoint{3.320030in}{1.551532in}}%
\pgfpathlineto{\pgfqpoint{3.320499in}{1.547458in}}%
\pgfpathlineto{\pgfqpoint{3.320882in}{1.547458in}}%
\pgfpathlineto{\pgfqpoint{3.321273in}{1.551532in}}%
\pgfpathlineto{\pgfqpoint{3.321657in}{1.535235in}}%
\pgfpathlineto{\pgfqpoint{3.322040in}{1.539309in}}%
\pgfpathlineto{\pgfqpoint{3.322424in}{1.555607in}}%
\pgfpathlineto{\pgfqpoint{3.322808in}{1.543384in}}%
\pgfpathlineto{\pgfqpoint{3.323191in}{1.527086in}}%
\pgfpathlineto{\pgfqpoint{3.323191in}{1.527086in}}%
\pgfpathlineto{\pgfqpoint{3.323191in}{1.527086in}}%
\pgfpathlineto{\pgfqpoint{3.324341in}{1.551532in}}%
\pgfpathlineto{\pgfqpoint{3.324724in}{1.535235in}}%
\pgfpathlineto{\pgfqpoint{3.325108in}{1.551532in}}%
\pgfpathlineto{\pgfqpoint{3.325492in}{1.559681in}}%
\pgfpathlineto{\pgfqpoint{3.326643in}{1.539309in}}%
\pgfpathlineto{\pgfqpoint{3.327793in}{1.555607in}}%
\pgfpathlineto{\pgfqpoint{3.328176in}{1.535235in}}%
\pgfpathlineto{\pgfqpoint{3.328560in}{1.551532in}}%
\pgfpathlineto{\pgfqpoint{3.328944in}{1.563756in}}%
\pgfpathlineto{\pgfqpoint{3.329327in}{1.539309in}}%
\pgfpathlineto{\pgfqpoint{3.330093in}{1.555607in}}%
\pgfpathlineto{\pgfqpoint{3.332010in}{1.535235in}}%
\pgfpathlineto{\pgfqpoint{3.332393in}{1.547458in}}%
\pgfpathlineto{\pgfqpoint{3.332393in}{1.547458in}}%
\pgfpathlineto{\pgfqpoint{3.332393in}{1.547458in}}%
\pgfpathlineto{\pgfqpoint{3.332776in}{1.527086in}}%
\pgfpathlineto{\pgfqpoint{3.333159in}{1.539309in}}%
\pgfpathlineto{\pgfqpoint{3.333542in}{1.563756in}}%
\pgfpathlineto{\pgfqpoint{3.333542in}{1.563756in}}%
\pgfpathlineto{\pgfqpoint{3.333542in}{1.563756in}}%
\pgfpathlineto{\pgfqpoint{3.333925in}{1.531161in}}%
\pgfpathlineto{\pgfqpoint{3.334692in}{1.551532in}}%
\pgfpathlineto{\pgfqpoint{3.335074in}{1.563756in}}%
\pgfpathlineto{\pgfqpoint{3.335457in}{1.551532in}}%
\pgfpathlineto{\pgfqpoint{3.336607in}{1.539309in}}%
\pgfpathlineto{\pgfqpoint{3.337758in}{1.555607in}}%
\pgfpathlineto{\pgfqpoint{3.338141in}{1.527086in}}%
\pgfpathlineto{\pgfqpoint{3.338524in}{1.535235in}}%
\pgfpathlineto{\pgfqpoint{3.338907in}{1.559681in}}%
\pgfpathlineto{\pgfqpoint{3.339674in}{1.539309in}}%
\pgfpathlineto{\pgfqpoint{3.340058in}{1.559681in}}%
\pgfpathlineto{\pgfqpoint{3.340058in}{1.559681in}}%
\pgfpathlineto{\pgfqpoint{3.340058in}{1.559681in}}%
\pgfpathlineto{\pgfqpoint{3.340441in}{1.527086in}}%
\pgfpathlineto{\pgfqpoint{3.341208in}{1.543384in}}%
\pgfpathlineto{\pgfqpoint{3.341591in}{1.535235in}}%
\pgfpathlineto{\pgfqpoint{3.341974in}{1.543384in}}%
\pgfpathlineto{\pgfqpoint{3.342357in}{1.547458in}}%
\pgfpathlineto{\pgfqpoint{3.342741in}{1.535235in}}%
\pgfpathlineto{\pgfqpoint{3.342741in}{1.535235in}}%
\pgfpathlineto{\pgfqpoint{3.342741in}{1.535235in}}%
\pgfpathlineto{\pgfqpoint{3.343125in}{1.555607in}}%
\pgfpathlineto{\pgfqpoint{3.343508in}{1.551532in}}%
\pgfpathlineto{\pgfqpoint{3.343892in}{1.531161in}}%
\pgfpathlineto{\pgfqpoint{3.343892in}{1.531161in}}%
\pgfpathlineto{\pgfqpoint{3.343892in}{1.531161in}}%
\pgfpathlineto{\pgfqpoint{3.344275in}{1.563756in}}%
\pgfpathlineto{\pgfqpoint{3.345041in}{1.551532in}}%
\pgfpathlineto{\pgfqpoint{3.345425in}{1.539309in}}%
\pgfpathlineto{\pgfqpoint{3.346191in}{1.543384in}}%
\pgfpathlineto{\pgfqpoint{3.346958in}{1.543384in}}%
\pgfpathlineto{\pgfqpoint{3.347342in}{1.535235in}}%
\pgfpathlineto{\pgfqpoint{3.348109in}{1.567830in}}%
\pgfpathlineto{\pgfqpoint{3.348492in}{1.551532in}}%
\pgfpathlineto{\pgfqpoint{3.349641in}{1.514863in}}%
\pgfpathlineto{\pgfqpoint{3.351176in}{1.555607in}}%
\pgfpathlineto{\pgfqpoint{3.351559in}{1.551532in}}%
\pgfpathlineto{\pgfqpoint{3.351941in}{1.531161in}}%
\pgfpathlineto{\pgfqpoint{3.352325in}{1.539309in}}%
\pgfpathlineto{\pgfqpoint{3.352708in}{1.555607in}}%
\pgfpathlineto{\pgfqpoint{3.353476in}{1.551532in}}%
\pgfpathlineto{\pgfqpoint{3.353858in}{1.551532in}}%
\pgfpathlineto{\pgfqpoint{3.354242in}{1.547458in}}%
\pgfpathlineto{\pgfqpoint{3.354625in}{1.531161in}}%
\pgfpathlineto{\pgfqpoint{3.354625in}{1.531161in}}%
\pgfpathlineto{\pgfqpoint{3.354625in}{1.531161in}}%
\pgfpathlineto{\pgfqpoint{3.355008in}{1.551532in}}%
\pgfpathlineto{\pgfqpoint{3.355775in}{1.543384in}}%
\pgfpathlineto{\pgfqpoint{3.356159in}{1.551532in}}%
\pgfpathlineto{\pgfqpoint{3.356542in}{1.535235in}}%
\pgfpathlineto{\pgfqpoint{3.357308in}{1.547458in}}%
\pgfpathlineto{\pgfqpoint{3.358074in}{1.563756in}}%
\pgfpathlineto{\pgfqpoint{3.358458in}{1.551532in}}%
\pgfpathlineto{\pgfqpoint{3.358841in}{1.535235in}}%
\pgfpathlineto{\pgfqpoint{3.359224in}{1.547458in}}%
\pgfpathlineto{\pgfqpoint{3.360375in}{1.563756in}}%
\pgfpathlineto{\pgfqpoint{3.360758in}{1.563756in}}%
\pgfpathlineto{\pgfqpoint{3.361916in}{1.531161in}}%
\pgfpathlineto{\pgfqpoint{3.363450in}{1.547458in}}%
\pgfpathlineto{\pgfqpoint{3.363834in}{1.531161in}}%
\pgfpathlineto{\pgfqpoint{3.364216in}{1.547458in}}%
\pgfpathlineto{\pgfqpoint{3.364599in}{1.547458in}}%
\pgfpathlineto{\pgfqpoint{3.364982in}{1.543384in}}%
\pgfpathlineto{\pgfqpoint{3.366135in}{1.551532in}}%
\pgfpathlineto{\pgfqpoint{3.367283in}{1.535235in}}%
\pgfpathlineto{\pgfqpoint{3.368049in}{1.551532in}}%
\pgfpathlineto{\pgfqpoint{3.368433in}{1.539309in}}%
\pgfpathlineto{\pgfqpoint{3.368817in}{1.535235in}}%
\pgfpathlineto{\pgfqpoint{3.369199in}{1.551532in}}%
\pgfpathlineto{\pgfqpoint{3.369667in}{1.547458in}}%
\pgfpathlineto{\pgfqpoint{3.370051in}{1.539309in}}%
\pgfpathlineto{\pgfqpoint{3.370434in}{1.547458in}}%
\pgfpathlineto{\pgfqpoint{3.370817in}{1.555607in}}%
\pgfpathlineto{\pgfqpoint{3.372349in}{1.527086in}}%
\pgfpathlineto{\pgfqpoint{3.372734in}{1.547458in}}%
\pgfpathlineto{\pgfqpoint{3.373500in}{1.543384in}}%
\pgfpathlineto{\pgfqpoint{3.373882in}{1.527086in}}%
\pgfpathlineto{\pgfqpoint{3.374266in}{1.543384in}}%
\pgfpathlineto{\pgfqpoint{3.374649in}{1.567830in}}%
\pgfpathlineto{\pgfqpoint{3.374649in}{1.567830in}}%
\pgfpathlineto{\pgfqpoint{3.374649in}{1.567830in}}%
\pgfpathlineto{\pgfqpoint{3.376183in}{1.523012in}}%
\pgfpathlineto{\pgfqpoint{3.377717in}{1.555607in}}%
\pgfpathlineto{\pgfqpoint{3.378483in}{1.547458in}}%
\pgfpathlineto{\pgfqpoint{3.379249in}{1.567830in}}%
\pgfpathlineto{\pgfqpoint{3.379633in}{1.539309in}}%
\pgfpathlineto{\pgfqpoint{3.380400in}{1.543384in}}%
\pgfpathlineto{\pgfqpoint{3.380784in}{1.567830in}}%
\pgfpathlineto{\pgfqpoint{3.380784in}{1.567830in}}%
\pgfpathlineto{\pgfqpoint{3.380784in}{1.567830in}}%
\pgfpathlineto{\pgfqpoint{3.381167in}{1.527086in}}%
\pgfpathlineto{\pgfqpoint{3.381933in}{1.551532in}}%
\pgfpathlineto{\pgfqpoint{3.382700in}{1.551532in}}%
\pgfpathlineto{\pgfqpoint{3.384232in}{1.523012in}}%
\pgfpathlineto{\pgfqpoint{3.385766in}{1.551532in}}%
\pgfpathlineto{\pgfqpoint{3.386534in}{1.527086in}}%
\pgfpathlineto{\pgfqpoint{3.387300in}{1.539309in}}%
\pgfpathlineto{\pgfqpoint{3.387684in}{1.567830in}}%
\pgfpathlineto{\pgfqpoint{3.388451in}{1.551532in}}%
\pgfpathlineto{\pgfqpoint{3.389218in}{1.535235in}}%
\pgfpathlineto{\pgfqpoint{3.389601in}{1.543384in}}%
\pgfpathlineto{\pgfqpoint{3.390751in}{1.563756in}}%
\pgfpathlineto{\pgfqpoint{3.391518in}{1.535235in}}%
\pgfpathlineto{\pgfqpoint{3.391901in}{1.543384in}}%
\pgfpathlineto{\pgfqpoint{3.392284in}{1.559681in}}%
\pgfpathlineto{\pgfqpoint{3.393051in}{1.551532in}}%
\pgfpathlineto{\pgfqpoint{3.393818in}{1.543384in}}%
\pgfpathlineto{\pgfqpoint{3.394202in}{1.547458in}}%
\pgfpathlineto{\pgfqpoint{3.395352in}{1.559681in}}%
\pgfpathlineto{\pgfqpoint{3.396120in}{1.543384in}}%
\pgfpathlineto{\pgfqpoint{3.396503in}{1.555607in}}%
\pgfpathlineto{\pgfqpoint{3.396886in}{1.559681in}}%
\pgfpathlineto{\pgfqpoint{3.397269in}{1.531161in}}%
\pgfpathlineto{\pgfqpoint{3.398036in}{1.555607in}}%
\pgfpathlineto{\pgfqpoint{3.398803in}{1.539309in}}%
\pgfpathlineto{\pgfqpoint{3.399186in}{1.559681in}}%
\pgfpathlineto{\pgfqpoint{3.399952in}{1.555607in}}%
\pgfpathlineto{\pgfqpoint{3.400335in}{1.559681in}}%
\pgfpathlineto{\pgfqpoint{3.401110in}{1.527086in}}%
\pgfpathlineto{\pgfqpoint{3.401877in}{1.543384in}}%
\pgfpathlineto{\pgfqpoint{3.403028in}{1.563756in}}%
\pgfpathlineto{\pgfqpoint{3.404560in}{1.539309in}}%
\pgfpathlineto{\pgfqpoint{3.404944in}{1.555607in}}%
\pgfpathlineto{\pgfqpoint{3.404944in}{1.555607in}}%
\pgfpathlineto{\pgfqpoint{3.404944in}{1.555607in}}%
\pgfpathlineto{\pgfqpoint{3.405328in}{1.527086in}}%
\pgfpathlineto{\pgfqpoint{3.406094in}{1.531161in}}%
\pgfpathlineto{\pgfqpoint{3.406860in}{1.543384in}}%
\pgfpathlineto{\pgfqpoint{3.407244in}{1.555607in}}%
\pgfpathlineto{\pgfqpoint{3.407628in}{1.547458in}}%
\pgfpathlineto{\pgfqpoint{3.408011in}{1.543384in}}%
\pgfpathlineto{\pgfqpoint{3.408394in}{1.547458in}}%
\pgfpathlineto{\pgfqpoint{3.409160in}{1.555607in}}%
\pgfpathlineto{\pgfqpoint{3.409544in}{1.551532in}}%
\pgfpathlineto{\pgfqpoint{3.409927in}{1.547458in}}%
\pgfpathlineto{\pgfqpoint{3.410311in}{1.551532in}}%
\pgfpathlineto{\pgfqpoint{3.410695in}{1.559681in}}%
\pgfpathlineto{\pgfqpoint{3.410695in}{1.559681in}}%
\pgfpathlineto{\pgfqpoint{3.410695in}{1.559681in}}%
\pgfpathlineto{\pgfqpoint{3.411842in}{1.543384in}}%
\pgfpathlineto{\pgfqpoint{3.412225in}{1.547458in}}%
\pgfpathlineto{\pgfqpoint{3.412609in}{1.527086in}}%
\pgfpathlineto{\pgfqpoint{3.413077in}{1.539309in}}%
\pgfpathlineto{\pgfqpoint{3.413460in}{1.563756in}}%
\pgfpathlineto{\pgfqpoint{3.413460in}{1.563756in}}%
\pgfpathlineto{\pgfqpoint{3.413460in}{1.563756in}}%
\pgfpathlineto{\pgfqpoint{3.413844in}{1.527086in}}%
\pgfpathlineto{\pgfqpoint{3.414612in}{1.535235in}}%
\pgfpathlineto{\pgfqpoint{3.414997in}{1.535235in}}%
\pgfpathlineto{\pgfqpoint{3.415764in}{1.547458in}}%
\pgfpathlineto{\pgfqpoint{3.416147in}{1.535235in}}%
\pgfpathlineto{\pgfqpoint{3.416531in}{1.547458in}}%
\pgfpathlineto{\pgfqpoint{3.416914in}{1.551532in}}%
\pgfpathlineto{\pgfqpoint{3.417297in}{1.539309in}}%
\pgfpathlineto{\pgfqpoint{3.417680in}{1.547458in}}%
\pgfpathlineto{\pgfqpoint{3.418063in}{1.555607in}}%
\pgfpathlineto{\pgfqpoint{3.418446in}{1.535235in}}%
\pgfpathlineto{\pgfqpoint{3.419214in}{1.551532in}}%
\pgfpathlineto{\pgfqpoint{3.420747in}{1.539309in}}%
\pgfpathlineto{\pgfqpoint{3.421897in}{1.571904in}}%
\pgfpathlineto{\pgfqpoint{3.423046in}{1.531161in}}%
\pgfpathlineto{\pgfqpoint{3.424578in}{1.551532in}}%
\pgfpathlineto{\pgfqpoint{3.425728in}{1.539309in}}%
\pgfpathlineto{\pgfqpoint{3.426494in}{1.567830in}}%
\pgfpathlineto{\pgfqpoint{3.426877in}{1.514863in}}%
\pgfpathlineto{\pgfqpoint{3.427644in}{1.551532in}}%
\pgfpathlineto{\pgfqpoint{3.428794in}{1.543384in}}%
\pgfpathlineto{\pgfqpoint{3.429176in}{1.555607in}}%
\pgfpathlineto{\pgfqpoint{3.429176in}{1.555607in}}%
\pgfpathlineto{\pgfqpoint{3.429176in}{1.555607in}}%
\pgfpathlineto{\pgfqpoint{3.429559in}{1.539309in}}%
\pgfpathlineto{\pgfqpoint{3.429559in}{1.539309in}}%
\pgfpathlineto{\pgfqpoint{3.429559in}{1.539309in}}%
\pgfpathlineto{\pgfqpoint{3.429944in}{1.567830in}}%
\pgfpathlineto{\pgfqpoint{3.429944in}{1.567830in}}%
\pgfpathlineto{\pgfqpoint{3.429944in}{1.567830in}}%
\pgfpathlineto{\pgfqpoint{3.430328in}{1.531161in}}%
\pgfpathlineto{\pgfqpoint{3.431095in}{1.547458in}}%
\pgfpathlineto{\pgfqpoint{3.431861in}{1.551532in}}%
\pgfpathlineto{\pgfqpoint{3.432628in}{1.531161in}}%
\pgfpathlineto{\pgfqpoint{3.433011in}{1.535235in}}%
\pgfpathlineto{\pgfqpoint{3.434162in}{1.543384in}}%
\pgfpathlineto{\pgfqpoint{3.434545in}{1.535235in}}%
\pgfpathlineto{\pgfqpoint{3.434929in}{1.551532in}}%
\pgfpathlineto{\pgfqpoint{3.435696in}{1.547458in}}%
\pgfpathlineto{\pgfqpoint{3.436079in}{1.543384in}}%
\pgfpathlineto{\pgfqpoint{3.436845in}{1.567830in}}%
\pgfpathlineto{\pgfqpoint{3.437228in}{1.539309in}}%
\pgfpathlineto{\pgfqpoint{3.437994in}{1.543384in}}%
\pgfpathlineto{\pgfqpoint{3.439911in}{1.563756in}}%
\pgfpathlineto{\pgfqpoint{3.441834in}{1.539309in}}%
\pgfpathlineto{\pgfqpoint{3.442217in}{1.543384in}}%
\pgfpathlineto{\pgfqpoint{3.442601in}{1.563756in}}%
\pgfpathlineto{\pgfqpoint{3.443368in}{1.547458in}}%
\pgfpathlineto{\pgfqpoint{3.444517in}{1.543384in}}%
\pgfpathlineto{\pgfqpoint{3.444899in}{1.547458in}}%
\pgfpathlineto{\pgfqpoint{3.445283in}{1.535235in}}%
\pgfpathlineto{\pgfqpoint{3.445666in}{1.559681in}}%
\pgfpathlineto{\pgfqpoint{3.445666in}{1.559681in}}%
\pgfpathlineto{\pgfqpoint{3.445666in}{1.559681in}}%
\pgfpathlineto{\pgfqpoint{3.446050in}{1.531161in}}%
\pgfpathlineto{\pgfqpoint{3.446432in}{1.539309in}}%
\pgfpathlineto{\pgfqpoint{3.446815in}{1.559681in}}%
\pgfpathlineto{\pgfqpoint{3.446815in}{1.559681in}}%
\pgfpathlineto{\pgfqpoint{3.446815in}{1.559681in}}%
\pgfpathlineto{\pgfqpoint{3.447963in}{1.514863in}}%
\pgfpathlineto{\pgfqpoint{3.449113in}{1.559681in}}%
\pgfpathlineto{\pgfqpoint{3.450262in}{1.535235in}}%
\pgfpathlineto{\pgfqpoint{3.450646in}{1.543384in}}%
\pgfpathlineto{\pgfqpoint{3.451029in}{1.563756in}}%
\pgfpathlineto{\pgfqpoint{3.451029in}{1.563756in}}%
\pgfpathlineto{\pgfqpoint{3.451029in}{1.563756in}}%
\pgfpathlineto{\pgfqpoint{3.451413in}{1.539309in}}%
\pgfpathlineto{\pgfqpoint{3.452178in}{1.547458in}}%
\pgfpathlineto{\pgfqpoint{3.452561in}{1.567830in}}%
\pgfpathlineto{\pgfqpoint{3.452944in}{1.551532in}}%
\pgfpathlineto{\pgfqpoint{3.454562in}{1.531161in}}%
\pgfpathlineto{\pgfqpoint{3.456095in}{1.563756in}}%
\pgfpathlineto{\pgfqpoint{3.457629in}{1.531161in}}%
\pgfpathlineto{\pgfqpoint{3.458394in}{1.559681in}}%
\pgfpathlineto{\pgfqpoint{3.458777in}{1.547458in}}%
\pgfpathlineto{\pgfqpoint{3.459926in}{1.539309in}}%
\pgfpathlineto{\pgfqpoint{3.460309in}{1.543384in}}%
\pgfpathlineto{\pgfqpoint{3.460693in}{1.559681in}}%
\pgfpathlineto{\pgfqpoint{3.461076in}{1.547458in}}%
\pgfpathlineto{\pgfqpoint{3.461458in}{1.523012in}}%
\pgfpathlineto{\pgfqpoint{3.461458in}{1.523012in}}%
\pgfpathlineto{\pgfqpoint{3.461458in}{1.523012in}}%
\pgfpathlineto{\pgfqpoint{3.462608in}{1.559681in}}%
\pgfpathlineto{\pgfqpoint{3.462991in}{1.531161in}}%
\pgfpathlineto{\pgfqpoint{3.463759in}{1.551532in}}%
\pgfpathlineto{\pgfqpoint{3.464142in}{1.551532in}}%
\pgfpathlineto{\pgfqpoint{3.464526in}{1.535235in}}%
\pgfpathlineto{\pgfqpoint{3.464909in}{1.547458in}}%
\pgfpathlineto{\pgfqpoint{3.465675in}{1.567830in}}%
\pgfpathlineto{\pgfqpoint{3.466443in}{1.531161in}}%
\pgfpathlineto{\pgfqpoint{3.466826in}{1.551532in}}%
\pgfpathlineto{\pgfqpoint{3.467209in}{1.539309in}}%
\pgfpathlineto{\pgfqpoint{3.467592in}{1.547458in}}%
\pgfpathlineto{\pgfqpoint{3.468359in}{1.559681in}}%
\pgfpathlineto{\pgfqpoint{3.469893in}{1.523012in}}%
\pgfpathlineto{\pgfqpoint{3.471810in}{1.551532in}}%
\pgfpathlineto{\pgfqpoint{3.472193in}{1.531161in}}%
\pgfpathlineto{\pgfqpoint{3.472193in}{1.531161in}}%
\pgfpathlineto{\pgfqpoint{3.472193in}{1.531161in}}%
\pgfpathlineto{\pgfqpoint{3.472576in}{1.555607in}}%
\pgfpathlineto{\pgfqpoint{3.473342in}{1.543384in}}%
\pgfpathlineto{\pgfqpoint{3.473727in}{1.518937in}}%
\pgfpathlineto{\pgfqpoint{3.473727in}{1.518937in}}%
\pgfpathlineto{\pgfqpoint{3.473727in}{1.518937in}}%
\pgfpathlineto{\pgfqpoint{3.474111in}{1.563756in}}%
\pgfpathlineto{\pgfqpoint{3.474877in}{1.551532in}}%
\pgfpathlineto{\pgfqpoint{3.475260in}{1.563756in}}%
\pgfpathlineto{\pgfqpoint{3.475643in}{1.551532in}}%
\pgfpathlineto{\pgfqpoint{3.476794in}{1.535235in}}%
\pgfpathlineto{\pgfqpoint{3.477944in}{1.551532in}}%
\pgfpathlineto{\pgfqpoint{3.478326in}{1.547458in}}%
\pgfpathlineto{\pgfqpoint{3.478709in}{1.531161in}}%
\pgfpathlineto{\pgfqpoint{3.479093in}{1.539309in}}%
\pgfpathlineto{\pgfqpoint{3.479477in}{1.563756in}}%
\pgfpathlineto{\pgfqpoint{3.480244in}{1.559681in}}%
\pgfpathlineto{\pgfqpoint{3.481017in}{1.535235in}}%
\pgfpathlineto{\pgfqpoint{3.481785in}{1.539309in}}%
\pgfpathlineto{\pgfqpoint{3.482168in}{1.535235in}}%
\pgfpathlineto{\pgfqpoint{3.483702in}{1.559681in}}%
\pgfpathlineto{\pgfqpoint{3.484468in}{1.551532in}}%
\pgfpathlineto{\pgfqpoint{3.484851in}{1.559681in}}%
\pgfpathlineto{\pgfqpoint{3.485617in}{1.535235in}}%
\pgfpathlineto{\pgfqpoint{3.486001in}{1.539309in}}%
\pgfpathlineto{\pgfqpoint{3.486385in}{1.543384in}}%
\pgfpathlineto{\pgfqpoint{3.486768in}{1.535235in}}%
\pgfpathlineto{\pgfqpoint{3.487151in}{1.563756in}}%
\pgfpathlineto{\pgfqpoint{3.487917in}{1.555607in}}%
\pgfpathlineto{\pgfqpoint{3.489068in}{1.531161in}}%
\pgfpathlineto{\pgfqpoint{3.490218in}{1.559681in}}%
\pgfpathlineto{\pgfqpoint{3.490601in}{1.543384in}}%
\pgfpathlineto{\pgfqpoint{3.491368in}{1.551532in}}%
\pgfpathlineto{\pgfqpoint{3.491752in}{1.559681in}}%
\pgfpathlineto{\pgfqpoint{3.491752in}{1.559681in}}%
\pgfpathlineto{\pgfqpoint{3.491752in}{1.559681in}}%
\pgfpathlineto{\pgfqpoint{3.492518in}{1.539309in}}%
\pgfpathlineto{\pgfqpoint{3.492901in}{1.547458in}}%
\pgfpathlineto{\pgfqpoint{3.493667in}{1.547458in}}%
\pgfpathlineto{\pgfqpoint{3.494818in}{1.555607in}}%
\pgfpathlineto{\pgfqpoint{3.495201in}{1.555607in}}%
\pgfpathlineto{\pgfqpoint{3.496051in}{1.518937in}}%
\pgfpathlineto{\pgfqpoint{3.496818in}{1.535235in}}%
\pgfpathlineto{\pgfqpoint{3.497584in}{1.539309in}}%
\pgfpathlineto{\pgfqpoint{3.497968in}{1.563756in}}%
\pgfpathlineto{\pgfqpoint{3.497968in}{1.563756in}}%
\pgfpathlineto{\pgfqpoint{3.497968in}{1.563756in}}%
\pgfpathlineto{\pgfqpoint{3.499117in}{1.523012in}}%
\pgfpathlineto{\pgfqpoint{3.500266in}{1.559681in}}%
\pgfpathlineto{\pgfqpoint{3.501416in}{1.535235in}}%
\pgfpathlineto{\pgfqpoint{3.501799in}{1.543384in}}%
\pgfpathlineto{\pgfqpoint{3.502564in}{1.559681in}}%
\pgfpathlineto{\pgfqpoint{3.503714in}{1.535235in}}%
\pgfpathlineto{\pgfqpoint{3.504097in}{1.567830in}}%
\pgfpathlineto{\pgfqpoint{3.504860in}{1.559681in}}%
\pgfpathlineto{\pgfqpoint{3.506395in}{1.518937in}}%
\pgfpathlineto{\pgfqpoint{3.507162in}{1.567830in}}%
\pgfpathlineto{\pgfqpoint{3.507545in}{1.559681in}}%
\pgfpathlineto{\pgfqpoint{3.508312in}{1.531161in}}%
\pgfpathlineto{\pgfqpoint{3.508695in}{1.551532in}}%
\pgfpathlineto{\pgfqpoint{3.509079in}{1.551532in}}%
\pgfpathlineto{\pgfqpoint{3.509463in}{1.539309in}}%
\pgfpathlineto{\pgfqpoint{3.509847in}{1.543384in}}%
\pgfpathlineto{\pgfqpoint{3.510613in}{1.551532in}}%
\pgfpathlineto{\pgfqpoint{3.512147in}{1.535235in}}%
\pgfpathlineto{\pgfqpoint{3.513296in}{1.551532in}}%
\pgfpathlineto{\pgfqpoint{3.514447in}{1.523012in}}%
\pgfpathlineto{\pgfqpoint{3.515595in}{1.551532in}}%
\pgfpathlineto{\pgfqpoint{3.515978in}{1.539309in}}%
\pgfpathlineto{\pgfqpoint{3.516744in}{1.567830in}}%
\pgfpathlineto{\pgfqpoint{3.517128in}{1.559681in}}%
\pgfpathlineto{\pgfqpoint{3.518661in}{1.543384in}}%
\pgfpathlineto{\pgfqpoint{3.519045in}{1.539309in}}%
\pgfpathlineto{\pgfqpoint{3.519429in}{1.563756in}}%
\pgfpathlineto{\pgfqpoint{3.519429in}{1.563756in}}%
\pgfpathlineto{\pgfqpoint{3.519429in}{1.563756in}}%
\pgfpathlineto{\pgfqpoint{3.519813in}{1.531161in}}%
\pgfpathlineto{\pgfqpoint{3.520579in}{1.543384in}}%
\pgfpathlineto{\pgfqpoint{3.521737in}{1.559681in}}%
\pgfpathlineto{\pgfqpoint{3.522503in}{1.543384in}}%
\pgfpathlineto{\pgfqpoint{3.522886in}{1.555607in}}%
\pgfpathlineto{\pgfqpoint{3.523652in}{1.535235in}}%
\pgfpathlineto{\pgfqpoint{3.524037in}{1.543384in}}%
\pgfpathlineto{\pgfqpoint{3.524421in}{1.551532in}}%
\pgfpathlineto{\pgfqpoint{3.524421in}{1.551532in}}%
\pgfpathlineto{\pgfqpoint{3.524421in}{1.551532in}}%
\pgfpathlineto{\pgfqpoint{3.524804in}{1.539309in}}%
\pgfpathlineto{\pgfqpoint{3.524804in}{1.539309in}}%
\pgfpathlineto{\pgfqpoint{3.524804in}{1.539309in}}%
\pgfpathlineto{\pgfqpoint{3.525187in}{1.559681in}}%
\pgfpathlineto{\pgfqpoint{3.525187in}{1.559681in}}%
\pgfpathlineto{\pgfqpoint{3.525187in}{1.559681in}}%
\pgfpathlineto{\pgfqpoint{3.525570in}{1.535235in}}%
\pgfpathlineto{\pgfqpoint{3.526336in}{1.547458in}}%
\pgfpathlineto{\pgfqpoint{3.527104in}{1.527086in}}%
\pgfpathlineto{\pgfqpoint{3.527487in}{1.539309in}}%
\pgfpathlineto{\pgfqpoint{3.528636in}{1.555607in}}%
\pgfpathlineto{\pgfqpoint{3.529402in}{1.539309in}}%
\pgfpathlineto{\pgfqpoint{3.530554in}{1.559681in}}%
\pgfpathlineto{\pgfqpoint{3.531703in}{1.531161in}}%
\pgfpathlineto{\pgfqpoint{3.532086in}{1.555607in}}%
\pgfpathlineto{\pgfqpoint{3.532853in}{1.543384in}}%
\pgfpathlineto{\pgfqpoint{3.533236in}{1.543384in}}%
\pgfpathlineto{\pgfqpoint{3.533619in}{1.539309in}}%
\pgfpathlineto{\pgfqpoint{3.534001in}{1.563756in}}%
\pgfpathlineto{\pgfqpoint{3.534001in}{1.563756in}}%
\pgfpathlineto{\pgfqpoint{3.534001in}{1.563756in}}%
\pgfpathlineto{\pgfqpoint{3.534384in}{1.535235in}}%
\pgfpathlineto{\pgfqpoint{3.535150in}{1.547458in}}%
\pgfpathlineto{\pgfqpoint{3.536003in}{1.535235in}}%
\pgfpathlineto{\pgfqpoint{3.536769in}{1.559681in}}%
\pgfpathlineto{\pgfqpoint{3.537536in}{1.555607in}}%
\pgfpathlineto{\pgfqpoint{3.539068in}{1.531161in}}%
\pgfpathlineto{\pgfqpoint{3.539834in}{1.555607in}}%
\pgfpathlineto{\pgfqpoint{3.540217in}{1.547458in}}%
\pgfpathlineto{\pgfqpoint{3.540600in}{1.539309in}}%
\pgfpathlineto{\pgfqpoint{3.540983in}{1.559681in}}%
\pgfpathlineto{\pgfqpoint{3.541750in}{1.547458in}}%
\pgfpathlineto{\pgfqpoint{3.542134in}{1.555607in}}%
\pgfpathlineto{\pgfqpoint{3.542517in}{1.539309in}}%
\pgfpathlineto{\pgfqpoint{3.542517in}{1.539309in}}%
\pgfpathlineto{\pgfqpoint{3.542517in}{1.539309in}}%
\pgfpathlineto{\pgfqpoint{3.542900in}{1.563756in}}%
\pgfpathlineto{\pgfqpoint{3.543667in}{1.551532in}}%
\pgfpathlineto{\pgfqpoint{3.544433in}{1.531161in}}%
\pgfpathlineto{\pgfqpoint{3.544816in}{1.555607in}}%
\pgfpathlineto{\pgfqpoint{3.545584in}{1.551532in}}%
\pgfpathlineto{\pgfqpoint{3.545966in}{1.543384in}}%
\pgfpathlineto{\pgfqpoint{3.545966in}{1.543384in}}%
\pgfpathlineto{\pgfqpoint{3.545966in}{1.543384in}}%
\pgfpathlineto{\pgfqpoint{3.546351in}{1.555607in}}%
\pgfpathlineto{\pgfqpoint{3.546351in}{1.555607in}}%
\pgfpathlineto{\pgfqpoint{3.546351in}{1.555607in}}%
\pgfpathlineto{\pgfqpoint{3.547118in}{1.535235in}}%
\pgfpathlineto{\pgfqpoint{3.547884in}{1.563756in}}%
\pgfpathlineto{\pgfqpoint{3.548267in}{1.551532in}}%
\pgfpathlineto{\pgfqpoint{3.549033in}{1.527086in}}%
\pgfpathlineto{\pgfqpoint{3.549799in}{1.551532in}}%
\pgfpathlineto{\pgfqpoint{3.550182in}{1.543384in}}%
\pgfpathlineto{\pgfqpoint{3.550565in}{1.551532in}}%
\pgfpathlineto{\pgfqpoint{3.550565in}{1.551532in}}%
\pgfpathlineto{\pgfqpoint{3.550565in}{1.551532in}}%
\pgfpathlineto{\pgfqpoint{3.550949in}{1.539309in}}%
\pgfpathlineto{\pgfqpoint{3.551332in}{1.543384in}}%
\pgfpathlineto{\pgfqpoint{3.551715in}{1.555607in}}%
\pgfpathlineto{\pgfqpoint{3.551715in}{1.555607in}}%
\pgfpathlineto{\pgfqpoint{3.551715in}{1.555607in}}%
\pgfpathlineto{\pgfqpoint{3.552098in}{1.539309in}}%
\pgfpathlineto{\pgfqpoint{3.552865in}{1.551532in}}%
\pgfpathlineto{\pgfqpoint{3.553249in}{1.539309in}}%
\pgfpathlineto{\pgfqpoint{3.553633in}{1.547458in}}%
\pgfpathlineto{\pgfqpoint{3.554781in}{1.567830in}}%
\pgfpathlineto{\pgfqpoint{3.555166in}{1.518937in}}%
\pgfpathlineto{\pgfqpoint{3.555933in}{1.539309in}}%
\pgfpathlineto{\pgfqpoint{3.556316in}{1.535235in}}%
\pgfpathlineto{\pgfqpoint{3.556699in}{1.547458in}}%
\pgfpathlineto{\pgfqpoint{3.557464in}{1.543384in}}%
\pgfpathlineto{\pgfqpoint{3.558616in}{1.535235in}}%
\pgfpathlineto{\pgfqpoint{3.558999in}{1.551532in}}%
\pgfpathlineto{\pgfqpoint{3.559382in}{1.539309in}}%
\pgfpathlineto{\pgfqpoint{3.559764in}{1.535235in}}%
\pgfpathlineto{\pgfqpoint{3.560149in}{1.555607in}}%
\pgfpathlineto{\pgfqpoint{3.560922in}{1.543384in}}%
\pgfpathlineto{\pgfqpoint{3.561306in}{1.543384in}}%
\pgfpathlineto{\pgfqpoint{3.561688in}{1.551532in}}%
\pgfpathlineto{\pgfqpoint{3.561688in}{1.551532in}}%
\pgfpathlineto{\pgfqpoint{3.561688in}{1.551532in}}%
\pgfpathlineto{\pgfqpoint{3.562840in}{1.531161in}}%
\pgfpathlineto{\pgfqpoint{3.563989in}{1.567830in}}%
\pgfpathlineto{\pgfqpoint{3.564372in}{1.543384in}}%
\pgfpathlineto{\pgfqpoint{3.565139in}{1.551532in}}%
\pgfpathlineto{\pgfqpoint{3.565905in}{1.539309in}}%
\pgfpathlineto{\pgfqpoint{3.566287in}{1.551532in}}%
\pgfpathlineto{\pgfqpoint{3.566287in}{1.551532in}}%
\pgfpathlineto{\pgfqpoint{3.566287in}{1.551532in}}%
\pgfpathlineto{\pgfqpoint{3.566670in}{1.535235in}}%
\pgfpathlineto{\pgfqpoint{3.566670in}{1.535235in}}%
\pgfpathlineto{\pgfqpoint{3.566670in}{1.535235in}}%
\pgfpathlineto{\pgfqpoint{3.567437in}{1.567830in}}%
\pgfpathlineto{\pgfqpoint{3.567821in}{1.535235in}}%
\pgfpathlineto{\pgfqpoint{3.568588in}{1.555607in}}%
\pgfpathlineto{\pgfqpoint{3.568970in}{1.539309in}}%
\pgfpathlineto{\pgfqpoint{3.569736in}{1.543384in}}%
\pgfpathlineto{\pgfqpoint{3.570119in}{1.555607in}}%
\pgfpathlineto{\pgfqpoint{3.570502in}{1.543384in}}%
\pgfpathlineto{\pgfqpoint{3.570886in}{1.535235in}}%
\pgfpathlineto{\pgfqpoint{3.571652in}{1.539309in}}%
\pgfpathlineto{\pgfqpoint{3.572419in}{1.547458in}}%
\pgfpathlineto{\pgfqpoint{3.572803in}{1.531161in}}%
\pgfpathlineto{\pgfqpoint{3.573187in}{1.547458in}}%
\pgfpathlineto{\pgfqpoint{3.573570in}{1.551532in}}%
\pgfpathlineto{\pgfqpoint{3.575102in}{1.518937in}}%
\pgfpathlineto{\pgfqpoint{3.575868in}{1.551532in}}%
\pgfpathlineto{\pgfqpoint{3.576338in}{1.539309in}}%
\pgfpathlineto{\pgfqpoint{3.577104in}{1.551532in}}%
\pgfpathlineto{\pgfqpoint{3.577489in}{1.547458in}}%
\pgfpathlineto{\pgfqpoint{3.577872in}{1.531161in}}%
\pgfpathlineto{\pgfqpoint{3.577872in}{1.531161in}}%
\pgfpathlineto{\pgfqpoint{3.577872in}{1.531161in}}%
\pgfpathlineto{\pgfqpoint{3.579023in}{1.555607in}}%
\pgfpathlineto{\pgfqpoint{3.579790in}{1.531161in}}%
\pgfpathlineto{\pgfqpoint{3.580173in}{1.543384in}}%
\pgfpathlineto{\pgfqpoint{3.580556in}{1.551532in}}%
\pgfpathlineto{\pgfqpoint{3.580940in}{1.543384in}}%
\pgfpathlineto{\pgfqpoint{3.581707in}{1.531161in}}%
\pgfpathlineto{\pgfqpoint{3.582091in}{1.559681in}}%
\pgfpathlineto{\pgfqpoint{3.582857in}{1.551532in}}%
\pgfpathlineto{\pgfqpoint{3.583241in}{1.555607in}}%
\pgfpathlineto{\pgfqpoint{3.584008in}{1.539309in}}%
\pgfpathlineto{\pgfqpoint{3.584391in}{1.543384in}}%
\pgfpathlineto{\pgfqpoint{3.584774in}{1.551532in}}%
\pgfpathlineto{\pgfqpoint{3.585158in}{1.543384in}}%
\pgfpathlineto{\pgfqpoint{3.585542in}{1.535235in}}%
\pgfpathlineto{\pgfqpoint{3.585925in}{1.543384in}}%
\pgfpathlineto{\pgfqpoint{3.586308in}{1.551532in}}%
\pgfpathlineto{\pgfqpoint{3.586692in}{1.531161in}}%
\pgfpathlineto{\pgfqpoint{3.587460in}{1.535235in}}%
\pgfpathlineto{\pgfqpoint{3.587843in}{1.563756in}}%
\pgfpathlineto{\pgfqpoint{3.588610in}{1.547458in}}%
\pgfpathlineto{\pgfqpoint{3.588994in}{1.555607in}}%
\pgfpathlineto{\pgfqpoint{3.588994in}{1.555607in}}%
\pgfpathlineto{\pgfqpoint{3.588994in}{1.555607in}}%
\pgfpathlineto{\pgfqpoint{3.589378in}{1.543384in}}%
\pgfpathlineto{\pgfqpoint{3.590145in}{1.547458in}}%
\pgfpathlineto{\pgfqpoint{3.590529in}{1.555607in}}%
\pgfpathlineto{\pgfqpoint{3.590529in}{1.555607in}}%
\pgfpathlineto{\pgfqpoint{3.590529in}{1.555607in}}%
\pgfpathlineto{\pgfqpoint{3.591294in}{1.539309in}}%
\pgfpathlineto{\pgfqpoint{3.591678in}{1.547458in}}%
\pgfpathlineto{\pgfqpoint{3.592063in}{1.555607in}}%
\pgfpathlineto{\pgfqpoint{3.592446in}{1.535235in}}%
\pgfpathlineto{\pgfqpoint{3.592446in}{1.535235in}}%
\pgfpathlineto{\pgfqpoint{3.592446in}{1.535235in}}%
\pgfpathlineto{\pgfqpoint{3.593595in}{1.563756in}}%
\pgfpathlineto{\pgfqpoint{3.594363in}{1.539309in}}%
\pgfpathlineto{\pgfqpoint{3.595129in}{1.543384in}}%
\pgfpathlineto{\pgfqpoint{3.595895in}{1.555607in}}%
\pgfpathlineto{\pgfqpoint{3.596278in}{1.547458in}}%
\pgfpathlineto{\pgfqpoint{3.596662in}{1.543384in}}%
\pgfpathlineto{\pgfqpoint{3.597046in}{1.551532in}}%
\pgfpathlineto{\pgfqpoint{3.597046in}{1.551532in}}%
\pgfpathlineto{\pgfqpoint{3.597046in}{1.551532in}}%
\pgfpathlineto{\pgfqpoint{3.598579in}{1.535235in}}%
\pgfpathlineto{\pgfqpoint{3.599731in}{1.555607in}}%
\pgfpathlineto{\pgfqpoint{3.600496in}{1.531161in}}%
\pgfpathlineto{\pgfqpoint{3.600887in}{1.551532in}}%
\pgfpathlineto{\pgfqpoint{3.601271in}{1.551532in}}%
\pgfpathlineto{\pgfqpoint{3.602039in}{1.539309in}}%
\pgfpathlineto{\pgfqpoint{3.602421in}{1.555607in}}%
\pgfpathlineto{\pgfqpoint{3.602804in}{1.547458in}}%
\pgfpathlineto{\pgfqpoint{3.603187in}{1.531161in}}%
\pgfpathlineto{\pgfqpoint{3.603187in}{1.531161in}}%
\pgfpathlineto{\pgfqpoint{3.603187in}{1.531161in}}%
\pgfpathlineto{\pgfqpoint{3.604337in}{1.563756in}}%
\pgfpathlineto{\pgfqpoint{3.604721in}{1.531161in}}%
\pgfpathlineto{\pgfqpoint{3.604721in}{1.531161in}}%
\pgfpathlineto{\pgfqpoint{3.604721in}{1.531161in}}%
\pgfpathlineto{\pgfqpoint{3.605104in}{1.567830in}}%
\pgfpathlineto{\pgfqpoint{3.605871in}{1.543384in}}%
\pgfpathlineto{\pgfqpoint{3.606255in}{1.555607in}}%
\pgfpathlineto{\pgfqpoint{3.606638in}{1.551532in}}%
\pgfpathlineto{\pgfqpoint{3.607022in}{1.531161in}}%
\pgfpathlineto{\pgfqpoint{3.607788in}{1.543384in}}%
\pgfpathlineto{\pgfqpoint{3.608171in}{1.535235in}}%
\pgfpathlineto{\pgfqpoint{3.608937in}{1.551532in}}%
\pgfpathlineto{\pgfqpoint{3.609320in}{1.531161in}}%
\pgfpathlineto{\pgfqpoint{3.610086in}{1.543384in}}%
\pgfpathlineto{\pgfqpoint{3.610470in}{1.535235in}}%
\pgfpathlineto{\pgfqpoint{3.610853in}{1.551532in}}%
\pgfpathlineto{\pgfqpoint{3.611619in}{1.539309in}}%
\pgfpathlineto{\pgfqpoint{3.612771in}{1.551532in}}%
\pgfpathlineto{\pgfqpoint{3.613154in}{1.543384in}}%
\pgfpathlineto{\pgfqpoint{3.613537in}{1.551532in}}%
\pgfpathlineto{\pgfqpoint{3.614303in}{1.559681in}}%
\pgfpathlineto{\pgfqpoint{3.615454in}{1.551532in}}%
\pgfpathlineto{\pgfqpoint{3.615837in}{1.559681in}}%
\pgfpathlineto{\pgfqpoint{3.616219in}{1.555607in}}%
\pgfpathlineto{\pgfqpoint{3.617753in}{1.535235in}}%
\pgfpathlineto{\pgfqpoint{3.618136in}{1.535235in}}%
\pgfpathlineto{\pgfqpoint{3.619285in}{1.571904in}}%
\pgfpathlineto{\pgfqpoint{3.620819in}{1.531161in}}%
\pgfpathlineto{\pgfqpoint{3.621202in}{1.551532in}}%
\pgfpathlineto{\pgfqpoint{3.621969in}{1.543384in}}%
\pgfpathlineto{\pgfqpoint{3.622735in}{1.559681in}}%
\pgfpathlineto{\pgfqpoint{3.623587in}{1.531161in}}%
\pgfpathlineto{\pgfqpoint{3.624355in}{1.559681in}}%
\pgfpathlineto{\pgfqpoint{3.624739in}{1.555607in}}%
\pgfpathlineto{\pgfqpoint{3.625122in}{1.531161in}}%
\pgfpathlineto{\pgfqpoint{3.625506in}{1.551532in}}%
\pgfpathlineto{\pgfqpoint{3.625890in}{1.555607in}}%
\pgfpathlineto{\pgfqpoint{3.626657in}{1.527086in}}%
\pgfpathlineto{\pgfqpoint{3.627040in}{1.567830in}}%
\pgfpathlineto{\pgfqpoint{3.627806in}{1.535235in}}%
\pgfpathlineto{\pgfqpoint{3.628573in}{1.551532in}}%
\pgfpathlineto{\pgfqpoint{3.628956in}{1.531161in}}%
\pgfpathlineto{\pgfqpoint{3.629723in}{1.539309in}}%
\pgfpathlineto{\pgfqpoint{3.630489in}{1.559681in}}%
\pgfpathlineto{\pgfqpoint{3.630873in}{1.555607in}}%
\pgfpathlineto{\pgfqpoint{3.632406in}{1.535235in}}%
\pgfpathlineto{\pgfqpoint{3.632789in}{1.535235in}}%
\pgfpathlineto{\pgfqpoint{3.633173in}{1.523012in}}%
\pgfpathlineto{\pgfqpoint{3.633173in}{1.523012in}}%
\pgfpathlineto{\pgfqpoint{3.633173in}{1.523012in}}%
\pgfpathlineto{\pgfqpoint{3.634705in}{1.543384in}}%
\pgfpathlineto{\pgfqpoint{3.635088in}{1.547458in}}%
\pgfpathlineto{\pgfqpoint{3.635855in}{1.535235in}}%
\pgfpathlineto{\pgfqpoint{3.637389in}{1.559681in}}%
\pgfpathlineto{\pgfqpoint{3.638921in}{1.539309in}}%
\pgfpathlineto{\pgfqpoint{3.639305in}{1.555607in}}%
\pgfpathlineto{\pgfqpoint{3.640073in}{1.547458in}}%
\pgfpathlineto{\pgfqpoint{3.641615in}{1.531161in}}%
\pgfpathlineto{\pgfqpoint{3.642381in}{1.551532in}}%
\pgfpathlineto{\pgfqpoint{3.642764in}{1.535235in}}%
\pgfpathlineto{\pgfqpoint{3.644680in}{1.559681in}}%
\pgfpathlineto{\pgfqpoint{3.645446in}{1.539309in}}%
\pgfpathlineto{\pgfqpoint{3.645829in}{1.559681in}}%
\pgfpathlineto{\pgfqpoint{3.646595in}{1.551532in}}%
\pgfpathlineto{\pgfqpoint{3.646979in}{1.551532in}}%
\pgfpathlineto{\pgfqpoint{3.647363in}{1.543384in}}%
\pgfpathlineto{\pgfqpoint{3.647363in}{1.543384in}}%
\pgfpathlineto{\pgfqpoint{3.647363in}{1.543384in}}%
\pgfpathlineto{\pgfqpoint{3.647746in}{1.555607in}}%
\pgfpathlineto{\pgfqpoint{3.648129in}{1.551532in}}%
\pgfpathlineto{\pgfqpoint{3.649279in}{1.535235in}}%
\pgfpathlineto{\pgfqpoint{3.650046in}{1.559681in}}%
\pgfpathlineto{\pgfqpoint{3.650429in}{1.547458in}}%
\pgfpathlineto{\pgfqpoint{3.651196in}{1.535235in}}%
\pgfpathlineto{\pgfqpoint{3.651579in}{1.543384in}}%
\pgfpathlineto{\pgfqpoint{3.652730in}{1.559681in}}%
\pgfpathlineto{\pgfqpoint{3.653496in}{1.527086in}}%
\pgfpathlineto{\pgfqpoint{3.653880in}{1.543384in}}%
\pgfpathlineto{\pgfqpoint{3.655414in}{1.567830in}}%
\pgfpathlineto{\pgfqpoint{3.655798in}{1.559681in}}%
\pgfpathlineto{\pgfqpoint{3.655798in}{1.559681in}}%
\pgfpathlineto{\pgfqpoint{3.655798in}{1.559681in}}%
\pgfpathlineto{\pgfqpoint{3.656180in}{1.571904in}}%
\pgfpathlineto{\pgfqpoint{3.656180in}{1.571904in}}%
\pgfpathlineto{\pgfqpoint{3.656180in}{1.571904in}}%
\pgfpathlineto{\pgfqpoint{3.656947in}{1.539309in}}%
\pgfpathlineto{\pgfqpoint{3.657715in}{1.551532in}}%
\pgfpathlineto{\pgfqpoint{3.658481in}{1.543384in}}%
\pgfpathlineto{\pgfqpoint{3.658864in}{1.555607in}}%
\pgfpathlineto{\pgfqpoint{3.659247in}{1.547458in}}%
\pgfpathlineto{\pgfqpoint{3.659630in}{1.527086in}}%
\pgfpathlineto{\pgfqpoint{3.660015in}{1.535235in}}%
\pgfpathlineto{\pgfqpoint{3.660398in}{1.571904in}}%
\pgfpathlineto{\pgfqpoint{3.661163in}{1.551532in}}%
\pgfpathlineto{\pgfqpoint{3.661546in}{1.543384in}}%
\pgfpathlineto{\pgfqpoint{3.661929in}{1.547458in}}%
\pgfpathlineto{\pgfqpoint{3.662313in}{1.555607in}}%
\pgfpathlineto{\pgfqpoint{3.663464in}{1.527086in}}%
\pgfpathlineto{\pgfqpoint{3.663846in}{1.535235in}}%
\pgfpathlineto{\pgfqpoint{3.664229in}{1.531161in}}%
\pgfpathlineto{\pgfqpoint{3.665764in}{1.555607in}}%
\pgfpathlineto{\pgfqpoint{3.666147in}{1.551532in}}%
\pgfpathlineto{\pgfqpoint{3.667296in}{1.535235in}}%
\pgfpathlineto{\pgfqpoint{3.667679in}{1.555607in}}%
\pgfpathlineto{\pgfqpoint{3.668147in}{1.547458in}}%
\pgfpathlineto{\pgfqpoint{3.668530in}{1.527086in}}%
\pgfpathlineto{\pgfqpoint{3.668530in}{1.527086in}}%
\pgfpathlineto{\pgfqpoint{3.668530in}{1.527086in}}%
\pgfpathlineto{\pgfqpoint{3.668914in}{1.563756in}}%
\pgfpathlineto{\pgfqpoint{3.669681in}{1.543384in}}%
\pgfpathlineto{\pgfqpoint{3.670064in}{1.518937in}}%
\pgfpathlineto{\pgfqpoint{3.670829in}{1.535235in}}%
\pgfpathlineto{\pgfqpoint{3.671212in}{1.547458in}}%
\pgfpathlineto{\pgfqpoint{3.671212in}{1.547458in}}%
\pgfpathlineto{\pgfqpoint{3.671212in}{1.547458in}}%
\pgfpathlineto{\pgfqpoint{3.671596in}{1.531161in}}%
\pgfpathlineto{\pgfqpoint{3.671596in}{1.531161in}}%
\pgfpathlineto{\pgfqpoint{3.671596in}{1.531161in}}%
\pgfpathlineto{\pgfqpoint{3.671980in}{1.551532in}}%
\pgfpathlineto{\pgfqpoint{3.672747in}{1.547458in}}%
\pgfpathlineto{\pgfqpoint{3.673130in}{1.543384in}}%
\pgfpathlineto{\pgfqpoint{3.673513in}{1.555607in}}%
\pgfpathlineto{\pgfqpoint{3.673513in}{1.555607in}}%
\pgfpathlineto{\pgfqpoint{3.673513in}{1.555607in}}%
\pgfpathlineto{\pgfqpoint{3.673897in}{1.539309in}}%
\pgfpathlineto{\pgfqpoint{3.674664in}{1.551532in}}%
\pgfpathlineto{\pgfqpoint{3.675430in}{1.543384in}}%
\pgfpathlineto{\pgfqpoint{3.675813in}{1.555607in}}%
\pgfpathlineto{\pgfqpoint{3.676196in}{1.543384in}}%
\pgfpathlineto{\pgfqpoint{3.676580in}{1.543384in}}%
\pgfpathlineto{\pgfqpoint{3.676963in}{1.555607in}}%
\pgfpathlineto{\pgfqpoint{3.677730in}{1.551532in}}%
\pgfpathlineto{\pgfqpoint{3.678113in}{1.527086in}}%
\pgfpathlineto{\pgfqpoint{3.678879in}{1.539309in}}%
\pgfpathlineto{\pgfqpoint{3.679645in}{1.535235in}}%
\pgfpathlineto{\pgfqpoint{3.680803in}{1.559681in}}%
\pgfpathlineto{\pgfqpoint{3.681187in}{1.551532in}}%
\pgfpathlineto{\pgfqpoint{3.681953in}{1.531161in}}%
\pgfpathlineto{\pgfqpoint{3.682336in}{1.543384in}}%
\pgfpathlineto{\pgfqpoint{3.683101in}{1.531161in}}%
\pgfpathlineto{\pgfqpoint{3.684250in}{1.563756in}}%
\pgfpathlineto{\pgfqpoint{3.684634in}{1.543384in}}%
\pgfpathlineto{\pgfqpoint{3.685401in}{1.547458in}}%
\pgfpathlineto{\pgfqpoint{3.685784in}{1.555607in}}%
\pgfpathlineto{\pgfqpoint{3.686167in}{1.535235in}}%
\pgfpathlineto{\pgfqpoint{3.686550in}{1.543384in}}%
\pgfpathlineto{\pgfqpoint{3.686933in}{1.563756in}}%
\pgfpathlineto{\pgfqpoint{3.687701in}{1.547458in}}%
\pgfpathlineto{\pgfqpoint{3.688850in}{1.559681in}}%
\pgfpathlineto{\pgfqpoint{3.690000in}{1.547458in}}%
\pgfpathlineto{\pgfqpoint{3.690383in}{1.559681in}}%
\pgfpathlineto{\pgfqpoint{3.690765in}{1.555607in}}%
\pgfpathlineto{\pgfqpoint{3.692300in}{1.527086in}}%
\pgfpathlineto{\pgfqpoint{3.692683in}{1.563756in}}%
\pgfpathlineto{\pgfqpoint{3.693448in}{1.543384in}}%
\pgfpathlineto{\pgfqpoint{3.693832in}{1.543384in}}%
\pgfpathlineto{\pgfqpoint{3.694215in}{1.535235in}}%
\pgfpathlineto{\pgfqpoint{3.694215in}{1.535235in}}%
\pgfpathlineto{\pgfqpoint{3.694215in}{1.535235in}}%
\pgfpathlineto{\pgfqpoint{3.694981in}{1.551532in}}%
\pgfpathlineto{\pgfqpoint{3.695365in}{1.535235in}}%
\pgfpathlineto{\pgfqpoint{3.696131in}{1.547458in}}%
\pgfpathlineto{\pgfqpoint{3.696515in}{1.543384in}}%
\pgfpathlineto{\pgfqpoint{3.696899in}{1.547458in}}%
\pgfpathlineto{\pgfqpoint{3.697282in}{1.555607in}}%
\pgfpathlineto{\pgfqpoint{3.698433in}{1.535235in}}%
\pgfpathlineto{\pgfqpoint{3.699200in}{1.555607in}}%
\pgfpathlineto{\pgfqpoint{3.699966in}{1.535235in}}%
\pgfpathlineto{\pgfqpoint{3.700349in}{1.539309in}}%
\pgfpathlineto{\pgfqpoint{3.700734in}{1.539309in}}%
\pgfpathlineto{\pgfqpoint{3.701117in}{1.551532in}}%
\pgfpathlineto{\pgfqpoint{3.701884in}{1.547458in}}%
\pgfpathlineto{\pgfqpoint{3.702266in}{1.547458in}}%
\pgfpathlineto{\pgfqpoint{3.702649in}{1.539309in}}%
\pgfpathlineto{\pgfqpoint{3.703034in}{1.543384in}}%
\pgfpathlineto{\pgfqpoint{3.703417in}{1.555607in}}%
\pgfpathlineto{\pgfqpoint{3.703801in}{1.527086in}}%
\pgfpathlineto{\pgfqpoint{3.704567in}{1.551532in}}%
\pgfpathlineto{\pgfqpoint{3.704950in}{1.555607in}}%
\pgfpathlineto{\pgfqpoint{3.706100in}{1.539309in}}%
\pgfpathlineto{\pgfqpoint{3.706484in}{1.559681in}}%
\pgfpathlineto{\pgfqpoint{3.707250in}{1.547458in}}%
\pgfpathlineto{\pgfqpoint{3.707634in}{1.551532in}}%
\pgfpathlineto{\pgfqpoint{3.708400in}{1.539309in}}%
\pgfpathlineto{\pgfqpoint{3.708783in}{1.567830in}}%
\pgfpathlineto{\pgfqpoint{3.709167in}{1.551532in}}%
\pgfpathlineto{\pgfqpoint{3.709551in}{1.531161in}}%
\pgfpathlineto{\pgfqpoint{3.709551in}{1.531161in}}%
\pgfpathlineto{\pgfqpoint{3.709551in}{1.531161in}}%
\pgfpathlineto{\pgfqpoint{3.710020in}{1.563756in}}%
\pgfpathlineto{\pgfqpoint{3.710787in}{1.551532in}}%
\pgfpathlineto{\pgfqpoint{3.711170in}{1.551532in}}%
\pgfpathlineto{\pgfqpoint{3.711553in}{1.523012in}}%
\pgfpathlineto{\pgfqpoint{3.712320in}{1.543384in}}%
\pgfpathlineto{\pgfqpoint{3.712704in}{1.535235in}}%
\pgfpathlineto{\pgfqpoint{3.713470in}{1.539309in}}%
\pgfpathlineto{\pgfqpoint{3.715001in}{1.563756in}}%
\pgfpathlineto{\pgfqpoint{3.716150in}{1.518937in}}%
\pgfpathlineto{\pgfqpoint{3.717299in}{1.555607in}}%
\pgfpathlineto{\pgfqpoint{3.717683in}{1.535235in}}%
\pgfpathlineto{\pgfqpoint{3.717683in}{1.535235in}}%
\pgfpathlineto{\pgfqpoint{3.717683in}{1.535235in}}%
\pgfpathlineto{\pgfqpoint{3.718065in}{1.559681in}}%
\pgfpathlineto{\pgfqpoint{3.718831in}{1.539309in}}%
\pgfpathlineto{\pgfqpoint{3.719213in}{1.547458in}}%
\pgfpathlineto{\pgfqpoint{3.719597in}{1.531161in}}%
\pgfpathlineto{\pgfqpoint{3.720364in}{1.539309in}}%
\pgfpathlineto{\pgfqpoint{3.720755in}{1.559681in}}%
\pgfpathlineto{\pgfqpoint{3.721139in}{1.551532in}}%
\pgfpathlineto{\pgfqpoint{3.721521in}{1.539309in}}%
\pgfpathlineto{\pgfqpoint{3.721905in}{1.551532in}}%
\pgfpathlineto{\pgfqpoint{3.723055in}{1.563756in}}%
\pgfpathlineto{\pgfqpoint{3.723439in}{1.535235in}}%
\pgfpathlineto{\pgfqpoint{3.724204in}{1.539309in}}%
\pgfpathlineto{\pgfqpoint{3.725740in}{1.555607in}}%
\pgfpathlineto{\pgfqpoint{3.726123in}{1.547458in}}%
\pgfpathlineto{\pgfqpoint{3.726505in}{1.571904in}}%
\pgfpathlineto{\pgfqpoint{3.727271in}{1.555607in}}%
\pgfpathlineto{\pgfqpoint{3.727655in}{1.543384in}}%
\pgfpathlineto{\pgfqpoint{3.728039in}{1.551532in}}%
\pgfpathlineto{\pgfqpoint{3.728423in}{1.567830in}}%
\pgfpathlineto{\pgfqpoint{3.728806in}{1.563756in}}%
\pgfpathlineto{\pgfqpoint{3.729571in}{1.539309in}}%
\pgfpathlineto{\pgfqpoint{3.729954in}{1.551532in}}%
\pgfpathlineto{\pgfqpoint{3.731106in}{1.535235in}}%
\pgfpathlineto{\pgfqpoint{3.731489in}{1.551532in}}%
\pgfpathlineto{\pgfqpoint{3.732254in}{1.539309in}}%
\pgfpathlineto{\pgfqpoint{3.733022in}{1.551532in}}%
\pgfpathlineto{\pgfqpoint{3.733405in}{1.535235in}}%
\pgfpathlineto{\pgfqpoint{3.733788in}{1.551532in}}%
\pgfpathlineto{\pgfqpoint{3.734171in}{1.559681in}}%
\pgfpathlineto{\pgfqpoint{3.734171in}{1.559681in}}%
\pgfpathlineto{\pgfqpoint{3.734171in}{1.559681in}}%
\pgfpathlineto{\pgfqpoint{3.734554in}{1.547458in}}%
\pgfpathlineto{\pgfqpoint{3.734937in}{1.559681in}}%
\pgfpathlineto{\pgfqpoint{3.735321in}{1.559681in}}%
\pgfpathlineto{\pgfqpoint{3.735705in}{1.531161in}}%
\pgfpathlineto{\pgfqpoint{3.736472in}{1.543384in}}%
\pgfpathlineto{\pgfqpoint{3.736854in}{1.535235in}}%
\pgfpathlineto{\pgfqpoint{3.736854in}{1.535235in}}%
\pgfpathlineto{\pgfqpoint{3.736854in}{1.535235in}}%
\pgfpathlineto{\pgfqpoint{3.737619in}{1.563756in}}%
\pgfpathlineto{\pgfqpoint{3.738002in}{1.551532in}}%
\pgfpathlineto{\pgfqpoint{3.738385in}{1.535235in}}%
\pgfpathlineto{\pgfqpoint{3.738385in}{1.535235in}}%
\pgfpathlineto{\pgfqpoint{3.738385in}{1.535235in}}%
\pgfpathlineto{\pgfqpoint{3.738769in}{1.555607in}}%
\pgfpathlineto{\pgfqpoint{3.739536in}{1.551532in}}%
\pgfpathlineto{\pgfqpoint{3.740302in}{1.531161in}}%
\pgfpathlineto{\pgfqpoint{3.740686in}{1.547458in}}%
\pgfpathlineto{\pgfqpoint{3.741069in}{1.547458in}}%
\pgfpathlineto{\pgfqpoint{3.741453in}{1.539309in}}%
\pgfpathlineto{\pgfqpoint{3.742220in}{1.563756in}}%
\pgfpathlineto{\pgfqpoint{3.742986in}{1.559681in}}%
\pgfpathlineto{\pgfqpoint{3.744519in}{1.535235in}}%
\pgfpathlineto{\pgfqpoint{3.744902in}{1.535235in}}%
\pgfpathlineto{\pgfqpoint{3.745284in}{1.543384in}}%
\pgfpathlineto{\pgfqpoint{3.745669in}{1.523012in}}%
\pgfpathlineto{\pgfqpoint{3.746052in}{1.531161in}}%
\pgfpathlineto{\pgfqpoint{3.747203in}{1.551532in}}%
\pgfpathlineto{\pgfqpoint{3.747586in}{1.551532in}}%
\pgfpathlineto{\pgfqpoint{3.747968in}{1.555607in}}%
\pgfpathlineto{\pgfqpoint{3.748351in}{1.543384in}}%
\pgfpathlineto{\pgfqpoint{3.749117in}{1.551532in}}%
\pgfpathlineto{\pgfqpoint{3.749501in}{1.567830in}}%
\pgfpathlineto{\pgfqpoint{3.751122in}{1.535235in}}%
\pgfpathlineto{\pgfqpoint{3.751504in}{1.539309in}}%
\pgfpathlineto{\pgfqpoint{3.751887in}{1.527086in}}%
\pgfpathlineto{\pgfqpoint{3.752270in}{1.567830in}}%
\pgfpathlineto{\pgfqpoint{3.753037in}{1.555607in}}%
\pgfpathlineto{\pgfqpoint{3.753421in}{1.539309in}}%
\pgfpathlineto{\pgfqpoint{3.754187in}{1.543384in}}%
\pgfpathlineto{\pgfqpoint{3.755336in}{1.527086in}}%
\pgfpathlineto{\pgfqpoint{3.756868in}{1.547458in}}%
\pgfpathlineto{\pgfqpoint{3.757251in}{1.547458in}}%
\pgfpathlineto{\pgfqpoint{3.757634in}{1.551532in}}%
\pgfpathlineto{\pgfqpoint{3.758401in}{1.531161in}}%
\pgfpathlineto{\pgfqpoint{3.758784in}{1.539309in}}%
\pgfpathlineto{\pgfqpoint{3.759167in}{1.551532in}}%
\pgfpathlineto{\pgfqpoint{3.759549in}{1.543384in}}%
\pgfpathlineto{\pgfqpoint{3.759932in}{1.539309in}}%
\pgfpathlineto{\pgfqpoint{3.760316in}{1.543384in}}%
\pgfpathlineto{\pgfqpoint{3.760707in}{1.551532in}}%
\pgfpathlineto{\pgfqpoint{3.760707in}{1.551532in}}%
\pgfpathlineto{\pgfqpoint{3.760707in}{1.551532in}}%
\pgfpathlineto{\pgfqpoint{3.761090in}{1.539309in}}%
\pgfpathlineto{\pgfqpoint{3.761856in}{1.547458in}}%
\pgfpathlineto{\pgfqpoint{3.762241in}{1.543384in}}%
\pgfpathlineto{\pgfqpoint{3.762625in}{1.551532in}}%
\pgfpathlineto{\pgfqpoint{3.762625in}{1.551532in}}%
\pgfpathlineto{\pgfqpoint{3.762625in}{1.551532in}}%
\pgfpathlineto{\pgfqpoint{3.763774in}{1.523012in}}%
\pgfpathlineto{\pgfqpoint{3.764924in}{1.551532in}}%
\pgfpathlineto{\pgfqpoint{3.766073in}{1.535235in}}%
\pgfpathlineto{\pgfqpoint{3.766455in}{1.563756in}}%
\pgfpathlineto{\pgfqpoint{3.767221in}{1.547458in}}%
\pgfpathlineto{\pgfqpoint{3.767988in}{1.527086in}}%
\pgfpathlineto{\pgfqpoint{3.769520in}{1.559681in}}%
\pgfpathlineto{\pgfqpoint{3.770668in}{1.539309in}}%
\pgfpathlineto{\pgfqpoint{3.771818in}{1.563756in}}%
\pgfpathlineto{\pgfqpoint{3.772966in}{1.527086in}}%
\pgfpathlineto{\pgfqpoint{3.774498in}{1.551532in}}%
\pgfpathlineto{\pgfqpoint{3.775648in}{1.531161in}}%
\pgfpathlineto{\pgfqpoint{3.776032in}{1.559681in}}%
\pgfpathlineto{\pgfqpoint{3.776798in}{1.539309in}}%
\pgfpathlineto{\pgfqpoint{3.777182in}{1.539309in}}%
\pgfpathlineto{\pgfqpoint{3.778332in}{1.563756in}}%
\pgfpathlineto{\pgfqpoint{3.779480in}{1.531161in}}%
\pgfpathlineto{\pgfqpoint{3.779863in}{1.551532in}}%
\pgfpathlineto{\pgfqpoint{3.780631in}{1.547458in}}%
\pgfpathlineto{\pgfqpoint{3.781014in}{1.531161in}}%
\pgfpathlineto{\pgfqpoint{3.781397in}{1.543384in}}%
\pgfpathlineto{\pgfqpoint{3.781780in}{1.559681in}}%
\pgfpathlineto{\pgfqpoint{3.781780in}{1.559681in}}%
\pgfpathlineto{\pgfqpoint{3.781780in}{1.559681in}}%
\pgfpathlineto{\pgfqpoint{3.782163in}{1.531161in}}%
\pgfpathlineto{\pgfqpoint{3.782931in}{1.543384in}}%
\pgfpathlineto{\pgfqpoint{3.783315in}{1.539309in}}%
\pgfpathlineto{\pgfqpoint{3.783696in}{1.547458in}}%
\pgfpathlineto{\pgfqpoint{3.784078in}{1.543384in}}%
\pgfpathlineto{\pgfqpoint{3.784461in}{1.535235in}}%
\pgfpathlineto{\pgfqpoint{3.784461in}{1.535235in}}%
\pgfpathlineto{\pgfqpoint{3.784461in}{1.535235in}}%
\pgfpathlineto{\pgfqpoint{3.785611in}{1.551532in}}%
\pgfpathlineto{\pgfqpoint{3.785994in}{1.535235in}}%
\pgfpathlineto{\pgfqpoint{3.786846in}{1.539309in}}%
\pgfpathlineto{\pgfqpoint{3.787230in}{1.563756in}}%
\pgfpathlineto{\pgfqpoint{3.787997in}{1.559681in}}%
\pgfpathlineto{\pgfqpoint{3.788380in}{1.531161in}}%
\pgfpathlineto{\pgfqpoint{3.788380in}{1.531161in}}%
\pgfpathlineto{\pgfqpoint{3.788380in}{1.531161in}}%
\pgfpathlineto{\pgfqpoint{3.788763in}{1.563756in}}%
\pgfpathlineto{\pgfqpoint{3.789530in}{1.551532in}}%
\pgfpathlineto{\pgfqpoint{3.789914in}{1.543384in}}%
\pgfpathlineto{\pgfqpoint{3.790680in}{1.547458in}}%
\pgfpathlineto{\pgfqpoint{3.791063in}{1.543384in}}%
\pgfpathlineto{\pgfqpoint{3.791446in}{1.547458in}}%
\pgfpathlineto{\pgfqpoint{3.791830in}{1.555607in}}%
\pgfpathlineto{\pgfqpoint{3.791830in}{1.555607in}}%
\pgfpathlineto{\pgfqpoint{3.791830in}{1.555607in}}%
\pgfpathlineto{\pgfqpoint{3.793362in}{1.510789in}}%
\pgfpathlineto{\pgfqpoint{3.794128in}{1.559681in}}%
\pgfpathlineto{\pgfqpoint{3.794512in}{1.547458in}}%
\pgfpathlineto{\pgfqpoint{3.794896in}{1.531161in}}%
\pgfpathlineto{\pgfqpoint{3.795279in}{1.547458in}}%
\pgfpathlineto{\pgfqpoint{3.795662in}{1.563756in}}%
\pgfpathlineto{\pgfqpoint{3.795662in}{1.563756in}}%
\pgfpathlineto{\pgfqpoint{3.795662in}{1.563756in}}%
\pgfpathlineto{\pgfqpoint{3.797193in}{1.535235in}}%
\pgfpathlineto{\pgfqpoint{3.797577in}{1.555607in}}%
\pgfpathlineto{\pgfqpoint{3.797577in}{1.555607in}}%
\pgfpathlineto{\pgfqpoint{3.797577in}{1.555607in}}%
\pgfpathlineto{\pgfqpoint{3.798342in}{1.531161in}}%
\pgfpathlineto{\pgfqpoint{3.798726in}{1.559681in}}%
\pgfpathlineto{\pgfqpoint{3.799492in}{1.539309in}}%
\pgfpathlineto{\pgfqpoint{3.799875in}{1.535235in}}%
\pgfpathlineto{\pgfqpoint{3.801799in}{1.571904in}}%
\pgfpathlineto{\pgfqpoint{3.802949in}{1.539309in}}%
\pgfpathlineto{\pgfqpoint{3.803333in}{1.547458in}}%
\pgfpathlineto{\pgfqpoint{3.804099in}{1.527086in}}%
\pgfpathlineto{\pgfqpoint{3.804482in}{1.543384in}}%
\pgfpathlineto{\pgfqpoint{3.805633in}{1.555607in}}%
\pgfpathlineto{\pgfqpoint{3.806782in}{1.527086in}}%
\pgfpathlineto{\pgfqpoint{3.808316in}{1.555607in}}%
\pgfpathlineto{\pgfqpoint{3.808699in}{1.539309in}}%
\pgfpathlineto{\pgfqpoint{3.808699in}{1.539309in}}%
\pgfpathlineto{\pgfqpoint{3.808699in}{1.539309in}}%
\pgfpathlineto{\pgfqpoint{3.809082in}{1.567830in}}%
\pgfpathlineto{\pgfqpoint{3.809082in}{1.567830in}}%
\pgfpathlineto{\pgfqpoint{3.809082in}{1.567830in}}%
\pgfpathlineto{\pgfqpoint{3.809465in}{1.535235in}}%
\pgfpathlineto{\pgfqpoint{3.810232in}{1.555607in}}%
\pgfpathlineto{\pgfqpoint{3.811382in}{1.543384in}}%
\pgfpathlineto{\pgfqpoint{3.812148in}{1.563756in}}%
\pgfpathlineto{\pgfqpoint{3.812916in}{1.555607in}}%
\pgfpathlineto{\pgfqpoint{3.813683in}{1.555607in}}%
\pgfpathlineto{\pgfqpoint{3.814065in}{1.559681in}}%
\pgfpathlineto{\pgfqpoint{3.814832in}{1.535235in}}%
\pgfpathlineto{\pgfqpoint{3.815216in}{1.551532in}}%
\pgfpathlineto{\pgfqpoint{3.815600in}{1.571904in}}%
\pgfpathlineto{\pgfqpoint{3.815983in}{1.551532in}}%
\pgfpathlineto{\pgfqpoint{3.816368in}{1.535235in}}%
\pgfpathlineto{\pgfqpoint{3.816368in}{1.535235in}}%
\pgfpathlineto{\pgfqpoint{3.816368in}{1.535235in}}%
\pgfpathlineto{\pgfqpoint{3.817518in}{1.571904in}}%
\pgfpathlineto{\pgfqpoint{3.817902in}{1.547458in}}%
\pgfpathlineto{\pgfqpoint{3.818668in}{1.555607in}}%
\pgfpathlineto{\pgfqpoint{3.819818in}{1.547458in}}%
\pgfpathlineto{\pgfqpoint{3.820202in}{1.559681in}}%
\pgfpathlineto{\pgfqpoint{3.820585in}{1.551532in}}%
\pgfpathlineto{\pgfqpoint{3.820969in}{1.547458in}}%
\pgfpathlineto{\pgfqpoint{3.822501in}{1.567830in}}%
\pgfpathlineto{\pgfqpoint{3.824035in}{1.531161in}}%
\pgfpathlineto{\pgfqpoint{3.824419in}{1.531161in}}%
\pgfpathlineto{\pgfqpoint{3.825952in}{1.559681in}}%
\pgfpathlineto{\pgfqpoint{3.826336in}{1.535235in}}%
\pgfpathlineto{\pgfqpoint{3.827102in}{1.551532in}}%
\pgfpathlineto{\pgfqpoint{3.827485in}{1.567830in}}%
\pgfpathlineto{\pgfqpoint{3.827485in}{1.567830in}}%
\pgfpathlineto{\pgfqpoint{3.827485in}{1.567830in}}%
\pgfpathlineto{\pgfqpoint{3.828636in}{1.539309in}}%
\pgfpathlineto{\pgfqpoint{3.829403in}{1.555607in}}%
\pgfpathlineto{\pgfqpoint{3.830253in}{1.527086in}}%
\pgfpathlineto{\pgfqpoint{3.830636in}{1.551532in}}%
\pgfpathlineto{\pgfqpoint{3.831020in}{1.543384in}}%
\pgfpathlineto{\pgfqpoint{3.831787in}{1.547458in}}%
\pgfpathlineto{\pgfqpoint{3.832936in}{1.543384in}}%
\pgfpathlineto{\pgfqpoint{3.833320in}{1.563756in}}%
\pgfpathlineto{\pgfqpoint{3.833320in}{1.563756in}}%
\pgfpathlineto{\pgfqpoint{3.833320in}{1.563756in}}%
\pgfpathlineto{\pgfqpoint{3.833703in}{1.531161in}}%
\pgfpathlineto{\pgfqpoint{3.834471in}{1.543384in}}%
\pgfpathlineto{\pgfqpoint{3.834854in}{1.567830in}}%
\pgfpathlineto{\pgfqpoint{3.835237in}{1.559681in}}%
\pgfpathlineto{\pgfqpoint{3.836005in}{1.535235in}}%
\pgfpathlineto{\pgfqpoint{3.837155in}{1.559681in}}%
\pgfpathlineto{\pgfqpoint{3.837538in}{1.559681in}}%
\pgfpathlineto{\pgfqpoint{3.839070in}{1.531161in}}%
\pgfpathlineto{\pgfqpoint{3.840221in}{1.555607in}}%
\pgfpathlineto{\pgfqpoint{3.840611in}{1.547458in}}%
\pgfpathlineto{\pgfqpoint{3.840994in}{1.547458in}}%
\pgfpathlineto{\pgfqpoint{3.841377in}{1.543384in}}%
\pgfpathlineto{\pgfqpoint{3.841761in}{1.547458in}}%
\pgfpathlineto{\pgfqpoint{3.842145in}{1.547458in}}%
\pgfpathlineto{\pgfqpoint{3.842529in}{1.535235in}}%
\pgfpathlineto{\pgfqpoint{3.842529in}{1.535235in}}%
\pgfpathlineto{\pgfqpoint{3.842529in}{1.535235in}}%
\pgfpathlineto{\pgfqpoint{3.842911in}{1.555607in}}%
\pgfpathlineto{\pgfqpoint{3.843678in}{1.539309in}}%
\pgfpathlineto{\pgfqpoint{3.844062in}{1.547458in}}%
\pgfpathlineto{\pgfqpoint{3.844446in}{1.539309in}}%
\pgfpathlineto{\pgfqpoint{3.845213in}{1.539309in}}%
\pgfpathlineto{\pgfqpoint{3.845978in}{1.555607in}}%
\pgfpathlineto{\pgfqpoint{3.846745in}{1.551532in}}%
\pgfpathlineto{\pgfqpoint{3.847513in}{1.531161in}}%
\pgfpathlineto{\pgfqpoint{3.847896in}{1.543384in}}%
\pgfpathlineto{\pgfqpoint{3.848661in}{1.567830in}}%
\pgfpathlineto{\pgfqpoint{3.849812in}{1.535235in}}%
\pgfpathlineto{\pgfqpoint{3.850196in}{1.555607in}}%
\pgfpathlineto{\pgfqpoint{3.850962in}{1.539309in}}%
\pgfpathlineto{\pgfqpoint{3.851345in}{1.555607in}}%
\pgfpathlineto{\pgfqpoint{3.852114in}{1.543384in}}%
\pgfpathlineto{\pgfqpoint{3.853647in}{1.555607in}}%
\pgfpathlineto{\pgfqpoint{3.854030in}{1.547458in}}%
\pgfpathlineto{\pgfqpoint{3.854414in}{1.555607in}}%
\pgfpathlineto{\pgfqpoint{3.854797in}{1.563756in}}%
\pgfpathlineto{\pgfqpoint{3.855947in}{1.531161in}}%
\pgfpathlineto{\pgfqpoint{3.857097in}{1.563756in}}%
\pgfpathlineto{\pgfqpoint{3.857480in}{1.555607in}}%
\pgfpathlineto{\pgfqpoint{3.857863in}{1.551532in}}%
\pgfpathlineto{\pgfqpoint{3.858246in}{1.531161in}}%
\pgfpathlineto{\pgfqpoint{3.859016in}{1.547458in}}%
\pgfpathlineto{\pgfqpoint{3.859400in}{1.543384in}}%
\pgfpathlineto{\pgfqpoint{3.859783in}{1.555607in}}%
\pgfpathlineto{\pgfqpoint{3.860166in}{1.547458in}}%
\pgfpathlineto{\pgfqpoint{3.860550in}{1.535235in}}%
\pgfpathlineto{\pgfqpoint{3.860933in}{1.559681in}}%
\pgfpathlineto{\pgfqpoint{3.861699in}{1.539309in}}%
\pgfpathlineto{\pgfqpoint{3.862083in}{1.535235in}}%
\pgfpathlineto{\pgfqpoint{3.863616in}{1.559681in}}%
\pgfpathlineto{\pgfqpoint{3.863999in}{1.555607in}}%
\pgfpathlineto{\pgfqpoint{3.865534in}{1.531161in}}%
\pgfpathlineto{\pgfqpoint{3.865918in}{1.559681in}}%
\pgfpathlineto{\pgfqpoint{3.866684in}{1.555607in}}%
\pgfpathlineto{\pgfqpoint{3.868219in}{1.531161in}}%
\pgfpathlineto{\pgfqpoint{3.869367in}{1.559681in}}%
\pgfpathlineto{\pgfqpoint{3.870519in}{1.543384in}}%
\pgfpathlineto{\pgfqpoint{3.870902in}{1.518937in}}%
\pgfpathlineto{\pgfqpoint{3.870902in}{1.518937in}}%
\pgfpathlineto{\pgfqpoint{3.870902in}{1.518937in}}%
\pgfpathlineto{\pgfqpoint{3.871668in}{1.563756in}}%
\pgfpathlineto{\pgfqpoint{3.872051in}{1.510789in}}%
\pgfpathlineto{\pgfqpoint{3.872819in}{1.551532in}}%
\pgfpathlineto{\pgfqpoint{3.873585in}{1.539309in}}%
\pgfpathlineto{\pgfqpoint{3.874352in}{1.555607in}}%
\pgfpathlineto{\pgfqpoint{3.875118in}{1.539309in}}%
\pgfpathlineto{\pgfqpoint{3.875502in}{1.543384in}}%
\pgfpathlineto{\pgfqpoint{3.876269in}{1.580053in}}%
\pgfpathlineto{\pgfqpoint{3.877119in}{1.559681in}}%
\pgfpathlineto{\pgfqpoint{3.877501in}{1.559681in}}%
\pgfpathlineto{\pgfqpoint{3.878652in}{1.535235in}}%
\pgfpathlineto{\pgfqpoint{3.880186in}{1.563756in}}%
\pgfpathlineto{\pgfqpoint{3.880961in}{1.535235in}}%
\pgfpathlineto{\pgfqpoint{3.881727in}{1.539309in}}%
\pgfpathlineto{\pgfqpoint{3.882493in}{1.539309in}}%
\pgfpathlineto{\pgfqpoint{3.882876in}{1.535235in}}%
\pgfpathlineto{\pgfqpoint{3.884027in}{1.547458in}}%
\pgfpathlineto{\pgfqpoint{3.884410in}{1.539309in}}%
\pgfpathlineto{\pgfqpoint{3.884795in}{1.543384in}}%
\pgfpathlineto{\pgfqpoint{3.885561in}{1.551532in}}%
\pgfpathlineto{\pgfqpoint{3.885946in}{1.547458in}}%
\pgfpathlineto{\pgfqpoint{3.886330in}{1.523012in}}%
\pgfpathlineto{\pgfqpoint{3.887096in}{1.539309in}}%
\pgfpathlineto{\pgfqpoint{3.887479in}{1.559681in}}%
\pgfpathlineto{\pgfqpoint{3.887861in}{1.514863in}}%
\pgfpathlineto{\pgfqpoint{3.888246in}{1.547458in}}%
\pgfpathlineto{\pgfqpoint{3.888629in}{1.563756in}}%
\pgfpathlineto{\pgfqpoint{3.889013in}{1.547458in}}%
\pgfpathlineto{\pgfqpoint{3.889396in}{1.535235in}}%
\pgfpathlineto{\pgfqpoint{3.889396in}{1.535235in}}%
\pgfpathlineto{\pgfqpoint{3.889396in}{1.535235in}}%
\pgfpathlineto{\pgfqpoint{3.889779in}{1.551532in}}%
\pgfpathlineto{\pgfqpoint{3.890163in}{1.539309in}}%
\pgfpathlineto{\pgfqpoint{3.890545in}{1.535235in}}%
\pgfpathlineto{\pgfqpoint{3.890929in}{1.551532in}}%
\pgfpathlineto{\pgfqpoint{3.890929in}{1.551532in}}%
\pgfpathlineto{\pgfqpoint{3.890929in}{1.551532in}}%
\pgfpathlineto{\pgfqpoint{3.891312in}{1.527086in}}%
\pgfpathlineto{\pgfqpoint{3.891695in}{1.547458in}}%
\pgfpathlineto{\pgfqpoint{3.892078in}{1.555607in}}%
\pgfpathlineto{\pgfqpoint{3.893227in}{1.535235in}}%
\pgfpathlineto{\pgfqpoint{3.894761in}{1.559681in}}%
\pgfpathlineto{\pgfqpoint{3.895144in}{1.555607in}}%
\pgfpathlineto{\pgfqpoint{3.895528in}{1.535235in}}%
\pgfpathlineto{\pgfqpoint{3.896294in}{1.543384in}}%
\pgfpathlineto{\pgfqpoint{3.897445in}{1.567830in}}%
\pgfpathlineto{\pgfqpoint{3.898979in}{1.539309in}}%
\pgfpathlineto{\pgfqpoint{3.899363in}{1.539309in}}%
\pgfpathlineto{\pgfqpoint{3.900129in}{1.555607in}}%
\pgfpathlineto{\pgfqpoint{3.900512in}{1.547458in}}%
\pgfpathlineto{\pgfqpoint{3.900895in}{1.531161in}}%
\pgfpathlineto{\pgfqpoint{3.901662in}{1.539309in}}%
\pgfpathlineto{\pgfqpoint{3.902045in}{1.555607in}}%
\pgfpathlineto{\pgfqpoint{3.902429in}{1.523012in}}%
\pgfpathlineto{\pgfqpoint{3.903194in}{1.543384in}}%
\pgfpathlineto{\pgfqpoint{3.903961in}{1.539309in}}%
\pgfpathlineto{\pgfqpoint{3.904728in}{1.547458in}}%
\pgfpathlineto{\pgfqpoint{3.905111in}{1.539309in}}%
\pgfpathlineto{\pgfqpoint{3.905494in}{1.559681in}}%
\pgfpathlineto{\pgfqpoint{3.906260in}{1.547458in}}%
\pgfpathlineto{\pgfqpoint{3.906644in}{1.551532in}}%
\pgfpathlineto{\pgfqpoint{3.907027in}{1.539309in}}%
\pgfpathlineto{\pgfqpoint{3.907027in}{1.539309in}}%
\pgfpathlineto{\pgfqpoint{3.907027in}{1.539309in}}%
\pgfpathlineto{\pgfqpoint{3.908177in}{1.559681in}}%
\pgfpathlineto{\pgfqpoint{3.908561in}{1.539309in}}%
\pgfpathlineto{\pgfqpoint{3.908944in}{1.543384in}}%
\pgfpathlineto{\pgfqpoint{3.910095in}{1.563756in}}%
\pgfpathlineto{\pgfqpoint{3.911627in}{1.539309in}}%
\pgfpathlineto{\pgfqpoint{3.912778in}{1.559681in}}%
\pgfpathlineto{\pgfqpoint{3.913545in}{1.551532in}}%
\pgfpathlineto{\pgfqpoint{3.913927in}{1.531161in}}%
\pgfpathlineto{\pgfqpoint{3.913927in}{1.531161in}}%
\pgfpathlineto{\pgfqpoint{3.913927in}{1.531161in}}%
\pgfpathlineto{\pgfqpoint{3.914310in}{1.563756in}}%
\pgfpathlineto{\pgfqpoint{3.915077in}{1.551532in}}%
\pgfpathlineto{\pgfqpoint{3.915462in}{1.547458in}}%
\pgfpathlineto{\pgfqpoint{3.915846in}{1.559681in}}%
\pgfpathlineto{\pgfqpoint{3.916612in}{1.551532in}}%
\pgfpathlineto{\pgfqpoint{3.917377in}{1.543384in}}%
\pgfpathlineto{\pgfqpoint{3.918145in}{1.563756in}}%
\pgfpathlineto{\pgfqpoint{3.918529in}{1.531161in}}%
\pgfpathlineto{\pgfqpoint{3.918529in}{1.531161in}}%
\pgfpathlineto{\pgfqpoint{3.918529in}{1.531161in}}%
\pgfpathlineto{\pgfqpoint{3.918912in}{1.567830in}}%
\pgfpathlineto{\pgfqpoint{3.919295in}{1.555607in}}%
\pgfpathlineto{\pgfqpoint{3.919678in}{1.531161in}}%
\pgfpathlineto{\pgfqpoint{3.920061in}{1.555607in}}%
\pgfpathlineto{\pgfqpoint{3.920452in}{1.555607in}}%
\pgfpathlineto{\pgfqpoint{3.921218in}{1.523012in}}%
\pgfpathlineto{\pgfqpoint{3.921984in}{1.543384in}}%
\pgfpathlineto{\pgfqpoint{3.922368in}{1.551532in}}%
\pgfpathlineto{\pgfqpoint{3.922751in}{1.543384in}}%
\pgfpathlineto{\pgfqpoint{3.923602in}{1.543384in}}%
\pgfpathlineto{\pgfqpoint{3.923985in}{1.563756in}}%
\pgfpathlineto{\pgfqpoint{3.923985in}{1.563756in}}%
\pgfpathlineto{\pgfqpoint{3.923985in}{1.563756in}}%
\pgfpathlineto{\pgfqpoint{3.924369in}{1.531161in}}%
\pgfpathlineto{\pgfqpoint{3.925136in}{1.547458in}}%
\pgfpathlineto{\pgfqpoint{3.926286in}{1.543384in}}%
\pgfpathlineto{\pgfqpoint{3.927055in}{1.547458in}}%
\pgfpathlineto{\pgfqpoint{3.927438in}{1.535235in}}%
\pgfpathlineto{\pgfqpoint{3.927821in}{1.563756in}}%
\pgfpathlineto{\pgfqpoint{3.927821in}{1.563756in}}%
\pgfpathlineto{\pgfqpoint{3.927821in}{1.563756in}}%
\pgfpathlineto{\pgfqpoint{3.928205in}{1.527086in}}%
\pgfpathlineto{\pgfqpoint{3.928972in}{1.535235in}}%
\pgfpathlineto{\pgfqpoint{3.929355in}{1.539309in}}%
\pgfpathlineto{\pgfqpoint{3.929738in}{1.559681in}}%
\pgfpathlineto{\pgfqpoint{3.929738in}{1.559681in}}%
\pgfpathlineto{\pgfqpoint{3.929738in}{1.559681in}}%
\pgfpathlineto{\pgfqpoint{3.930121in}{1.531161in}}%
\pgfpathlineto{\pgfqpoint{3.930887in}{1.539309in}}%
\pgfpathlineto{\pgfqpoint{3.931271in}{1.543384in}}%
\pgfpathlineto{\pgfqpoint{3.931656in}{1.523012in}}%
\pgfpathlineto{\pgfqpoint{3.932039in}{1.543384in}}%
\pgfpathlineto{\pgfqpoint{3.932805in}{1.547458in}}%
\pgfpathlineto{\pgfqpoint{3.933189in}{1.535235in}}%
\pgfpathlineto{\pgfqpoint{3.933189in}{1.535235in}}%
\pgfpathlineto{\pgfqpoint{3.933189in}{1.535235in}}%
\pgfpathlineto{\pgfqpoint{3.933572in}{1.551532in}}%
\pgfpathlineto{\pgfqpoint{3.934339in}{1.539309in}}%
\pgfpathlineto{\pgfqpoint{3.935105in}{1.547458in}}%
\pgfpathlineto{\pgfqpoint{3.935488in}{1.531161in}}%
\pgfpathlineto{\pgfqpoint{3.935488in}{1.531161in}}%
\pgfpathlineto{\pgfqpoint{3.935488in}{1.531161in}}%
\pgfpathlineto{\pgfqpoint{3.935873in}{1.559681in}}%
\pgfpathlineto{\pgfqpoint{3.936639in}{1.551532in}}%
\pgfpathlineto{\pgfqpoint{3.937022in}{1.551532in}}%
\pgfpathlineto{\pgfqpoint{3.938556in}{1.539309in}}%
\pgfpathlineto{\pgfqpoint{3.938940in}{1.580053in}}%
\pgfpathlineto{\pgfqpoint{3.938940in}{1.580053in}}%
\pgfpathlineto{\pgfqpoint{3.938940in}{1.580053in}}%
\pgfpathlineto{\pgfqpoint{3.939322in}{1.531161in}}%
\pgfpathlineto{\pgfqpoint{3.940089in}{1.543384in}}%
\pgfpathlineto{\pgfqpoint{3.940472in}{1.539309in}}%
\pgfpathlineto{\pgfqpoint{3.941239in}{1.559681in}}%
\pgfpathlineto{\pgfqpoint{3.941623in}{1.535235in}}%
\pgfpathlineto{\pgfqpoint{3.942389in}{1.547458in}}%
\pgfpathlineto{\pgfqpoint{3.942773in}{1.563756in}}%
\pgfpathlineto{\pgfqpoint{3.942773in}{1.563756in}}%
\pgfpathlineto{\pgfqpoint{3.942773in}{1.563756in}}%
\pgfpathlineto{\pgfqpoint{3.943923in}{1.539309in}}%
\pgfpathlineto{\pgfqpoint{3.944306in}{1.551532in}}%
\pgfpathlineto{\pgfqpoint{3.945075in}{1.543384in}}%
\pgfpathlineto{\pgfqpoint{3.945458in}{1.551532in}}%
\pgfpathlineto{\pgfqpoint{3.945841in}{1.535235in}}%
\pgfpathlineto{\pgfqpoint{3.945841in}{1.535235in}}%
\pgfpathlineto{\pgfqpoint{3.945841in}{1.535235in}}%
\pgfpathlineto{\pgfqpoint{3.946224in}{1.555607in}}%
\pgfpathlineto{\pgfqpoint{3.946991in}{1.543384in}}%
\pgfpathlineto{\pgfqpoint{3.947375in}{1.551532in}}%
\pgfpathlineto{\pgfqpoint{3.948141in}{1.547458in}}%
\pgfpathlineto{\pgfqpoint{3.949290in}{1.543384in}}%
\pgfpathlineto{\pgfqpoint{3.950057in}{1.547458in}}%
\pgfpathlineto{\pgfqpoint{3.950441in}{1.539309in}}%
\pgfpathlineto{\pgfqpoint{3.950441in}{1.539309in}}%
\pgfpathlineto{\pgfqpoint{3.950441in}{1.539309in}}%
\pgfpathlineto{\pgfqpoint{3.950823in}{1.551532in}}%
\pgfpathlineto{\pgfqpoint{3.951589in}{1.543384in}}%
\pgfpathlineto{\pgfqpoint{3.951974in}{1.527086in}}%
\pgfpathlineto{\pgfqpoint{3.952358in}{1.539309in}}%
\pgfpathlineto{\pgfqpoint{3.953507in}{1.543384in}}%
\pgfpathlineto{\pgfqpoint{3.953890in}{1.535235in}}%
\pgfpathlineto{\pgfqpoint{3.954274in}{1.543384in}}%
\pgfpathlineto{\pgfqpoint{3.955808in}{1.551532in}}%
\pgfpathlineto{\pgfqpoint{3.956574in}{1.539309in}}%
\pgfpathlineto{\pgfqpoint{3.956957in}{1.547458in}}%
\pgfpathlineto{\pgfqpoint{3.958108in}{1.531161in}}%
\pgfpathlineto{\pgfqpoint{3.959258in}{1.555607in}}%
\pgfpathlineto{\pgfqpoint{3.959641in}{1.551532in}}%
\pgfpathlineto{\pgfqpoint{3.960799in}{1.535235in}}%
\pgfpathlineto{\pgfqpoint{3.961183in}{1.559681in}}%
\pgfpathlineto{\pgfqpoint{3.961950in}{1.547458in}}%
\pgfpathlineto{\pgfqpoint{3.962334in}{1.555607in}}%
\pgfpathlineto{\pgfqpoint{3.963100in}{1.551532in}}%
\pgfpathlineto{\pgfqpoint{3.963867in}{1.535235in}}%
\pgfpathlineto{\pgfqpoint{3.964251in}{1.547458in}}%
\pgfpathlineto{\pgfqpoint{3.964634in}{1.543384in}}%
\pgfpathlineto{\pgfqpoint{3.965019in}{1.551532in}}%
\pgfpathlineto{\pgfqpoint{3.965402in}{1.543384in}}%
\pgfpathlineto{\pgfqpoint{3.965786in}{1.543384in}}%
\pgfpathlineto{\pgfqpoint{3.966552in}{1.551532in}}%
\pgfpathlineto{\pgfqpoint{3.966936in}{1.547458in}}%
\pgfpathlineto{\pgfqpoint{3.967319in}{1.547458in}}%
\pgfpathlineto{\pgfqpoint{3.968087in}{1.539309in}}%
\pgfpathlineto{\pgfqpoint{3.968470in}{1.543384in}}%
\pgfpathlineto{\pgfqpoint{3.968853in}{1.555607in}}%
\pgfpathlineto{\pgfqpoint{3.969237in}{1.547458in}}%
\pgfpathlineto{\pgfqpoint{3.969619in}{1.527086in}}%
\pgfpathlineto{\pgfqpoint{3.970387in}{1.543384in}}%
\pgfpathlineto{\pgfqpoint{3.970770in}{1.543384in}}%
\pgfpathlineto{\pgfqpoint{3.971153in}{1.555607in}}%
\pgfpathlineto{\pgfqpoint{3.971153in}{1.555607in}}%
\pgfpathlineto{\pgfqpoint{3.971153in}{1.555607in}}%
\pgfpathlineto{\pgfqpoint{3.971920in}{1.523012in}}%
\pgfpathlineto{\pgfqpoint{3.972773in}{1.527086in}}%
\pgfpathlineto{\pgfqpoint{3.973537in}{1.559681in}}%
\pgfpathlineto{\pgfqpoint{3.973920in}{1.551532in}}%
\pgfpathlineto{\pgfqpoint{3.975838in}{1.531161in}}%
\pgfpathlineto{\pgfqpoint{3.976986in}{1.555607in}}%
\pgfpathlineto{\pgfqpoint{3.977370in}{1.543384in}}%
\pgfpathlineto{\pgfqpoint{3.978138in}{1.551532in}}%
\pgfpathlineto{\pgfqpoint{3.978906in}{1.543384in}}%
\pgfpathlineto{\pgfqpoint{3.979290in}{1.563756in}}%
\pgfpathlineto{\pgfqpoint{3.979673in}{1.547458in}}%
\pgfpathlineto{\pgfqpoint{3.980057in}{1.527086in}}%
\pgfpathlineto{\pgfqpoint{3.980057in}{1.527086in}}%
\pgfpathlineto{\pgfqpoint{3.980057in}{1.527086in}}%
\pgfpathlineto{\pgfqpoint{3.981207in}{1.567830in}}%
\pgfpathlineto{\pgfqpoint{3.981973in}{1.539309in}}%
\pgfpathlineto{\pgfqpoint{3.982356in}{1.543384in}}%
\pgfpathlineto{\pgfqpoint{3.982740in}{1.551532in}}%
\pgfpathlineto{\pgfqpoint{3.983124in}{1.547458in}}%
\pgfpathlineto{\pgfqpoint{3.984274in}{1.531161in}}%
\pgfpathlineto{\pgfqpoint{3.986191in}{1.567830in}}%
\pgfpathlineto{\pgfqpoint{3.988107in}{1.531161in}}%
\pgfpathlineto{\pgfqpoint{3.989258in}{1.555607in}}%
\pgfpathlineto{\pgfqpoint{3.989642in}{1.547458in}}%
\pgfpathlineto{\pgfqpoint{3.990025in}{1.547458in}}%
\pgfpathlineto{\pgfqpoint{3.990408in}{1.531161in}}%
\pgfpathlineto{\pgfqpoint{3.991176in}{1.543384in}}%
\pgfpathlineto{\pgfqpoint{3.991559in}{1.543384in}}%
\pgfpathlineto{\pgfqpoint{3.991942in}{1.539309in}}%
\pgfpathlineto{\pgfqpoint{3.992325in}{1.518937in}}%
\pgfpathlineto{\pgfqpoint{3.992325in}{1.518937in}}%
\pgfpathlineto{\pgfqpoint{3.992325in}{1.518937in}}%
\pgfpathlineto{\pgfqpoint{3.993475in}{1.563756in}}%
\pgfpathlineto{\pgfqpoint{3.994625in}{1.527086in}}%
\pgfpathlineto{\pgfqpoint{3.996158in}{1.551532in}}%
\pgfpathlineto{\pgfqpoint{3.996541in}{1.551532in}}%
\pgfpathlineto{\pgfqpoint{3.997307in}{1.531161in}}%
\pgfpathlineto{\pgfqpoint{3.997691in}{1.547458in}}%
\pgfpathlineto{\pgfqpoint{3.998074in}{1.555607in}}%
\pgfpathlineto{\pgfqpoint{3.998840in}{1.551532in}}%
\pgfpathlineto{\pgfqpoint{3.999224in}{1.551532in}}%
\pgfpathlineto{\pgfqpoint{3.999606in}{1.523012in}}%
\pgfpathlineto{\pgfqpoint{3.999606in}{1.523012in}}%
\pgfpathlineto{\pgfqpoint{3.999606in}{1.523012in}}%
\pgfpathlineto{\pgfqpoint{3.999989in}{1.555607in}}%
\pgfpathlineto{\pgfqpoint{4.000764in}{1.535235in}}%
\pgfpathlineto{\pgfqpoint{4.001914in}{1.559681in}}%
\pgfpathlineto{\pgfqpoint{4.002681in}{1.531161in}}%
\pgfpathlineto{\pgfqpoint{4.003064in}{1.547458in}}%
\pgfpathlineto{\pgfqpoint{4.004215in}{1.567830in}}%
\pgfpathlineto{\pgfqpoint{4.005363in}{1.539309in}}%
\pgfpathlineto{\pgfqpoint{4.005746in}{1.555607in}}%
\pgfpathlineto{\pgfqpoint{4.006513in}{1.551532in}}%
\pgfpathlineto{\pgfqpoint{4.006897in}{1.527086in}}%
\pgfpathlineto{\pgfqpoint{4.007663in}{1.535235in}}%
\pgfpathlineto{\pgfqpoint{4.008046in}{1.547458in}}%
\pgfpathlineto{\pgfqpoint{4.008815in}{1.543384in}}%
\pgfpathlineto{\pgfqpoint{4.009199in}{1.535235in}}%
\pgfpathlineto{\pgfqpoint{4.009965in}{1.555607in}}%
\pgfpathlineto{\pgfqpoint{4.010348in}{1.551532in}}%
\pgfpathlineto{\pgfqpoint{4.010731in}{1.531161in}}%
\pgfpathlineto{\pgfqpoint{4.010731in}{1.531161in}}%
\pgfpathlineto{\pgfqpoint{4.010731in}{1.531161in}}%
\pgfpathlineto{\pgfqpoint{4.011113in}{1.563756in}}%
\pgfpathlineto{\pgfqpoint{4.011882in}{1.539309in}}%
\pgfpathlineto{\pgfqpoint{4.013413in}{1.559681in}}%
\pgfpathlineto{\pgfqpoint{4.014181in}{1.531161in}}%
\pgfpathlineto{\pgfqpoint{4.015331in}{1.567830in}}%
\pgfpathlineto{\pgfqpoint{4.017329in}{1.527086in}}%
\pgfpathlineto{\pgfqpoint{4.018096in}{1.559681in}}%
\pgfpathlineto{\pgfqpoint{4.018864in}{1.547458in}}%
\pgfpathlineto{\pgfqpoint{4.020012in}{1.539309in}}%
\pgfpathlineto{\pgfqpoint{4.020396in}{1.559681in}}%
\pgfpathlineto{\pgfqpoint{4.021162in}{1.551532in}}%
\pgfpathlineto{\pgfqpoint{4.021545in}{1.551532in}}%
\pgfpathlineto{\pgfqpoint{4.023078in}{1.531161in}}%
\pgfpathlineto{\pgfqpoint{4.023462in}{1.539309in}}%
\pgfpathlineto{\pgfqpoint{4.024228in}{1.563756in}}%
\pgfpathlineto{\pgfqpoint{4.024611in}{1.547458in}}%
\pgfpathlineto{\pgfqpoint{4.024994in}{1.539309in}}%
\pgfpathlineto{\pgfqpoint{4.025378in}{1.547458in}}%
\pgfpathlineto{\pgfqpoint{4.025762in}{1.551532in}}%
\pgfpathlineto{\pgfqpoint{4.026146in}{1.539309in}}%
\pgfpathlineto{\pgfqpoint{4.026912in}{1.547458in}}%
\pgfpathlineto{\pgfqpoint{4.028062in}{1.543384in}}%
\pgfpathlineto{\pgfqpoint{4.028829in}{1.559681in}}%
\pgfpathlineto{\pgfqpoint{4.029212in}{1.543384in}}%
\pgfpathlineto{\pgfqpoint{4.029979in}{1.547458in}}%
\pgfpathlineto{\pgfqpoint{4.030745in}{1.563756in}}%
\pgfpathlineto{\pgfqpoint{4.031128in}{1.518937in}}%
\pgfpathlineto{\pgfqpoint{4.031896in}{1.547458in}}%
\pgfpathlineto{\pgfqpoint{4.032279in}{1.571904in}}%
\pgfpathlineto{\pgfqpoint{4.032663in}{1.551532in}}%
\pgfpathlineto{\pgfqpoint{4.033812in}{1.539309in}}%
\pgfpathlineto{\pgfqpoint{4.034963in}{1.547458in}}%
\pgfpathlineto{\pgfqpoint{4.036880in}{1.535235in}}%
\pgfpathlineto{\pgfqpoint{4.037648in}{1.555607in}}%
\pgfpathlineto{\pgfqpoint{4.038031in}{1.551532in}}%
\pgfpathlineto{\pgfqpoint{4.038414in}{1.551532in}}%
\pgfpathlineto{\pgfqpoint{4.039565in}{1.535235in}}%
\pgfpathlineto{\pgfqpoint{4.039947in}{1.547458in}}%
\pgfpathlineto{\pgfqpoint{4.040720in}{1.539309in}}%
\pgfpathlineto{\pgfqpoint{4.041105in}{1.531161in}}%
\pgfpathlineto{\pgfqpoint{4.042254in}{1.551532in}}%
\pgfpathlineto{\pgfqpoint{4.043020in}{1.523012in}}%
\pgfpathlineto{\pgfqpoint{4.043405in}{1.543384in}}%
\pgfpathlineto{\pgfqpoint{4.043789in}{1.547458in}}%
\pgfpathlineto{\pgfqpoint{4.044172in}{1.539309in}}%
\pgfpathlineto{\pgfqpoint{4.044939in}{1.563756in}}%
\pgfpathlineto{\pgfqpoint{4.045322in}{1.539309in}}%
\pgfpathlineto{\pgfqpoint{4.046090in}{1.543384in}}%
\pgfpathlineto{\pgfqpoint{4.046473in}{1.559681in}}%
\pgfpathlineto{\pgfqpoint{4.046856in}{1.543384in}}%
\pgfpathlineto{\pgfqpoint{4.047239in}{1.535235in}}%
\pgfpathlineto{\pgfqpoint{4.048389in}{1.567830in}}%
\pgfpathlineto{\pgfqpoint{4.048773in}{1.535235in}}%
\pgfpathlineto{\pgfqpoint{4.049539in}{1.539309in}}%
\pgfpathlineto{\pgfqpoint{4.049922in}{1.539309in}}%
\pgfpathlineto{\pgfqpoint{4.050306in}{1.559681in}}%
\pgfpathlineto{\pgfqpoint{4.050306in}{1.559681in}}%
\pgfpathlineto{\pgfqpoint{4.050306in}{1.559681in}}%
\pgfpathlineto{\pgfqpoint{4.050689in}{1.527086in}}%
\pgfpathlineto{\pgfqpoint{4.051072in}{1.547458in}}%
\pgfpathlineto{\pgfqpoint{4.051456in}{1.559681in}}%
\pgfpathlineto{\pgfqpoint{4.051840in}{1.547458in}}%
\pgfpathlineto{\pgfqpoint{4.052606in}{1.523012in}}%
\pgfpathlineto{\pgfqpoint{4.052989in}{1.584128in}}%
\pgfpathlineto{\pgfqpoint{4.053755in}{1.543384in}}%
\pgfpathlineto{\pgfqpoint{4.054139in}{1.523012in}}%
\pgfpathlineto{\pgfqpoint{4.054523in}{1.567830in}}%
\pgfpathlineto{\pgfqpoint{4.055292in}{1.551532in}}%
\pgfpathlineto{\pgfqpoint{4.056058in}{1.539309in}}%
\pgfpathlineto{\pgfqpoint{4.056442in}{1.555607in}}%
\pgfpathlineto{\pgfqpoint{4.056442in}{1.555607in}}%
\pgfpathlineto{\pgfqpoint{4.056442in}{1.555607in}}%
\pgfpathlineto{\pgfqpoint{4.056826in}{1.531161in}}%
\pgfpathlineto{\pgfqpoint{4.057592in}{1.539309in}}%
\pgfpathlineto{\pgfqpoint{4.057975in}{1.551532in}}%
\pgfpathlineto{\pgfqpoint{4.057975in}{1.551532in}}%
\pgfpathlineto{\pgfqpoint{4.057975in}{1.551532in}}%
\pgfpathlineto{\pgfqpoint{4.058358in}{1.531161in}}%
\pgfpathlineto{\pgfqpoint{4.059125in}{1.535235in}}%
\pgfpathlineto{\pgfqpoint{4.059510in}{1.551532in}}%
\pgfpathlineto{\pgfqpoint{4.060276in}{1.547458in}}%
\pgfpathlineto{\pgfqpoint{4.060659in}{1.531161in}}%
\pgfpathlineto{\pgfqpoint{4.061042in}{1.535235in}}%
\pgfpathlineto{\pgfqpoint{4.061425in}{1.555607in}}%
\pgfpathlineto{\pgfqpoint{4.062278in}{1.547458in}}%
\pgfpathlineto{\pgfqpoint{4.062658in}{1.551532in}}%
\pgfpathlineto{\pgfqpoint{4.063425in}{1.539309in}}%
\pgfpathlineto{\pgfqpoint{4.063808in}{1.563756in}}%
\pgfpathlineto{\pgfqpoint{4.064574in}{1.543384in}}%
\pgfpathlineto{\pgfqpoint{4.065724in}{1.547458in}}%
\pgfpathlineto{\pgfqpoint{4.067257in}{1.535235in}}%
\pgfpathlineto{\pgfqpoint{4.068023in}{1.547458in}}%
\pgfpathlineto{\pgfqpoint{4.069174in}{1.531161in}}%
\pgfpathlineto{\pgfqpoint{4.069940in}{1.543384in}}%
\pgfpathlineto{\pgfqpoint{4.070706in}{1.571904in}}%
\pgfpathlineto{\pgfqpoint{4.071091in}{1.547458in}}%
\pgfpathlineto{\pgfqpoint{4.071857in}{1.551532in}}%
\pgfpathlineto{\pgfqpoint{4.072623in}{1.543384in}}%
\pgfpathlineto{\pgfqpoint{4.073006in}{1.547458in}}%
\pgfpathlineto{\pgfqpoint{4.073390in}{1.551532in}}%
\pgfpathlineto{\pgfqpoint{4.073774in}{1.567830in}}%
\pgfpathlineto{\pgfqpoint{4.074157in}{1.551532in}}%
\pgfpathlineto{\pgfqpoint{4.074540in}{1.531161in}}%
\pgfpathlineto{\pgfqpoint{4.075306in}{1.547458in}}%
\pgfpathlineto{\pgfqpoint{4.075689in}{1.551532in}}%
\pgfpathlineto{\pgfqpoint{4.076840in}{1.539309in}}%
\pgfpathlineto{\pgfqpoint{4.077223in}{1.555607in}}%
\pgfpathlineto{\pgfqpoint{4.077606in}{1.539309in}}%
\pgfpathlineto{\pgfqpoint{4.077989in}{1.527086in}}%
\pgfpathlineto{\pgfqpoint{4.078756in}{1.535235in}}%
\pgfpathlineto{\pgfqpoint{4.079524in}{1.551532in}}%
\pgfpathlineto{\pgfqpoint{4.079907in}{1.535235in}}%
\pgfpathlineto{\pgfqpoint{4.080296in}{1.551532in}}%
\pgfpathlineto{\pgfqpoint{4.080680in}{1.551532in}}%
\pgfpathlineto{\pgfqpoint{4.081064in}{1.535235in}}%
\pgfpathlineto{\pgfqpoint{4.081064in}{1.535235in}}%
\pgfpathlineto{\pgfqpoint{4.081064in}{1.535235in}}%
\pgfpathlineto{\pgfqpoint{4.081448in}{1.559681in}}%
\pgfpathlineto{\pgfqpoint{4.082215in}{1.539309in}}%
\pgfpathlineto{\pgfqpoint{4.082598in}{1.539309in}}%
\pgfpathlineto{\pgfqpoint{4.083365in}{1.563756in}}%
\pgfpathlineto{\pgfqpoint{4.083748in}{1.551532in}}%
\pgfpathlineto{\pgfqpoint{4.084131in}{1.551532in}}%
\pgfpathlineto{\pgfqpoint{4.084515in}{1.535235in}}%
\pgfpathlineto{\pgfqpoint{4.084898in}{1.543384in}}%
\pgfpathlineto{\pgfqpoint{4.085281in}{1.555607in}}%
\pgfpathlineto{\pgfqpoint{4.086047in}{1.547458in}}%
\pgfpathlineto{\pgfqpoint{4.086430in}{1.539309in}}%
\pgfpathlineto{\pgfqpoint{4.086813in}{1.547458in}}%
\pgfpathlineto{\pgfqpoint{4.087196in}{1.547458in}}%
\pgfpathlineto{\pgfqpoint{4.087961in}{1.559681in}}%
\pgfpathlineto{\pgfqpoint{4.089878in}{1.523012in}}%
\pgfpathlineto{\pgfqpoint{4.091413in}{1.559681in}}%
\pgfpathlineto{\pgfqpoint{4.091797in}{1.547458in}}%
\pgfpathlineto{\pgfqpoint{4.092181in}{1.575979in}}%
\pgfpathlineto{\pgfqpoint{4.092181in}{1.575979in}}%
\pgfpathlineto{\pgfqpoint{4.092181in}{1.575979in}}%
\pgfpathlineto{\pgfqpoint{4.092947in}{1.539309in}}%
\pgfpathlineto{\pgfqpoint{4.093330in}{1.543384in}}%
\pgfpathlineto{\pgfqpoint{4.093714in}{1.555607in}}%
\pgfpathlineto{\pgfqpoint{4.094098in}{1.531161in}}%
\pgfpathlineto{\pgfqpoint{4.094865in}{1.543384in}}%
\pgfpathlineto{\pgfqpoint{4.095631in}{1.559681in}}%
\pgfpathlineto{\pgfqpoint{4.096014in}{1.551532in}}%
\pgfpathlineto{\pgfqpoint{4.096396in}{1.551532in}}%
\pgfpathlineto{\pgfqpoint{4.097546in}{1.543384in}}%
\pgfpathlineto{\pgfqpoint{4.097929in}{1.547458in}}%
\pgfpathlineto{\pgfqpoint{4.098312in}{1.543384in}}%
\pgfpathlineto{\pgfqpoint{4.098695in}{1.531161in}}%
\pgfpathlineto{\pgfqpoint{4.099846in}{1.555607in}}%
\pgfpathlineto{\pgfqpoint{4.100996in}{1.535235in}}%
\pgfpathlineto{\pgfqpoint{4.101762in}{1.559681in}}%
\pgfpathlineto{\pgfqpoint{4.102146in}{1.539309in}}%
\pgfpathlineto{\pgfqpoint{4.103680in}{1.551532in}}%
\pgfpathlineto{\pgfqpoint{4.104149in}{1.551532in}}%
\pgfpathlineto{\pgfqpoint{4.104532in}{1.539309in}}%
\pgfpathlineto{\pgfqpoint{4.104916in}{1.563756in}}%
\pgfpathlineto{\pgfqpoint{4.105684in}{1.543384in}}%
\pgfpathlineto{\pgfqpoint{4.106067in}{1.539309in}}%
\pgfpathlineto{\pgfqpoint{4.106450in}{1.563756in}}%
\pgfpathlineto{\pgfqpoint{4.106833in}{1.547458in}}%
\pgfpathlineto{\pgfqpoint{4.107217in}{1.535235in}}%
\pgfpathlineto{\pgfqpoint{4.107217in}{1.535235in}}%
\pgfpathlineto{\pgfqpoint{4.107217in}{1.535235in}}%
\pgfpathlineto{\pgfqpoint{4.107601in}{1.555607in}}%
\pgfpathlineto{\pgfqpoint{4.108368in}{1.547458in}}%
\pgfpathlineto{\pgfqpoint{4.109132in}{1.539309in}}%
\pgfpathlineto{\pgfqpoint{4.110282in}{1.551532in}}%
\pgfpathlineto{\pgfqpoint{4.110667in}{1.547458in}}%
\pgfpathlineto{\pgfqpoint{4.111049in}{1.551532in}}%
\pgfpathlineto{\pgfqpoint{4.111433in}{1.563756in}}%
\pgfpathlineto{\pgfqpoint{4.111433in}{1.563756in}}%
\pgfpathlineto{\pgfqpoint{4.111433in}{1.563756in}}%
\pgfpathlineto{\pgfqpoint{4.112584in}{1.539309in}}%
\pgfpathlineto{\pgfqpoint{4.113351in}{1.555607in}}%
\pgfpathlineto{\pgfqpoint{4.113734in}{1.539309in}}%
\pgfpathlineto{\pgfqpoint{4.114501in}{1.547458in}}%
\pgfpathlineto{\pgfqpoint{4.114884in}{1.559681in}}%
\pgfpathlineto{\pgfqpoint{4.115267in}{1.555607in}}%
\pgfpathlineto{\pgfqpoint{4.116036in}{1.535235in}}%
\pgfpathlineto{\pgfqpoint{4.116418in}{1.547458in}}%
\pgfpathlineto{\pgfqpoint{4.117567in}{1.563756in}}%
\pgfpathlineto{\pgfqpoint{4.119101in}{1.535235in}}%
\pgfpathlineto{\pgfqpoint{4.119485in}{1.567830in}}%
\pgfpathlineto{\pgfqpoint{4.119868in}{1.547458in}}%
\pgfpathlineto{\pgfqpoint{4.120260in}{1.518937in}}%
\pgfpathlineto{\pgfqpoint{4.120260in}{1.518937in}}%
\pgfpathlineto{\pgfqpoint{4.120260in}{1.518937in}}%
\pgfpathlineto{\pgfqpoint{4.121027in}{1.563756in}}%
\pgfpathlineto{\pgfqpoint{4.121411in}{1.559681in}}%
\pgfpathlineto{\pgfqpoint{4.122945in}{1.543384in}}%
\pgfpathlineto{\pgfqpoint{4.123327in}{1.547458in}}%
\pgfpathlineto{\pgfqpoint{4.123711in}{1.543384in}}%
\pgfpathlineto{\pgfqpoint{4.124094in}{1.539309in}}%
\pgfpathlineto{\pgfqpoint{4.124860in}{1.559681in}}%
\pgfpathlineto{\pgfqpoint{4.125244in}{1.543384in}}%
\pgfpathlineto{\pgfqpoint{4.126394in}{1.547458in}}%
\pgfpathlineto{\pgfqpoint{4.126777in}{1.543384in}}%
\pgfpathlineto{\pgfqpoint{4.127160in}{1.555607in}}%
\pgfpathlineto{\pgfqpoint{4.127160in}{1.555607in}}%
\pgfpathlineto{\pgfqpoint{4.127160in}{1.555607in}}%
\pgfpathlineto{\pgfqpoint{4.128694in}{1.535235in}}%
\pgfpathlineto{\pgfqpoint{4.129460in}{1.559681in}}%
\pgfpathlineto{\pgfqpoint{4.129844in}{1.539309in}}%
\pgfpathlineto{\pgfqpoint{4.130227in}{1.543384in}}%
\pgfpathlineto{\pgfqpoint{4.130611in}{1.531161in}}%
\pgfpathlineto{\pgfqpoint{4.130611in}{1.531161in}}%
\pgfpathlineto{\pgfqpoint{4.130611in}{1.531161in}}%
\pgfpathlineto{\pgfqpoint{4.130995in}{1.551532in}}%
\pgfpathlineto{\pgfqpoint{4.131762in}{1.547458in}}%
\pgfpathlineto{\pgfqpoint{4.132912in}{1.535235in}}%
\pgfpathlineto{\pgfqpoint{4.134444in}{1.551532in}}%
\pgfpathlineto{\pgfqpoint{4.134827in}{1.551532in}}%
\pgfpathlineto{\pgfqpoint{4.135211in}{1.539309in}}%
\pgfpathlineto{\pgfqpoint{4.135594in}{1.551532in}}%
\pgfpathlineto{\pgfqpoint{4.136362in}{1.567830in}}%
\pgfpathlineto{\pgfqpoint{4.137128in}{1.547458in}}%
\pgfpathlineto{\pgfqpoint{4.137512in}{1.567830in}}%
\pgfpathlineto{\pgfqpoint{4.137894in}{1.555607in}}%
\pgfpathlineto{\pgfqpoint{4.139045in}{1.535235in}}%
\pgfpathlineto{\pgfqpoint{4.140578in}{1.551532in}}%
\pgfpathlineto{\pgfqpoint{4.141728in}{1.535235in}}%
\pgfpathlineto{\pgfqpoint{4.142494in}{1.555607in}}%
\pgfpathlineto{\pgfqpoint{4.143261in}{1.527086in}}%
\pgfpathlineto{\pgfqpoint{4.143644in}{1.555607in}}%
\pgfpathlineto{\pgfqpoint{4.144411in}{1.539309in}}%
\pgfpathlineto{\pgfqpoint{4.146327in}{1.555607in}}%
\pgfpathlineto{\pgfqpoint{4.147861in}{1.547458in}}%
\pgfpathlineto{\pgfqpoint{4.148244in}{1.547458in}}%
\pgfpathlineto{\pgfqpoint{4.148627in}{1.559681in}}%
\pgfpathlineto{\pgfqpoint{4.149010in}{1.547458in}}%
\pgfpathlineto{\pgfqpoint{4.149394in}{1.543384in}}%
\pgfpathlineto{\pgfqpoint{4.149863in}{1.547458in}}%
\pgfpathlineto{\pgfqpoint{4.150247in}{1.555607in}}%
\pgfpathlineto{\pgfqpoint{4.150630in}{1.535235in}}%
\pgfpathlineto{\pgfqpoint{4.151013in}{1.551532in}}%
\pgfpathlineto{\pgfqpoint{4.151778in}{1.571904in}}%
\pgfpathlineto{\pgfqpoint{4.152161in}{1.559681in}}%
\pgfpathlineto{\pgfqpoint{4.153696in}{1.539309in}}%
\pgfpathlineto{\pgfqpoint{4.154845in}{1.551532in}}%
\pgfpathlineto{\pgfqpoint{4.155228in}{1.539309in}}%
\pgfpathlineto{\pgfqpoint{4.155610in}{1.567830in}}%
\pgfpathlineto{\pgfqpoint{4.156377in}{1.543384in}}%
\pgfpathlineto{\pgfqpoint{4.156760in}{1.539309in}}%
\pgfpathlineto{\pgfqpoint{4.157527in}{1.559681in}}%
\pgfpathlineto{\pgfqpoint{4.158295in}{1.551532in}}%
\pgfpathlineto{\pgfqpoint{4.158678in}{1.551532in}}%
\pgfpathlineto{\pgfqpoint{4.160220in}{1.531161in}}%
\pgfpathlineto{\pgfqpoint{4.160986in}{1.555607in}}%
\pgfpathlineto{\pgfqpoint{4.161369in}{1.539309in}}%
\pgfpathlineto{\pgfqpoint{4.162136in}{1.555607in}}%
\pgfpathlineto{\pgfqpoint{4.162520in}{1.543384in}}%
\pgfpathlineto{\pgfqpoint{4.163286in}{1.543384in}}%
\pgfpathlineto{\pgfqpoint{4.164053in}{1.547458in}}%
\pgfpathlineto{\pgfqpoint{4.164436in}{1.539309in}}%
\pgfpathlineto{\pgfqpoint{4.164819in}{1.547458in}}%
\pgfpathlineto{\pgfqpoint{4.165203in}{1.567830in}}%
\pgfpathlineto{\pgfqpoint{4.165203in}{1.567830in}}%
\pgfpathlineto{\pgfqpoint{4.165203in}{1.567830in}}%
\pgfpathlineto{\pgfqpoint{4.166736in}{1.539309in}}%
\pgfpathlineto{\pgfqpoint{4.167118in}{1.551532in}}%
\pgfpathlineto{\pgfqpoint{4.167886in}{1.547458in}}%
\pgfpathlineto{\pgfqpoint{4.168270in}{1.543384in}}%
\pgfpathlineto{\pgfqpoint{4.168654in}{1.551532in}}%
\pgfpathlineto{\pgfqpoint{4.168654in}{1.551532in}}%
\pgfpathlineto{\pgfqpoint{4.168654in}{1.551532in}}%
\pgfpathlineto{\pgfqpoint{4.169037in}{1.539309in}}%
\pgfpathlineto{\pgfqpoint{4.169037in}{1.539309in}}%
\pgfpathlineto{\pgfqpoint{4.169037in}{1.539309in}}%
\pgfpathlineto{\pgfqpoint{4.169420in}{1.555607in}}%
\pgfpathlineto{\pgfqpoint{4.170185in}{1.551532in}}%
\pgfpathlineto{\pgfqpoint{4.171337in}{1.543384in}}%
\pgfpathlineto{\pgfqpoint{4.171720in}{1.551532in}}%
\pgfpathlineto{\pgfqpoint{4.171720in}{1.551532in}}%
\pgfpathlineto{\pgfqpoint{4.171720in}{1.551532in}}%
\pgfpathlineto{\pgfqpoint{4.172870in}{1.539309in}}%
\pgfpathlineto{\pgfqpoint{4.173254in}{1.539309in}}%
\pgfpathlineto{\pgfqpoint{4.174021in}{1.555607in}}%
\pgfpathlineto{\pgfqpoint{4.174404in}{1.543384in}}%
\pgfpathlineto{\pgfqpoint{4.174787in}{1.539309in}}%
\pgfpathlineto{\pgfqpoint{4.175171in}{1.555607in}}%
\pgfpathlineto{\pgfqpoint{4.175554in}{1.539309in}}%
\pgfpathlineto{\pgfqpoint{4.175938in}{1.535235in}}%
\pgfpathlineto{\pgfqpoint{4.176704in}{1.563756in}}%
\pgfpathlineto{\pgfqpoint{4.177087in}{1.527086in}}%
\pgfpathlineto{\pgfqpoint{4.177854in}{1.547458in}}%
\pgfpathlineto{\pgfqpoint{4.178237in}{1.547458in}}%
\pgfpathlineto{\pgfqpoint{4.178621in}{1.539309in}}%
\pgfpathlineto{\pgfqpoint{4.178621in}{1.539309in}}%
\pgfpathlineto{\pgfqpoint{4.178621in}{1.539309in}}%
\pgfpathlineto{\pgfqpoint{4.179772in}{1.563756in}}%
\pgfpathlineto{\pgfqpoint{4.180538in}{1.531161in}}%
\pgfpathlineto{\pgfqpoint{4.181306in}{1.543384in}}%
\pgfpathlineto{\pgfqpoint{4.181689in}{1.539309in}}%
\pgfpathlineto{\pgfqpoint{4.182072in}{1.543384in}}%
\pgfpathlineto{\pgfqpoint{4.182455in}{1.547458in}}%
\pgfpathlineto{\pgfqpoint{4.183989in}{1.527086in}}%
\pgfpathlineto{\pgfqpoint{4.184372in}{1.551532in}}%
\pgfpathlineto{\pgfqpoint{4.185138in}{1.543384in}}%
\pgfpathlineto{\pgfqpoint{4.185521in}{1.547458in}}%
\pgfpathlineto{\pgfqpoint{4.187056in}{1.527086in}}%
\pgfpathlineto{\pgfqpoint{4.187823in}{1.563756in}}%
\pgfpathlineto{\pgfqpoint{4.188206in}{1.551532in}}%
\pgfpathlineto{\pgfqpoint{4.188589in}{1.535235in}}%
\pgfpathlineto{\pgfqpoint{4.189357in}{1.539309in}}%
\pgfpathlineto{\pgfqpoint{4.190123in}{1.559681in}}%
\pgfpathlineto{\pgfqpoint{4.190889in}{1.527086in}}%
\pgfpathlineto{\pgfqpoint{4.191272in}{1.551532in}}%
\pgfpathlineto{\pgfqpoint{4.192423in}{1.535235in}}%
\pgfpathlineto{\pgfqpoint{4.192806in}{1.559681in}}%
\pgfpathlineto{\pgfqpoint{4.193573in}{1.539309in}}%
\pgfpathlineto{\pgfqpoint{4.194340in}{1.543384in}}%
\pgfpathlineto{\pgfqpoint{4.194724in}{1.531161in}}%
\pgfpathlineto{\pgfqpoint{4.194724in}{1.531161in}}%
\pgfpathlineto{\pgfqpoint{4.194724in}{1.531161in}}%
\pgfpathlineto{\pgfqpoint{4.195576in}{1.551532in}}%
\pgfpathlineto{\pgfqpoint{4.195960in}{1.535235in}}%
\pgfpathlineto{\pgfqpoint{4.196726in}{1.543384in}}%
\pgfpathlineto{\pgfqpoint{4.197492in}{1.543384in}}%
\pgfpathlineto{\pgfqpoint{4.197875in}{1.563756in}}%
\pgfpathlineto{\pgfqpoint{4.198257in}{1.551532in}}%
\pgfpathlineto{\pgfqpoint{4.198640in}{1.527086in}}%
\pgfpathlineto{\pgfqpoint{4.199024in}{1.543384in}}%
\pgfpathlineto{\pgfqpoint{4.199406in}{1.563756in}}%
\pgfpathlineto{\pgfqpoint{4.199406in}{1.563756in}}%
\pgfpathlineto{\pgfqpoint{4.199406in}{1.563756in}}%
\pgfpathlineto{\pgfqpoint{4.199789in}{1.535235in}}%
\pgfpathlineto{\pgfqpoint{4.200564in}{1.547458in}}%
\pgfpathlineto{\pgfqpoint{4.200947in}{1.543384in}}%
\pgfpathlineto{\pgfqpoint{4.201330in}{1.567830in}}%
\pgfpathlineto{\pgfqpoint{4.202095in}{1.551532in}}%
\pgfpathlineto{\pgfqpoint{4.202480in}{1.559681in}}%
\pgfpathlineto{\pgfqpoint{4.202864in}{1.555607in}}%
\pgfpathlineto{\pgfqpoint{4.203630in}{1.535235in}}%
\pgfpathlineto{\pgfqpoint{4.204014in}{1.539309in}}%
\pgfpathlineto{\pgfqpoint{4.204397in}{1.575979in}}%
\pgfpathlineto{\pgfqpoint{4.205165in}{1.555607in}}%
\pgfpathlineto{\pgfqpoint{4.205549in}{1.527086in}}%
\pgfpathlineto{\pgfqpoint{4.205932in}{1.551532in}}%
\pgfpathlineto{\pgfqpoint{4.206698in}{1.563756in}}%
\pgfpathlineto{\pgfqpoint{4.207083in}{1.559681in}}%
\pgfpathlineto{\pgfqpoint{4.207850in}{1.535235in}}%
\pgfpathlineto{\pgfqpoint{4.208233in}{1.543384in}}%
\pgfpathlineto{\pgfqpoint{4.208998in}{1.559681in}}%
\pgfpathlineto{\pgfqpoint{4.209382in}{1.555607in}}%
\pgfpathlineto{\pgfqpoint{4.209766in}{1.531161in}}%
\pgfpathlineto{\pgfqpoint{4.210533in}{1.551532in}}%
\pgfpathlineto{\pgfqpoint{4.210916in}{1.559681in}}%
\pgfpathlineto{\pgfqpoint{4.211299in}{1.555607in}}%
\pgfpathlineto{\pgfqpoint{4.211683in}{1.543384in}}%
\pgfpathlineto{\pgfqpoint{4.212067in}{1.571904in}}%
\pgfpathlineto{\pgfqpoint{4.212067in}{1.571904in}}%
\pgfpathlineto{\pgfqpoint{4.212067in}{1.571904in}}%
\pgfpathlineto{\pgfqpoint{4.212450in}{1.518937in}}%
\pgfpathlineto{\pgfqpoint{4.213217in}{1.531161in}}%
\pgfpathlineto{\pgfqpoint{4.213983in}{1.547458in}}%
\pgfpathlineto{\pgfqpoint{4.214366in}{1.531161in}}%
\pgfpathlineto{\pgfqpoint{4.214366in}{1.531161in}}%
\pgfpathlineto{\pgfqpoint{4.214366in}{1.531161in}}%
\pgfpathlineto{\pgfqpoint{4.214749in}{1.559681in}}%
\pgfpathlineto{\pgfqpoint{4.215133in}{1.543384in}}%
\pgfpathlineto{\pgfqpoint{4.215517in}{1.531161in}}%
\pgfpathlineto{\pgfqpoint{4.215900in}{1.539309in}}%
\pgfpathlineto{\pgfqpoint{4.216283in}{1.551532in}}%
\pgfpathlineto{\pgfqpoint{4.216283in}{1.551532in}}%
\pgfpathlineto{\pgfqpoint{4.216283in}{1.551532in}}%
\pgfpathlineto{\pgfqpoint{4.216667in}{1.535235in}}%
\pgfpathlineto{\pgfqpoint{4.216667in}{1.535235in}}%
\pgfpathlineto{\pgfqpoint{4.216667in}{1.535235in}}%
\pgfpathlineto{\pgfqpoint{4.217050in}{1.563756in}}%
\pgfpathlineto{\pgfqpoint{4.217433in}{1.551532in}}%
\pgfpathlineto{\pgfqpoint{4.217816in}{1.535235in}}%
\pgfpathlineto{\pgfqpoint{4.218200in}{1.539309in}}%
\pgfpathlineto{\pgfqpoint{4.218967in}{1.555607in}}%
\pgfpathlineto{\pgfqpoint{4.219351in}{1.543384in}}%
\pgfpathlineto{\pgfqpoint{4.219734in}{1.523012in}}%
\pgfpathlineto{\pgfqpoint{4.220501in}{1.535235in}}%
\pgfpathlineto{\pgfqpoint{4.221268in}{1.567830in}}%
\pgfpathlineto{\pgfqpoint{4.221651in}{1.551532in}}%
\pgfpathlineto{\pgfqpoint{4.223185in}{1.514863in}}%
\pgfpathlineto{\pgfqpoint{4.224335in}{1.547458in}}%
\pgfpathlineto{\pgfqpoint{4.225101in}{1.543384in}}%
\pgfpathlineto{\pgfqpoint{4.225869in}{1.551532in}}%
\pgfpathlineto{\pgfqpoint{4.226635in}{1.535235in}}%
\pgfpathlineto{\pgfqpoint{4.227401in}{1.539309in}}%
\pgfpathlineto{\pgfqpoint{4.228936in}{1.559681in}}%
\pgfpathlineto{\pgfqpoint{4.229703in}{1.539309in}}%
\pgfpathlineto{\pgfqpoint{4.230469in}{1.559681in}}%
\pgfpathlineto{\pgfqpoint{4.230854in}{1.543384in}}%
\pgfpathlineto{\pgfqpoint{4.231237in}{1.543384in}}%
\pgfpathlineto{\pgfqpoint{4.231621in}{1.531161in}}%
\pgfpathlineto{\pgfqpoint{4.231621in}{1.531161in}}%
\pgfpathlineto{\pgfqpoint{4.231621in}{1.531161in}}%
\pgfpathlineto{\pgfqpoint{4.232771in}{1.555607in}}%
\pgfpathlineto{\pgfqpoint{4.233154in}{1.535235in}}%
\pgfpathlineto{\pgfqpoint{4.233921in}{1.543384in}}%
\pgfpathlineto{\pgfqpoint{4.234304in}{1.543384in}}%
\pgfpathlineto{\pgfqpoint{4.234688in}{1.551532in}}%
\pgfpathlineto{\pgfqpoint{4.235071in}{1.531161in}}%
\pgfpathlineto{\pgfqpoint{4.235837in}{1.547458in}}%
\pgfpathlineto{\pgfqpoint{4.236605in}{1.539309in}}%
\pgfpathlineto{\pgfqpoint{4.237372in}{1.555607in}}%
\pgfpathlineto{\pgfqpoint{4.237754in}{1.535235in}}%
\pgfpathlineto{\pgfqpoint{4.237754in}{1.535235in}}%
\pgfpathlineto{\pgfqpoint{4.237754in}{1.535235in}}%
\pgfpathlineto{\pgfqpoint{4.238905in}{1.559681in}}%
\pgfpathlineto{\pgfqpoint{4.240830in}{1.531161in}}%
\pgfpathlineto{\pgfqpoint{4.242363in}{1.555607in}}%
\pgfpathlineto{\pgfqpoint{4.242746in}{1.543384in}}%
\pgfpathlineto{\pgfqpoint{4.243129in}{1.567830in}}%
\pgfpathlineto{\pgfqpoint{4.243129in}{1.567830in}}%
\pgfpathlineto{\pgfqpoint{4.243129in}{1.567830in}}%
\pgfpathlineto{\pgfqpoint{4.243896in}{1.535235in}}%
\pgfpathlineto{\pgfqpoint{4.244280in}{1.543384in}}%
\pgfpathlineto{\pgfqpoint{4.244663in}{1.551532in}}%
\pgfpathlineto{\pgfqpoint{4.245046in}{1.547458in}}%
\pgfpathlineto{\pgfqpoint{4.245429in}{1.531161in}}%
\pgfpathlineto{\pgfqpoint{4.245897in}{1.539309in}}%
\pgfpathlineto{\pgfqpoint{4.246282in}{1.547458in}}%
\pgfpathlineto{\pgfqpoint{4.246666in}{1.539309in}}%
\pgfpathlineto{\pgfqpoint{4.247050in}{1.535235in}}%
\pgfpathlineto{\pgfqpoint{4.248581in}{1.555607in}}%
\pgfpathlineto{\pgfqpoint{4.248966in}{1.563756in}}%
\pgfpathlineto{\pgfqpoint{4.250116in}{1.518937in}}%
\pgfpathlineto{\pgfqpoint{4.250499in}{1.527086in}}%
\pgfpathlineto{\pgfqpoint{4.250883in}{1.527086in}}%
\pgfpathlineto{\pgfqpoint{4.251266in}{1.531161in}}%
\pgfpathlineto{\pgfqpoint{4.251651in}{1.547458in}}%
\pgfpathlineto{\pgfqpoint{4.252034in}{1.539309in}}%
\pgfpathlineto{\pgfqpoint{4.252418in}{1.527086in}}%
\pgfpathlineto{\pgfqpoint{4.252800in}{1.539309in}}%
\pgfpathlineto{\pgfqpoint{4.253567in}{1.543384in}}%
\pgfpathlineto{\pgfqpoint{4.253951in}{1.571904in}}%
\pgfpathlineto{\pgfqpoint{4.254335in}{1.551532in}}%
\pgfpathlineto{\pgfqpoint{4.255101in}{1.543384in}}%
\pgfpathlineto{\pgfqpoint{4.256633in}{1.559681in}}%
\pgfpathlineto{\pgfqpoint{4.258550in}{1.527086in}}%
\pgfpathlineto{\pgfqpoint{4.259316in}{1.551532in}}%
\pgfpathlineto{\pgfqpoint{4.259700in}{1.543384in}}%
\pgfpathlineto{\pgfqpoint{4.260084in}{1.547458in}}%
\pgfpathlineto{\pgfqpoint{4.260467in}{1.543384in}}%
\pgfpathlineto{\pgfqpoint{4.261232in}{1.543384in}}%
\pgfpathlineto{\pgfqpoint{4.262000in}{1.559681in}}%
\pgfpathlineto{\pgfqpoint{4.262767in}{1.555607in}}%
\pgfpathlineto{\pgfqpoint{4.263151in}{1.539309in}}%
\pgfpathlineto{\pgfqpoint{4.263151in}{1.539309in}}%
\pgfpathlineto{\pgfqpoint{4.263151in}{1.539309in}}%
\pgfpathlineto{\pgfqpoint{4.263533in}{1.559681in}}%
\pgfpathlineto{\pgfqpoint{4.263916in}{1.518937in}}%
\pgfpathlineto{\pgfqpoint{4.263916in}{1.518937in}}%
\pgfpathlineto{\pgfqpoint{4.263916in}{1.518937in}}%
\pgfpathlineto{\pgfqpoint{4.264300in}{1.563756in}}%
\pgfpathlineto{\pgfqpoint{4.265068in}{1.539309in}}%
\pgfpathlineto{\pgfqpoint{4.265451in}{1.543384in}}%
\pgfpathlineto{\pgfqpoint{4.265834in}{1.535235in}}%
\pgfpathlineto{\pgfqpoint{4.266217in}{1.555607in}}%
\pgfpathlineto{\pgfqpoint{4.266984in}{1.551532in}}%
\pgfpathlineto{\pgfqpoint{4.267367in}{1.551532in}}%
\pgfpathlineto{\pgfqpoint{4.268134in}{1.539309in}}%
\pgfpathlineto{\pgfqpoint{4.268517in}{1.563756in}}%
\pgfpathlineto{\pgfqpoint{4.268901in}{1.539309in}}%
\pgfpathlineto{\pgfqpoint{4.269284in}{1.535235in}}%
\pgfpathlineto{\pgfqpoint{4.270435in}{1.551532in}}%
\pgfpathlineto{\pgfqpoint{4.271201in}{1.543384in}}%
\pgfpathlineto{\pgfqpoint{4.271584in}{1.551532in}}%
\pgfpathlineto{\pgfqpoint{4.271967in}{1.527086in}}%
\pgfpathlineto{\pgfqpoint{4.271967in}{1.527086in}}%
\pgfpathlineto{\pgfqpoint{4.271967in}{1.527086in}}%
\pgfpathlineto{\pgfqpoint{4.272350in}{1.555607in}}%
\pgfpathlineto{\pgfqpoint{4.273118in}{1.535235in}}%
\pgfpathlineto{\pgfqpoint{4.273501in}{1.535235in}}%
\pgfpathlineto{\pgfqpoint{4.273885in}{1.531161in}}%
\pgfpathlineto{\pgfqpoint{4.275419in}{1.567830in}}%
\pgfpathlineto{\pgfqpoint{4.276952in}{1.531161in}}%
\pgfpathlineto{\pgfqpoint{4.279252in}{1.559681in}}%
\pgfpathlineto{\pgfqpoint{4.279635in}{1.531161in}}%
\pgfpathlineto{\pgfqpoint{4.280506in}{1.547458in}}%
\pgfpathlineto{\pgfqpoint{4.280890in}{1.543384in}}%
\pgfpathlineto{\pgfqpoint{4.281273in}{1.555607in}}%
\pgfpathlineto{\pgfqpoint{4.281273in}{1.555607in}}%
\pgfpathlineto{\pgfqpoint{4.281273in}{1.555607in}}%
\pgfpathlineto{\pgfqpoint{4.282423in}{1.535235in}}%
\pgfpathlineto{\pgfqpoint{4.282808in}{1.539309in}}%
\pgfpathlineto{\pgfqpoint{4.283191in}{1.559681in}}%
\pgfpathlineto{\pgfqpoint{4.283957in}{1.551532in}}%
\pgfpathlineto{\pgfqpoint{4.285492in}{1.543384in}}%
\pgfpathlineto{\pgfqpoint{4.285874in}{1.543384in}}%
\pgfpathlineto{\pgfqpoint{4.286640in}{1.527086in}}%
\pgfpathlineto{\pgfqpoint{4.288174in}{1.555607in}}%
\pgfpathlineto{\pgfqpoint{4.290090in}{1.535235in}}%
\pgfpathlineto{\pgfqpoint{4.290858in}{1.559681in}}%
\pgfpathlineto{\pgfqpoint{4.291241in}{1.551532in}}%
\pgfpathlineto{\pgfqpoint{4.292391in}{1.531161in}}%
\pgfpathlineto{\pgfqpoint{4.293925in}{1.551532in}}%
\pgfpathlineto{\pgfqpoint{4.294308in}{1.555607in}}%
\pgfpathlineto{\pgfqpoint{4.295074in}{1.531161in}}%
\pgfpathlineto{\pgfqpoint{4.295457in}{1.547458in}}%
\pgfpathlineto{\pgfqpoint{4.295840in}{1.547458in}}%
\pgfpathlineto{\pgfqpoint{4.296225in}{1.551532in}}%
\pgfpathlineto{\pgfqpoint{4.296608in}{1.535235in}}%
\pgfpathlineto{\pgfqpoint{4.297374in}{1.543384in}}%
\pgfpathlineto{\pgfqpoint{4.298142in}{1.531161in}}%
\pgfpathlineto{\pgfqpoint{4.298526in}{1.539309in}}%
\pgfpathlineto{\pgfqpoint{4.298909in}{1.547458in}}%
\pgfpathlineto{\pgfqpoint{4.299293in}{1.527086in}}%
\pgfpathlineto{\pgfqpoint{4.299675in}{1.543384in}}%
\pgfpathlineto{\pgfqpoint{4.300058in}{1.547458in}}%
\pgfpathlineto{\pgfqpoint{4.300443in}{1.543384in}}%
\pgfpathlineto{\pgfqpoint{4.300826in}{1.567830in}}%
\pgfpathlineto{\pgfqpoint{4.300826in}{1.567830in}}%
\pgfpathlineto{\pgfqpoint{4.300826in}{1.567830in}}%
\pgfpathlineto{\pgfqpoint{4.301209in}{1.531161in}}%
\pgfpathlineto{\pgfqpoint{4.301978in}{1.551532in}}%
\pgfpathlineto{\pgfqpoint{4.303511in}{1.527086in}}%
\pgfpathlineto{\pgfqpoint{4.303894in}{1.539309in}}%
\pgfpathlineto{\pgfqpoint{4.304277in}{1.539309in}}%
\pgfpathlineto{\pgfqpoint{4.306195in}{1.551532in}}%
\pgfpathlineto{\pgfqpoint{4.306961in}{1.527086in}}%
\pgfpathlineto{\pgfqpoint{4.308496in}{1.555607in}}%
\pgfpathlineto{\pgfqpoint{4.308879in}{1.563756in}}%
\pgfpathlineto{\pgfqpoint{4.309262in}{1.543384in}}%
\pgfpathlineto{\pgfqpoint{4.310028in}{1.555607in}}%
\pgfpathlineto{\pgfqpoint{4.310411in}{1.559681in}}%
\pgfpathlineto{\pgfqpoint{4.311178in}{1.535235in}}%
\pgfpathlineto{\pgfqpoint{4.311561in}{1.559681in}}%
\pgfpathlineto{\pgfqpoint{4.312327in}{1.539309in}}%
\pgfpathlineto{\pgfqpoint{4.313095in}{1.555607in}}%
\pgfpathlineto{\pgfqpoint{4.313478in}{1.539309in}}%
\pgfpathlineto{\pgfqpoint{4.314244in}{1.547458in}}%
\pgfpathlineto{\pgfqpoint{4.315394in}{1.547458in}}%
\pgfpathlineto{\pgfqpoint{4.316162in}{1.531161in}}%
\pgfpathlineto{\pgfqpoint{4.316545in}{1.559681in}}%
\pgfpathlineto{\pgfqpoint{4.317311in}{1.547458in}}%
\pgfpathlineto{\pgfqpoint{4.318461in}{1.527086in}}%
\pgfpathlineto{\pgfqpoint{4.318844in}{1.555607in}}%
\pgfpathlineto{\pgfqpoint{4.319612in}{1.547458in}}%
\pgfpathlineto{\pgfqpoint{4.320003in}{1.531161in}}%
\pgfpathlineto{\pgfqpoint{4.320003in}{1.531161in}}%
\pgfpathlineto{\pgfqpoint{4.320003in}{1.531161in}}%
\pgfpathlineto{\pgfqpoint{4.321153in}{1.559681in}}%
\pgfpathlineto{\pgfqpoint{4.321537in}{1.539309in}}%
\pgfpathlineto{\pgfqpoint{4.322302in}{1.547458in}}%
\pgfpathlineto{\pgfqpoint{4.322685in}{1.539309in}}%
\pgfpathlineto{\pgfqpoint{4.323069in}{1.543384in}}%
\pgfpathlineto{\pgfqpoint{4.323836in}{1.559681in}}%
\pgfpathlineto{\pgfqpoint{4.324220in}{1.555607in}}%
\pgfpathlineto{\pgfqpoint{4.324602in}{1.531161in}}%
\pgfpathlineto{\pgfqpoint{4.325368in}{1.539309in}}%
\pgfpathlineto{\pgfqpoint{4.325752in}{1.543384in}}%
\pgfpathlineto{\pgfqpoint{4.326135in}{1.535235in}}%
\pgfpathlineto{\pgfqpoint{4.326135in}{1.535235in}}%
\pgfpathlineto{\pgfqpoint{4.326135in}{1.535235in}}%
\pgfpathlineto{\pgfqpoint{4.327284in}{1.547458in}}%
\pgfpathlineto{\pgfqpoint{4.327666in}{1.543384in}}%
\pgfpathlineto{\pgfqpoint{4.328049in}{1.547458in}}%
\pgfpathlineto{\pgfqpoint{4.328433in}{1.547458in}}%
\pgfpathlineto{\pgfqpoint{4.328817in}{1.543384in}}%
\pgfpathlineto{\pgfqpoint{4.329201in}{1.555607in}}%
\pgfpathlineto{\pgfqpoint{4.329584in}{1.527086in}}%
\pgfpathlineto{\pgfqpoint{4.329967in}{1.551532in}}%
\pgfpathlineto{\pgfqpoint{4.330733in}{1.563756in}}%
\pgfpathlineto{\pgfqpoint{4.332269in}{1.539309in}}%
\pgfpathlineto{\pgfqpoint{4.332652in}{1.551532in}}%
\pgfpathlineto{\pgfqpoint{4.333035in}{1.518937in}}%
\pgfpathlineto{\pgfqpoint{4.333803in}{1.535235in}}%
\pgfpathlineto{\pgfqpoint{4.334570in}{1.555607in}}%
\pgfpathlineto{\pgfqpoint{4.334953in}{1.551532in}}%
\pgfpathlineto{\pgfqpoint{4.335336in}{1.535235in}}%
\pgfpathlineto{\pgfqpoint{4.336102in}{1.539309in}}%
\pgfpathlineto{\pgfqpoint{4.336868in}{1.551532in}}%
\pgfpathlineto{\pgfqpoint{4.337252in}{1.580053in}}%
\pgfpathlineto{\pgfqpoint{4.337252in}{1.580053in}}%
\pgfpathlineto{\pgfqpoint{4.337252in}{1.580053in}}%
\pgfpathlineto{\pgfqpoint{4.337636in}{1.535235in}}%
\pgfpathlineto{\pgfqpoint{4.338402in}{1.547458in}}%
\pgfpathlineto{\pgfqpoint{4.339552in}{1.535235in}}%
\pgfpathlineto{\pgfqpoint{4.339935in}{1.559681in}}%
\pgfpathlineto{\pgfqpoint{4.340701in}{1.547458in}}%
\pgfpathlineto{\pgfqpoint{4.341467in}{1.531161in}}%
\pgfpathlineto{\pgfqpoint{4.342700in}{1.555607in}}%
\pgfpathlineto{\pgfqpoint{4.343466in}{1.539309in}}%
\pgfpathlineto{\pgfqpoint{4.343849in}{1.551532in}}%
\pgfpathlineto{\pgfqpoint{4.344615in}{1.543384in}}%
\pgfpathlineto{\pgfqpoint{4.344997in}{1.547458in}}%
\pgfpathlineto{\pgfqpoint{4.345380in}{1.547458in}}%
\pgfpathlineto{\pgfqpoint{4.345763in}{1.539309in}}%
\pgfpathlineto{\pgfqpoint{4.346147in}{1.555607in}}%
\pgfpathlineto{\pgfqpoint{4.346147in}{1.555607in}}%
\pgfpathlineto{\pgfqpoint{4.346147in}{1.555607in}}%
\pgfpathlineto{\pgfqpoint{4.346530in}{1.535235in}}%
\pgfpathlineto{\pgfqpoint{4.346530in}{1.535235in}}%
\pgfpathlineto{\pgfqpoint{4.346530in}{1.535235in}}%
\pgfpathlineto{\pgfqpoint{4.346913in}{1.567830in}}%
\pgfpathlineto{\pgfqpoint{4.346913in}{1.567830in}}%
\pgfpathlineto{\pgfqpoint{4.346913in}{1.567830in}}%
\pgfpathlineto{\pgfqpoint{4.347295in}{1.531161in}}%
\pgfpathlineto{\pgfqpoint{4.348061in}{1.551532in}}%
\pgfpathlineto{\pgfqpoint{4.348443in}{1.531161in}}%
\pgfpathlineto{\pgfqpoint{4.348827in}{1.547458in}}%
\pgfpathlineto{\pgfqpoint{4.349210in}{1.551532in}}%
\pgfpathlineto{\pgfqpoint{4.349593in}{1.539309in}}%
\pgfpathlineto{\pgfqpoint{4.349976in}{1.551532in}}%
\pgfpathlineto{\pgfqpoint{4.350358in}{1.551532in}}%
\pgfpathlineto{\pgfqpoint{4.351507in}{1.523012in}}%
\pgfpathlineto{\pgfqpoint{4.352660in}{1.563756in}}%
\pgfpathlineto{\pgfqpoint{4.353043in}{1.547458in}}%
\pgfpathlineto{\pgfqpoint{4.353427in}{1.551532in}}%
\pgfpathlineto{\pgfqpoint{4.354193in}{1.531161in}}%
\pgfpathlineto{\pgfqpoint{4.354577in}{1.535235in}}%
\pgfpathlineto{\pgfqpoint{4.354960in}{1.539309in}}%
\pgfpathlineto{\pgfqpoint{4.355344in}{1.531161in}}%
\pgfpathlineto{\pgfqpoint{4.356110in}{1.535235in}}%
\pgfpathlineto{\pgfqpoint{4.356493in}{1.551532in}}%
\pgfpathlineto{\pgfqpoint{4.357260in}{1.543384in}}%
\pgfpathlineto{\pgfqpoint{4.357644in}{1.531161in}}%
\pgfpathlineto{\pgfqpoint{4.357644in}{1.531161in}}%
\pgfpathlineto{\pgfqpoint{4.357644in}{1.531161in}}%
\pgfpathlineto{\pgfqpoint{4.358409in}{1.559681in}}%
\pgfpathlineto{\pgfqpoint{4.358792in}{1.539309in}}%
\pgfpathlineto{\pgfqpoint{4.359175in}{1.543384in}}%
\pgfpathlineto{\pgfqpoint{4.359558in}{1.531161in}}%
\pgfpathlineto{\pgfqpoint{4.359951in}{1.539309in}}%
\pgfpathlineto{\pgfqpoint{4.360335in}{1.547458in}}%
\pgfpathlineto{\pgfqpoint{4.361101in}{1.543384in}}%
\pgfpathlineto{\pgfqpoint{4.361485in}{1.535235in}}%
\pgfpathlineto{\pgfqpoint{4.362252in}{1.559681in}}%
\pgfpathlineto{\pgfqpoint{4.362636in}{1.547458in}}%
\pgfpathlineto{\pgfqpoint{4.363019in}{1.527086in}}%
\pgfpathlineto{\pgfqpoint{4.363403in}{1.543384in}}%
\pgfpathlineto{\pgfqpoint{4.363787in}{1.551532in}}%
\pgfpathlineto{\pgfqpoint{4.364555in}{1.518937in}}%
\pgfpathlineto{\pgfqpoint{4.364938in}{1.547458in}}%
\pgfpathlineto{\pgfqpoint{4.365704in}{1.547458in}}%
\pgfpathlineto{\pgfqpoint{4.366087in}{1.555607in}}%
\pgfpathlineto{\pgfqpoint{4.366471in}{1.539309in}}%
\pgfpathlineto{\pgfqpoint{4.367238in}{1.551532in}}%
\pgfpathlineto{\pgfqpoint{4.367621in}{1.559681in}}%
\pgfpathlineto{\pgfqpoint{4.368004in}{1.555607in}}%
\pgfpathlineto{\pgfqpoint{4.368770in}{1.527086in}}%
\pgfpathlineto{\pgfqpoint{4.369155in}{1.551532in}}%
\pgfpathlineto{\pgfqpoint{4.369921in}{1.535235in}}%
\pgfpathlineto{\pgfqpoint{4.370688in}{1.543384in}}%
\pgfpathlineto{\pgfqpoint{4.371071in}{1.547458in}}%
\pgfpathlineto{\pgfqpoint{4.371454in}{1.543384in}}%
\pgfpathlineto{\pgfqpoint{4.371838in}{1.543384in}}%
\pgfpathlineto{\pgfqpoint{4.372221in}{1.539309in}}%
\pgfpathlineto{\pgfqpoint{4.372604in}{1.559681in}}%
\pgfpathlineto{\pgfqpoint{4.372604in}{1.559681in}}%
\pgfpathlineto{\pgfqpoint{4.372604in}{1.559681in}}%
\pgfpathlineto{\pgfqpoint{4.372987in}{1.535235in}}%
\pgfpathlineto{\pgfqpoint{4.373754in}{1.555607in}}%
\pgfpathlineto{\pgfqpoint{4.374138in}{1.567830in}}%
\pgfpathlineto{\pgfqpoint{4.374522in}{1.535235in}}%
\pgfpathlineto{\pgfqpoint{4.375288in}{1.539309in}}%
\pgfpathlineto{\pgfqpoint{4.375671in}{1.539309in}}%
\pgfpathlineto{\pgfqpoint{4.377205in}{1.551532in}}%
\pgfpathlineto{\pgfqpoint{4.378355in}{1.523012in}}%
\pgfpathlineto{\pgfqpoint{4.378738in}{1.551532in}}%
\pgfpathlineto{\pgfqpoint{4.379121in}{1.543384in}}%
\pgfpathlineto{\pgfqpoint{4.379505in}{1.523012in}}%
\pgfpathlineto{\pgfqpoint{4.379505in}{1.523012in}}%
\pgfpathlineto{\pgfqpoint{4.379505in}{1.523012in}}%
\pgfpathlineto{\pgfqpoint{4.380272in}{1.555607in}}%
\pgfpathlineto{\pgfqpoint{4.380655in}{1.539309in}}%
\pgfpathlineto{\pgfqpoint{4.381421in}{1.551532in}}%
\pgfpathlineto{\pgfqpoint{4.381804in}{1.535235in}}%
\pgfpathlineto{\pgfqpoint{4.382188in}{1.543384in}}%
\pgfpathlineto{\pgfqpoint{4.382956in}{1.555607in}}%
\pgfpathlineto{\pgfqpoint{4.383339in}{1.539309in}}%
\pgfpathlineto{\pgfqpoint{4.384189in}{1.547458in}}%
\pgfpathlineto{\pgfqpoint{4.384955in}{1.535235in}}%
\pgfpathlineto{\pgfqpoint{4.385338in}{1.551532in}}%
\pgfpathlineto{\pgfqpoint{4.385338in}{1.551532in}}%
\pgfpathlineto{\pgfqpoint{4.385338in}{1.551532in}}%
\pgfpathlineto{\pgfqpoint{4.385722in}{1.531161in}}%
\pgfpathlineto{\pgfqpoint{4.386488in}{1.547458in}}%
\pgfpathlineto{\pgfqpoint{4.387254in}{1.555607in}}%
\pgfpathlineto{\pgfqpoint{4.388019in}{1.527086in}}%
\pgfpathlineto{\pgfqpoint{4.388786in}{1.535235in}}%
\pgfpathlineto{\pgfqpoint{4.389169in}{1.547458in}}%
\pgfpathlineto{\pgfqpoint{4.389169in}{1.547458in}}%
\pgfpathlineto{\pgfqpoint{4.389169in}{1.547458in}}%
\pgfpathlineto{\pgfqpoint{4.389552in}{1.527086in}}%
\pgfpathlineto{\pgfqpoint{4.389552in}{1.527086in}}%
\pgfpathlineto{\pgfqpoint{4.389552in}{1.527086in}}%
\pgfpathlineto{\pgfqpoint{4.390700in}{1.555607in}}%
\pgfpathlineto{\pgfqpoint{4.391084in}{1.551532in}}%
\pgfpathlineto{\pgfqpoint{4.391466in}{1.535235in}}%
\pgfpathlineto{\pgfqpoint{4.391850in}{1.539309in}}%
\pgfpathlineto{\pgfqpoint{4.392235in}{1.559681in}}%
\pgfpathlineto{\pgfqpoint{4.393000in}{1.547458in}}%
\pgfpathlineto{\pgfqpoint{4.394148in}{1.567830in}}%
\pgfpathlineto{\pgfqpoint{4.395299in}{1.539309in}}%
\pgfpathlineto{\pgfqpoint{4.395681in}{1.543384in}}%
\pgfpathlineto{\pgfqpoint{4.396064in}{1.535235in}}%
\pgfpathlineto{\pgfqpoint{4.396447in}{1.555607in}}%
\pgfpathlineto{\pgfqpoint{4.397215in}{1.539309in}}%
\pgfpathlineto{\pgfqpoint{4.397599in}{1.543384in}}%
\pgfpathlineto{\pgfqpoint{4.397981in}{1.527086in}}%
\pgfpathlineto{\pgfqpoint{4.397981in}{1.527086in}}%
\pgfpathlineto{\pgfqpoint{4.397981in}{1.527086in}}%
\pgfpathlineto{\pgfqpoint{4.399513in}{1.559681in}}%
\pgfpathlineto{\pgfqpoint{4.401054in}{1.543384in}}%
\pgfpathlineto{\pgfqpoint{4.402205in}{1.551532in}}%
\pgfpathlineto{\pgfqpoint{4.402971in}{1.539309in}}%
\pgfpathlineto{\pgfqpoint{4.403355in}{1.551532in}}%
\pgfpathlineto{\pgfqpoint{4.404120in}{1.547458in}}%
\pgfpathlineto{\pgfqpoint{4.405272in}{1.539309in}}%
\pgfpathlineto{\pgfqpoint{4.405655in}{1.555607in}}%
\pgfpathlineto{\pgfqpoint{4.406038in}{1.551532in}}%
\pgfpathlineto{\pgfqpoint{4.406421in}{1.539309in}}%
\pgfpathlineto{\pgfqpoint{4.406421in}{1.539309in}}%
\pgfpathlineto{\pgfqpoint{4.406421in}{1.539309in}}%
\pgfpathlineto{\pgfqpoint{4.406803in}{1.559681in}}%
\pgfpathlineto{\pgfqpoint{4.407571in}{1.555607in}}%
\pgfpathlineto{\pgfqpoint{4.407955in}{1.547458in}}%
\pgfpathlineto{\pgfqpoint{4.408338in}{1.563756in}}%
\pgfpathlineto{\pgfqpoint{4.408721in}{1.531161in}}%
\pgfpathlineto{\pgfqpoint{4.409487in}{1.535235in}}%
\pgfpathlineto{\pgfqpoint{4.409871in}{1.547458in}}%
\pgfpathlineto{\pgfqpoint{4.410639in}{1.539309in}}%
\pgfpathlineto{\pgfqpoint{4.411788in}{1.547458in}}%
\pgfpathlineto{\pgfqpoint{4.412939in}{1.527086in}}%
\pgfpathlineto{\pgfqpoint{4.413705in}{1.563756in}}%
\pgfpathlineto{\pgfqpoint{4.414087in}{1.543384in}}%
\pgfpathlineto{\pgfqpoint{4.414470in}{1.547458in}}%
\pgfpathlineto{\pgfqpoint{4.414854in}{1.563756in}}%
\pgfpathlineto{\pgfqpoint{4.414854in}{1.563756in}}%
\pgfpathlineto{\pgfqpoint{4.414854in}{1.563756in}}%
\pgfpathlineto{\pgfqpoint{4.416004in}{1.535235in}}%
\pgfpathlineto{\pgfqpoint{4.417536in}{1.551532in}}%
\pgfpathlineto{\pgfqpoint{4.417918in}{1.555607in}}%
\pgfpathlineto{\pgfqpoint{4.418302in}{1.551532in}}%
\pgfpathlineto{\pgfqpoint{4.419068in}{1.523012in}}%
\pgfpathlineto{\pgfqpoint{4.419451in}{1.539309in}}%
\pgfpathlineto{\pgfqpoint{4.420983in}{1.551532in}}%
\pgfpathlineto{\pgfqpoint{4.421369in}{1.551532in}}%
\pgfpathlineto{\pgfqpoint{4.421753in}{1.527086in}}%
\pgfpathlineto{\pgfqpoint{4.422519in}{1.547458in}}%
\pgfpathlineto{\pgfqpoint{4.423370in}{1.543384in}}%
\pgfpathlineto{\pgfqpoint{4.423753in}{1.567830in}}%
\pgfpathlineto{\pgfqpoint{4.424521in}{1.551532in}}%
\pgfpathlineto{\pgfqpoint{4.424904in}{1.555607in}}%
\pgfpathlineto{\pgfqpoint{4.426436in}{1.527086in}}%
\pgfpathlineto{\pgfqpoint{4.427586in}{1.559681in}}%
\pgfpathlineto{\pgfqpoint{4.428352in}{1.539309in}}%
\pgfpathlineto{\pgfqpoint{4.428736in}{1.551532in}}%
\pgfpathlineto{\pgfqpoint{4.429887in}{1.518937in}}%
\pgfpathlineto{\pgfqpoint{4.431037in}{1.543384in}}%
\pgfpathlineto{\pgfqpoint{4.431420in}{1.543384in}}%
\pgfpathlineto{\pgfqpoint{4.431803in}{1.547458in}}%
\pgfpathlineto{\pgfqpoint{4.432569in}{1.567830in}}%
\pgfpathlineto{\pgfqpoint{4.433337in}{1.527086in}}%
\pgfpathlineto{\pgfqpoint{4.434102in}{1.539309in}}%
\pgfpathlineto{\pgfqpoint{4.434866in}{1.555607in}}%
\pgfpathlineto{\pgfqpoint{4.435250in}{1.543384in}}%
\pgfpathlineto{\pgfqpoint{4.436400in}{1.543384in}}%
\pgfpathlineto{\pgfqpoint{4.436783in}{1.555607in}}%
\pgfpathlineto{\pgfqpoint{4.437165in}{1.543384in}}%
\pgfpathlineto{\pgfqpoint{4.437932in}{1.531161in}}%
\pgfpathlineto{\pgfqpoint{4.438698in}{1.555607in}}%
\pgfpathlineto{\pgfqpoint{4.439082in}{1.551532in}}%
\pgfpathlineto{\pgfqpoint{4.440624in}{1.535235in}}%
\pgfpathlineto{\pgfqpoint{4.441006in}{1.551532in}}%
\pgfpathlineto{\pgfqpoint{4.441006in}{1.551532in}}%
\pgfpathlineto{\pgfqpoint{4.441006in}{1.551532in}}%
\pgfpathlineto{\pgfqpoint{4.441389in}{1.531161in}}%
\pgfpathlineto{\pgfqpoint{4.441772in}{1.547458in}}%
\pgfpathlineto{\pgfqpoint{4.442156in}{1.551532in}}%
\pgfpathlineto{\pgfqpoint{4.442156in}{1.551532in}}%
\pgfusepath{stroke}%
\end{pgfscope}%
\begin{pgfscope}%
\pgfpathrectangle{\pgfqpoint{0.812440in}{0.565123in}}{\pgfqpoint{3.802560in}{2.249877in}} %
\pgfusepath{clip}%
\pgfsetrectcap%
\pgfsetroundjoin%
\pgfsetlinewidth{1.505625pt}%
\definecolor{currentstroke}{rgb}{1.000000,0.498039,0.054902}%
\pgfsetstrokecolor{currentstroke}%
\pgfsetdash{}{0pt}%
\pgfpathmoveto{\pgfqpoint{0.985283in}{1.673764in}}%
\pgfpathlineto{\pgfqpoint{0.985666in}{1.685987in}}%
\pgfpathlineto{\pgfqpoint{0.986050in}{1.661541in}}%
\pgfpathlineto{\pgfqpoint{0.986818in}{1.669690in}}%
\pgfpathlineto{\pgfqpoint{0.987585in}{1.685987in}}%
\pgfpathlineto{\pgfqpoint{0.988734in}{1.665615in}}%
\pgfpathlineto{\pgfqpoint{0.989118in}{1.669690in}}%
\pgfpathlineto{\pgfqpoint{0.989500in}{1.665615in}}%
\pgfpathlineto{\pgfqpoint{0.989883in}{1.694136in}}%
\pgfpathlineto{\pgfqpoint{0.990267in}{1.677838in}}%
\pgfpathlineto{\pgfqpoint{0.990650in}{1.661541in}}%
\pgfpathlineto{\pgfqpoint{0.991415in}{1.665615in}}%
\pgfpathlineto{\pgfqpoint{0.992567in}{1.690062in}}%
\pgfpathlineto{\pgfqpoint{0.992948in}{1.677838in}}%
\pgfpathlineto{\pgfqpoint{0.993330in}{1.673764in}}%
\pgfpathlineto{\pgfqpoint{0.993713in}{1.685987in}}%
\pgfpathlineto{\pgfqpoint{0.994863in}{1.661541in}}%
\pgfpathlineto{\pgfqpoint{0.995246in}{1.690062in}}%
\pgfpathlineto{\pgfqpoint{0.995629in}{1.681913in}}%
\pgfpathlineto{\pgfqpoint{0.996012in}{1.653392in}}%
\pgfpathlineto{\pgfqpoint{0.996012in}{1.653392in}}%
\pgfpathlineto{\pgfqpoint{0.996012in}{1.653392in}}%
\pgfpathlineto{\pgfqpoint{0.996395in}{1.694136in}}%
\pgfpathlineto{\pgfqpoint{0.997162in}{1.669690in}}%
\pgfpathlineto{\pgfqpoint{0.997929in}{1.685987in}}%
\pgfpathlineto{\pgfqpoint{0.998312in}{1.681913in}}%
\pgfpathlineto{\pgfqpoint{0.998695in}{1.665615in}}%
\pgfpathlineto{\pgfqpoint{0.998695in}{1.665615in}}%
\pgfpathlineto{\pgfqpoint{0.998695in}{1.665615in}}%
\pgfpathlineto{\pgfqpoint{0.999078in}{1.685987in}}%
\pgfpathlineto{\pgfqpoint{0.999845in}{1.677838in}}%
\pgfpathlineto{\pgfqpoint{1.000228in}{1.677838in}}%
\pgfpathlineto{\pgfqpoint{1.000612in}{1.657467in}}%
\pgfpathlineto{\pgfqpoint{1.001378in}{1.665615in}}%
\pgfpathlineto{\pgfqpoint{1.001762in}{1.661541in}}%
\pgfpathlineto{\pgfqpoint{1.002911in}{1.681913in}}%
\pgfpathlineto{\pgfqpoint{1.003294in}{1.665615in}}%
\pgfpathlineto{\pgfqpoint{1.003294in}{1.665615in}}%
\pgfpathlineto{\pgfqpoint{1.003294in}{1.665615in}}%
\pgfpathlineto{\pgfqpoint{1.003677in}{1.694136in}}%
\pgfpathlineto{\pgfqpoint{1.003677in}{1.694136in}}%
\pgfpathlineto{\pgfqpoint{1.003677in}{1.694136in}}%
\pgfpathlineto{\pgfqpoint{1.004067in}{1.661541in}}%
\pgfpathlineto{\pgfqpoint{1.004835in}{1.673764in}}%
\pgfpathlineto{\pgfqpoint{1.005218in}{1.661541in}}%
\pgfpathlineto{\pgfqpoint{1.005602in}{1.665615in}}%
\pgfpathlineto{\pgfqpoint{1.006368in}{1.690062in}}%
\pgfpathlineto{\pgfqpoint{1.006751in}{1.677838in}}%
\pgfpathlineto{\pgfqpoint{1.007134in}{1.657467in}}%
\pgfpathlineto{\pgfqpoint{1.007134in}{1.657467in}}%
\pgfpathlineto{\pgfqpoint{1.007134in}{1.657467in}}%
\pgfpathlineto{\pgfqpoint{1.008283in}{1.690062in}}%
\pgfpathlineto{\pgfqpoint{1.009050in}{1.665615in}}%
\pgfpathlineto{\pgfqpoint{1.009432in}{1.681913in}}%
\pgfpathlineto{\pgfqpoint{1.009815in}{1.681913in}}%
\pgfpathlineto{\pgfqpoint{1.010198in}{1.665615in}}%
\pgfpathlineto{\pgfqpoint{1.010966in}{1.669690in}}%
\pgfpathlineto{\pgfqpoint{1.011349in}{1.685987in}}%
\pgfpathlineto{\pgfqpoint{1.012114in}{1.677838in}}%
\pgfpathlineto{\pgfqpoint{1.012965in}{1.673764in}}%
\pgfpathlineto{\pgfqpoint{1.013349in}{1.681913in}}%
\pgfpathlineto{\pgfqpoint{1.014116in}{1.657467in}}%
\pgfpathlineto{\pgfqpoint{1.014499in}{1.669690in}}%
\pgfpathlineto{\pgfqpoint{1.015265in}{1.685987in}}%
\pgfpathlineto{\pgfqpoint{1.015649in}{1.641169in}}%
\pgfpathlineto{\pgfqpoint{1.016415in}{1.665615in}}%
\pgfpathlineto{\pgfqpoint{1.016799in}{1.665615in}}%
\pgfpathlineto{\pgfqpoint{1.017182in}{1.685987in}}%
\pgfpathlineto{\pgfqpoint{1.017565in}{1.677838in}}%
\pgfpathlineto{\pgfqpoint{1.017949in}{1.665615in}}%
\pgfpathlineto{\pgfqpoint{1.018332in}{1.690062in}}%
\pgfpathlineto{\pgfqpoint{1.019098in}{1.677838in}}%
\pgfpathlineto{\pgfqpoint{1.019482in}{1.690062in}}%
\pgfpathlineto{\pgfqpoint{1.019865in}{1.649318in}}%
\pgfpathlineto{\pgfqpoint{1.020632in}{1.669690in}}%
\pgfpathlineto{\pgfqpoint{1.021014in}{1.665615in}}%
\pgfpathlineto{\pgfqpoint{1.021397in}{1.669690in}}%
\pgfpathlineto{\pgfqpoint{1.021780in}{1.669690in}}%
\pgfpathlineto{\pgfqpoint{1.022548in}{1.690062in}}%
\pgfpathlineto{\pgfqpoint{1.022931in}{1.677838in}}%
\pgfpathlineto{\pgfqpoint{1.023314in}{1.677838in}}%
\pgfpathlineto{\pgfqpoint{1.023697in}{1.685987in}}%
\pgfpathlineto{\pgfqpoint{1.024080in}{1.661541in}}%
\pgfpathlineto{\pgfqpoint{1.024847in}{1.677838in}}%
\pgfpathlineto{\pgfqpoint{1.025230in}{1.690062in}}%
\pgfpathlineto{\pgfqpoint{1.025230in}{1.690062in}}%
\pgfpathlineto{\pgfqpoint{1.025230in}{1.690062in}}%
\pgfpathlineto{\pgfqpoint{1.025613in}{1.669690in}}%
\pgfpathlineto{\pgfqpoint{1.026378in}{1.677838in}}%
\pgfpathlineto{\pgfqpoint{1.026763in}{1.690062in}}%
\pgfpathlineto{\pgfqpoint{1.026763in}{1.690062in}}%
\pgfpathlineto{\pgfqpoint{1.026763in}{1.690062in}}%
\pgfpathlineto{\pgfqpoint{1.027913in}{1.669690in}}%
\pgfpathlineto{\pgfqpoint{1.028679in}{1.681913in}}%
\pgfpathlineto{\pgfqpoint{1.029062in}{1.677838in}}%
\pgfpathlineto{\pgfqpoint{1.029445in}{1.657467in}}%
\pgfpathlineto{\pgfqpoint{1.029828in}{1.677838in}}%
\pgfpathlineto{\pgfqpoint{1.030211in}{1.685987in}}%
\pgfpathlineto{\pgfqpoint{1.030978in}{1.665615in}}%
\pgfpathlineto{\pgfqpoint{1.031362in}{1.681913in}}%
\pgfpathlineto{\pgfqpoint{1.031745in}{1.685987in}}%
\pgfpathlineto{\pgfqpoint{1.032129in}{1.661541in}}%
\pgfpathlineto{\pgfqpoint{1.032894in}{1.673764in}}%
\pgfpathlineto{\pgfqpoint{1.033279in}{1.661541in}}%
\pgfpathlineto{\pgfqpoint{1.033279in}{1.661541in}}%
\pgfpathlineto{\pgfqpoint{1.033279in}{1.661541in}}%
\pgfpathlineto{\pgfqpoint{1.034428in}{1.690062in}}%
\pgfpathlineto{\pgfqpoint{1.034811in}{1.685987in}}%
\pgfpathlineto{\pgfqpoint{1.035193in}{1.657467in}}%
\pgfpathlineto{\pgfqpoint{1.035961in}{1.681913in}}%
\pgfpathlineto{\pgfqpoint{1.036345in}{1.685987in}}%
\pgfpathlineto{\pgfqpoint{1.036728in}{1.649318in}}%
\pgfpathlineto{\pgfqpoint{1.037494in}{1.681913in}}%
\pgfpathlineto{\pgfqpoint{1.038644in}{1.661541in}}%
\pgfpathlineto{\pgfqpoint{1.040177in}{1.694136in}}%
\pgfpathlineto{\pgfqpoint{1.040561in}{1.694136in}}%
\pgfpathlineto{\pgfqpoint{1.040944in}{1.698210in}}%
\pgfpathlineto{\pgfqpoint{1.042477in}{1.673764in}}%
\pgfpathlineto{\pgfqpoint{1.043244in}{1.685987in}}%
\pgfpathlineto{\pgfqpoint{1.043627in}{1.677838in}}%
\pgfpathlineto{\pgfqpoint{1.044017in}{1.661541in}}%
\pgfpathlineto{\pgfqpoint{1.044400in}{1.694136in}}%
\pgfpathlineto{\pgfqpoint{1.045166in}{1.665615in}}%
\pgfpathlineto{\pgfqpoint{1.045550in}{1.665615in}}%
\pgfpathlineto{\pgfqpoint{1.045933in}{1.653392in}}%
\pgfpathlineto{\pgfqpoint{1.046317in}{1.665615in}}%
\pgfpathlineto{\pgfqpoint{1.047467in}{1.673764in}}%
\pgfpathlineto{\pgfqpoint{1.048614in}{1.653392in}}%
\pgfpathlineto{\pgfqpoint{1.050145in}{1.681913in}}%
\pgfpathlineto{\pgfqpoint{1.050994in}{1.669690in}}%
\pgfpathlineto{\pgfqpoint{1.051376in}{1.681913in}}%
\pgfpathlineto{\pgfqpoint{1.051758in}{1.669690in}}%
\pgfpathlineto{\pgfqpoint{1.052141in}{1.657467in}}%
\pgfpathlineto{\pgfqpoint{1.052523in}{1.665615in}}%
\pgfpathlineto{\pgfqpoint{1.053672in}{1.702285in}}%
\pgfpathlineto{\pgfqpoint{1.054820in}{1.669690in}}%
\pgfpathlineto{\pgfqpoint{1.055203in}{1.685987in}}%
\pgfpathlineto{\pgfqpoint{1.055967in}{1.677838in}}%
\pgfpathlineto{\pgfqpoint{1.056349in}{1.661541in}}%
\pgfpathlineto{\pgfqpoint{1.056734in}{1.665615in}}%
\pgfpathlineto{\pgfqpoint{1.057117in}{1.690062in}}%
\pgfpathlineto{\pgfqpoint{1.057882in}{1.673764in}}%
\pgfpathlineto{\pgfqpoint{1.058648in}{1.665615in}}%
\pgfpathlineto{\pgfqpoint{1.059797in}{1.681913in}}%
\pgfpathlineto{\pgfqpoint{1.060944in}{1.669690in}}%
\pgfpathlineto{\pgfqpoint{1.061709in}{1.685987in}}%
\pgfpathlineto{\pgfqpoint{1.062092in}{1.681913in}}%
\pgfpathlineto{\pgfqpoint{1.062474in}{1.685987in}}%
\pgfpathlineto{\pgfqpoint{1.063241in}{1.665615in}}%
\pgfpathlineto{\pgfqpoint{1.063625in}{1.669690in}}%
\pgfpathlineto{\pgfqpoint{1.065155in}{1.698210in}}%
\pgfpathlineto{\pgfqpoint{1.065537in}{1.657467in}}%
\pgfpathlineto{\pgfqpoint{1.066303in}{1.665615in}}%
\pgfpathlineto{\pgfqpoint{1.066686in}{1.690062in}}%
\pgfpathlineto{\pgfqpoint{1.067069in}{1.677838in}}%
\pgfpathlineto{\pgfqpoint{1.067838in}{1.653392in}}%
\pgfpathlineto{\pgfqpoint{1.068217in}{1.669690in}}%
\pgfpathlineto{\pgfqpoint{1.068597in}{1.669690in}}%
\pgfpathlineto{\pgfqpoint{1.068980in}{1.657467in}}%
\pgfpathlineto{\pgfqpoint{1.069362in}{1.669690in}}%
\pgfpathlineto{\pgfqpoint{1.069745in}{1.677838in}}%
\pgfpathlineto{\pgfqpoint{1.070514in}{1.673764in}}%
\pgfpathlineto{\pgfqpoint{1.071280in}{1.677838in}}%
\pgfpathlineto{\pgfqpoint{1.071663in}{1.649318in}}%
\pgfpathlineto{\pgfqpoint{1.071663in}{1.649318in}}%
\pgfpathlineto{\pgfqpoint{1.071663in}{1.649318in}}%
\pgfpathlineto{\pgfqpoint{1.072430in}{1.685987in}}%
\pgfpathlineto{\pgfqpoint{1.072815in}{1.673764in}}%
\pgfpathlineto{\pgfqpoint{1.073198in}{1.669690in}}%
\pgfpathlineto{\pgfqpoint{1.073580in}{1.673764in}}%
\pgfpathlineto{\pgfqpoint{1.074730in}{1.681913in}}%
\pgfpathlineto{\pgfqpoint{1.075496in}{1.669690in}}%
\pgfpathlineto{\pgfqpoint{1.075965in}{1.673764in}}%
\pgfpathlineto{\pgfqpoint{1.076348in}{1.690062in}}%
\pgfpathlineto{\pgfqpoint{1.076731in}{1.673764in}}%
\pgfpathlineto{\pgfqpoint{1.077498in}{1.657467in}}%
\pgfpathlineto{\pgfqpoint{1.079030in}{1.690062in}}%
\pgfpathlineto{\pgfqpoint{1.080179in}{1.657467in}}%
\pgfpathlineto{\pgfqpoint{1.080562in}{1.677838in}}%
\pgfpathlineto{\pgfqpoint{1.081329in}{1.665615in}}%
\pgfpathlineto{\pgfqpoint{1.081713in}{1.669690in}}%
\pgfpathlineto{\pgfqpoint{1.082095in}{1.649318in}}%
\pgfpathlineto{\pgfqpoint{1.082095in}{1.649318in}}%
\pgfpathlineto{\pgfqpoint{1.082095in}{1.649318in}}%
\pgfpathlineto{\pgfqpoint{1.082478in}{1.685987in}}%
\pgfpathlineto{\pgfqpoint{1.083243in}{1.665615in}}%
\pgfpathlineto{\pgfqpoint{1.083626in}{1.677838in}}%
\pgfpathlineto{\pgfqpoint{1.084018in}{1.665615in}}%
\pgfpathlineto{\pgfqpoint{1.084401in}{1.665615in}}%
\pgfpathlineto{\pgfqpoint{1.084784in}{1.690062in}}%
\pgfpathlineto{\pgfqpoint{1.085166in}{1.665615in}}%
\pgfpathlineto{\pgfqpoint{1.085549in}{1.661541in}}%
\pgfpathlineto{\pgfqpoint{1.087081in}{1.685987in}}%
\pgfpathlineto{\pgfqpoint{1.087847in}{1.653392in}}%
\pgfpathlineto{\pgfqpoint{1.088230in}{1.669690in}}%
\pgfpathlineto{\pgfqpoint{1.088995in}{1.681913in}}%
\pgfpathlineto{\pgfqpoint{1.090146in}{1.661541in}}%
\pgfpathlineto{\pgfqpoint{1.091295in}{1.673764in}}%
\pgfpathlineto{\pgfqpoint{1.091678in}{1.669690in}}%
\pgfpathlineto{\pgfqpoint{1.092061in}{1.653392in}}%
\pgfpathlineto{\pgfqpoint{1.092061in}{1.653392in}}%
\pgfpathlineto{\pgfqpoint{1.092061in}{1.653392in}}%
\pgfpathlineto{\pgfqpoint{1.092830in}{1.694136in}}%
\pgfpathlineto{\pgfqpoint{1.093213in}{1.661541in}}%
\pgfpathlineto{\pgfqpoint{1.093596in}{1.673764in}}%
\pgfpathlineto{\pgfqpoint{1.093596in}{1.673764in}}%
\pgfpathlineto{\pgfqpoint{1.093596in}{1.673764in}}%
\pgfpathlineto{\pgfqpoint{1.093979in}{1.657467in}}%
\pgfpathlineto{\pgfqpoint{1.094361in}{1.665615in}}%
\pgfpathlineto{\pgfqpoint{1.095129in}{1.694136in}}%
\pgfpathlineto{\pgfqpoint{1.095513in}{1.681913in}}%
\pgfpathlineto{\pgfqpoint{1.095895in}{1.661541in}}%
\pgfpathlineto{\pgfqpoint{1.096662in}{1.677838in}}%
\pgfpathlineto{\pgfqpoint{1.097045in}{1.694136in}}%
\pgfpathlineto{\pgfqpoint{1.097812in}{1.685987in}}%
\pgfpathlineto{\pgfqpoint{1.098195in}{1.685987in}}%
\pgfpathlineto{\pgfqpoint{1.099343in}{1.669690in}}%
\pgfpathlineto{\pgfqpoint{1.099726in}{1.673764in}}%
\pgfpathlineto{\pgfqpoint{1.100109in}{1.681913in}}%
\pgfpathlineto{\pgfqpoint{1.100109in}{1.681913in}}%
\pgfpathlineto{\pgfqpoint{1.100109in}{1.681913in}}%
\pgfpathlineto{\pgfqpoint{1.100493in}{1.669690in}}%
\pgfpathlineto{\pgfqpoint{1.101259in}{1.673764in}}%
\pgfpathlineto{\pgfqpoint{1.102409in}{1.657467in}}%
\pgfpathlineto{\pgfqpoint{1.103942in}{1.681913in}}%
\pgfpathlineto{\pgfqpoint{1.104325in}{1.677838in}}%
\pgfpathlineto{\pgfqpoint{1.104707in}{1.690062in}}%
\pgfpathlineto{\pgfqpoint{1.104707in}{1.690062in}}%
\pgfpathlineto{\pgfqpoint{1.104707in}{1.690062in}}%
\pgfpathlineto{\pgfqpoint{1.105090in}{1.669690in}}%
\pgfpathlineto{\pgfqpoint{1.105856in}{1.673764in}}%
\pgfpathlineto{\pgfqpoint{1.106240in}{1.673764in}}%
\pgfpathlineto{\pgfqpoint{1.106624in}{1.681913in}}%
\pgfpathlineto{\pgfqpoint{1.107007in}{1.661541in}}%
\pgfpathlineto{\pgfqpoint{1.107391in}{1.681913in}}%
\pgfpathlineto{\pgfqpoint{1.107774in}{1.685987in}}%
\pgfpathlineto{\pgfqpoint{1.108542in}{1.669690in}}%
\pgfpathlineto{\pgfqpoint{1.108925in}{1.681913in}}%
\pgfpathlineto{\pgfqpoint{1.110456in}{1.669690in}}%
\pgfpathlineto{\pgfqpoint{1.111222in}{1.690062in}}%
\pgfpathlineto{\pgfqpoint{1.111690in}{1.673764in}}%
\pgfpathlineto{\pgfqpoint{1.112073in}{1.673764in}}%
\pgfpathlineto{\pgfqpoint{1.112456in}{1.685987in}}%
\pgfpathlineto{\pgfqpoint{1.112456in}{1.685987in}}%
\pgfpathlineto{\pgfqpoint{1.112456in}{1.685987in}}%
\pgfpathlineto{\pgfqpoint{1.113607in}{1.661541in}}%
\pgfpathlineto{\pgfqpoint{1.114756in}{1.685987in}}%
\pgfpathlineto{\pgfqpoint{1.115139in}{1.681913in}}%
\pgfpathlineto{\pgfqpoint{1.116672in}{1.665615in}}%
\pgfpathlineto{\pgfqpoint{1.117437in}{1.681913in}}%
\pgfpathlineto{\pgfqpoint{1.118205in}{1.690062in}}%
\pgfpathlineto{\pgfqpoint{1.118970in}{1.665615in}}%
\pgfpathlineto{\pgfqpoint{1.119353in}{1.690062in}}%
\pgfpathlineto{\pgfqpoint{1.119736in}{1.665615in}}%
\pgfpathlineto{\pgfqpoint{1.120119in}{1.665615in}}%
\pgfpathlineto{\pgfqpoint{1.120502in}{1.681913in}}%
\pgfpathlineto{\pgfqpoint{1.120885in}{1.673764in}}%
\pgfpathlineto{\pgfqpoint{1.121268in}{1.661541in}}%
\pgfpathlineto{\pgfqpoint{1.121651in}{1.685987in}}%
\pgfpathlineto{\pgfqpoint{1.122034in}{1.673764in}}%
\pgfpathlineto{\pgfqpoint{1.122417in}{1.657467in}}%
\pgfpathlineto{\pgfqpoint{1.122801in}{1.690062in}}%
\pgfpathlineto{\pgfqpoint{1.123568in}{1.665615in}}%
\pgfpathlineto{\pgfqpoint{1.125108in}{1.690062in}}%
\pgfpathlineto{\pgfqpoint{1.126643in}{1.665615in}}%
\pgfpathlineto{\pgfqpoint{1.127410in}{1.681913in}}%
\pgfpathlineto{\pgfqpoint{1.127794in}{1.665615in}}%
\pgfpathlineto{\pgfqpoint{1.127794in}{1.665615in}}%
\pgfpathlineto{\pgfqpoint{1.127794in}{1.665615in}}%
\pgfpathlineto{\pgfqpoint{1.128178in}{1.694136in}}%
\pgfpathlineto{\pgfqpoint{1.128944in}{1.673764in}}%
\pgfpathlineto{\pgfqpoint{1.129327in}{1.690062in}}%
\pgfpathlineto{\pgfqpoint{1.130092in}{1.681913in}}%
\pgfpathlineto{\pgfqpoint{1.130476in}{1.665615in}}%
\pgfpathlineto{\pgfqpoint{1.130859in}{1.681913in}}%
\pgfpathlineto{\pgfqpoint{1.131244in}{1.681913in}}%
\pgfpathlineto{\pgfqpoint{1.132009in}{1.665615in}}%
\pgfpathlineto{\pgfqpoint{1.132391in}{1.673764in}}%
\pgfpathlineto{\pgfqpoint{1.133156in}{1.685987in}}%
\pgfpathlineto{\pgfqpoint{1.133924in}{1.649318in}}%
\pgfpathlineto{\pgfqpoint{1.134307in}{1.681913in}}%
\pgfpathlineto{\pgfqpoint{1.135455in}{1.665615in}}%
\pgfpathlineto{\pgfqpoint{1.135838in}{1.669690in}}%
\pgfpathlineto{\pgfqpoint{1.136221in}{1.673764in}}%
\pgfpathlineto{\pgfqpoint{1.136605in}{1.661541in}}%
\pgfpathlineto{\pgfqpoint{1.136989in}{1.665615in}}%
\pgfpathlineto{\pgfqpoint{1.137372in}{1.681913in}}%
\pgfpathlineto{\pgfqpoint{1.138139in}{1.669690in}}%
\pgfpathlineto{\pgfqpoint{1.138522in}{1.681913in}}%
\pgfpathlineto{\pgfqpoint{1.139288in}{1.677838in}}%
\pgfpathlineto{\pgfqpoint{1.139671in}{1.681913in}}%
\pgfpathlineto{\pgfqpoint{1.140820in}{1.673764in}}%
\pgfpathlineto{\pgfqpoint{1.141203in}{1.673764in}}%
\pgfpathlineto{\pgfqpoint{1.142354in}{1.657467in}}%
\pgfpathlineto{\pgfqpoint{1.143503in}{1.681913in}}%
\pgfpathlineto{\pgfqpoint{1.144651in}{1.665615in}}%
\pgfpathlineto{\pgfqpoint{1.145802in}{1.677838in}}%
\pgfpathlineto{\pgfqpoint{1.146185in}{1.661541in}}%
\pgfpathlineto{\pgfqpoint{1.146951in}{1.673764in}}%
\pgfpathlineto{\pgfqpoint{1.147335in}{1.673764in}}%
\pgfpathlineto{\pgfqpoint{1.147718in}{1.681913in}}%
\pgfpathlineto{\pgfqpoint{1.147718in}{1.681913in}}%
\pgfpathlineto{\pgfqpoint{1.147718in}{1.681913in}}%
\pgfpathlineto{\pgfqpoint{1.148868in}{1.661541in}}%
\pgfpathlineto{\pgfqpoint{1.149336in}{1.690062in}}%
\pgfpathlineto{\pgfqpoint{1.150104in}{1.673764in}}%
\pgfpathlineto{\pgfqpoint{1.150486in}{1.669690in}}%
\pgfpathlineto{\pgfqpoint{1.150870in}{1.677838in}}%
\pgfpathlineto{\pgfqpoint{1.150870in}{1.677838in}}%
\pgfpathlineto{\pgfqpoint{1.150870in}{1.677838in}}%
\pgfpathlineto{\pgfqpoint{1.152022in}{1.653392in}}%
\pgfpathlineto{\pgfqpoint{1.153555in}{1.681913in}}%
\pgfpathlineto{\pgfqpoint{1.153938in}{1.681913in}}%
\pgfpathlineto{\pgfqpoint{1.154705in}{1.677838in}}%
\pgfpathlineto{\pgfqpoint{1.155088in}{1.657467in}}%
\pgfpathlineto{\pgfqpoint{1.155853in}{1.665615in}}%
\pgfpathlineto{\pgfqpoint{1.156239in}{1.685987in}}%
\pgfpathlineto{\pgfqpoint{1.157005in}{1.669690in}}%
\pgfpathlineto{\pgfqpoint{1.157771in}{1.685987in}}%
\pgfpathlineto{\pgfqpoint{1.158154in}{1.673764in}}%
\pgfpathlineto{\pgfqpoint{1.158537in}{1.657467in}}%
\pgfpathlineto{\pgfqpoint{1.159304in}{1.669690in}}%
\pgfpathlineto{\pgfqpoint{1.160454in}{1.690062in}}%
\pgfpathlineto{\pgfqpoint{1.160838in}{1.685987in}}%
\pgfpathlineto{\pgfqpoint{1.161988in}{1.657467in}}%
\pgfpathlineto{\pgfqpoint{1.163520in}{1.677838in}}%
\pgfpathlineto{\pgfqpoint{1.164295in}{1.685987in}}%
\pgfpathlineto{\pgfqpoint{1.165829in}{1.657467in}}%
\pgfpathlineto{\pgfqpoint{1.166979in}{1.677838in}}%
\pgfpathlineto{\pgfqpoint{1.167361in}{1.673764in}}%
\pgfpathlineto{\pgfqpoint{1.167745in}{1.669690in}}%
\pgfpathlineto{\pgfqpoint{1.168129in}{1.677838in}}%
\pgfpathlineto{\pgfqpoint{1.168512in}{1.661541in}}%
\pgfpathlineto{\pgfqpoint{1.168512in}{1.661541in}}%
\pgfpathlineto{\pgfqpoint{1.168512in}{1.661541in}}%
\pgfpathlineto{\pgfqpoint{1.168896in}{1.685987in}}%
\pgfpathlineto{\pgfqpoint{1.169662in}{1.673764in}}%
\pgfpathlineto{\pgfqpoint{1.170428in}{1.661541in}}%
\pgfpathlineto{\pgfqpoint{1.171195in}{1.681913in}}%
\pgfpathlineto{\pgfqpoint{1.171579in}{1.673764in}}%
\pgfpathlineto{\pgfqpoint{1.171962in}{1.669690in}}%
\pgfpathlineto{\pgfqpoint{1.172345in}{1.677838in}}%
\pgfpathlineto{\pgfqpoint{1.172728in}{1.673764in}}%
\pgfpathlineto{\pgfqpoint{1.173494in}{1.661541in}}%
\pgfpathlineto{\pgfqpoint{1.173876in}{1.669690in}}%
\pgfpathlineto{\pgfqpoint{1.175027in}{1.681913in}}%
\pgfpathlineto{\pgfqpoint{1.175410in}{1.661541in}}%
\pgfpathlineto{\pgfqpoint{1.176177in}{1.669690in}}%
\pgfpathlineto{\pgfqpoint{1.176562in}{1.669690in}}%
\pgfpathlineto{\pgfqpoint{1.176947in}{1.661541in}}%
\pgfpathlineto{\pgfqpoint{1.178096in}{1.685987in}}%
\pgfpathlineto{\pgfqpoint{1.178479in}{1.657467in}}%
\pgfpathlineto{\pgfqpoint{1.179247in}{1.673764in}}%
\pgfpathlineto{\pgfqpoint{1.179630in}{1.681913in}}%
\pgfpathlineto{\pgfqpoint{1.180013in}{1.677838in}}%
\pgfpathlineto{\pgfqpoint{1.180396in}{1.657467in}}%
\pgfpathlineto{\pgfqpoint{1.181162in}{1.673764in}}%
\pgfpathlineto{\pgfqpoint{1.181928in}{1.669690in}}%
\pgfpathlineto{\pgfqpoint{1.182313in}{1.681913in}}%
\pgfpathlineto{\pgfqpoint{1.182695in}{1.673764in}}%
\pgfpathlineto{\pgfqpoint{1.183078in}{1.653392in}}%
\pgfpathlineto{\pgfqpoint{1.183078in}{1.653392in}}%
\pgfpathlineto{\pgfqpoint{1.183078in}{1.653392in}}%
\pgfpathlineto{\pgfqpoint{1.184227in}{1.690062in}}%
\pgfpathlineto{\pgfqpoint{1.184996in}{1.665615in}}%
\pgfpathlineto{\pgfqpoint{1.185378in}{1.698210in}}%
\pgfpathlineto{\pgfqpoint{1.185378in}{1.698210in}}%
\pgfpathlineto{\pgfqpoint{1.185378in}{1.698210in}}%
\pgfpathlineto{\pgfqpoint{1.185762in}{1.653392in}}%
\pgfpathlineto{\pgfqpoint{1.186528in}{1.673764in}}%
\pgfpathlineto{\pgfqpoint{1.187295in}{1.681913in}}%
\pgfpathlineto{\pgfqpoint{1.188062in}{1.649318in}}%
\pgfpathlineto{\pgfqpoint{1.188445in}{1.677838in}}%
\pgfpathlineto{\pgfqpoint{1.189595in}{1.657467in}}%
\pgfpathlineto{\pgfqpoint{1.189978in}{1.665615in}}%
\pgfpathlineto{\pgfqpoint{1.190363in}{1.694136in}}%
\pgfpathlineto{\pgfqpoint{1.191129in}{1.673764in}}%
\pgfpathlineto{\pgfqpoint{1.191513in}{1.677838in}}%
\pgfpathlineto{\pgfqpoint{1.191895in}{1.665615in}}%
\pgfpathlineto{\pgfqpoint{1.191895in}{1.665615in}}%
\pgfpathlineto{\pgfqpoint{1.191895in}{1.665615in}}%
\pgfpathlineto{\pgfqpoint{1.192278in}{1.681913in}}%
\pgfpathlineto{\pgfqpoint{1.193046in}{1.669690in}}%
\pgfpathlineto{\pgfqpoint{1.193428in}{1.677838in}}%
\pgfpathlineto{\pgfqpoint{1.193811in}{1.657467in}}%
\pgfpathlineto{\pgfqpoint{1.194279in}{1.698210in}}%
\pgfpathlineto{\pgfqpoint{1.194663in}{1.690062in}}%
\pgfpathlineto{\pgfqpoint{1.195046in}{1.661541in}}%
\pgfpathlineto{\pgfqpoint{1.195812in}{1.665615in}}%
\pgfpathlineto{\pgfqpoint{1.196578in}{1.677838in}}%
\pgfpathlineto{\pgfqpoint{1.196961in}{1.673764in}}%
\pgfpathlineto{\pgfqpoint{1.197344in}{1.673764in}}%
\pgfpathlineto{\pgfqpoint{1.198492in}{1.690062in}}%
\pgfpathlineto{\pgfqpoint{1.198876in}{1.685987in}}%
\pgfpathlineto{\pgfqpoint{1.199259in}{1.673764in}}%
\pgfpathlineto{\pgfqpoint{1.200027in}{1.681913in}}%
\pgfpathlineto{\pgfqpoint{1.200410in}{1.685987in}}%
\pgfpathlineto{\pgfqpoint{1.201177in}{1.661541in}}%
\pgfpathlineto{\pgfqpoint{1.201560in}{1.677838in}}%
\pgfpathlineto{\pgfqpoint{1.201944in}{1.669690in}}%
\pgfpathlineto{\pgfqpoint{1.201944in}{1.669690in}}%
\pgfpathlineto{\pgfqpoint{1.201944in}{1.669690in}}%
\pgfpathlineto{\pgfqpoint{1.202327in}{1.681913in}}%
\pgfpathlineto{\pgfqpoint{1.203092in}{1.673764in}}%
\pgfpathlineto{\pgfqpoint{1.204251in}{1.673764in}}%
\pgfpathlineto{\pgfqpoint{1.204634in}{1.657467in}}%
\pgfpathlineto{\pgfqpoint{1.204634in}{1.657467in}}%
\pgfpathlineto{\pgfqpoint{1.204634in}{1.657467in}}%
\pgfpathlineto{\pgfqpoint{1.205783in}{1.694136in}}%
\pgfpathlineto{\pgfqpoint{1.206166in}{1.665615in}}%
\pgfpathlineto{\pgfqpoint{1.206550in}{1.669690in}}%
\pgfpathlineto{\pgfqpoint{1.206933in}{1.698210in}}%
\pgfpathlineto{\pgfqpoint{1.207699in}{1.673764in}}%
\pgfpathlineto{\pgfqpoint{1.208082in}{1.673764in}}%
\pgfpathlineto{\pgfqpoint{1.209999in}{1.694136in}}%
\pgfpathlineto{\pgfqpoint{1.211531in}{1.661541in}}%
\pgfpathlineto{\pgfqpoint{1.212298in}{1.694136in}}%
\pgfpathlineto{\pgfqpoint{1.212682in}{1.669690in}}%
\pgfpathlineto{\pgfqpoint{1.213448in}{1.657467in}}%
\pgfpathlineto{\pgfqpoint{1.213832in}{1.665615in}}%
\pgfpathlineto{\pgfqpoint{1.215366in}{1.690062in}}%
\pgfpathlineto{\pgfqpoint{1.215749in}{1.685987in}}%
\pgfpathlineto{\pgfqpoint{1.216132in}{1.669690in}}%
\pgfpathlineto{\pgfqpoint{1.216515in}{1.685987in}}%
\pgfpathlineto{\pgfqpoint{1.216898in}{1.694136in}}%
\pgfpathlineto{\pgfqpoint{1.218815in}{1.653392in}}%
\pgfpathlineto{\pgfqpoint{1.219966in}{1.673764in}}%
\pgfpathlineto{\pgfqpoint{1.220350in}{1.657467in}}%
\pgfpathlineto{\pgfqpoint{1.220350in}{1.657467in}}%
\pgfpathlineto{\pgfqpoint{1.220350in}{1.657467in}}%
\pgfpathlineto{\pgfqpoint{1.221116in}{1.690062in}}%
\pgfpathlineto{\pgfqpoint{1.221499in}{1.673764in}}%
\pgfpathlineto{\pgfqpoint{1.222265in}{1.661541in}}%
\pgfpathlineto{\pgfqpoint{1.222649in}{1.677838in}}%
\pgfpathlineto{\pgfqpoint{1.223033in}{1.669690in}}%
\pgfpathlineto{\pgfqpoint{1.224181in}{1.653392in}}%
\pgfpathlineto{\pgfqpoint{1.225332in}{1.669690in}}%
\pgfpathlineto{\pgfqpoint{1.225716in}{1.669690in}}%
\pgfpathlineto{\pgfqpoint{1.226100in}{1.665615in}}%
\pgfpathlineto{\pgfqpoint{1.227249in}{1.681913in}}%
\pgfpathlineto{\pgfqpoint{1.227632in}{1.661541in}}%
\pgfpathlineto{\pgfqpoint{1.228016in}{1.669690in}}%
\pgfpathlineto{\pgfqpoint{1.229166in}{1.685987in}}%
\pgfpathlineto{\pgfqpoint{1.229549in}{1.653392in}}%
\pgfpathlineto{\pgfqpoint{1.230315in}{1.677838in}}%
\pgfpathlineto{\pgfqpoint{1.230699in}{1.698210in}}%
\pgfpathlineto{\pgfqpoint{1.230699in}{1.698210in}}%
\pgfpathlineto{\pgfqpoint{1.230699in}{1.698210in}}%
\pgfpathlineto{\pgfqpoint{1.231083in}{1.661541in}}%
\pgfpathlineto{\pgfqpoint{1.231849in}{1.669690in}}%
\pgfpathlineto{\pgfqpoint{1.232232in}{1.669690in}}%
\pgfpathlineto{\pgfqpoint{1.233766in}{1.690062in}}%
\pgfpathlineto{\pgfqpoint{1.234915in}{1.673764in}}%
\pgfpathlineto{\pgfqpoint{1.235384in}{1.677838in}}%
\pgfpathlineto{\pgfqpoint{1.236151in}{1.669690in}}%
\pgfpathlineto{\pgfqpoint{1.236534in}{1.681913in}}%
\pgfpathlineto{\pgfqpoint{1.236534in}{1.681913in}}%
\pgfpathlineto{\pgfqpoint{1.236534in}{1.681913in}}%
\pgfpathlineto{\pgfqpoint{1.236918in}{1.665615in}}%
\pgfpathlineto{\pgfqpoint{1.236918in}{1.665615in}}%
\pgfpathlineto{\pgfqpoint{1.236918in}{1.665615in}}%
\pgfpathlineto{\pgfqpoint{1.238452in}{1.702285in}}%
\pgfpathlineto{\pgfqpoint{1.238836in}{1.661541in}}%
\pgfpathlineto{\pgfqpoint{1.239603in}{1.665615in}}%
\pgfpathlineto{\pgfqpoint{1.239986in}{1.665615in}}%
\pgfpathlineto{\pgfqpoint{1.240370in}{1.673764in}}%
\pgfpathlineto{\pgfqpoint{1.240753in}{1.653392in}}%
\pgfpathlineto{\pgfqpoint{1.241136in}{1.669690in}}%
\pgfpathlineto{\pgfqpoint{1.242670in}{1.690062in}}%
\pgfpathlineto{\pgfqpoint{1.243054in}{1.661541in}}%
\pgfpathlineto{\pgfqpoint{1.243829in}{1.669690in}}%
\pgfpathlineto{\pgfqpoint{1.244978in}{1.677838in}}%
\pgfpathlineto{\pgfqpoint{1.245362in}{1.673764in}}%
\pgfpathlineto{\pgfqpoint{1.246512in}{1.685987in}}%
\pgfpathlineto{\pgfqpoint{1.246895in}{1.669690in}}%
\pgfpathlineto{\pgfqpoint{1.247662in}{1.681913in}}%
\pgfpathlineto{\pgfqpoint{1.248429in}{1.669690in}}%
\pgfpathlineto{\pgfqpoint{1.248812in}{1.681913in}}%
\pgfpathlineto{\pgfqpoint{1.249578in}{1.673764in}}%
\pgfpathlineto{\pgfqpoint{1.249961in}{1.677838in}}%
\pgfpathlineto{\pgfqpoint{1.250343in}{1.665615in}}%
\pgfpathlineto{\pgfqpoint{1.250726in}{1.677838in}}%
\pgfpathlineto{\pgfqpoint{1.251108in}{1.685987in}}%
\pgfpathlineto{\pgfqpoint{1.251492in}{1.665615in}}%
\pgfpathlineto{\pgfqpoint{1.252260in}{1.681913in}}%
\pgfpathlineto{\pgfqpoint{1.253792in}{1.665615in}}%
\pgfpathlineto{\pgfqpoint{1.254942in}{1.685987in}}%
\pgfpathlineto{\pgfqpoint{1.255325in}{1.661541in}}%
\pgfpathlineto{\pgfqpoint{1.256090in}{1.665615in}}%
\pgfpathlineto{\pgfqpoint{1.257240in}{1.681913in}}%
\pgfpathlineto{\pgfqpoint{1.258390in}{1.649318in}}%
\pgfpathlineto{\pgfqpoint{1.258774in}{1.657467in}}%
\pgfpathlineto{\pgfqpoint{1.259922in}{1.685987in}}%
\pgfpathlineto{\pgfqpoint{1.260306in}{1.681913in}}%
\pgfpathlineto{\pgfqpoint{1.260689in}{1.661541in}}%
\pgfpathlineto{\pgfqpoint{1.261072in}{1.681913in}}%
\pgfpathlineto{\pgfqpoint{1.261455in}{1.690062in}}%
\pgfpathlineto{\pgfqpoint{1.261455in}{1.690062in}}%
\pgfpathlineto{\pgfqpoint{1.261455in}{1.690062in}}%
\pgfpathlineto{\pgfqpoint{1.262221in}{1.657467in}}%
\pgfpathlineto{\pgfqpoint{1.262604in}{1.681913in}}%
\pgfpathlineto{\pgfqpoint{1.262987in}{1.690062in}}%
\pgfpathlineto{\pgfqpoint{1.263371in}{1.673764in}}%
\pgfpathlineto{\pgfqpoint{1.264137in}{1.677838in}}%
\pgfpathlineto{\pgfqpoint{1.264520in}{1.673764in}}%
\pgfpathlineto{\pgfqpoint{1.264902in}{1.685987in}}%
\pgfpathlineto{\pgfqpoint{1.265285in}{1.661541in}}%
\pgfpathlineto{\pgfqpoint{1.266053in}{1.669690in}}%
\pgfpathlineto{\pgfqpoint{1.267201in}{1.690062in}}%
\pgfpathlineto{\pgfqpoint{1.267583in}{1.669690in}}%
\pgfpathlineto{\pgfqpoint{1.268351in}{1.685987in}}%
\pgfpathlineto{\pgfqpoint{1.268735in}{1.685987in}}%
\pgfpathlineto{\pgfqpoint{1.269118in}{1.681913in}}%
\pgfpathlineto{\pgfqpoint{1.269501in}{1.661541in}}%
\pgfpathlineto{\pgfqpoint{1.270268in}{1.677838in}}%
\pgfpathlineto{\pgfqpoint{1.270651in}{1.690062in}}%
\pgfpathlineto{\pgfqpoint{1.271034in}{1.657467in}}%
\pgfpathlineto{\pgfqpoint{1.271801in}{1.673764in}}%
\pgfpathlineto{\pgfqpoint{1.272183in}{1.698210in}}%
\pgfpathlineto{\pgfqpoint{1.272567in}{1.673764in}}%
\pgfpathlineto{\pgfqpoint{1.273332in}{1.673764in}}%
\pgfpathlineto{\pgfqpoint{1.273714in}{1.665615in}}%
\pgfpathlineto{\pgfqpoint{1.274565in}{1.669690in}}%
\pgfpathlineto{\pgfqpoint{1.274950in}{1.665615in}}%
\pgfpathlineto{\pgfqpoint{1.275333in}{1.669690in}}%
\pgfpathlineto{\pgfqpoint{1.275716in}{1.673764in}}%
\pgfpathlineto{\pgfqpoint{1.276099in}{1.661541in}}%
\pgfpathlineto{\pgfqpoint{1.276482in}{1.665615in}}%
\pgfpathlineto{\pgfqpoint{1.277249in}{1.681913in}}%
\pgfpathlineto{\pgfqpoint{1.277632in}{1.677838in}}%
\pgfpathlineto{\pgfqpoint{1.278398in}{1.649318in}}%
\pgfpathlineto{\pgfqpoint{1.278781in}{1.669690in}}%
\pgfpathlineto{\pgfqpoint{1.279547in}{1.677838in}}%
\pgfpathlineto{\pgfqpoint{1.279930in}{1.673764in}}%
\pgfpathlineto{\pgfqpoint{1.280313in}{1.645243in}}%
\pgfpathlineto{\pgfqpoint{1.280313in}{1.645243in}}%
\pgfpathlineto{\pgfqpoint{1.280313in}{1.645243in}}%
\pgfpathlineto{\pgfqpoint{1.280698in}{1.677838in}}%
\pgfpathlineto{\pgfqpoint{1.281464in}{1.665615in}}%
\pgfpathlineto{\pgfqpoint{1.281850in}{1.665615in}}%
\pgfpathlineto{\pgfqpoint{1.282616in}{1.673764in}}%
\pgfpathlineto{\pgfqpoint{1.282999in}{1.661541in}}%
\pgfpathlineto{\pgfqpoint{1.282999in}{1.661541in}}%
\pgfpathlineto{\pgfqpoint{1.282999in}{1.661541in}}%
\pgfpathlineto{\pgfqpoint{1.284157in}{1.677838in}}%
\pgfpathlineto{\pgfqpoint{1.284924in}{1.677838in}}%
\pgfpathlineto{\pgfqpoint{1.285308in}{1.694136in}}%
\pgfpathlineto{\pgfqpoint{1.285308in}{1.694136in}}%
\pgfpathlineto{\pgfqpoint{1.285308in}{1.694136in}}%
\pgfpathlineto{\pgfqpoint{1.286457in}{1.665615in}}%
\pgfpathlineto{\pgfqpoint{1.286840in}{1.694136in}}%
\pgfpathlineto{\pgfqpoint{1.287224in}{1.669690in}}%
\pgfpathlineto{\pgfqpoint{1.287607in}{1.653392in}}%
\pgfpathlineto{\pgfqpoint{1.287607in}{1.653392in}}%
\pgfpathlineto{\pgfqpoint{1.287607in}{1.653392in}}%
\pgfpathlineto{\pgfqpoint{1.289141in}{1.694136in}}%
\pgfpathlineto{\pgfqpoint{1.289524in}{1.694136in}}%
\pgfpathlineto{\pgfqpoint{1.289907in}{1.677838in}}%
\pgfpathlineto{\pgfqpoint{1.290291in}{1.681913in}}%
\pgfpathlineto{\pgfqpoint{1.291059in}{1.694136in}}%
\pgfpathlineto{\pgfqpoint{1.292207in}{1.665615in}}%
\pgfpathlineto{\pgfqpoint{1.293359in}{1.685987in}}%
\pgfpathlineto{\pgfqpoint{1.293742in}{1.685987in}}%
\pgfpathlineto{\pgfqpoint{1.294125in}{1.657467in}}%
\pgfpathlineto{\pgfqpoint{1.294890in}{1.673764in}}%
\pgfpathlineto{\pgfqpoint{1.296425in}{1.653392in}}%
\pgfpathlineto{\pgfqpoint{1.297191in}{1.673764in}}%
\pgfpathlineto{\pgfqpoint{1.297574in}{1.669690in}}%
\pgfpathlineto{\pgfqpoint{1.297958in}{1.669690in}}%
\pgfpathlineto{\pgfqpoint{1.299490in}{1.681913in}}%
\pgfpathlineto{\pgfqpoint{1.300257in}{1.665615in}}%
\pgfpathlineto{\pgfqpoint{1.300640in}{1.681913in}}%
\pgfpathlineto{\pgfqpoint{1.301407in}{1.673764in}}%
\pgfpathlineto{\pgfqpoint{1.302173in}{1.690062in}}%
\pgfpathlineto{\pgfqpoint{1.302556in}{1.677838in}}%
\pgfpathlineto{\pgfqpoint{1.302939in}{1.665615in}}%
\pgfpathlineto{\pgfqpoint{1.303322in}{1.669690in}}%
\pgfpathlineto{\pgfqpoint{1.303706in}{1.681913in}}%
\pgfpathlineto{\pgfqpoint{1.304090in}{1.649318in}}%
\pgfpathlineto{\pgfqpoint{1.304474in}{1.669690in}}%
\pgfpathlineto{\pgfqpoint{1.304857in}{1.690062in}}%
\pgfpathlineto{\pgfqpoint{1.305623in}{1.677838in}}%
\pgfpathlineto{\pgfqpoint{1.306389in}{1.685987in}}%
\pgfpathlineto{\pgfqpoint{1.307156in}{1.669690in}}%
\pgfpathlineto{\pgfqpoint{1.307539in}{1.685987in}}%
\pgfpathlineto{\pgfqpoint{1.307539in}{1.685987in}}%
\pgfpathlineto{\pgfqpoint{1.307539in}{1.685987in}}%
\pgfpathlineto{\pgfqpoint{1.308305in}{1.653392in}}%
\pgfpathlineto{\pgfqpoint{1.308688in}{1.677838in}}%
\pgfpathlineto{\pgfqpoint{1.310220in}{1.661541in}}%
\pgfpathlineto{\pgfqpoint{1.311369in}{1.681913in}}%
\pgfpathlineto{\pgfqpoint{1.311753in}{1.657467in}}%
\pgfpathlineto{\pgfqpoint{1.312136in}{1.681913in}}%
\pgfpathlineto{\pgfqpoint{1.312520in}{1.681913in}}%
\pgfpathlineto{\pgfqpoint{1.313668in}{1.665615in}}%
\pgfpathlineto{\pgfqpoint{1.314434in}{1.677838in}}%
\pgfpathlineto{\pgfqpoint{1.314817in}{1.665615in}}%
\pgfpathlineto{\pgfqpoint{1.314817in}{1.665615in}}%
\pgfpathlineto{\pgfqpoint{1.314817in}{1.665615in}}%
\pgfpathlineto{\pgfqpoint{1.315201in}{1.681913in}}%
\pgfpathlineto{\pgfqpoint{1.315584in}{1.677838in}}%
\pgfpathlineto{\pgfqpoint{1.316435in}{1.657467in}}%
\pgfpathlineto{\pgfqpoint{1.316818in}{1.685987in}}%
\pgfpathlineto{\pgfqpoint{1.317584in}{1.669690in}}%
\pgfpathlineto{\pgfqpoint{1.317966in}{1.669690in}}%
\pgfpathlineto{\pgfqpoint{1.318350in}{1.665615in}}%
\pgfpathlineto{\pgfqpoint{1.320264in}{1.690062in}}%
\pgfpathlineto{\pgfqpoint{1.321413in}{1.661541in}}%
\pgfpathlineto{\pgfqpoint{1.323328in}{1.685987in}}%
\pgfpathlineto{\pgfqpoint{1.324486in}{1.673764in}}%
\pgfpathlineto{\pgfqpoint{1.324869in}{1.673764in}}%
\pgfpathlineto{\pgfqpoint{1.326402in}{1.661541in}}%
\pgfpathlineto{\pgfqpoint{1.326785in}{1.673764in}}%
\pgfpathlineto{\pgfqpoint{1.327168in}{1.661541in}}%
\pgfpathlineto{\pgfqpoint{1.327551in}{1.661541in}}%
\pgfpathlineto{\pgfqpoint{1.328317in}{1.694136in}}%
\pgfpathlineto{\pgfqpoint{1.328699in}{1.681913in}}%
\pgfpathlineto{\pgfqpoint{1.329083in}{1.657467in}}%
\pgfpathlineto{\pgfqpoint{1.329467in}{1.681913in}}%
\pgfpathlineto{\pgfqpoint{1.329850in}{1.681913in}}%
\pgfpathlineto{\pgfqpoint{1.330615in}{1.661541in}}%
\pgfpathlineto{\pgfqpoint{1.330998in}{1.681913in}}%
\pgfpathlineto{\pgfqpoint{1.331381in}{1.673764in}}%
\pgfpathlineto{\pgfqpoint{1.331764in}{1.661541in}}%
\pgfpathlineto{\pgfqpoint{1.332531in}{1.669690in}}%
\pgfpathlineto{\pgfqpoint{1.332914in}{1.681913in}}%
\pgfpathlineto{\pgfqpoint{1.333297in}{1.677838in}}%
\pgfpathlineto{\pgfqpoint{1.333681in}{1.661541in}}%
\pgfpathlineto{\pgfqpoint{1.334064in}{1.669690in}}%
\pgfpathlineto{\pgfqpoint{1.334830in}{1.690062in}}%
\pgfpathlineto{\pgfqpoint{1.335213in}{1.685987in}}%
\pgfpathlineto{\pgfqpoint{1.335596in}{1.681913in}}%
\pgfpathlineto{\pgfqpoint{1.336363in}{1.657467in}}%
\pgfpathlineto{\pgfqpoint{1.337512in}{1.690062in}}%
\pgfpathlineto{\pgfqpoint{1.339428in}{1.657467in}}%
\pgfpathlineto{\pgfqpoint{1.340961in}{1.677838in}}%
\pgfpathlineto{\pgfqpoint{1.341345in}{1.653392in}}%
\pgfpathlineto{\pgfqpoint{1.342111in}{1.657467in}}%
\pgfpathlineto{\pgfqpoint{1.343260in}{1.677838in}}%
\pgfpathlineto{\pgfqpoint{1.343644in}{1.665615in}}%
\pgfpathlineto{\pgfqpoint{1.343644in}{1.665615in}}%
\pgfpathlineto{\pgfqpoint{1.343644in}{1.665615in}}%
\pgfpathlineto{\pgfqpoint{1.344410in}{1.685987in}}%
\pgfpathlineto{\pgfqpoint{1.344793in}{1.669690in}}%
\pgfpathlineto{\pgfqpoint{1.345559in}{1.685987in}}%
\pgfpathlineto{\pgfqpoint{1.345943in}{1.681913in}}%
\pgfpathlineto{\pgfqpoint{1.346326in}{1.665615in}}%
\pgfpathlineto{\pgfqpoint{1.347093in}{1.669690in}}%
\pgfpathlineto{\pgfqpoint{1.347860in}{1.681913in}}%
\pgfpathlineto{\pgfqpoint{1.349010in}{1.669690in}}%
\pgfpathlineto{\pgfqpoint{1.349393in}{1.677838in}}%
\pgfpathlineto{\pgfqpoint{1.349393in}{1.677838in}}%
\pgfpathlineto{\pgfqpoint{1.349393in}{1.677838in}}%
\pgfpathlineto{\pgfqpoint{1.349776in}{1.665615in}}%
\pgfpathlineto{\pgfqpoint{1.350160in}{1.669690in}}%
\pgfpathlineto{\pgfqpoint{1.350542in}{1.681913in}}%
\pgfpathlineto{\pgfqpoint{1.350542in}{1.681913in}}%
\pgfpathlineto{\pgfqpoint{1.350542in}{1.681913in}}%
\pgfpathlineto{\pgfqpoint{1.350925in}{1.665615in}}%
\pgfpathlineto{\pgfqpoint{1.351393in}{1.681913in}}%
\pgfpathlineto{\pgfqpoint{1.351776in}{1.685987in}}%
\pgfpathlineto{\pgfqpoint{1.352927in}{1.669690in}}%
\pgfpathlineto{\pgfqpoint{1.353310in}{1.677838in}}%
\pgfpathlineto{\pgfqpoint{1.353310in}{1.677838in}}%
\pgfpathlineto{\pgfqpoint{1.353310in}{1.677838in}}%
\pgfpathlineto{\pgfqpoint{1.354459in}{1.661541in}}%
\pgfpathlineto{\pgfqpoint{1.354842in}{1.690062in}}%
\pgfpathlineto{\pgfqpoint{1.355609in}{1.673764in}}%
\pgfpathlineto{\pgfqpoint{1.356375in}{1.665615in}}%
\pgfpathlineto{\pgfqpoint{1.356757in}{1.681913in}}%
\pgfpathlineto{\pgfqpoint{1.356757in}{1.681913in}}%
\pgfpathlineto{\pgfqpoint{1.356757in}{1.681913in}}%
\pgfpathlineto{\pgfqpoint{1.357140in}{1.653392in}}%
\pgfpathlineto{\pgfqpoint{1.357523in}{1.657467in}}%
\pgfpathlineto{\pgfqpoint{1.357906in}{1.685987in}}%
\pgfpathlineto{\pgfqpoint{1.358672in}{1.677838in}}%
\pgfpathlineto{\pgfqpoint{1.359440in}{1.657467in}}%
\pgfpathlineto{\pgfqpoint{1.360973in}{1.694136in}}%
\pgfpathlineto{\pgfqpoint{1.362122in}{1.669690in}}%
\pgfpathlineto{\pgfqpoint{1.362888in}{1.681913in}}%
\pgfpathlineto{\pgfqpoint{1.363662in}{1.653392in}}%
\pgfpathlineto{\pgfqpoint{1.364045in}{1.673764in}}%
\pgfpathlineto{\pgfqpoint{1.365192in}{1.661541in}}%
\pgfpathlineto{\pgfqpoint{1.366726in}{1.677838in}}%
\pgfpathlineto{\pgfqpoint{1.367110in}{1.677838in}}%
\pgfpathlineto{\pgfqpoint{1.368259in}{1.665615in}}%
\pgfpathlineto{\pgfqpoint{1.368643in}{1.677838in}}%
\pgfpathlineto{\pgfqpoint{1.368643in}{1.677838in}}%
\pgfpathlineto{\pgfqpoint{1.368643in}{1.677838in}}%
\pgfpathlineto{\pgfqpoint{1.369025in}{1.661541in}}%
\pgfpathlineto{\pgfqpoint{1.369408in}{1.669690in}}%
\pgfpathlineto{\pgfqpoint{1.369791in}{1.677838in}}%
\pgfpathlineto{\pgfqpoint{1.370173in}{1.669690in}}%
\pgfpathlineto{\pgfqpoint{1.370557in}{1.669690in}}%
\pgfpathlineto{\pgfqpoint{1.372090in}{1.661541in}}%
\pgfpathlineto{\pgfqpoint{1.372857in}{1.673764in}}%
\pgfpathlineto{\pgfqpoint{1.373240in}{1.669690in}}%
\pgfpathlineto{\pgfqpoint{1.373623in}{1.669690in}}%
\pgfpathlineto{\pgfqpoint{1.374391in}{1.681913in}}%
\pgfpathlineto{\pgfqpoint{1.374773in}{1.702285in}}%
\pgfpathlineto{\pgfqpoint{1.374773in}{1.702285in}}%
\pgfpathlineto{\pgfqpoint{1.374773in}{1.702285in}}%
\pgfpathlineto{\pgfqpoint{1.375157in}{1.665615in}}%
\pgfpathlineto{\pgfqpoint{1.375922in}{1.677838in}}%
\pgfpathlineto{\pgfqpoint{1.376305in}{1.681913in}}%
\pgfpathlineto{\pgfqpoint{1.377457in}{1.657467in}}%
\pgfpathlineto{\pgfqpoint{1.378989in}{1.685987in}}%
\pgfpathlineto{\pgfqpoint{1.380523in}{1.661541in}}%
\pgfpathlineto{\pgfqpoint{1.380906in}{1.669690in}}%
\pgfpathlineto{\pgfqpoint{1.381672in}{1.681913in}}%
\pgfpathlineto{\pgfqpoint{1.382056in}{1.669690in}}%
\pgfpathlineto{\pgfqpoint{1.382823in}{1.677838in}}%
\pgfpathlineto{\pgfqpoint{1.383590in}{1.657467in}}%
\pgfpathlineto{\pgfqpoint{1.384739in}{1.677838in}}%
\pgfpathlineto{\pgfqpoint{1.385505in}{1.665615in}}%
\pgfpathlineto{\pgfqpoint{1.385888in}{1.681913in}}%
\pgfpathlineto{\pgfqpoint{1.386271in}{1.673764in}}%
\pgfpathlineto{\pgfqpoint{1.386653in}{1.657467in}}%
\pgfpathlineto{\pgfqpoint{1.386653in}{1.657467in}}%
\pgfpathlineto{\pgfqpoint{1.386653in}{1.657467in}}%
\pgfpathlineto{\pgfqpoint{1.387418in}{1.685987in}}%
\pgfpathlineto{\pgfqpoint{1.387887in}{1.673764in}}%
\pgfpathlineto{\pgfqpoint{1.388270in}{1.677838in}}%
\pgfpathlineto{\pgfqpoint{1.389037in}{1.669690in}}%
\pgfpathlineto{\pgfqpoint{1.389420in}{1.690062in}}%
\pgfpathlineto{\pgfqpoint{1.390186in}{1.681913in}}%
\pgfpathlineto{\pgfqpoint{1.390951in}{1.669690in}}%
\pgfpathlineto{\pgfqpoint{1.391334in}{1.677838in}}%
\pgfpathlineto{\pgfqpoint{1.392101in}{1.677838in}}%
\pgfpathlineto{\pgfqpoint{1.393634in}{1.661541in}}%
\pgfpathlineto{\pgfqpoint{1.394018in}{1.677838in}}%
\pgfpathlineto{\pgfqpoint{1.394018in}{1.677838in}}%
\pgfpathlineto{\pgfqpoint{1.394018in}{1.677838in}}%
\pgfpathlineto{\pgfqpoint{1.394401in}{1.649318in}}%
\pgfpathlineto{\pgfqpoint{1.394401in}{1.649318in}}%
\pgfpathlineto{\pgfqpoint{1.394401in}{1.649318in}}%
\pgfpathlineto{\pgfqpoint{1.395935in}{1.690062in}}%
\pgfpathlineto{\pgfqpoint{1.396318in}{1.673764in}}%
\pgfpathlineto{\pgfqpoint{1.396318in}{1.673764in}}%
\pgfpathlineto{\pgfqpoint{1.396318in}{1.673764in}}%
\pgfpathlineto{\pgfqpoint{1.396701in}{1.698210in}}%
\pgfpathlineto{\pgfqpoint{1.396701in}{1.698210in}}%
\pgfpathlineto{\pgfqpoint{1.396701in}{1.698210in}}%
\pgfpathlineto{\pgfqpoint{1.397468in}{1.669690in}}%
\pgfpathlineto{\pgfqpoint{1.397851in}{1.681913in}}%
\pgfpathlineto{\pgfqpoint{1.398235in}{1.685987in}}%
\pgfpathlineto{\pgfqpoint{1.398618in}{1.665615in}}%
\pgfpathlineto{\pgfqpoint{1.399001in}{1.669690in}}%
\pgfpathlineto{\pgfqpoint{1.399384in}{1.685987in}}%
\pgfpathlineto{\pgfqpoint{1.400151in}{1.681913in}}%
\pgfpathlineto{\pgfqpoint{1.400918in}{1.673764in}}%
\pgfpathlineto{\pgfqpoint{1.401683in}{1.685987in}}%
\pgfpathlineto{\pgfqpoint{1.402450in}{1.669690in}}%
\pgfpathlineto{\pgfqpoint{1.402833in}{1.673764in}}%
\pgfpathlineto{\pgfqpoint{1.403218in}{1.685987in}}%
\pgfpathlineto{\pgfqpoint{1.403218in}{1.685987in}}%
\pgfpathlineto{\pgfqpoint{1.403218in}{1.685987in}}%
\pgfpathlineto{\pgfqpoint{1.404375in}{1.669690in}}%
\pgfpathlineto{\pgfqpoint{1.404759in}{1.685987in}}%
\pgfpathlineto{\pgfqpoint{1.405525in}{1.681913in}}%
\pgfpathlineto{\pgfqpoint{1.405908in}{1.661541in}}%
\pgfpathlineto{\pgfqpoint{1.406674in}{1.673764in}}%
\pgfpathlineto{\pgfqpoint{1.407058in}{1.677838in}}%
\pgfpathlineto{\pgfqpoint{1.407440in}{1.673764in}}%
\pgfpathlineto{\pgfqpoint{1.407823in}{1.669690in}}%
\pgfpathlineto{\pgfqpoint{1.408588in}{1.685987in}}%
\pgfpathlineto{\pgfqpoint{1.408971in}{1.649318in}}%
\pgfpathlineto{\pgfqpoint{1.409738in}{1.669690in}}%
\pgfpathlineto{\pgfqpoint{1.410122in}{1.685987in}}%
\pgfpathlineto{\pgfqpoint{1.410122in}{1.685987in}}%
\pgfpathlineto{\pgfqpoint{1.410122in}{1.685987in}}%
\pgfpathlineto{\pgfqpoint{1.410887in}{1.665615in}}%
\pgfpathlineto{\pgfqpoint{1.411270in}{1.677838in}}%
\pgfpathlineto{\pgfqpoint{1.411651in}{1.685987in}}%
\pgfpathlineto{\pgfqpoint{1.411651in}{1.685987in}}%
\pgfpathlineto{\pgfqpoint{1.411651in}{1.685987in}}%
\pgfpathlineto{\pgfqpoint{1.412034in}{1.673764in}}%
\pgfpathlineto{\pgfqpoint{1.412416in}{1.677838in}}%
\pgfpathlineto{\pgfqpoint{1.412799in}{1.698210in}}%
\pgfpathlineto{\pgfqpoint{1.412799in}{1.698210in}}%
\pgfpathlineto{\pgfqpoint{1.412799in}{1.698210in}}%
\pgfpathlineto{\pgfqpoint{1.414334in}{1.665615in}}%
\pgfpathlineto{\pgfqpoint{1.414717in}{1.694136in}}%
\pgfpathlineto{\pgfqpoint{1.415484in}{1.690062in}}%
\pgfpathlineto{\pgfqpoint{1.416252in}{1.661541in}}%
\pgfpathlineto{\pgfqpoint{1.416635in}{1.665615in}}%
\pgfpathlineto{\pgfqpoint{1.417401in}{1.685987in}}%
\pgfpathlineto{\pgfqpoint{1.418167in}{1.645243in}}%
\pgfpathlineto{\pgfqpoint{1.418552in}{1.681913in}}%
\pgfpathlineto{\pgfqpoint{1.418935in}{1.669690in}}%
\pgfpathlineto{\pgfqpoint{1.419701in}{1.677838in}}%
\pgfpathlineto{\pgfqpoint{1.420084in}{1.685987in}}%
\pgfpathlineto{\pgfqpoint{1.420851in}{1.681913in}}%
\pgfpathlineto{\pgfqpoint{1.421617in}{1.669690in}}%
\pgfpathlineto{\pgfqpoint{1.422001in}{1.690062in}}%
\pgfpathlineto{\pgfqpoint{1.422384in}{1.673764in}}%
\pgfpathlineto{\pgfqpoint{1.423918in}{1.665615in}}%
\pgfpathlineto{\pgfqpoint{1.425068in}{1.690062in}}%
\pgfpathlineto{\pgfqpoint{1.425834in}{1.669690in}}%
\pgfpathlineto{\pgfqpoint{1.426217in}{1.677838in}}%
\pgfpathlineto{\pgfqpoint{1.426601in}{1.681913in}}%
\pgfpathlineto{\pgfqpoint{1.427836in}{1.673764in}}%
\pgfpathlineto{\pgfqpoint{1.429372in}{1.685987in}}%
\pgfpathlineto{\pgfqpoint{1.430522in}{1.673764in}}%
\pgfpathlineto{\pgfqpoint{1.430905in}{1.698210in}}%
\pgfpathlineto{\pgfqpoint{1.431289in}{1.681913in}}%
\pgfpathlineto{\pgfqpoint{1.432057in}{1.661541in}}%
\pgfpathlineto{\pgfqpoint{1.432441in}{1.665615in}}%
\pgfpathlineto{\pgfqpoint{1.432825in}{1.690062in}}%
\pgfpathlineto{\pgfqpoint{1.432825in}{1.690062in}}%
\pgfpathlineto{\pgfqpoint{1.432825in}{1.690062in}}%
\pgfpathlineto{\pgfqpoint{1.433208in}{1.653392in}}%
\pgfpathlineto{\pgfqpoint{1.433974in}{1.669690in}}%
\pgfpathlineto{\pgfqpoint{1.434358in}{1.661541in}}%
\pgfpathlineto{\pgfqpoint{1.434742in}{1.690062in}}%
\pgfpathlineto{\pgfqpoint{1.435508in}{1.677838in}}%
\pgfpathlineto{\pgfqpoint{1.436657in}{1.657467in}}%
\pgfpathlineto{\pgfqpoint{1.437424in}{1.685987in}}%
\pgfpathlineto{\pgfqpoint{1.438191in}{1.669690in}}%
\pgfpathlineto{\pgfqpoint{1.438574in}{1.657467in}}%
\pgfpathlineto{\pgfqpoint{1.438957in}{1.685987in}}%
\pgfpathlineto{\pgfqpoint{1.439724in}{1.665615in}}%
\pgfpathlineto{\pgfqpoint{1.440107in}{1.653392in}}%
\pgfpathlineto{\pgfqpoint{1.440875in}{1.685987in}}%
\pgfpathlineto{\pgfqpoint{1.441257in}{1.681913in}}%
\pgfpathlineto{\pgfqpoint{1.441640in}{1.681913in}}%
\pgfpathlineto{\pgfqpoint{1.442790in}{1.669690in}}%
\pgfpathlineto{\pgfqpoint{1.443565in}{1.694136in}}%
\pgfpathlineto{\pgfqpoint{1.443949in}{1.677838in}}%
\pgfpathlineto{\pgfqpoint{1.445099in}{1.657467in}}%
\pgfpathlineto{\pgfqpoint{1.445483in}{1.694136in}}%
\pgfpathlineto{\pgfqpoint{1.446249in}{1.681913in}}%
\pgfpathlineto{\pgfqpoint{1.446632in}{1.669690in}}%
\pgfpathlineto{\pgfqpoint{1.446632in}{1.669690in}}%
\pgfpathlineto{\pgfqpoint{1.446632in}{1.669690in}}%
\pgfpathlineto{\pgfqpoint{1.447016in}{1.685987in}}%
\pgfpathlineto{\pgfqpoint{1.447782in}{1.681913in}}%
\pgfpathlineto{\pgfqpoint{1.448165in}{1.681913in}}%
\pgfpathlineto{\pgfqpoint{1.448547in}{1.677838in}}%
\pgfpathlineto{\pgfqpoint{1.448930in}{1.661541in}}%
\pgfpathlineto{\pgfqpoint{1.449695in}{1.673764in}}%
\pgfpathlineto{\pgfqpoint{1.450081in}{1.677838in}}%
\pgfpathlineto{\pgfqpoint{1.450464in}{1.673764in}}%
\pgfpathlineto{\pgfqpoint{1.450847in}{1.673764in}}%
\pgfpathlineto{\pgfqpoint{1.451230in}{1.669690in}}%
\pgfpathlineto{\pgfqpoint{1.451613in}{1.698210in}}%
\pgfpathlineto{\pgfqpoint{1.452380in}{1.673764in}}%
\pgfpathlineto{\pgfqpoint{1.453529in}{1.694136in}}%
\pgfpathlineto{\pgfqpoint{1.454296in}{1.665615in}}%
\pgfpathlineto{\pgfqpoint{1.454679in}{1.669690in}}%
\pgfpathlineto{\pgfqpoint{1.456595in}{1.694136in}}%
\pgfpathlineto{\pgfqpoint{1.458512in}{1.657467in}}%
\pgfpathlineto{\pgfqpoint{1.461195in}{1.690062in}}%
\pgfpathlineto{\pgfqpoint{1.462729in}{1.661541in}}%
\pgfpathlineto{\pgfqpoint{1.463112in}{1.677838in}}%
\pgfpathlineto{\pgfqpoint{1.463496in}{1.673764in}}%
\pgfpathlineto{\pgfqpoint{1.463880in}{1.661541in}}%
\pgfpathlineto{\pgfqpoint{1.464263in}{1.673764in}}%
\pgfpathlineto{\pgfqpoint{1.464645in}{1.694136in}}%
\pgfpathlineto{\pgfqpoint{1.465411in}{1.677838in}}%
\pgfpathlineto{\pgfqpoint{1.466945in}{1.665615in}}%
\pgfpathlineto{\pgfqpoint{1.468094in}{1.685987in}}%
\pgfpathlineto{\pgfqpoint{1.468478in}{1.681913in}}%
\pgfpathlineto{\pgfqpoint{1.468861in}{1.661541in}}%
\pgfpathlineto{\pgfqpoint{1.469629in}{1.673764in}}%
\pgfpathlineto{\pgfqpoint{1.470011in}{1.677838in}}%
\pgfpathlineto{\pgfqpoint{1.470778in}{1.661541in}}%
\pgfpathlineto{\pgfqpoint{1.471546in}{1.694136in}}%
\pgfpathlineto{\pgfqpoint{1.471929in}{1.685987in}}%
\pgfpathlineto{\pgfqpoint{1.472398in}{1.685987in}}%
\pgfpathlineto{\pgfqpoint{1.473548in}{1.681913in}}%
\pgfpathlineto{\pgfqpoint{1.473930in}{1.657467in}}%
\pgfpathlineto{\pgfqpoint{1.474313in}{1.665615in}}%
\pgfpathlineto{\pgfqpoint{1.475080in}{1.685987in}}%
\pgfpathlineto{\pgfqpoint{1.475463in}{1.677838in}}%
\pgfpathlineto{\pgfqpoint{1.476994in}{1.669690in}}%
\pgfpathlineto{\pgfqpoint{1.477378in}{1.681913in}}%
\pgfpathlineto{\pgfqpoint{1.477762in}{1.657467in}}%
\pgfpathlineto{\pgfqpoint{1.478527in}{1.665615in}}%
\pgfpathlineto{\pgfqpoint{1.479293in}{1.677838in}}%
\pgfpathlineto{\pgfqpoint{1.479675in}{1.661541in}}%
\pgfpathlineto{\pgfqpoint{1.480442in}{1.665615in}}%
\pgfpathlineto{\pgfqpoint{1.481974in}{1.685987in}}%
\pgfpathlineto{\pgfqpoint{1.482357in}{1.698210in}}%
\pgfpathlineto{\pgfqpoint{1.482740in}{1.685987in}}%
\pgfpathlineto{\pgfqpoint{1.483123in}{1.685987in}}%
\pgfpathlineto{\pgfqpoint{1.483514in}{1.669690in}}%
\pgfpathlineto{\pgfqpoint{1.483514in}{1.669690in}}%
\pgfpathlineto{\pgfqpoint{1.483514in}{1.669690in}}%
\pgfpathlineto{\pgfqpoint{1.483897in}{1.690062in}}%
\pgfpathlineto{\pgfqpoint{1.484663in}{1.677838in}}%
\pgfpathlineto{\pgfqpoint{1.485429in}{1.669690in}}%
\pgfpathlineto{\pgfqpoint{1.485813in}{1.681913in}}%
\pgfpathlineto{\pgfqpoint{1.486579in}{1.673764in}}%
\pgfpathlineto{\pgfqpoint{1.487345in}{1.690062in}}%
\pgfpathlineto{\pgfqpoint{1.487727in}{1.685987in}}%
\pgfpathlineto{\pgfqpoint{1.488111in}{1.665615in}}%
\pgfpathlineto{\pgfqpoint{1.488878in}{1.669690in}}%
\pgfpathlineto{\pgfqpoint{1.490026in}{1.685987in}}%
\pgfpathlineto{\pgfqpoint{1.490409in}{1.661541in}}%
\pgfpathlineto{\pgfqpoint{1.491176in}{1.665615in}}%
\pgfpathlineto{\pgfqpoint{1.492709in}{1.685987in}}%
\pgfpathlineto{\pgfqpoint{1.493475in}{1.669690in}}%
\pgfpathlineto{\pgfqpoint{1.493858in}{1.673764in}}%
\pgfpathlineto{\pgfqpoint{1.494242in}{1.677838in}}%
\pgfpathlineto{\pgfqpoint{1.494624in}{1.669690in}}%
\pgfpathlineto{\pgfqpoint{1.495008in}{1.685987in}}%
\pgfpathlineto{\pgfqpoint{1.495391in}{1.681913in}}%
\pgfpathlineto{\pgfqpoint{1.495774in}{1.665615in}}%
\pgfpathlineto{\pgfqpoint{1.496542in}{1.677838in}}%
\pgfpathlineto{\pgfqpoint{1.497308in}{1.690062in}}%
\pgfpathlineto{\pgfqpoint{1.498074in}{1.669690in}}%
\pgfpathlineto{\pgfqpoint{1.498457in}{1.677838in}}%
\pgfpathlineto{\pgfqpoint{1.499615in}{1.677838in}}%
\pgfpathlineto{\pgfqpoint{1.501153in}{1.657467in}}%
\pgfpathlineto{\pgfqpoint{1.501536in}{1.685987in}}%
\pgfpathlineto{\pgfqpoint{1.502302in}{1.677838in}}%
\pgfpathlineto{\pgfqpoint{1.503453in}{1.681913in}}%
\pgfpathlineto{\pgfqpoint{1.503837in}{1.681913in}}%
\pgfpathlineto{\pgfqpoint{1.504600in}{1.665615in}}%
\pgfpathlineto{\pgfqpoint{1.504984in}{1.677838in}}%
\pgfpathlineto{\pgfqpoint{1.506133in}{1.669690in}}%
\pgfpathlineto{\pgfqpoint{1.507285in}{1.677838in}}%
\pgfpathlineto{\pgfqpoint{1.508051in}{1.665615in}}%
\pgfpathlineto{\pgfqpoint{1.508434in}{1.669690in}}%
\pgfpathlineto{\pgfqpoint{1.509201in}{1.690062in}}%
\pgfpathlineto{\pgfqpoint{1.510351in}{1.657467in}}%
\pgfpathlineto{\pgfqpoint{1.511886in}{1.685987in}}%
\pgfpathlineto{\pgfqpoint{1.512269in}{1.677838in}}%
\pgfpathlineto{\pgfqpoint{1.512269in}{1.677838in}}%
\pgfpathlineto{\pgfqpoint{1.512269in}{1.677838in}}%
\pgfpathlineto{\pgfqpoint{1.513418in}{1.694136in}}%
\pgfpathlineto{\pgfqpoint{1.513801in}{1.657467in}}%
\pgfpathlineto{\pgfqpoint{1.514570in}{1.665615in}}%
\pgfpathlineto{\pgfqpoint{1.514952in}{1.694136in}}%
\pgfpathlineto{\pgfqpoint{1.515804in}{1.677838in}}%
\pgfpathlineto{\pgfqpoint{1.516186in}{1.694136in}}%
\pgfpathlineto{\pgfqpoint{1.516186in}{1.694136in}}%
\pgfpathlineto{\pgfqpoint{1.516186in}{1.694136in}}%
\pgfpathlineto{\pgfqpoint{1.517721in}{1.665615in}}%
\pgfpathlineto{\pgfqpoint{1.518104in}{1.673764in}}%
\pgfpathlineto{\pgfqpoint{1.518487in}{1.653392in}}%
\pgfpathlineto{\pgfqpoint{1.518487in}{1.653392in}}%
\pgfpathlineto{\pgfqpoint{1.518487in}{1.653392in}}%
\pgfpathlineto{\pgfqpoint{1.518870in}{1.681913in}}%
\pgfpathlineto{\pgfqpoint{1.519636in}{1.673764in}}%
\pgfpathlineto{\pgfqpoint{1.520019in}{1.665615in}}%
\pgfpathlineto{\pgfqpoint{1.520403in}{1.685987in}}%
\pgfpathlineto{\pgfqpoint{1.520786in}{1.673764in}}%
\pgfpathlineto{\pgfqpoint{1.521171in}{1.665615in}}%
\pgfpathlineto{\pgfqpoint{1.521554in}{1.694136in}}%
\pgfpathlineto{\pgfqpoint{1.521937in}{1.677838in}}%
\pgfpathlineto{\pgfqpoint{1.523087in}{1.665615in}}%
\pgfpathlineto{\pgfqpoint{1.523478in}{1.665615in}}%
\pgfpathlineto{\pgfqpoint{1.524244in}{1.677838in}}%
\pgfpathlineto{\pgfqpoint{1.524628in}{1.657467in}}%
\pgfpathlineto{\pgfqpoint{1.525012in}{1.665615in}}%
\pgfpathlineto{\pgfqpoint{1.525395in}{1.681913in}}%
\pgfpathlineto{\pgfqpoint{1.526161in}{1.677838in}}%
\pgfpathlineto{\pgfqpoint{1.526544in}{1.681913in}}%
\pgfpathlineto{\pgfqpoint{1.526927in}{1.665615in}}%
\pgfpathlineto{\pgfqpoint{1.527311in}{1.681913in}}%
\pgfpathlineto{\pgfqpoint{1.527695in}{1.690062in}}%
\pgfpathlineto{\pgfqpoint{1.528078in}{1.685987in}}%
\pgfpathlineto{\pgfqpoint{1.529226in}{1.669690in}}%
\pgfpathlineto{\pgfqpoint{1.530376in}{1.694136in}}%
\pgfpathlineto{\pgfqpoint{1.531525in}{1.657467in}}%
\pgfpathlineto{\pgfqpoint{1.531907in}{1.669690in}}%
\pgfpathlineto{\pgfqpoint{1.533057in}{1.681913in}}%
\pgfpathlineto{\pgfqpoint{1.534973in}{1.661541in}}%
\pgfpathlineto{\pgfqpoint{1.535356in}{1.657467in}}%
\pgfpathlineto{\pgfqpoint{1.536889in}{1.702285in}}%
\pgfpathlineto{\pgfqpoint{1.538037in}{1.669690in}}%
\pgfpathlineto{\pgfqpoint{1.538804in}{1.681913in}}%
\pgfpathlineto{\pgfqpoint{1.540337in}{1.657467in}}%
\pgfpathlineto{\pgfqpoint{1.541487in}{1.677838in}}%
\pgfpathlineto{\pgfqpoint{1.542637in}{1.653392in}}%
\pgfpathlineto{\pgfqpoint{1.544172in}{1.681913in}}%
\pgfpathlineto{\pgfqpoint{1.545321in}{1.669690in}}%
\pgfpathlineto{\pgfqpoint{1.545703in}{1.685987in}}%
\pgfpathlineto{\pgfqpoint{1.546087in}{1.653392in}}%
\pgfpathlineto{\pgfqpoint{1.546854in}{1.657467in}}%
\pgfpathlineto{\pgfqpoint{1.547237in}{1.690062in}}%
\pgfpathlineto{\pgfqpoint{1.548003in}{1.681913in}}%
\pgfpathlineto{\pgfqpoint{1.549153in}{1.661541in}}%
\pgfpathlineto{\pgfqpoint{1.550303in}{1.673764in}}%
\pgfpathlineto{\pgfqpoint{1.550687in}{1.665615in}}%
\pgfpathlineto{\pgfqpoint{1.551067in}{1.669690in}}%
\pgfpathlineto{\pgfqpoint{1.551833in}{1.694136in}}%
\pgfpathlineto{\pgfqpoint{1.552217in}{1.677838in}}%
\pgfpathlineto{\pgfqpoint{1.552601in}{1.665615in}}%
\pgfpathlineto{\pgfqpoint{1.553750in}{1.698210in}}%
\pgfpathlineto{\pgfqpoint{1.555367in}{1.657467in}}%
\pgfpathlineto{\pgfqpoint{1.556135in}{1.673764in}}%
\pgfpathlineto{\pgfqpoint{1.556518in}{1.669690in}}%
\pgfpathlineto{\pgfqpoint{1.558049in}{1.681913in}}%
\pgfpathlineto{\pgfqpoint{1.559582in}{1.661541in}}%
\pgfpathlineto{\pgfqpoint{1.560732in}{1.677838in}}%
\pgfpathlineto{\pgfqpoint{1.561115in}{1.690062in}}%
\pgfpathlineto{\pgfqpoint{1.561498in}{1.653392in}}%
\pgfpathlineto{\pgfqpoint{1.562265in}{1.673764in}}%
\pgfpathlineto{\pgfqpoint{1.562649in}{1.669690in}}%
\pgfpathlineto{\pgfqpoint{1.563423in}{1.685987in}}%
\pgfpathlineto{\pgfqpoint{1.563806in}{1.673764in}}%
\pgfpathlineto{\pgfqpoint{1.564189in}{1.673764in}}%
\pgfpathlineto{\pgfqpoint{1.564573in}{1.669690in}}%
\pgfpathlineto{\pgfqpoint{1.564956in}{1.690062in}}%
\pgfpathlineto{\pgfqpoint{1.565339in}{1.673764in}}%
\pgfpathlineto{\pgfqpoint{1.565722in}{1.661541in}}%
\pgfpathlineto{\pgfqpoint{1.566105in}{1.685987in}}%
\pgfpathlineto{\pgfqpoint{1.566872in}{1.673764in}}%
\pgfpathlineto{\pgfqpoint{1.567256in}{1.673764in}}%
\pgfpathlineto{\pgfqpoint{1.568022in}{1.685987in}}%
\pgfpathlineto{\pgfqpoint{1.569556in}{1.649318in}}%
\pgfpathlineto{\pgfqpoint{1.569940in}{1.690062in}}%
\pgfpathlineto{\pgfqpoint{1.570705in}{1.685987in}}%
\pgfpathlineto{\pgfqpoint{1.571089in}{1.685987in}}%
\pgfpathlineto{\pgfqpoint{1.571474in}{1.694136in}}%
\pgfpathlineto{\pgfqpoint{1.571474in}{1.694136in}}%
\pgfpathlineto{\pgfqpoint{1.571474in}{1.694136in}}%
\pgfpathlineto{\pgfqpoint{1.572622in}{1.673764in}}%
\pgfpathlineto{\pgfqpoint{1.573773in}{1.694136in}}%
\pgfpathlineto{\pgfqpoint{1.574539in}{1.661541in}}%
\pgfpathlineto{\pgfqpoint{1.574922in}{1.677838in}}%
\pgfpathlineto{\pgfqpoint{1.575305in}{1.665615in}}%
\pgfpathlineto{\pgfqpoint{1.576072in}{1.669690in}}%
\pgfpathlineto{\pgfqpoint{1.576455in}{1.685987in}}%
\pgfpathlineto{\pgfqpoint{1.577222in}{1.681913in}}%
\pgfpathlineto{\pgfqpoint{1.578373in}{1.690062in}}%
\pgfpathlineto{\pgfqpoint{1.578755in}{1.673764in}}%
\pgfpathlineto{\pgfqpoint{1.578755in}{1.673764in}}%
\pgfpathlineto{\pgfqpoint{1.578755in}{1.673764in}}%
\pgfpathlineto{\pgfqpoint{1.579138in}{1.698210in}}%
\pgfpathlineto{\pgfqpoint{1.579138in}{1.698210in}}%
\pgfpathlineto{\pgfqpoint{1.579138in}{1.698210in}}%
\pgfpathlineto{\pgfqpoint{1.579521in}{1.665615in}}%
\pgfpathlineto{\pgfqpoint{1.580289in}{1.685987in}}%
\pgfpathlineto{\pgfqpoint{1.580673in}{1.690062in}}%
\pgfpathlineto{\pgfqpoint{1.581821in}{1.669690in}}%
\pgfpathlineto{\pgfqpoint{1.582972in}{1.685987in}}%
\pgfpathlineto{\pgfqpoint{1.583355in}{1.677838in}}%
\pgfpathlineto{\pgfqpoint{1.583739in}{1.649318in}}%
\pgfpathlineto{\pgfqpoint{1.584122in}{1.665615in}}%
\pgfpathlineto{\pgfqpoint{1.585273in}{1.694136in}}%
\pgfpathlineto{\pgfqpoint{1.585655in}{1.690062in}}%
\pgfpathlineto{\pgfqpoint{1.586804in}{1.653392in}}%
\pgfpathlineto{\pgfqpoint{1.587955in}{1.685987in}}%
\pgfpathlineto{\pgfqpoint{1.589105in}{1.661541in}}%
\pgfpathlineto{\pgfqpoint{1.589488in}{1.665615in}}%
\pgfpathlineto{\pgfqpoint{1.590637in}{1.681913in}}%
\pgfpathlineto{\pgfqpoint{1.591020in}{1.673764in}}%
\pgfpathlineto{\pgfqpoint{1.591787in}{1.677838in}}%
\pgfpathlineto{\pgfqpoint{1.592170in}{1.677838in}}%
\pgfpathlineto{\pgfqpoint{1.592553in}{1.657467in}}%
\pgfpathlineto{\pgfqpoint{1.592936in}{1.665615in}}%
\pgfpathlineto{\pgfqpoint{1.593703in}{1.694136in}}%
\pgfpathlineto{\pgfqpoint{1.594469in}{1.661541in}}%
\pgfpathlineto{\pgfqpoint{1.594937in}{1.665615in}}%
\pgfpathlineto{\pgfqpoint{1.596469in}{1.681913in}}%
\pgfpathlineto{\pgfqpoint{1.596853in}{1.657467in}}%
\pgfpathlineto{\pgfqpoint{1.596853in}{1.657467in}}%
\pgfpathlineto{\pgfqpoint{1.596853in}{1.657467in}}%
\pgfpathlineto{\pgfqpoint{1.597236in}{1.690062in}}%
\pgfpathlineto{\pgfqpoint{1.598002in}{1.673764in}}%
\pgfpathlineto{\pgfqpoint{1.598768in}{1.685987in}}%
\pgfpathlineto{\pgfqpoint{1.599535in}{1.669690in}}%
\pgfpathlineto{\pgfqpoint{1.599918in}{1.681913in}}%
\pgfpathlineto{\pgfqpoint{1.600302in}{1.681913in}}%
\pgfpathlineto{\pgfqpoint{1.600685in}{1.641169in}}%
\pgfpathlineto{\pgfqpoint{1.600685in}{1.641169in}}%
\pgfpathlineto{\pgfqpoint{1.600685in}{1.641169in}}%
\pgfpathlineto{\pgfqpoint{1.601068in}{1.690062in}}%
\pgfpathlineto{\pgfqpoint{1.601833in}{1.669690in}}%
\pgfpathlineto{\pgfqpoint{1.602601in}{1.685987in}}%
\pgfpathlineto{\pgfqpoint{1.602984in}{1.673764in}}%
\pgfpathlineto{\pgfqpoint{1.603374in}{1.681913in}}%
\pgfpathlineto{\pgfqpoint{1.603758in}{1.661541in}}%
\pgfpathlineto{\pgfqpoint{1.603758in}{1.661541in}}%
\pgfpathlineto{\pgfqpoint{1.603758in}{1.661541in}}%
\pgfpathlineto{\pgfqpoint{1.604142in}{1.690062in}}%
\pgfpathlineto{\pgfqpoint{1.604909in}{1.677838in}}%
\pgfpathlineto{\pgfqpoint{1.605292in}{1.690062in}}%
\pgfpathlineto{\pgfqpoint{1.605675in}{1.685987in}}%
\pgfpathlineto{\pgfqpoint{1.606443in}{1.669690in}}%
\pgfpathlineto{\pgfqpoint{1.606827in}{1.681913in}}%
\pgfpathlineto{\pgfqpoint{1.607594in}{1.690062in}}%
\pgfpathlineto{\pgfqpoint{1.608743in}{1.653392in}}%
\pgfpathlineto{\pgfqpoint{1.610277in}{1.681913in}}%
\pgfpathlineto{\pgfqpoint{1.610660in}{1.661541in}}%
\pgfpathlineto{\pgfqpoint{1.611043in}{1.677838in}}%
\pgfpathlineto{\pgfqpoint{1.611426in}{1.681913in}}%
\pgfpathlineto{\pgfqpoint{1.612192in}{1.653392in}}%
\pgfpathlineto{\pgfqpoint{1.612575in}{1.665615in}}%
\pgfpathlineto{\pgfqpoint{1.612958in}{1.694136in}}%
\pgfpathlineto{\pgfqpoint{1.613341in}{1.669690in}}%
\pgfpathlineto{\pgfqpoint{1.613724in}{1.661541in}}%
\pgfpathlineto{\pgfqpoint{1.613724in}{1.661541in}}%
\pgfpathlineto{\pgfqpoint{1.613724in}{1.661541in}}%
\pgfpathlineto{\pgfqpoint{1.614875in}{1.677838in}}%
\pgfpathlineto{\pgfqpoint{1.615259in}{1.669690in}}%
\pgfpathlineto{\pgfqpoint{1.615643in}{1.673764in}}%
\pgfpathlineto{\pgfqpoint{1.616026in}{1.690062in}}%
\pgfpathlineto{\pgfqpoint{1.616409in}{1.673764in}}%
\pgfpathlineto{\pgfqpoint{1.616793in}{1.673764in}}%
\pgfpathlineto{\pgfqpoint{1.617177in}{1.657467in}}%
\pgfpathlineto{\pgfqpoint{1.617177in}{1.657467in}}%
\pgfpathlineto{\pgfqpoint{1.617177in}{1.657467in}}%
\pgfpathlineto{\pgfqpoint{1.617943in}{1.685987in}}%
\pgfpathlineto{\pgfqpoint{1.618326in}{1.669690in}}%
\pgfpathlineto{\pgfqpoint{1.618709in}{1.669690in}}%
\pgfpathlineto{\pgfqpoint{1.619093in}{1.661541in}}%
\pgfpathlineto{\pgfqpoint{1.619859in}{1.665615in}}%
\pgfpathlineto{\pgfqpoint{1.620626in}{1.698210in}}%
\pgfpathlineto{\pgfqpoint{1.621009in}{1.673764in}}%
\pgfpathlineto{\pgfqpoint{1.621393in}{1.665615in}}%
\pgfpathlineto{\pgfqpoint{1.621776in}{1.673764in}}%
\pgfpathlineto{\pgfqpoint{1.622542in}{1.690062in}}%
\pgfpathlineto{\pgfqpoint{1.622928in}{1.669690in}}%
\pgfpathlineto{\pgfqpoint{1.623695in}{1.673764in}}%
\pgfpathlineto{\pgfqpoint{1.624077in}{1.677838in}}%
\pgfpathlineto{\pgfqpoint{1.624460in}{1.665615in}}%
\pgfpathlineto{\pgfqpoint{1.624460in}{1.665615in}}%
\pgfpathlineto{\pgfqpoint{1.624460in}{1.665615in}}%
\pgfpathlineto{\pgfqpoint{1.625610in}{1.690062in}}%
\pgfpathlineto{\pgfqpoint{1.626759in}{1.657467in}}%
\pgfpathlineto{\pgfqpoint{1.627909in}{1.694136in}}%
\pgfpathlineto{\pgfqpoint{1.629061in}{1.681913in}}%
\pgfpathlineto{\pgfqpoint{1.629444in}{1.690062in}}%
\pgfpathlineto{\pgfqpoint{1.629826in}{1.685987in}}%
\pgfpathlineto{\pgfqpoint{1.630593in}{1.669690in}}%
\pgfpathlineto{\pgfqpoint{1.630976in}{0.667390in}}%
\pgfpathlineto{\pgfqpoint{1.631359in}{1.669690in}}%
\pgfpathlineto{\pgfqpoint{1.632126in}{1.690062in}}%
\pgfpathlineto{\pgfqpoint{1.632509in}{1.661541in}}%
\pgfpathlineto{\pgfqpoint{1.633276in}{1.673764in}}%
\pgfpathlineto{\pgfqpoint{1.634042in}{1.665615in}}%
\pgfpathlineto{\pgfqpoint{1.635192in}{1.681913in}}%
\pgfpathlineto{\pgfqpoint{1.636725in}{1.665615in}}%
\pgfpathlineto{\pgfqpoint{1.637578in}{1.681913in}}%
\pgfpathlineto{\pgfqpoint{1.638726in}{1.649318in}}%
\pgfpathlineto{\pgfqpoint{1.639109in}{1.661541in}}%
\pgfpathlineto{\pgfqpoint{1.639493in}{1.685987in}}%
\pgfpathlineto{\pgfqpoint{1.640260in}{1.669690in}}%
\pgfpathlineto{\pgfqpoint{1.641026in}{1.665615in}}%
\pgfpathlineto{\pgfqpoint{1.641409in}{1.685987in}}%
\pgfpathlineto{\pgfqpoint{1.642175in}{1.681913in}}%
\pgfpathlineto{\pgfqpoint{1.642559in}{1.665615in}}%
\pgfpathlineto{\pgfqpoint{1.643332in}{1.677838in}}%
\pgfpathlineto{\pgfqpoint{1.643715in}{1.669690in}}%
\pgfpathlineto{\pgfqpoint{1.644096in}{1.673764in}}%
\pgfpathlineto{\pgfqpoint{1.644481in}{1.690062in}}%
\pgfpathlineto{\pgfqpoint{1.645247in}{1.681913in}}%
\pgfpathlineto{\pgfqpoint{1.645629in}{1.685987in}}%
\pgfpathlineto{\pgfqpoint{1.646011in}{1.673764in}}%
\pgfpathlineto{\pgfqpoint{1.646011in}{1.673764in}}%
\pgfpathlineto{\pgfqpoint{1.646011in}{1.673764in}}%
\pgfpathlineto{\pgfqpoint{1.646393in}{1.690062in}}%
\pgfpathlineto{\pgfqpoint{1.646393in}{1.690062in}}%
\pgfpathlineto{\pgfqpoint{1.646393in}{1.690062in}}%
\pgfpathlineto{\pgfqpoint{1.646776in}{1.665615in}}%
\pgfpathlineto{\pgfqpoint{1.647541in}{1.677838in}}%
\pgfpathlineto{\pgfqpoint{1.647923in}{1.677838in}}%
\pgfpathlineto{\pgfqpoint{1.648307in}{1.685987in}}%
\pgfpathlineto{\pgfqpoint{1.648689in}{1.669690in}}%
\pgfpathlineto{\pgfqpoint{1.649454in}{1.673764in}}%
\pgfpathlineto{\pgfqpoint{1.649837in}{1.690062in}}%
\pgfpathlineto{\pgfqpoint{1.649837in}{1.690062in}}%
\pgfpathlineto{\pgfqpoint{1.649837in}{1.690062in}}%
\pgfpathlineto{\pgfqpoint{1.651368in}{1.665615in}}%
\pgfpathlineto{\pgfqpoint{1.651751in}{1.685987in}}%
\pgfpathlineto{\pgfqpoint{1.652515in}{1.677838in}}%
\pgfpathlineto{\pgfqpoint{1.652898in}{1.677838in}}%
\pgfpathlineto{\pgfqpoint{1.653280in}{1.657467in}}%
\pgfpathlineto{\pgfqpoint{1.653663in}{1.677838in}}%
\pgfpathlineto{\pgfqpoint{1.654046in}{1.681913in}}%
\pgfpathlineto{\pgfqpoint{1.654429in}{1.677838in}}%
\pgfpathlineto{\pgfqpoint{1.655577in}{1.665615in}}%
\pgfpathlineto{\pgfqpoint{1.657106in}{1.681913in}}%
\pgfpathlineto{\pgfqpoint{1.657872in}{1.669690in}}%
\pgfpathlineto{\pgfqpoint{1.658639in}{1.702285in}}%
\pgfpathlineto{\pgfqpoint{1.659786in}{1.649318in}}%
\pgfpathlineto{\pgfqpoint{1.660169in}{1.661541in}}%
\pgfpathlineto{\pgfqpoint{1.662083in}{1.694136in}}%
\pgfpathlineto{\pgfqpoint{1.662848in}{1.665615in}}%
\pgfpathlineto{\pgfqpoint{1.663230in}{1.673764in}}%
\pgfpathlineto{\pgfqpoint{1.664080in}{1.673764in}}%
\pgfpathlineto{\pgfqpoint{1.664462in}{1.669690in}}%
\pgfpathlineto{\pgfqpoint{1.664846in}{1.681913in}}%
\pgfpathlineto{\pgfqpoint{1.665228in}{1.669690in}}%
\pgfpathlineto{\pgfqpoint{1.665613in}{1.669690in}}%
\pgfpathlineto{\pgfqpoint{1.666378in}{1.677838in}}%
\pgfpathlineto{\pgfqpoint{1.667144in}{1.661541in}}%
\pgfpathlineto{\pgfqpoint{1.667526in}{1.673764in}}%
\pgfpathlineto{\pgfqpoint{1.667909in}{1.677838in}}%
\pgfpathlineto{\pgfqpoint{1.668293in}{1.669690in}}%
\pgfpathlineto{\pgfqpoint{1.668676in}{1.690062in}}%
\pgfpathlineto{\pgfqpoint{1.669442in}{1.677838in}}%
\pgfpathlineto{\pgfqpoint{1.670591in}{1.649318in}}%
\pgfpathlineto{\pgfqpoint{1.672508in}{1.685987in}}%
\pgfpathlineto{\pgfqpoint{1.672891in}{1.669690in}}%
\pgfpathlineto{\pgfqpoint{1.672891in}{1.669690in}}%
\pgfpathlineto{\pgfqpoint{1.672891in}{1.669690in}}%
\pgfpathlineto{\pgfqpoint{1.673274in}{1.690062in}}%
\pgfpathlineto{\pgfqpoint{1.673657in}{1.669690in}}%
\pgfpathlineto{\pgfqpoint{1.674042in}{1.653392in}}%
\pgfpathlineto{\pgfqpoint{1.674042in}{1.653392in}}%
\pgfpathlineto{\pgfqpoint{1.674042in}{1.653392in}}%
\pgfpathlineto{\pgfqpoint{1.674425in}{1.673764in}}%
\pgfpathlineto{\pgfqpoint{1.674425in}{1.673764in}}%
\pgfpathlineto{\pgfqpoint{1.674425in}{1.673764in}}%
\pgfpathlineto{\pgfqpoint{1.674808in}{1.649318in}}%
\pgfpathlineto{\pgfqpoint{1.674808in}{1.649318in}}%
\pgfpathlineto{\pgfqpoint{1.674808in}{1.649318in}}%
\pgfpathlineto{\pgfqpoint{1.675191in}{1.694136in}}%
\pgfpathlineto{\pgfqpoint{1.675957in}{1.677838in}}%
\pgfpathlineto{\pgfqpoint{1.676342in}{1.677838in}}%
\pgfpathlineto{\pgfqpoint{1.677108in}{1.690062in}}%
\pgfpathlineto{\pgfqpoint{1.678641in}{1.665615in}}%
\pgfpathlineto{\pgfqpoint{1.679409in}{1.665615in}}%
\pgfpathlineto{\pgfqpoint{1.679792in}{1.677838in}}%
\pgfpathlineto{\pgfqpoint{1.680175in}{1.669690in}}%
\pgfpathlineto{\pgfqpoint{1.680941in}{1.653392in}}%
\pgfpathlineto{\pgfqpoint{1.682475in}{1.685987in}}%
\pgfpathlineto{\pgfqpoint{1.683248in}{1.673764in}}%
\pgfpathlineto{\pgfqpoint{1.683632in}{1.661541in}}%
\pgfpathlineto{\pgfqpoint{1.683632in}{1.661541in}}%
\pgfpathlineto{\pgfqpoint{1.683632in}{1.661541in}}%
\pgfpathlineto{\pgfqpoint{1.685164in}{1.690062in}}%
\pgfpathlineto{\pgfqpoint{1.685547in}{1.685987in}}%
\pgfpathlineto{\pgfqpoint{1.685930in}{1.665615in}}%
\pgfpathlineto{\pgfqpoint{1.686697in}{1.681913in}}%
\pgfpathlineto{\pgfqpoint{1.687463in}{1.661541in}}%
\pgfpathlineto{\pgfqpoint{1.687846in}{1.673764in}}%
\pgfpathlineto{\pgfqpoint{1.688229in}{1.673764in}}%
\pgfpathlineto{\pgfqpoint{1.688612in}{1.657467in}}%
\pgfpathlineto{\pgfqpoint{1.688995in}{1.665615in}}%
\pgfpathlineto{\pgfqpoint{1.689378in}{1.677838in}}%
\pgfpathlineto{\pgfqpoint{1.689762in}{1.669690in}}%
\pgfpathlineto{\pgfqpoint{1.690145in}{1.665615in}}%
\pgfpathlineto{\pgfqpoint{1.690527in}{1.681913in}}%
\pgfpathlineto{\pgfqpoint{1.690909in}{1.673764in}}%
\pgfpathlineto{\pgfqpoint{1.691293in}{1.661541in}}%
\pgfpathlineto{\pgfqpoint{1.691293in}{1.661541in}}%
\pgfpathlineto{\pgfqpoint{1.691293in}{1.661541in}}%
\pgfpathlineto{\pgfqpoint{1.691676in}{1.677838in}}%
\pgfpathlineto{\pgfqpoint{1.692444in}{1.665615in}}%
\pgfpathlineto{\pgfqpoint{1.692827in}{1.677838in}}%
\pgfpathlineto{\pgfqpoint{1.693211in}{1.673764in}}%
\pgfpathlineto{\pgfqpoint{1.693593in}{1.657467in}}%
\pgfpathlineto{\pgfqpoint{1.693593in}{1.657467in}}%
\pgfpathlineto{\pgfqpoint{1.693593in}{1.657467in}}%
\pgfpathlineto{\pgfqpoint{1.695127in}{1.685987in}}%
\pgfpathlineto{\pgfqpoint{1.695511in}{1.681913in}}%
\pgfpathlineto{\pgfqpoint{1.695894in}{1.665615in}}%
\pgfpathlineto{\pgfqpoint{1.696660in}{1.673764in}}%
\pgfpathlineto{\pgfqpoint{1.697043in}{1.677838in}}%
\pgfpathlineto{\pgfqpoint{1.697426in}{1.673764in}}%
\pgfpathlineto{\pgfqpoint{1.697809in}{1.673764in}}%
\pgfpathlineto{\pgfqpoint{1.698575in}{1.657467in}}%
\pgfpathlineto{\pgfqpoint{1.699341in}{1.698210in}}%
\pgfpathlineto{\pgfqpoint{1.699724in}{1.665615in}}%
\pgfpathlineto{\pgfqpoint{1.700192in}{1.669690in}}%
\pgfpathlineto{\pgfqpoint{1.700575in}{1.653392in}}%
\pgfpathlineto{\pgfqpoint{1.701341in}{1.685987in}}%
\pgfpathlineto{\pgfqpoint{1.701725in}{1.677838in}}%
\pgfpathlineto{\pgfqpoint{1.702109in}{1.681913in}}%
\pgfpathlineto{\pgfqpoint{1.702492in}{1.665615in}}%
\pgfpathlineto{\pgfqpoint{1.702492in}{1.665615in}}%
\pgfpathlineto{\pgfqpoint{1.702492in}{1.665615in}}%
\pgfpathlineto{\pgfqpoint{1.702875in}{1.685987in}}%
\pgfpathlineto{\pgfqpoint{1.703259in}{1.677838in}}%
\pgfpathlineto{\pgfqpoint{1.703642in}{1.665615in}}%
\pgfpathlineto{\pgfqpoint{1.704408in}{1.673764in}}%
\pgfpathlineto{\pgfqpoint{1.704792in}{1.673764in}}%
\pgfpathlineto{\pgfqpoint{1.705175in}{1.677838in}}%
\pgfpathlineto{\pgfqpoint{1.705559in}{1.665615in}}%
\pgfpathlineto{\pgfqpoint{1.705559in}{1.665615in}}%
\pgfpathlineto{\pgfqpoint{1.705559in}{1.665615in}}%
\pgfpathlineto{\pgfqpoint{1.705942in}{1.681913in}}%
\pgfpathlineto{\pgfqpoint{1.706325in}{1.665615in}}%
\pgfpathlineto{\pgfqpoint{1.706708in}{1.657467in}}%
\pgfpathlineto{\pgfqpoint{1.708242in}{1.690062in}}%
\pgfpathlineto{\pgfqpoint{1.708625in}{1.698210in}}%
\pgfpathlineto{\pgfqpoint{1.710159in}{1.673764in}}%
\pgfpathlineto{\pgfqpoint{1.710542in}{1.669690in}}%
\pgfpathlineto{\pgfqpoint{1.710925in}{1.698210in}}%
\pgfpathlineto{\pgfqpoint{1.711690in}{1.673764in}}%
\pgfpathlineto{\pgfqpoint{1.712074in}{1.661541in}}%
\pgfpathlineto{\pgfqpoint{1.712458in}{1.665615in}}%
\pgfpathlineto{\pgfqpoint{1.712843in}{1.681913in}}%
\pgfpathlineto{\pgfqpoint{1.713226in}{1.677838in}}%
\pgfpathlineto{\pgfqpoint{1.713609in}{1.665615in}}%
\pgfpathlineto{\pgfqpoint{1.714374in}{1.673764in}}%
\pgfpathlineto{\pgfqpoint{1.714757in}{1.685987in}}%
\pgfpathlineto{\pgfqpoint{1.715141in}{1.673764in}}%
\pgfpathlineto{\pgfqpoint{1.715525in}{1.665615in}}%
\pgfpathlineto{\pgfqpoint{1.716291in}{1.690062in}}%
\pgfpathlineto{\pgfqpoint{1.716675in}{1.681913in}}%
\pgfpathlineto{\pgfqpoint{1.717824in}{1.657467in}}%
\pgfpathlineto{\pgfqpoint{1.719357in}{1.685987in}}%
\pgfpathlineto{\pgfqpoint{1.720892in}{1.661541in}}%
\pgfpathlineto{\pgfqpoint{1.721658in}{1.669690in}}%
\pgfpathlineto{\pgfqpoint{1.722041in}{1.665615in}}%
\pgfpathlineto{\pgfqpoint{1.722807in}{1.665615in}}%
\pgfpathlineto{\pgfqpoint{1.724348in}{1.685987in}}%
\pgfpathlineto{\pgfqpoint{1.724732in}{1.665615in}}%
\pgfpathlineto{\pgfqpoint{1.725498in}{1.677838in}}%
\pgfpathlineto{\pgfqpoint{1.725882in}{1.669690in}}%
\pgfpathlineto{\pgfqpoint{1.725882in}{1.669690in}}%
\pgfpathlineto{\pgfqpoint{1.725882in}{1.669690in}}%
\pgfpathlineto{\pgfqpoint{1.726265in}{1.681913in}}%
\pgfpathlineto{\pgfqpoint{1.727031in}{1.673764in}}%
\pgfpathlineto{\pgfqpoint{1.727415in}{1.673764in}}%
\pgfpathlineto{\pgfqpoint{1.727798in}{1.669690in}}%
\pgfpathlineto{\pgfqpoint{1.728182in}{1.673764in}}%
\pgfpathlineto{\pgfqpoint{1.728565in}{1.673764in}}%
\pgfpathlineto{\pgfqpoint{1.728947in}{1.669690in}}%
\pgfpathlineto{\pgfqpoint{1.729330in}{1.673764in}}%
\pgfpathlineto{\pgfqpoint{1.729713in}{1.677838in}}%
\pgfpathlineto{\pgfqpoint{1.730096in}{1.661541in}}%
\pgfpathlineto{\pgfqpoint{1.730863in}{1.665615in}}%
\pgfpathlineto{\pgfqpoint{1.731630in}{1.685987in}}%
\pgfpathlineto{\pgfqpoint{1.732013in}{1.681913in}}%
\pgfpathlineto{\pgfqpoint{1.732396in}{1.661541in}}%
\pgfpathlineto{\pgfqpoint{1.732780in}{1.673764in}}%
\pgfpathlineto{\pgfqpoint{1.733930in}{1.690062in}}%
\pgfpathlineto{\pgfqpoint{1.736230in}{1.657467in}}%
\pgfpathlineto{\pgfqpoint{1.736613in}{1.661541in}}%
\pgfpathlineto{\pgfqpoint{1.737761in}{1.685987in}}%
\pgfpathlineto{\pgfqpoint{1.738910in}{1.661541in}}%
\pgfpathlineto{\pgfqpoint{1.739294in}{1.690062in}}%
\pgfpathlineto{\pgfqpoint{1.740061in}{1.685987in}}%
\pgfpathlineto{\pgfqpoint{1.740443in}{1.685987in}}%
\pgfpathlineto{\pgfqpoint{1.741678in}{1.669690in}}%
\pgfpathlineto{\pgfqpoint{1.742063in}{1.673764in}}%
\pgfpathlineto{\pgfqpoint{1.742445in}{1.685987in}}%
\pgfpathlineto{\pgfqpoint{1.742445in}{1.685987in}}%
\pgfpathlineto{\pgfqpoint{1.742445in}{1.685987in}}%
\pgfpathlineto{\pgfqpoint{1.743594in}{1.661541in}}%
\pgfpathlineto{\pgfqpoint{1.745129in}{1.694136in}}%
\pgfpathlineto{\pgfqpoint{1.745894in}{1.690062in}}%
\pgfpathlineto{\pgfqpoint{1.747427in}{1.657467in}}%
\pgfpathlineto{\pgfqpoint{1.748193in}{1.681913in}}%
\pgfpathlineto{\pgfqpoint{1.748575in}{1.649318in}}%
\pgfpathlineto{\pgfqpoint{1.749342in}{1.661541in}}%
\pgfpathlineto{\pgfqpoint{1.750493in}{1.694136in}}%
\pgfpathlineto{\pgfqpoint{1.750876in}{1.685987in}}%
\pgfpathlineto{\pgfqpoint{1.751259in}{1.653392in}}%
\pgfpathlineto{\pgfqpoint{1.751641in}{1.685987in}}%
\pgfpathlineto{\pgfqpoint{1.752407in}{1.698210in}}%
\pgfpathlineto{\pgfqpoint{1.752790in}{1.690062in}}%
\pgfpathlineto{\pgfqpoint{1.753174in}{1.669690in}}%
\pgfpathlineto{\pgfqpoint{1.753174in}{1.669690in}}%
\pgfpathlineto{\pgfqpoint{1.753174in}{1.669690in}}%
\pgfpathlineto{\pgfqpoint{1.753557in}{1.694136in}}%
\pgfpathlineto{\pgfqpoint{1.754323in}{1.673764in}}%
\pgfpathlineto{\pgfqpoint{1.755473in}{1.661541in}}%
\pgfpathlineto{\pgfqpoint{1.755856in}{1.681913in}}%
\pgfpathlineto{\pgfqpoint{1.755856in}{1.681913in}}%
\pgfpathlineto{\pgfqpoint{1.755856in}{1.681913in}}%
\pgfpathlineto{\pgfqpoint{1.756239in}{1.657467in}}%
\pgfpathlineto{\pgfqpoint{1.757005in}{1.673764in}}%
\pgfpathlineto{\pgfqpoint{1.757387in}{1.677838in}}%
\pgfpathlineto{\pgfqpoint{1.757771in}{1.657467in}}%
\pgfpathlineto{\pgfqpoint{1.758154in}{1.669690in}}%
\pgfpathlineto{\pgfqpoint{1.758537in}{1.681913in}}%
\pgfpathlineto{\pgfqpoint{1.758920in}{1.669690in}}%
\pgfpathlineto{\pgfqpoint{1.759303in}{1.665615in}}%
\pgfpathlineto{\pgfqpoint{1.759686in}{1.685987in}}%
\pgfpathlineto{\pgfqpoint{1.760453in}{1.673764in}}%
\pgfpathlineto{\pgfqpoint{1.761601in}{1.657467in}}%
\pgfpathlineto{\pgfqpoint{1.763144in}{1.690062in}}%
\pgfpathlineto{\pgfqpoint{1.763528in}{1.685987in}}%
\pgfpathlineto{\pgfqpoint{1.763912in}{1.661541in}}%
\pgfpathlineto{\pgfqpoint{1.764294in}{1.665615in}}%
\pgfpathlineto{\pgfqpoint{1.765443in}{1.685987in}}%
\pgfpathlineto{\pgfqpoint{1.766594in}{1.661541in}}%
\pgfpathlineto{\pgfqpoint{1.766977in}{1.665615in}}%
\pgfpathlineto{\pgfqpoint{1.768126in}{1.694136in}}%
\pgfpathlineto{\pgfqpoint{1.768510in}{1.661541in}}%
\pgfpathlineto{\pgfqpoint{1.769277in}{1.665615in}}%
\pgfpathlineto{\pgfqpoint{1.770809in}{1.685987in}}%
\pgfpathlineto{\pgfqpoint{1.771193in}{1.669690in}}%
\pgfpathlineto{\pgfqpoint{1.771193in}{1.669690in}}%
\pgfpathlineto{\pgfqpoint{1.771193in}{1.669690in}}%
\pgfpathlineto{\pgfqpoint{1.771577in}{1.690062in}}%
\pgfpathlineto{\pgfqpoint{1.771960in}{1.673764in}}%
\pgfpathlineto{\pgfqpoint{1.772726in}{1.657467in}}%
\pgfpathlineto{\pgfqpoint{1.774258in}{1.677838in}}%
\pgfpathlineto{\pgfqpoint{1.775407in}{1.665615in}}%
\pgfpathlineto{\pgfqpoint{1.775790in}{1.681913in}}%
\pgfpathlineto{\pgfqpoint{1.776556in}{1.673764in}}%
\pgfpathlineto{\pgfqpoint{1.777322in}{1.657467in}}%
\pgfpathlineto{\pgfqpoint{1.778557in}{1.690062in}}%
\pgfpathlineto{\pgfqpoint{1.779323in}{1.645243in}}%
\pgfpathlineto{\pgfqpoint{1.779706in}{1.681913in}}%
\pgfpathlineto{\pgfqpoint{1.780472in}{1.669690in}}%
\pgfpathlineto{\pgfqpoint{1.780855in}{1.677838in}}%
\pgfpathlineto{\pgfqpoint{1.781239in}{1.690062in}}%
\pgfpathlineto{\pgfqpoint{1.781239in}{1.690062in}}%
\pgfpathlineto{\pgfqpoint{1.781239in}{1.690062in}}%
\pgfpathlineto{\pgfqpoint{1.782771in}{1.665615in}}%
\pgfpathlineto{\pgfqpoint{1.783155in}{1.677838in}}%
\pgfpathlineto{\pgfqpoint{1.783537in}{1.669690in}}%
\pgfpathlineto{\pgfqpoint{1.783920in}{1.657467in}}%
\pgfpathlineto{\pgfqpoint{1.783920in}{1.657467in}}%
\pgfpathlineto{\pgfqpoint{1.783920in}{1.657467in}}%
\pgfpathlineto{\pgfqpoint{1.785454in}{1.677838in}}%
\pgfpathlineto{\pgfqpoint{1.785836in}{1.657467in}}%
\pgfpathlineto{\pgfqpoint{1.785836in}{1.657467in}}%
\pgfpathlineto{\pgfqpoint{1.785836in}{1.657467in}}%
\pgfpathlineto{\pgfqpoint{1.786219in}{1.681913in}}%
\pgfpathlineto{\pgfqpoint{1.786986in}{1.665615in}}%
\pgfpathlineto{\pgfqpoint{1.787370in}{1.673764in}}%
\pgfpathlineto{\pgfqpoint{1.787370in}{1.673764in}}%
\pgfpathlineto{\pgfqpoint{1.787370in}{1.673764in}}%
\pgfpathlineto{\pgfqpoint{1.787754in}{1.661541in}}%
\pgfpathlineto{\pgfqpoint{1.787754in}{1.661541in}}%
\pgfpathlineto{\pgfqpoint{1.787754in}{1.661541in}}%
\pgfpathlineto{\pgfqpoint{1.788520in}{1.677838in}}%
\pgfpathlineto{\pgfqpoint{1.788902in}{1.665615in}}%
\pgfpathlineto{\pgfqpoint{1.789286in}{1.661541in}}%
\pgfpathlineto{\pgfqpoint{1.790054in}{1.677838in}}%
\pgfpathlineto{\pgfqpoint{1.790820in}{1.673764in}}%
\pgfpathlineto{\pgfqpoint{1.791587in}{1.653392in}}%
\pgfpathlineto{\pgfqpoint{1.792738in}{1.690062in}}%
\pgfpathlineto{\pgfqpoint{1.793121in}{1.685987in}}%
\pgfpathlineto{\pgfqpoint{1.793504in}{1.665615in}}%
\pgfpathlineto{\pgfqpoint{1.793504in}{1.665615in}}%
\pgfpathlineto{\pgfqpoint{1.793504in}{1.665615in}}%
\pgfpathlineto{\pgfqpoint{1.793887in}{1.690062in}}%
\pgfpathlineto{\pgfqpoint{1.794271in}{1.677838in}}%
\pgfpathlineto{\pgfqpoint{1.794655in}{1.665615in}}%
\pgfpathlineto{\pgfqpoint{1.795421in}{1.694136in}}%
\pgfpathlineto{\pgfqpoint{1.795804in}{1.681913in}}%
\pgfpathlineto{\pgfqpoint{1.796187in}{1.661541in}}%
\pgfpathlineto{\pgfqpoint{1.796954in}{1.677838in}}%
\pgfpathlineto{\pgfqpoint{1.797338in}{1.673764in}}%
\pgfpathlineto{\pgfqpoint{1.797721in}{1.657467in}}%
\pgfpathlineto{\pgfqpoint{1.798487in}{1.665615in}}%
\pgfpathlineto{\pgfqpoint{1.798870in}{1.681913in}}%
\pgfpathlineto{\pgfqpoint{1.798870in}{1.681913in}}%
\pgfpathlineto{\pgfqpoint{1.798870in}{1.681913in}}%
\pgfpathlineto{\pgfqpoint{1.799253in}{1.661541in}}%
\pgfpathlineto{\pgfqpoint{1.800021in}{1.673764in}}%
\pgfpathlineto{\pgfqpoint{1.801171in}{1.694136in}}%
\pgfpathlineto{\pgfqpoint{1.801937in}{1.649318in}}%
\pgfpathlineto{\pgfqpoint{1.802704in}{1.673764in}}%
\pgfpathlineto{\pgfqpoint{1.803095in}{1.677838in}}%
\pgfpathlineto{\pgfqpoint{1.803480in}{1.665615in}}%
\pgfpathlineto{\pgfqpoint{1.803864in}{1.669690in}}%
\pgfpathlineto{\pgfqpoint{1.804631in}{1.681913in}}%
\pgfpathlineto{\pgfqpoint{1.805014in}{1.677838in}}%
\pgfpathlineto{\pgfqpoint{1.806163in}{1.657467in}}%
\pgfpathlineto{\pgfqpoint{1.806546in}{1.681913in}}%
\pgfpathlineto{\pgfqpoint{1.807313in}{1.677838in}}%
\pgfpathlineto{\pgfqpoint{1.807696in}{1.653392in}}%
\pgfpathlineto{\pgfqpoint{1.808463in}{1.673764in}}%
\pgfpathlineto{\pgfqpoint{1.809613in}{1.690062in}}%
\pgfpathlineto{\pgfqpoint{1.810763in}{1.669690in}}%
\pgfpathlineto{\pgfqpoint{1.811145in}{1.685987in}}%
\pgfpathlineto{\pgfqpoint{1.811528in}{1.669690in}}%
\pgfpathlineto{\pgfqpoint{1.812679in}{1.657467in}}%
\pgfpathlineto{\pgfqpoint{1.814211in}{1.673764in}}%
\pgfpathlineto{\pgfqpoint{1.814594in}{1.673764in}}%
\pgfpathlineto{\pgfqpoint{1.814977in}{1.669690in}}%
\pgfpathlineto{\pgfqpoint{1.815361in}{1.673764in}}%
\pgfpathlineto{\pgfqpoint{1.815745in}{1.685987in}}%
\pgfpathlineto{\pgfqpoint{1.816510in}{1.677838in}}%
\pgfpathlineto{\pgfqpoint{1.816893in}{1.661541in}}%
\pgfpathlineto{\pgfqpoint{1.817660in}{1.673764in}}%
\pgfpathlineto{\pgfqpoint{1.818044in}{1.665615in}}%
\pgfpathlineto{\pgfqpoint{1.818044in}{1.665615in}}%
\pgfpathlineto{\pgfqpoint{1.818044in}{1.665615in}}%
\pgfpathlineto{\pgfqpoint{1.818810in}{1.681913in}}%
\pgfpathlineto{\pgfqpoint{1.819193in}{1.661541in}}%
\pgfpathlineto{\pgfqpoint{1.819960in}{1.669690in}}%
\pgfpathlineto{\pgfqpoint{1.820811in}{1.690062in}}%
\pgfpathlineto{\pgfqpoint{1.821577in}{1.681913in}}%
\pgfpathlineto{\pgfqpoint{1.821962in}{1.677838in}}%
\pgfpathlineto{\pgfqpoint{1.822345in}{1.661541in}}%
\pgfpathlineto{\pgfqpoint{1.822345in}{1.661541in}}%
\pgfpathlineto{\pgfqpoint{1.822345in}{1.661541in}}%
\pgfpathlineto{\pgfqpoint{1.822728in}{1.681913in}}%
\pgfpathlineto{\pgfqpoint{1.823493in}{1.677838in}}%
\pgfpathlineto{\pgfqpoint{1.824645in}{1.661541in}}%
\pgfpathlineto{\pgfqpoint{1.825028in}{1.665615in}}%
\pgfpathlineto{\pgfqpoint{1.825411in}{1.681913in}}%
\pgfpathlineto{\pgfqpoint{1.825794in}{1.665615in}}%
\pgfpathlineto{\pgfqpoint{1.826176in}{1.665615in}}%
\pgfpathlineto{\pgfqpoint{1.826943in}{1.681913in}}%
\pgfpathlineto{\pgfqpoint{1.827328in}{1.673764in}}%
\pgfpathlineto{\pgfqpoint{1.827711in}{1.649318in}}%
\pgfpathlineto{\pgfqpoint{1.827711in}{1.649318in}}%
\pgfpathlineto{\pgfqpoint{1.827711in}{1.649318in}}%
\pgfpathlineto{\pgfqpoint{1.828860in}{1.690062in}}%
\pgfpathlineto{\pgfqpoint{1.830010in}{1.665615in}}%
\pgfpathlineto{\pgfqpoint{1.830394in}{1.673764in}}%
\pgfpathlineto{\pgfqpoint{1.830777in}{1.673764in}}%
\pgfpathlineto{\pgfqpoint{1.831161in}{1.681913in}}%
\pgfpathlineto{\pgfqpoint{1.831544in}{1.677838in}}%
\pgfpathlineto{\pgfqpoint{1.831927in}{1.657467in}}%
\pgfpathlineto{\pgfqpoint{1.832311in}{1.665615in}}%
\pgfpathlineto{\pgfqpoint{1.832694in}{1.690062in}}%
\pgfpathlineto{\pgfqpoint{1.833461in}{1.685987in}}%
\pgfpathlineto{\pgfqpoint{1.834610in}{1.669690in}}%
\pgfpathlineto{\pgfqpoint{1.834993in}{1.694136in}}%
\pgfpathlineto{\pgfqpoint{1.834993in}{1.694136in}}%
\pgfpathlineto{\pgfqpoint{1.834993in}{1.694136in}}%
\pgfpathlineto{\pgfqpoint{1.835378in}{1.661541in}}%
\pgfpathlineto{\pgfqpoint{1.836144in}{1.673764in}}%
\pgfpathlineto{\pgfqpoint{1.836527in}{1.673764in}}%
\pgfpathlineto{\pgfqpoint{1.836909in}{1.677838in}}%
\pgfpathlineto{\pgfqpoint{1.837290in}{1.694136in}}%
\pgfpathlineto{\pgfqpoint{1.837290in}{1.694136in}}%
\pgfpathlineto{\pgfqpoint{1.837290in}{1.694136in}}%
\pgfpathlineto{\pgfqpoint{1.838057in}{1.657467in}}%
\pgfpathlineto{\pgfqpoint{1.838441in}{1.702285in}}%
\pgfpathlineto{\pgfqpoint{1.839207in}{1.677838in}}%
\pgfpathlineto{\pgfqpoint{1.839593in}{1.665615in}}%
\pgfpathlineto{\pgfqpoint{1.839977in}{1.669690in}}%
\pgfpathlineto{\pgfqpoint{1.840360in}{1.681913in}}%
\pgfpathlineto{\pgfqpoint{1.840360in}{1.681913in}}%
\pgfpathlineto{\pgfqpoint{1.840360in}{1.681913in}}%
\pgfpathlineto{\pgfqpoint{1.841127in}{1.665615in}}%
\pgfpathlineto{\pgfqpoint{1.841510in}{1.677838in}}%
\pgfpathlineto{\pgfqpoint{1.841893in}{1.677838in}}%
\pgfpathlineto{\pgfqpoint{1.842277in}{1.690062in}}%
\pgfpathlineto{\pgfqpoint{1.842660in}{1.685987in}}%
\pgfpathlineto{\pgfqpoint{1.843816in}{1.669690in}}%
\pgfpathlineto{\pgfqpoint{1.844967in}{1.685987in}}%
\pgfpathlineto{\pgfqpoint{1.845351in}{1.681913in}}%
\pgfpathlineto{\pgfqpoint{1.846500in}{1.665615in}}%
\pgfpathlineto{\pgfqpoint{1.846883in}{1.669690in}}%
\pgfpathlineto{\pgfqpoint{1.847266in}{1.665615in}}%
\pgfpathlineto{\pgfqpoint{1.848032in}{1.677838in}}%
\pgfpathlineto{\pgfqpoint{1.848416in}{1.669690in}}%
\pgfpathlineto{\pgfqpoint{1.848799in}{1.673764in}}%
\pgfpathlineto{\pgfqpoint{1.849182in}{1.657467in}}%
\pgfpathlineto{\pgfqpoint{1.849182in}{1.657467in}}%
\pgfpathlineto{\pgfqpoint{1.849182in}{1.657467in}}%
\pgfpathlineto{\pgfqpoint{1.850715in}{1.681913in}}%
\pgfpathlineto{\pgfqpoint{1.851098in}{1.673764in}}%
\pgfpathlineto{\pgfqpoint{1.851482in}{1.681913in}}%
\pgfpathlineto{\pgfqpoint{1.851866in}{1.685987in}}%
\pgfpathlineto{\pgfqpoint{1.853398in}{1.661541in}}%
\pgfpathlineto{\pgfqpoint{1.854165in}{1.690062in}}%
\pgfpathlineto{\pgfqpoint{1.855312in}{1.681913in}}%
\pgfpathlineto{\pgfqpoint{1.856845in}{1.665615in}}%
\pgfpathlineto{\pgfqpoint{1.857993in}{1.685987in}}%
\pgfpathlineto{\pgfqpoint{1.858761in}{1.665615in}}%
\pgfpathlineto{\pgfqpoint{1.859145in}{1.669690in}}%
\pgfpathlineto{\pgfqpoint{1.859528in}{1.669690in}}%
\pgfpathlineto{\pgfqpoint{1.859910in}{1.649318in}}%
\pgfpathlineto{\pgfqpoint{1.860378in}{1.653392in}}%
\pgfpathlineto{\pgfqpoint{1.861145in}{1.673764in}}%
\pgfpathlineto{\pgfqpoint{1.861528in}{1.669690in}}%
\pgfpathlineto{\pgfqpoint{1.862296in}{1.673764in}}%
\pgfpathlineto{\pgfqpoint{1.862679in}{1.657467in}}%
\pgfpathlineto{\pgfqpoint{1.863061in}{1.673764in}}%
\pgfpathlineto{\pgfqpoint{1.863444in}{1.694136in}}%
\pgfpathlineto{\pgfqpoint{1.863827in}{1.673764in}}%
\pgfpathlineto{\pgfqpoint{1.864210in}{1.665615in}}%
\pgfpathlineto{\pgfqpoint{1.864592in}{1.702285in}}%
\pgfpathlineto{\pgfqpoint{1.865359in}{1.673764in}}%
\pgfpathlineto{\pgfqpoint{1.865743in}{1.673764in}}%
\pgfpathlineto{\pgfqpoint{1.866126in}{1.669690in}}%
\pgfpathlineto{\pgfqpoint{1.867275in}{1.690062in}}%
\pgfpathlineto{\pgfqpoint{1.868042in}{1.669690in}}%
\pgfpathlineto{\pgfqpoint{1.868807in}{1.673764in}}%
\pgfpathlineto{\pgfqpoint{1.869573in}{1.685987in}}%
\pgfpathlineto{\pgfqpoint{1.869956in}{1.677838in}}%
\pgfpathlineto{\pgfqpoint{1.870339in}{1.673764in}}%
\pgfpathlineto{\pgfqpoint{1.871106in}{1.698210in}}%
\pgfpathlineto{\pgfqpoint{1.871872in}{1.669690in}}%
\pgfpathlineto{\pgfqpoint{1.872255in}{1.677838in}}%
\pgfpathlineto{\pgfqpoint{1.872638in}{1.690062in}}%
\pgfpathlineto{\pgfqpoint{1.872638in}{1.690062in}}%
\pgfpathlineto{\pgfqpoint{1.872638in}{1.690062in}}%
\pgfpathlineto{\pgfqpoint{1.873405in}{1.665615in}}%
\pgfpathlineto{\pgfqpoint{1.873788in}{1.673764in}}%
\pgfpathlineto{\pgfqpoint{1.874939in}{1.681913in}}%
\pgfpathlineto{\pgfqpoint{1.875322in}{1.661541in}}%
\pgfpathlineto{\pgfqpoint{1.876088in}{1.673764in}}%
\pgfpathlineto{\pgfqpoint{1.876856in}{1.665615in}}%
\pgfpathlineto{\pgfqpoint{1.877239in}{1.681913in}}%
\pgfpathlineto{\pgfqpoint{1.878005in}{1.673764in}}%
\pgfpathlineto{\pgfqpoint{1.878387in}{1.669690in}}%
\pgfpathlineto{\pgfqpoint{1.879538in}{1.694136in}}%
\pgfpathlineto{\pgfqpoint{1.880303in}{1.673764in}}%
\pgfpathlineto{\pgfqpoint{1.880686in}{1.681913in}}%
\pgfpathlineto{\pgfqpoint{1.881453in}{1.669690in}}%
\pgfpathlineto{\pgfqpoint{1.881836in}{1.673764in}}%
\pgfpathlineto{\pgfqpoint{1.882219in}{1.681913in}}%
\pgfpathlineto{\pgfqpoint{1.882219in}{1.681913in}}%
\pgfpathlineto{\pgfqpoint{1.882219in}{1.681913in}}%
\pgfpathlineto{\pgfqpoint{1.882602in}{1.669690in}}%
\pgfpathlineto{\pgfqpoint{1.882602in}{1.669690in}}%
\pgfpathlineto{\pgfqpoint{1.882602in}{1.669690in}}%
\pgfpathlineto{\pgfqpoint{1.882992in}{1.690062in}}%
\pgfpathlineto{\pgfqpoint{1.883376in}{1.673764in}}%
\pgfpathlineto{\pgfqpoint{1.883760in}{1.669690in}}%
\pgfpathlineto{\pgfqpoint{1.884143in}{1.649318in}}%
\pgfpathlineto{\pgfqpoint{1.884143in}{1.649318in}}%
\pgfpathlineto{\pgfqpoint{1.884143in}{1.649318in}}%
\pgfpathlineto{\pgfqpoint{1.884909in}{1.681913in}}%
\pgfpathlineto{\pgfqpoint{1.885292in}{1.673764in}}%
\pgfpathlineto{\pgfqpoint{1.885676in}{1.673764in}}%
\pgfpathlineto{\pgfqpoint{1.886058in}{1.665615in}}%
\pgfpathlineto{\pgfqpoint{1.886058in}{1.665615in}}%
\pgfpathlineto{\pgfqpoint{1.886058in}{1.665615in}}%
\pgfpathlineto{\pgfqpoint{1.887208in}{1.685987in}}%
\pgfpathlineto{\pgfqpoint{1.887973in}{1.665615in}}%
\pgfpathlineto{\pgfqpoint{1.888356in}{1.690062in}}%
\pgfpathlineto{\pgfqpoint{1.889121in}{1.669690in}}%
\pgfpathlineto{\pgfqpoint{1.889504in}{1.685987in}}%
\pgfpathlineto{\pgfqpoint{1.889888in}{1.681913in}}%
\pgfpathlineto{\pgfqpoint{1.890273in}{1.657467in}}%
\pgfpathlineto{\pgfqpoint{1.890273in}{1.657467in}}%
\pgfpathlineto{\pgfqpoint{1.890273in}{1.657467in}}%
\pgfpathlineto{\pgfqpoint{1.891422in}{1.685987in}}%
\pgfpathlineto{\pgfqpoint{1.891804in}{1.673764in}}%
\pgfpathlineto{\pgfqpoint{1.892572in}{1.677838in}}%
\pgfpathlineto{\pgfqpoint{1.892956in}{1.685987in}}%
\pgfpathlineto{\pgfqpoint{1.893339in}{1.681913in}}%
\pgfpathlineto{\pgfqpoint{1.894488in}{1.669690in}}%
\pgfpathlineto{\pgfqpoint{1.894872in}{1.673764in}}%
\pgfpathlineto{\pgfqpoint{1.895255in}{1.673764in}}%
\pgfpathlineto{\pgfqpoint{1.895639in}{1.657467in}}%
\pgfpathlineto{\pgfqpoint{1.896405in}{1.669690in}}%
\pgfpathlineto{\pgfqpoint{1.896787in}{1.681913in}}%
\pgfpathlineto{\pgfqpoint{1.897638in}{1.673764in}}%
\pgfpathlineto{\pgfqpoint{1.898789in}{1.681913in}}%
\pgfpathlineto{\pgfqpoint{1.899173in}{1.677838in}}%
\pgfpathlineto{\pgfqpoint{1.899556in}{1.690062in}}%
\pgfpathlineto{\pgfqpoint{1.899938in}{1.661541in}}%
\pgfpathlineto{\pgfqpoint{1.900704in}{1.685987in}}%
\pgfpathlineto{\pgfqpoint{1.901087in}{1.673764in}}%
\pgfpathlineto{\pgfqpoint{1.901087in}{1.673764in}}%
\pgfpathlineto{\pgfqpoint{1.901087in}{1.673764in}}%
\pgfpathlineto{\pgfqpoint{1.901472in}{1.690062in}}%
\pgfpathlineto{\pgfqpoint{1.901855in}{1.681913in}}%
\pgfpathlineto{\pgfqpoint{1.903387in}{1.665615in}}%
\pgfpathlineto{\pgfqpoint{1.904154in}{1.665615in}}%
\pgfpathlineto{\pgfqpoint{1.904538in}{1.690062in}}%
\pgfpathlineto{\pgfqpoint{1.905304in}{1.685987in}}%
\pgfpathlineto{\pgfqpoint{1.906070in}{1.661541in}}%
\pgfpathlineto{\pgfqpoint{1.906454in}{1.669690in}}%
\pgfpathlineto{\pgfqpoint{1.907221in}{1.673764in}}%
\pgfpathlineto{\pgfqpoint{1.907987in}{1.649318in}}%
\pgfpathlineto{\pgfqpoint{1.908754in}{1.685987in}}%
\pgfpathlineto{\pgfqpoint{1.909137in}{1.673764in}}%
\pgfpathlineto{\pgfqpoint{1.909904in}{1.681913in}}%
\pgfpathlineto{\pgfqpoint{1.910670in}{1.649318in}}%
\pgfpathlineto{\pgfqpoint{1.911053in}{1.673764in}}%
\pgfpathlineto{\pgfqpoint{1.912589in}{1.690062in}}%
\pgfpathlineto{\pgfqpoint{1.913355in}{1.661541in}}%
\pgfpathlineto{\pgfqpoint{1.913738in}{1.669690in}}%
\pgfpathlineto{\pgfqpoint{1.914504in}{1.681913in}}%
\pgfpathlineto{\pgfqpoint{1.915655in}{1.661541in}}%
\pgfpathlineto{\pgfqpoint{1.916038in}{1.694136in}}%
\pgfpathlineto{\pgfqpoint{1.916803in}{1.665615in}}%
\pgfpathlineto{\pgfqpoint{1.917954in}{1.681913in}}%
\pgfpathlineto{\pgfqpoint{1.918337in}{1.661541in}}%
\pgfpathlineto{\pgfqpoint{1.919103in}{1.677838in}}%
\pgfpathlineto{\pgfqpoint{1.920254in}{1.690062in}}%
\pgfpathlineto{\pgfqpoint{1.920637in}{1.661541in}}%
\pgfpathlineto{\pgfqpoint{1.921403in}{1.681913in}}%
\pgfpathlineto{\pgfqpoint{1.921785in}{1.690062in}}%
\pgfpathlineto{\pgfqpoint{1.922168in}{1.681913in}}%
\pgfpathlineto{\pgfqpoint{1.922551in}{1.681913in}}%
\pgfpathlineto{\pgfqpoint{1.924091in}{1.657467in}}%
\pgfpathlineto{\pgfqpoint{1.925625in}{1.685987in}}%
\pgfpathlineto{\pgfqpoint{1.926008in}{1.685987in}}%
\pgfpathlineto{\pgfqpoint{1.926391in}{1.653392in}}%
\pgfpathlineto{\pgfqpoint{1.927157in}{1.665615in}}%
\pgfpathlineto{\pgfqpoint{1.927924in}{1.690062in}}%
\pgfpathlineto{\pgfqpoint{1.928307in}{1.673764in}}%
\pgfpathlineto{\pgfqpoint{1.929074in}{1.665615in}}%
\pgfpathlineto{\pgfqpoint{1.929840in}{1.690062in}}%
\pgfpathlineto{\pgfqpoint{1.930223in}{1.677838in}}%
\pgfpathlineto{\pgfqpoint{1.931757in}{1.669690in}}%
\pgfpathlineto{\pgfqpoint{1.932523in}{1.690062in}}%
\pgfpathlineto{\pgfqpoint{1.932906in}{1.661541in}}%
\pgfpathlineto{\pgfqpoint{1.933673in}{1.665615in}}%
\pgfpathlineto{\pgfqpoint{1.934057in}{1.661541in}}%
\pgfpathlineto{\pgfqpoint{1.934440in}{1.665615in}}%
\pgfpathlineto{\pgfqpoint{1.935206in}{1.685987in}}%
\pgfpathlineto{\pgfqpoint{1.935591in}{1.673764in}}%
\pgfpathlineto{\pgfqpoint{1.936358in}{1.681913in}}%
\pgfpathlineto{\pgfqpoint{1.936741in}{1.669690in}}%
\pgfpathlineto{\pgfqpoint{1.936741in}{1.669690in}}%
\pgfpathlineto{\pgfqpoint{1.936741in}{1.669690in}}%
\pgfpathlineto{\pgfqpoint{1.937507in}{1.694136in}}%
\pgfpathlineto{\pgfqpoint{1.937890in}{1.685987in}}%
\pgfpathlineto{\pgfqpoint{1.938358in}{1.685987in}}%
\pgfpathlineto{\pgfqpoint{1.939891in}{1.665615in}}%
\pgfpathlineto{\pgfqpoint{1.940657in}{1.685987in}}%
\pgfpathlineto{\pgfqpoint{1.941040in}{1.681913in}}%
\pgfpathlineto{\pgfqpoint{1.942190in}{1.657467in}}%
\pgfpathlineto{\pgfqpoint{1.943721in}{1.694136in}}%
\pgfpathlineto{\pgfqpoint{1.944869in}{1.657467in}}%
\pgfpathlineto{\pgfqpoint{1.945637in}{1.677838in}}%
\pgfpathlineto{\pgfqpoint{1.946020in}{1.673764in}}%
\pgfpathlineto{\pgfqpoint{1.946404in}{1.673764in}}%
\pgfpathlineto{\pgfqpoint{1.946787in}{1.661541in}}%
\pgfpathlineto{\pgfqpoint{1.946787in}{1.661541in}}%
\pgfpathlineto{\pgfqpoint{1.946787in}{1.661541in}}%
\pgfpathlineto{\pgfqpoint{1.947169in}{1.677838in}}%
\pgfpathlineto{\pgfqpoint{1.947937in}{1.669690in}}%
\pgfpathlineto{\pgfqpoint{1.948320in}{1.677838in}}%
\pgfpathlineto{\pgfqpoint{1.948320in}{1.677838in}}%
\pgfpathlineto{\pgfqpoint{1.948320in}{1.677838in}}%
\pgfpathlineto{\pgfqpoint{1.949087in}{1.665615in}}%
\pgfpathlineto{\pgfqpoint{1.950618in}{1.685987in}}%
\pgfpathlineto{\pgfqpoint{1.951768in}{1.665615in}}%
\pgfpathlineto{\pgfqpoint{1.952151in}{1.698210in}}%
\pgfpathlineto{\pgfqpoint{1.952916in}{1.669690in}}%
\pgfpathlineto{\pgfqpoint{1.953300in}{1.673764in}}%
\pgfpathlineto{\pgfqpoint{1.953683in}{1.665615in}}%
\pgfpathlineto{\pgfqpoint{1.954066in}{1.681913in}}%
\pgfpathlineto{\pgfqpoint{1.954066in}{1.681913in}}%
\pgfpathlineto{\pgfqpoint{1.954066in}{1.681913in}}%
\pgfpathlineto{\pgfqpoint{1.954449in}{1.653392in}}%
\pgfpathlineto{\pgfqpoint{1.955214in}{1.669690in}}%
\pgfpathlineto{\pgfqpoint{1.956365in}{1.681913in}}%
\pgfpathlineto{\pgfqpoint{1.956749in}{1.653392in}}%
\pgfpathlineto{\pgfqpoint{1.957514in}{1.669690in}}%
\pgfpathlineto{\pgfqpoint{1.957897in}{1.669690in}}%
\pgfpathlineto{\pgfqpoint{1.958280in}{1.685987in}}%
\pgfpathlineto{\pgfqpoint{1.958280in}{1.685987in}}%
\pgfpathlineto{\pgfqpoint{1.958280in}{1.685987in}}%
\pgfpathlineto{\pgfqpoint{1.959814in}{1.657467in}}%
\pgfpathlineto{\pgfqpoint{1.960197in}{1.653392in}}%
\pgfpathlineto{\pgfqpoint{1.960580in}{1.690062in}}%
\pgfpathlineto{\pgfqpoint{1.961346in}{1.669690in}}%
\pgfpathlineto{\pgfqpoint{1.961729in}{1.669690in}}%
\pgfpathlineto{\pgfqpoint{1.962113in}{1.698210in}}%
\pgfpathlineto{\pgfqpoint{1.962113in}{1.698210in}}%
\pgfpathlineto{\pgfqpoint{1.962113in}{1.698210in}}%
\pgfpathlineto{\pgfqpoint{1.962886in}{1.661541in}}%
\pgfpathlineto{\pgfqpoint{1.963269in}{1.665615in}}%
\pgfpathlineto{\pgfqpoint{1.963652in}{1.681913in}}%
\pgfpathlineto{\pgfqpoint{1.964036in}{1.665615in}}%
\pgfpathlineto{\pgfqpoint{1.964419in}{1.665615in}}%
\pgfpathlineto{\pgfqpoint{1.965186in}{1.685987in}}%
\pgfpathlineto{\pgfqpoint{1.965569in}{1.673764in}}%
\pgfpathlineto{\pgfqpoint{1.966335in}{1.661541in}}%
\pgfpathlineto{\pgfqpoint{1.966720in}{1.681913in}}%
\pgfpathlineto{\pgfqpoint{1.967487in}{1.673764in}}%
\pgfpathlineto{\pgfqpoint{1.967870in}{1.653392in}}%
\pgfpathlineto{\pgfqpoint{1.968252in}{1.665615in}}%
\pgfpathlineto{\pgfqpoint{1.968635in}{1.694136in}}%
\pgfpathlineto{\pgfqpoint{1.969404in}{1.677838in}}%
\pgfpathlineto{\pgfqpoint{1.969787in}{1.677838in}}%
\pgfpathlineto{\pgfqpoint{1.970170in}{1.685987in}}%
\pgfpathlineto{\pgfqpoint{1.970553in}{1.681913in}}%
\pgfpathlineto{\pgfqpoint{1.970936in}{1.677838in}}%
\pgfpathlineto{\pgfqpoint{1.971318in}{1.681913in}}%
\pgfpathlineto{\pgfqpoint{1.971703in}{1.681913in}}%
\pgfpathlineto{\pgfqpoint{1.972087in}{1.690062in}}%
\pgfpathlineto{\pgfqpoint{1.972470in}{1.669690in}}%
\pgfpathlineto{\pgfqpoint{1.973236in}{1.685987in}}%
\pgfpathlineto{\pgfqpoint{1.974386in}{1.665615in}}%
\pgfpathlineto{\pgfqpoint{1.974770in}{1.685987in}}%
\pgfpathlineto{\pgfqpoint{1.975536in}{1.673764in}}%
\pgfpathlineto{\pgfqpoint{1.976771in}{1.657467in}}%
\pgfpathlineto{\pgfqpoint{1.977920in}{1.685987in}}%
\pgfpathlineto{\pgfqpoint{1.979069in}{1.677838in}}%
\pgfpathlineto{\pgfqpoint{1.979453in}{1.677838in}}%
\pgfpathlineto{\pgfqpoint{1.979835in}{1.657467in}}%
\pgfpathlineto{\pgfqpoint{1.979835in}{1.657467in}}%
\pgfpathlineto{\pgfqpoint{1.979835in}{1.657467in}}%
\pgfpathlineto{\pgfqpoint{1.980218in}{1.685987in}}%
\pgfpathlineto{\pgfqpoint{1.980985in}{1.677838in}}%
\pgfpathlineto{\pgfqpoint{1.981368in}{1.665615in}}%
\pgfpathlineto{\pgfqpoint{1.981751in}{1.673764in}}%
\pgfpathlineto{\pgfqpoint{1.982134in}{1.690062in}}%
\pgfpathlineto{\pgfqpoint{1.982517in}{1.685987in}}%
\pgfpathlineto{\pgfqpoint{1.982900in}{1.673764in}}%
\pgfpathlineto{\pgfqpoint{1.983284in}{1.677838in}}%
\pgfpathlineto{\pgfqpoint{1.983668in}{1.690062in}}%
\pgfpathlineto{\pgfqpoint{1.983668in}{1.690062in}}%
\pgfpathlineto{\pgfqpoint{1.983668in}{1.690062in}}%
\pgfpathlineto{\pgfqpoint{1.985199in}{1.669690in}}%
\pgfpathlineto{\pgfqpoint{1.985583in}{1.681913in}}%
\pgfpathlineto{\pgfqpoint{1.985583in}{1.681913in}}%
\pgfpathlineto{\pgfqpoint{1.985583in}{1.681913in}}%
\pgfpathlineto{\pgfqpoint{1.985966in}{1.661541in}}%
\pgfpathlineto{\pgfqpoint{1.986734in}{1.665615in}}%
\pgfpathlineto{\pgfqpoint{1.987500in}{1.690062in}}%
\pgfpathlineto{\pgfqpoint{1.987882in}{1.677838in}}%
\pgfpathlineto{\pgfqpoint{1.988265in}{1.669690in}}%
\pgfpathlineto{\pgfqpoint{1.989031in}{1.685987in}}%
\pgfpathlineto{\pgfqpoint{1.989415in}{1.665615in}}%
\pgfpathlineto{\pgfqpoint{1.990182in}{1.677838in}}%
\pgfpathlineto{\pgfqpoint{1.990565in}{1.665615in}}%
\pgfpathlineto{\pgfqpoint{1.990565in}{1.665615in}}%
\pgfpathlineto{\pgfqpoint{1.990565in}{1.665615in}}%
\pgfpathlineto{\pgfqpoint{1.990948in}{1.681913in}}%
\pgfpathlineto{\pgfqpoint{1.991715in}{1.677838in}}%
\pgfpathlineto{\pgfqpoint{1.992483in}{1.677838in}}%
\pgfpathlineto{\pgfqpoint{1.992866in}{1.649318in}}%
\pgfpathlineto{\pgfqpoint{1.993249in}{1.677838in}}%
\pgfpathlineto{\pgfqpoint{1.993632in}{1.681913in}}%
\pgfpathlineto{\pgfqpoint{1.994783in}{1.661541in}}%
\pgfpathlineto{\pgfqpoint{1.995166in}{1.681913in}}%
\pgfpathlineto{\pgfqpoint{1.995932in}{1.669690in}}%
\pgfpathlineto{\pgfqpoint{1.996698in}{1.669690in}}%
\pgfpathlineto{\pgfqpoint{1.997465in}{1.677838in}}%
\pgfpathlineto{\pgfqpoint{1.997849in}{1.673764in}}%
\pgfpathlineto{\pgfqpoint{1.998999in}{1.673764in}}%
\pgfpathlineto{\pgfqpoint{1.999382in}{1.665615in}}%
\pgfpathlineto{\pgfqpoint{1.999766in}{1.681913in}}%
\pgfpathlineto{\pgfqpoint{1.999766in}{1.681913in}}%
\pgfpathlineto{\pgfqpoint{1.999766in}{1.681913in}}%
\pgfpathlineto{\pgfqpoint{2.000150in}{1.657467in}}%
\pgfpathlineto{\pgfqpoint{2.000916in}{1.665615in}}%
\pgfpathlineto{\pgfqpoint{2.001299in}{1.677838in}}%
\pgfpathlineto{\pgfqpoint{2.002065in}{1.669690in}}%
\pgfpathlineto{\pgfqpoint{2.002449in}{1.669690in}}%
\pgfpathlineto{\pgfqpoint{2.002839in}{1.698210in}}%
\pgfpathlineto{\pgfqpoint{2.003605in}{1.673764in}}%
\pgfpathlineto{\pgfqpoint{2.003990in}{1.685987in}}%
\pgfpathlineto{\pgfqpoint{2.003990in}{1.685987in}}%
\pgfpathlineto{\pgfqpoint{2.003990in}{1.685987in}}%
\pgfpathlineto{\pgfqpoint{2.004374in}{1.665615in}}%
\pgfpathlineto{\pgfqpoint{2.004374in}{1.665615in}}%
\pgfpathlineto{\pgfqpoint{2.004374in}{1.665615in}}%
\pgfpathlineto{\pgfqpoint{2.004757in}{1.690062in}}%
\pgfpathlineto{\pgfqpoint{2.005523in}{1.673764in}}%
\pgfpathlineto{\pgfqpoint{2.005906in}{1.653392in}}%
\pgfpathlineto{\pgfqpoint{2.005906in}{1.653392in}}%
\pgfpathlineto{\pgfqpoint{2.005906in}{1.653392in}}%
\pgfpathlineto{\pgfqpoint{2.006289in}{1.677838in}}%
\pgfpathlineto{\pgfqpoint{2.007056in}{1.665615in}}%
\pgfpathlineto{\pgfqpoint{2.007439in}{1.661541in}}%
\pgfpathlineto{\pgfqpoint{2.008971in}{1.690062in}}%
\pgfpathlineto{\pgfqpoint{2.010121in}{1.677838in}}%
\pgfpathlineto{\pgfqpoint{2.010887in}{1.685987in}}%
\pgfpathlineto{\pgfqpoint{2.012035in}{1.665615in}}%
\pgfpathlineto{\pgfqpoint{2.012418in}{1.690062in}}%
\pgfpathlineto{\pgfqpoint{2.013187in}{1.681913in}}%
\pgfpathlineto{\pgfqpoint{2.014335in}{1.661541in}}%
\pgfpathlineto{\pgfqpoint{2.015953in}{1.702285in}}%
\pgfpathlineto{\pgfqpoint{2.017104in}{1.645243in}}%
\pgfpathlineto{\pgfqpoint{2.018636in}{1.690062in}}%
\pgfpathlineto{\pgfqpoint{2.019020in}{1.685987in}}%
\pgfpathlineto{\pgfqpoint{2.019405in}{1.669690in}}%
\pgfpathlineto{\pgfqpoint{2.020171in}{1.681913in}}%
\pgfpathlineto{\pgfqpoint{2.020554in}{1.681913in}}%
\pgfpathlineto{\pgfqpoint{2.021704in}{1.665615in}}%
\pgfpathlineto{\pgfqpoint{2.022088in}{1.677838in}}%
\pgfpathlineto{\pgfqpoint{2.022471in}{1.669690in}}%
\pgfpathlineto{\pgfqpoint{2.022853in}{1.661541in}}%
\pgfpathlineto{\pgfqpoint{2.023236in}{1.685987in}}%
\pgfpathlineto{\pgfqpoint{2.024003in}{1.665615in}}%
\pgfpathlineto{\pgfqpoint{2.024387in}{1.657467in}}%
\pgfpathlineto{\pgfqpoint{2.025918in}{1.690062in}}%
\pgfpathlineto{\pgfqpoint{2.026685in}{1.677838in}}%
\pgfpathlineto{\pgfqpoint{2.027069in}{1.685987in}}%
\pgfpathlineto{\pgfqpoint{2.027453in}{1.681913in}}%
\pgfpathlineto{\pgfqpoint{2.027836in}{1.657467in}}%
\pgfpathlineto{\pgfqpoint{2.028602in}{1.677838in}}%
\pgfpathlineto{\pgfqpoint{2.029368in}{1.669690in}}%
\pgfpathlineto{\pgfqpoint{2.029753in}{1.677838in}}%
\pgfpathlineto{\pgfqpoint{2.030136in}{1.657467in}}%
\pgfpathlineto{\pgfqpoint{2.030519in}{1.673764in}}%
\pgfpathlineto{\pgfqpoint{2.030903in}{1.685987in}}%
\pgfpathlineto{\pgfqpoint{2.031669in}{1.677838in}}%
\pgfpathlineto{\pgfqpoint{2.032053in}{1.669690in}}%
\pgfpathlineto{\pgfqpoint{2.032821in}{1.673764in}}%
\pgfpathlineto{\pgfqpoint{2.033204in}{1.690062in}}%
\pgfpathlineto{\pgfqpoint{2.033204in}{1.690062in}}%
\pgfpathlineto{\pgfqpoint{2.033204in}{1.690062in}}%
\pgfpathlineto{\pgfqpoint{2.033587in}{1.669690in}}%
\pgfpathlineto{\pgfqpoint{2.034353in}{1.681913in}}%
\pgfpathlineto{\pgfqpoint{2.034736in}{1.681913in}}%
\pgfpathlineto{\pgfqpoint{2.035120in}{1.673764in}}%
\pgfpathlineto{\pgfqpoint{2.035505in}{1.681913in}}%
\pgfpathlineto{\pgfqpoint{2.035888in}{1.681913in}}%
\pgfpathlineto{\pgfqpoint{2.036653in}{1.665615in}}%
\pgfpathlineto{\pgfqpoint{2.037036in}{1.673764in}}%
\pgfpathlineto{\pgfqpoint{2.037804in}{1.690062in}}%
\pgfpathlineto{\pgfqpoint{2.038188in}{1.665615in}}%
\pgfpathlineto{\pgfqpoint{2.038954in}{1.677838in}}%
\pgfpathlineto{\pgfqpoint{2.039337in}{1.665615in}}%
\pgfpathlineto{\pgfqpoint{2.039337in}{1.665615in}}%
\pgfpathlineto{\pgfqpoint{2.039337in}{1.665615in}}%
\pgfpathlineto{\pgfqpoint{2.039719in}{1.685987in}}%
\pgfpathlineto{\pgfqpoint{2.040103in}{1.665615in}}%
\pgfpathlineto{\pgfqpoint{2.040486in}{1.665615in}}%
\pgfpathlineto{\pgfqpoint{2.041635in}{1.681913in}}%
\pgfpathlineto{\pgfqpoint{2.042017in}{1.677838in}}%
\pgfpathlineto{\pgfqpoint{2.043175in}{1.661541in}}%
\pgfpathlineto{\pgfqpoint{2.043558in}{1.669690in}}%
\pgfpathlineto{\pgfqpoint{2.043942in}{1.677838in}}%
\pgfpathlineto{\pgfqpoint{2.044325in}{1.673764in}}%
\pgfpathlineto{\pgfqpoint{2.044709in}{1.661541in}}%
\pgfpathlineto{\pgfqpoint{2.044709in}{1.661541in}}%
\pgfpathlineto{\pgfqpoint{2.044709in}{1.661541in}}%
\pgfpathlineto{\pgfqpoint{2.045859in}{1.677838in}}%
\pgfpathlineto{\pgfqpoint{2.046241in}{1.661541in}}%
\pgfpathlineto{\pgfqpoint{2.046241in}{1.661541in}}%
\pgfpathlineto{\pgfqpoint{2.046241in}{1.661541in}}%
\pgfpathlineto{\pgfqpoint{2.046624in}{1.681913in}}%
\pgfpathlineto{\pgfqpoint{2.047390in}{1.665615in}}%
\pgfpathlineto{\pgfqpoint{2.047774in}{1.681913in}}%
\pgfpathlineto{\pgfqpoint{2.047774in}{1.681913in}}%
\pgfpathlineto{\pgfqpoint{2.047774in}{1.681913in}}%
\pgfpathlineto{\pgfqpoint{2.049307in}{1.653392in}}%
\pgfpathlineto{\pgfqpoint{2.049690in}{1.677838in}}%
\pgfpathlineto{\pgfqpoint{2.050450in}{1.673764in}}%
\pgfpathlineto{\pgfqpoint{2.050832in}{1.669690in}}%
\pgfpathlineto{\pgfqpoint{2.051600in}{1.681913in}}%
\pgfpathlineto{\pgfqpoint{2.051984in}{1.649318in}}%
\pgfpathlineto{\pgfqpoint{2.052366in}{1.673764in}}%
\pgfpathlineto{\pgfqpoint{2.052749in}{1.698210in}}%
\pgfpathlineto{\pgfqpoint{2.053132in}{1.694136in}}%
\pgfpathlineto{\pgfqpoint{2.053600in}{1.657467in}}%
\pgfpathlineto{\pgfqpoint{2.054367in}{1.673764in}}%
\pgfpathlineto{\pgfqpoint{2.054751in}{1.661541in}}%
\pgfpathlineto{\pgfqpoint{2.055134in}{1.669690in}}%
\pgfpathlineto{\pgfqpoint{2.055517in}{1.673764in}}%
\pgfpathlineto{\pgfqpoint{2.055900in}{1.690062in}}%
\pgfpathlineto{\pgfqpoint{2.055900in}{1.690062in}}%
\pgfpathlineto{\pgfqpoint{2.055900in}{1.690062in}}%
\pgfpathlineto{\pgfqpoint{2.056283in}{1.665615in}}%
\pgfpathlineto{\pgfqpoint{2.057049in}{1.681913in}}%
\pgfpathlineto{\pgfqpoint{2.057432in}{1.677838in}}%
\pgfpathlineto{\pgfqpoint{2.057816in}{1.685987in}}%
\pgfpathlineto{\pgfqpoint{2.058582in}{1.665615in}}%
\pgfpathlineto{\pgfqpoint{2.058964in}{1.690062in}}%
\pgfpathlineto{\pgfqpoint{2.059347in}{1.677838in}}%
\pgfpathlineto{\pgfqpoint{2.059730in}{1.657467in}}%
\pgfpathlineto{\pgfqpoint{2.060113in}{1.669690in}}%
\pgfpathlineto{\pgfqpoint{2.060881in}{1.685987in}}%
\pgfpathlineto{\pgfqpoint{2.061264in}{1.665615in}}%
\pgfpathlineto{\pgfqpoint{2.061264in}{1.665615in}}%
\pgfpathlineto{\pgfqpoint{2.061264in}{1.665615in}}%
\pgfpathlineto{\pgfqpoint{2.061647in}{1.690062in}}%
\pgfpathlineto{\pgfqpoint{2.062413in}{1.669690in}}%
\pgfpathlineto{\pgfqpoint{2.062804in}{1.669690in}}%
\pgfpathlineto{\pgfqpoint{2.063574in}{1.681913in}}%
\pgfpathlineto{\pgfqpoint{2.063957in}{1.677838in}}%
\pgfpathlineto{\pgfqpoint{2.065491in}{1.661541in}}%
\pgfpathlineto{\pgfqpoint{2.066640in}{1.685987in}}%
\pgfpathlineto{\pgfqpoint{2.067023in}{1.665615in}}%
\pgfpathlineto{\pgfqpoint{2.067791in}{1.677838in}}%
\pgfpathlineto{\pgfqpoint{2.068174in}{1.669690in}}%
\pgfpathlineto{\pgfqpoint{2.068556in}{1.690062in}}%
\pgfpathlineto{\pgfqpoint{2.069322in}{1.681913in}}%
\pgfpathlineto{\pgfqpoint{2.069705in}{1.685987in}}%
\pgfpathlineto{\pgfqpoint{2.071240in}{1.661541in}}%
\pgfpathlineto{\pgfqpoint{2.071623in}{1.690062in}}%
\pgfpathlineto{\pgfqpoint{2.072389in}{1.669690in}}%
\pgfpathlineto{\pgfqpoint{2.072773in}{1.673764in}}%
\pgfpathlineto{\pgfqpoint{2.073157in}{1.694136in}}%
\pgfpathlineto{\pgfqpoint{2.073157in}{1.694136in}}%
\pgfpathlineto{\pgfqpoint{2.073157in}{1.694136in}}%
\pgfpathlineto{\pgfqpoint{2.074307in}{1.665615in}}%
\pgfpathlineto{\pgfqpoint{2.075456in}{1.694136in}}%
\pgfpathlineto{\pgfqpoint{2.076606in}{1.657467in}}%
\pgfpathlineto{\pgfqpoint{2.076989in}{1.698210in}}%
\pgfpathlineto{\pgfqpoint{2.077755in}{1.681913in}}%
\pgfpathlineto{\pgfqpoint{2.078138in}{1.669690in}}%
\pgfpathlineto{\pgfqpoint{2.078138in}{1.669690in}}%
\pgfpathlineto{\pgfqpoint{2.078138in}{1.669690in}}%
\pgfpathlineto{\pgfqpoint{2.078522in}{1.685987in}}%
\pgfpathlineto{\pgfqpoint{2.078905in}{1.677838in}}%
\pgfpathlineto{\pgfqpoint{2.079289in}{1.661541in}}%
\pgfpathlineto{\pgfqpoint{2.079671in}{1.673764in}}%
\pgfpathlineto{\pgfqpoint{2.080054in}{1.694136in}}%
\pgfpathlineto{\pgfqpoint{2.080054in}{1.694136in}}%
\pgfpathlineto{\pgfqpoint{2.080054in}{1.694136in}}%
\pgfpathlineto{\pgfqpoint{2.081589in}{1.649318in}}%
\pgfpathlineto{\pgfqpoint{2.082356in}{1.661541in}}%
\pgfpathlineto{\pgfqpoint{2.083515in}{1.681913in}}%
\pgfpathlineto{\pgfqpoint{2.084664in}{1.665615in}}%
\pgfpathlineto{\pgfqpoint{2.085047in}{1.702285in}}%
\pgfpathlineto{\pgfqpoint{2.085815in}{1.677838in}}%
\pgfpathlineto{\pgfqpoint{2.086580in}{1.681913in}}%
\pgfpathlineto{\pgfqpoint{2.088114in}{1.661541in}}%
\pgfpathlineto{\pgfqpoint{2.088880in}{1.694136in}}%
\pgfpathlineto{\pgfqpoint{2.089646in}{1.681913in}}%
\pgfpathlineto{\pgfqpoint{2.090029in}{1.681913in}}%
\pgfpathlineto{\pgfqpoint{2.090796in}{1.665615in}}%
\pgfpathlineto{\pgfqpoint{2.091562in}{1.669690in}}%
\pgfpathlineto{\pgfqpoint{2.092712in}{1.685987in}}%
\pgfpathlineto{\pgfqpoint{2.093096in}{1.681913in}}%
\pgfpathlineto{\pgfqpoint{2.093479in}{1.665615in}}%
\pgfpathlineto{\pgfqpoint{2.093479in}{1.665615in}}%
\pgfpathlineto{\pgfqpoint{2.093479in}{1.665615in}}%
\pgfpathlineto{\pgfqpoint{2.094245in}{1.685987in}}%
\pgfpathlineto{\pgfqpoint{2.094628in}{1.669690in}}%
\pgfpathlineto{\pgfqpoint{2.095778in}{1.669690in}}%
\pgfpathlineto{\pgfqpoint{2.096161in}{1.665615in}}%
\pgfpathlineto{\pgfqpoint{2.097311in}{1.677838in}}%
\pgfpathlineto{\pgfqpoint{2.097694in}{1.649318in}}%
\pgfpathlineto{\pgfqpoint{2.098077in}{1.665615in}}%
\pgfpathlineto{\pgfqpoint{2.099310in}{1.677838in}}%
\pgfpathlineto{\pgfqpoint{2.100460in}{1.669690in}}%
\pgfpathlineto{\pgfqpoint{2.101226in}{1.685987in}}%
\pgfpathlineto{\pgfqpoint{2.101609in}{1.669690in}}%
\pgfpathlineto{\pgfqpoint{2.101609in}{1.669690in}}%
\pgfpathlineto{\pgfqpoint{2.101609in}{1.669690in}}%
\pgfpathlineto{\pgfqpoint{2.101992in}{1.690062in}}%
\pgfpathlineto{\pgfqpoint{2.102375in}{1.673764in}}%
\pgfpathlineto{\pgfqpoint{2.102759in}{1.669690in}}%
\pgfpathlineto{\pgfqpoint{2.103142in}{1.673764in}}%
\pgfpathlineto{\pgfqpoint{2.103525in}{1.673764in}}%
\pgfpathlineto{\pgfqpoint{2.103908in}{1.677838in}}%
\pgfpathlineto{\pgfqpoint{2.104292in}{1.657467in}}%
\pgfpathlineto{\pgfqpoint{2.104675in}{1.673764in}}%
\pgfpathlineto{\pgfqpoint{2.105058in}{1.698210in}}%
\pgfpathlineto{\pgfqpoint{2.105442in}{1.685987in}}%
\pgfpathlineto{\pgfqpoint{2.106208in}{1.661541in}}%
\pgfpathlineto{\pgfqpoint{2.106592in}{1.673764in}}%
\pgfpathlineto{\pgfqpoint{2.106975in}{1.685987in}}%
\pgfpathlineto{\pgfqpoint{2.106975in}{1.685987in}}%
\pgfpathlineto{\pgfqpoint{2.106975in}{1.685987in}}%
\pgfpathlineto{\pgfqpoint{2.108123in}{1.661541in}}%
\pgfpathlineto{\pgfqpoint{2.108507in}{1.673764in}}%
\pgfpathlineto{\pgfqpoint{2.108507in}{1.673764in}}%
\pgfpathlineto{\pgfqpoint{2.108507in}{1.673764in}}%
\pgfpathlineto{\pgfqpoint{2.108890in}{1.657467in}}%
\pgfpathlineto{\pgfqpoint{2.108890in}{1.657467in}}%
\pgfpathlineto{\pgfqpoint{2.108890in}{1.657467in}}%
\pgfpathlineto{\pgfqpoint{2.110038in}{1.694136in}}%
\pgfpathlineto{\pgfqpoint{2.111187in}{1.657467in}}%
\pgfpathlineto{\pgfqpoint{2.112718in}{1.677838in}}%
\pgfpathlineto{\pgfqpoint{2.113102in}{1.681913in}}%
\pgfpathlineto{\pgfqpoint{2.113485in}{1.653392in}}%
\pgfpathlineto{\pgfqpoint{2.114251in}{1.673764in}}%
\pgfpathlineto{\pgfqpoint{2.115402in}{1.649318in}}%
\pgfpathlineto{\pgfqpoint{2.115784in}{1.690062in}}%
\pgfpathlineto{\pgfqpoint{2.116551in}{1.681913in}}%
\pgfpathlineto{\pgfqpoint{2.116935in}{1.673764in}}%
\pgfpathlineto{\pgfqpoint{2.116935in}{1.673764in}}%
\pgfpathlineto{\pgfqpoint{2.116935in}{1.673764in}}%
\pgfpathlineto{\pgfqpoint{2.117318in}{1.685987in}}%
\pgfpathlineto{\pgfqpoint{2.117702in}{1.681913in}}%
\pgfpathlineto{\pgfqpoint{2.118085in}{1.661541in}}%
\pgfpathlineto{\pgfqpoint{2.118468in}{1.669690in}}%
\pgfpathlineto{\pgfqpoint{2.119233in}{1.694136in}}%
\pgfpathlineto{\pgfqpoint{2.119617in}{1.653392in}}%
\pgfpathlineto{\pgfqpoint{2.120385in}{1.673764in}}%
\pgfpathlineto{\pgfqpoint{2.120768in}{1.681913in}}%
\pgfpathlineto{\pgfqpoint{2.121534in}{1.661541in}}%
\pgfpathlineto{\pgfqpoint{2.121917in}{1.694136in}}%
\pgfpathlineto{\pgfqpoint{2.122691in}{1.665615in}}%
\pgfpathlineto{\pgfqpoint{2.123074in}{1.681913in}}%
\pgfpathlineto{\pgfqpoint{2.123074in}{1.681913in}}%
\pgfpathlineto{\pgfqpoint{2.123074in}{1.681913in}}%
\pgfpathlineto{\pgfqpoint{2.123457in}{1.661541in}}%
\pgfpathlineto{\pgfqpoint{2.124223in}{1.677838in}}%
\pgfpathlineto{\pgfqpoint{2.124607in}{1.681913in}}%
\pgfpathlineto{\pgfqpoint{2.125756in}{1.657467in}}%
\pgfpathlineto{\pgfqpoint{2.126521in}{1.661541in}}%
\pgfpathlineto{\pgfqpoint{2.126905in}{1.665615in}}%
\pgfpathlineto{\pgfqpoint{2.128054in}{1.681913in}}%
\pgfpathlineto{\pgfqpoint{2.128821in}{1.677838in}}%
\pgfpathlineto{\pgfqpoint{2.129204in}{1.694136in}}%
\pgfpathlineto{\pgfqpoint{2.129587in}{1.677838in}}%
\pgfpathlineto{\pgfqpoint{2.129970in}{1.665615in}}%
\pgfpathlineto{\pgfqpoint{2.130353in}{1.690062in}}%
\pgfpathlineto{\pgfqpoint{2.131119in}{1.673764in}}%
\pgfpathlineto{\pgfqpoint{2.131503in}{1.685987in}}%
\pgfpathlineto{\pgfqpoint{2.131886in}{1.681913in}}%
\pgfpathlineto{\pgfqpoint{2.132269in}{1.673764in}}%
\pgfpathlineto{\pgfqpoint{2.133036in}{1.677838in}}%
\pgfpathlineto{\pgfqpoint{2.133419in}{1.677838in}}%
\pgfpathlineto{\pgfqpoint{2.134654in}{1.685987in}}%
\pgfpathlineto{\pgfqpoint{2.135037in}{1.673764in}}%
\pgfpathlineto{\pgfqpoint{2.135037in}{1.673764in}}%
\pgfpathlineto{\pgfqpoint{2.135037in}{1.673764in}}%
\pgfpathlineto{\pgfqpoint{2.135420in}{1.694136in}}%
\pgfpathlineto{\pgfqpoint{2.135420in}{1.694136in}}%
\pgfpathlineto{\pgfqpoint{2.135420in}{1.694136in}}%
\pgfpathlineto{\pgfqpoint{2.136187in}{1.661541in}}%
\pgfpathlineto{\pgfqpoint{2.136571in}{1.673764in}}%
\pgfpathlineto{\pgfqpoint{2.136954in}{1.681913in}}%
\pgfpathlineto{\pgfqpoint{2.137336in}{1.677838in}}%
\pgfpathlineto{\pgfqpoint{2.138869in}{1.653392in}}%
\pgfpathlineto{\pgfqpoint{2.139635in}{1.685987in}}%
\pgfpathlineto{\pgfqpoint{2.140018in}{1.681913in}}%
\pgfpathlineto{\pgfqpoint{2.140401in}{1.665615in}}%
\pgfpathlineto{\pgfqpoint{2.141167in}{1.669690in}}%
\pgfpathlineto{\pgfqpoint{2.141550in}{1.669690in}}%
\pgfpathlineto{\pgfqpoint{2.143083in}{1.685987in}}%
\pgfpathlineto{\pgfqpoint{2.143466in}{1.681913in}}%
\pgfpathlineto{\pgfqpoint{2.143849in}{1.706359in}}%
\pgfpathlineto{\pgfqpoint{2.144998in}{1.657467in}}%
\pgfpathlineto{\pgfqpoint{2.145765in}{1.685987in}}%
\pgfpathlineto{\pgfqpoint{2.146148in}{1.681913in}}%
\pgfpathlineto{\pgfqpoint{2.146530in}{1.673764in}}%
\pgfpathlineto{\pgfqpoint{2.147297in}{1.677838in}}%
\pgfpathlineto{\pgfqpoint{2.147681in}{1.677838in}}%
\pgfpathlineto{\pgfqpoint{2.148064in}{1.669690in}}%
\pgfpathlineto{\pgfqpoint{2.148447in}{1.677838in}}%
\pgfpathlineto{\pgfqpoint{2.148831in}{1.677838in}}%
\pgfpathlineto{\pgfqpoint{2.149597in}{1.673764in}}%
\pgfpathlineto{\pgfqpoint{2.149980in}{1.685987in}}%
\pgfpathlineto{\pgfqpoint{2.149980in}{1.685987in}}%
\pgfpathlineto{\pgfqpoint{2.149980in}{1.685987in}}%
\pgfpathlineto{\pgfqpoint{2.151131in}{1.665615in}}%
\pgfpathlineto{\pgfqpoint{2.152280in}{1.681913in}}%
\pgfpathlineto{\pgfqpoint{2.152663in}{1.665615in}}%
\pgfpathlineto{\pgfqpoint{2.153046in}{1.681913in}}%
\pgfpathlineto{\pgfqpoint{2.153429in}{1.685987in}}%
\pgfpathlineto{\pgfqpoint{2.153813in}{1.681913in}}%
\pgfpathlineto{\pgfqpoint{2.154579in}{1.681913in}}%
\pgfpathlineto{\pgfqpoint{2.155345in}{1.665615in}}%
\pgfpathlineto{\pgfqpoint{2.155726in}{1.677838in}}%
\pgfpathlineto{\pgfqpoint{2.156110in}{1.685987in}}%
\pgfpathlineto{\pgfqpoint{2.156493in}{1.665615in}}%
\pgfpathlineto{\pgfqpoint{2.156877in}{1.677838in}}%
\pgfpathlineto{\pgfqpoint{2.157643in}{1.694136in}}%
\pgfpathlineto{\pgfqpoint{2.158026in}{1.681913in}}%
\pgfpathlineto{\pgfqpoint{2.159559in}{1.669690in}}%
\pgfpathlineto{\pgfqpoint{2.159942in}{1.669690in}}%
\pgfpathlineto{\pgfqpoint{2.160707in}{1.665615in}}%
\pgfpathlineto{\pgfqpoint{2.161091in}{1.673764in}}%
\pgfpathlineto{\pgfqpoint{2.161474in}{1.653392in}}%
\pgfpathlineto{\pgfqpoint{2.161474in}{1.653392in}}%
\pgfpathlineto{\pgfqpoint{2.161474in}{1.653392in}}%
\pgfpathlineto{\pgfqpoint{2.163013in}{1.685987in}}%
\pgfpathlineto{\pgfqpoint{2.163397in}{1.661541in}}%
\pgfpathlineto{\pgfqpoint{2.164164in}{1.673764in}}%
\pgfpathlineto{\pgfqpoint{2.164547in}{1.673764in}}%
\pgfpathlineto{\pgfqpoint{2.165696in}{1.690062in}}%
\pgfpathlineto{\pgfqpoint{2.166080in}{1.665615in}}%
\pgfpathlineto{\pgfqpoint{2.166846in}{1.677838in}}%
\pgfpathlineto{\pgfqpoint{2.167229in}{1.681913in}}%
\pgfpathlineto{\pgfqpoint{2.167613in}{1.677838in}}%
\pgfpathlineto{\pgfqpoint{2.168762in}{1.657467in}}%
\pgfpathlineto{\pgfqpoint{2.169145in}{1.665615in}}%
\pgfpathlineto{\pgfqpoint{2.169529in}{1.681913in}}%
\pgfpathlineto{\pgfqpoint{2.169529in}{1.681913in}}%
\pgfpathlineto{\pgfqpoint{2.169529in}{1.681913in}}%
\pgfpathlineto{\pgfqpoint{2.169913in}{1.661541in}}%
\pgfpathlineto{\pgfqpoint{2.170296in}{1.677838in}}%
\pgfpathlineto{\pgfqpoint{2.170764in}{1.677838in}}%
\pgfpathlineto{\pgfqpoint{2.171146in}{1.681913in}}%
\pgfpathlineto{\pgfqpoint{2.171529in}{1.661541in}}%
\pgfpathlineto{\pgfqpoint{2.171912in}{1.673764in}}%
\pgfpathlineto{\pgfqpoint{2.172295in}{1.681913in}}%
\pgfpathlineto{\pgfqpoint{2.173062in}{1.677838in}}%
\pgfpathlineto{\pgfqpoint{2.173445in}{1.669690in}}%
\pgfpathlineto{\pgfqpoint{2.173827in}{1.673764in}}%
\pgfpathlineto{\pgfqpoint{2.174210in}{1.690062in}}%
\pgfpathlineto{\pgfqpoint{2.174210in}{1.690062in}}%
\pgfpathlineto{\pgfqpoint{2.174210in}{1.690062in}}%
\pgfpathlineto{\pgfqpoint{2.174593in}{1.661541in}}%
\pgfpathlineto{\pgfqpoint{2.175357in}{1.677838in}}%
\pgfpathlineto{\pgfqpoint{2.176507in}{1.681913in}}%
\pgfpathlineto{\pgfqpoint{2.176891in}{1.665615in}}%
\pgfpathlineto{\pgfqpoint{2.177275in}{1.681913in}}%
\pgfpathlineto{\pgfqpoint{2.178041in}{1.685987in}}%
\pgfpathlineto{\pgfqpoint{2.178808in}{1.653392in}}%
\pgfpathlineto{\pgfqpoint{2.179575in}{1.661541in}}%
\pgfpathlineto{\pgfqpoint{2.179958in}{1.661541in}}%
\pgfpathlineto{\pgfqpoint{2.180724in}{1.685987in}}%
\pgfpathlineto{\pgfqpoint{2.181108in}{1.673764in}}%
\pgfpathlineto{\pgfqpoint{2.181876in}{1.661541in}}%
\pgfpathlineto{\pgfqpoint{2.182261in}{1.665615in}}%
\pgfpathlineto{\pgfqpoint{2.183793in}{1.690062in}}%
\pgfpathlineto{\pgfqpoint{2.184558in}{1.681913in}}%
\pgfpathlineto{\pgfqpoint{2.185325in}{1.669690in}}%
\pgfpathlineto{\pgfqpoint{2.185708in}{1.677838in}}%
\pgfpathlineto{\pgfqpoint{2.186091in}{1.702285in}}%
\pgfpathlineto{\pgfqpoint{2.186475in}{1.677838in}}%
\pgfpathlineto{\pgfqpoint{2.186858in}{1.677838in}}%
\pgfpathlineto{\pgfqpoint{2.187241in}{1.685987in}}%
\pgfpathlineto{\pgfqpoint{2.187625in}{1.677838in}}%
\pgfpathlineto{\pgfqpoint{2.188008in}{1.677838in}}%
\pgfpathlineto{\pgfqpoint{2.188774in}{1.690062in}}%
\pgfpathlineto{\pgfqpoint{2.189541in}{1.665615in}}%
\pgfpathlineto{\pgfqpoint{2.189924in}{1.677838in}}%
\pgfpathlineto{\pgfqpoint{2.190692in}{1.661541in}}%
\pgfpathlineto{\pgfqpoint{2.191075in}{1.673764in}}%
\pgfpathlineto{\pgfqpoint{2.191841in}{1.677838in}}%
\pgfpathlineto{\pgfqpoint{2.192225in}{1.665615in}}%
\pgfpathlineto{\pgfqpoint{2.192991in}{1.673764in}}%
\pgfpathlineto{\pgfqpoint{2.193375in}{1.673764in}}%
\pgfpathlineto{\pgfqpoint{2.193757in}{1.665615in}}%
\pgfpathlineto{\pgfqpoint{2.194140in}{1.673764in}}%
\pgfpathlineto{\pgfqpoint{2.194523in}{1.673764in}}%
\pgfpathlineto{\pgfqpoint{2.194907in}{1.669690in}}%
\pgfpathlineto{\pgfqpoint{2.195674in}{1.694136in}}%
\pgfpathlineto{\pgfqpoint{2.197207in}{1.649318in}}%
\pgfpathlineto{\pgfqpoint{2.198358in}{1.690062in}}%
\pgfpathlineto{\pgfqpoint{2.198741in}{1.661541in}}%
\pgfpathlineto{\pgfqpoint{2.199507in}{1.665615in}}%
\pgfpathlineto{\pgfqpoint{2.200275in}{1.685987in}}%
\pgfpathlineto{\pgfqpoint{2.200657in}{1.673764in}}%
\pgfpathlineto{\pgfqpoint{2.202189in}{1.657467in}}%
\pgfpathlineto{\pgfqpoint{2.203729in}{1.677838in}}%
\pgfpathlineto{\pgfqpoint{2.204112in}{1.685987in}}%
\pgfpathlineto{\pgfqpoint{2.205261in}{1.669690in}}%
\pgfpathlineto{\pgfqpoint{2.205644in}{1.673764in}}%
\pgfpathlineto{\pgfqpoint{2.206026in}{1.657467in}}%
\pgfpathlineto{\pgfqpoint{2.206026in}{1.657467in}}%
\pgfpathlineto{\pgfqpoint{2.206026in}{1.657467in}}%
\pgfpathlineto{\pgfqpoint{2.207559in}{1.690062in}}%
\pgfpathlineto{\pgfqpoint{2.208795in}{1.665615in}}%
\pgfpathlineto{\pgfqpoint{2.209177in}{1.673764in}}%
\pgfpathlineto{\pgfqpoint{2.210326in}{1.685987in}}%
\pgfpathlineto{\pgfqpoint{2.211859in}{1.673764in}}%
\pgfpathlineto{\pgfqpoint{2.212242in}{1.685987in}}%
\pgfpathlineto{\pgfqpoint{2.212625in}{1.681913in}}%
\pgfpathlineto{\pgfqpoint{2.213775in}{1.661541in}}%
\pgfpathlineto{\pgfqpoint{2.214159in}{1.690062in}}%
\pgfpathlineto{\pgfqpoint{2.214925in}{1.685987in}}%
\pgfpathlineto{\pgfqpoint{2.215308in}{1.669690in}}%
\pgfpathlineto{\pgfqpoint{2.215308in}{1.669690in}}%
\pgfpathlineto{\pgfqpoint{2.215308in}{1.669690in}}%
\pgfpathlineto{\pgfqpoint{2.215692in}{1.690062in}}%
\pgfpathlineto{\pgfqpoint{2.216458in}{1.673764in}}%
\pgfpathlineto{\pgfqpoint{2.216842in}{1.681913in}}%
\pgfpathlineto{\pgfqpoint{2.217226in}{1.657467in}}%
\pgfpathlineto{\pgfqpoint{2.217609in}{1.673764in}}%
\pgfpathlineto{\pgfqpoint{2.217991in}{1.690062in}}%
\pgfpathlineto{\pgfqpoint{2.218375in}{1.673764in}}%
\pgfpathlineto{\pgfqpoint{2.218757in}{1.653392in}}%
\pgfpathlineto{\pgfqpoint{2.218757in}{1.653392in}}%
\pgfpathlineto{\pgfqpoint{2.218757in}{1.653392in}}%
\pgfpathlineto{\pgfqpoint{2.219140in}{1.681913in}}%
\pgfpathlineto{\pgfqpoint{2.219908in}{1.677838in}}%
\pgfpathlineto{\pgfqpoint{2.220290in}{1.681913in}}%
\pgfpathlineto{\pgfqpoint{2.220673in}{1.677838in}}%
\pgfpathlineto{\pgfqpoint{2.221056in}{1.665615in}}%
\pgfpathlineto{\pgfqpoint{2.221438in}{1.694136in}}%
\pgfpathlineto{\pgfqpoint{2.222204in}{1.669690in}}%
\pgfpathlineto{\pgfqpoint{2.222588in}{1.669690in}}%
\pgfpathlineto{\pgfqpoint{2.222971in}{1.665615in}}%
\pgfpathlineto{\pgfqpoint{2.224504in}{1.685987in}}%
\pgfpathlineto{\pgfqpoint{2.224887in}{1.653392in}}%
\pgfpathlineto{\pgfqpoint{2.225654in}{1.677838in}}%
\pgfpathlineto{\pgfqpoint{2.226037in}{1.673764in}}%
\pgfpathlineto{\pgfqpoint{2.226420in}{1.685987in}}%
\pgfpathlineto{\pgfqpoint{2.226803in}{1.645243in}}%
\pgfpathlineto{\pgfqpoint{2.227569in}{1.673764in}}%
\pgfpathlineto{\pgfqpoint{2.227952in}{1.669690in}}%
\pgfpathlineto{\pgfqpoint{2.228335in}{1.673764in}}%
\pgfpathlineto{\pgfqpoint{2.228718in}{1.677838in}}%
\pgfpathlineto{\pgfqpoint{2.229485in}{1.657467in}}%
\pgfpathlineto{\pgfqpoint{2.229868in}{1.665615in}}%
\pgfpathlineto{\pgfqpoint{2.230633in}{1.690062in}}%
\pgfpathlineto{\pgfqpoint{2.231017in}{1.681913in}}%
\pgfpathlineto{\pgfqpoint{2.231785in}{1.653392in}}%
\pgfpathlineto{\pgfqpoint{2.232168in}{1.665615in}}%
\pgfpathlineto{\pgfqpoint{2.232551in}{1.677838in}}%
\pgfpathlineto{\pgfqpoint{2.232934in}{1.669690in}}%
\pgfpathlineto{\pgfqpoint{2.233317in}{1.665615in}}%
\pgfpathlineto{\pgfqpoint{2.234850in}{1.685987in}}%
\pgfpathlineto{\pgfqpoint{2.235615in}{1.657467in}}%
\pgfpathlineto{\pgfqpoint{2.235999in}{1.673764in}}%
\pgfpathlineto{\pgfqpoint{2.236382in}{1.673764in}}%
\pgfpathlineto{\pgfqpoint{2.236765in}{1.685987in}}%
\pgfpathlineto{\pgfqpoint{2.236765in}{1.685987in}}%
\pgfpathlineto{\pgfqpoint{2.236765in}{1.685987in}}%
\pgfpathlineto{\pgfqpoint{2.237532in}{1.653392in}}%
\pgfpathlineto{\pgfqpoint{2.237915in}{1.665615in}}%
\pgfpathlineto{\pgfqpoint{2.238681in}{1.694136in}}%
\pgfpathlineto{\pgfqpoint{2.239449in}{1.685987in}}%
\pgfpathlineto{\pgfqpoint{2.239832in}{1.661541in}}%
\pgfpathlineto{\pgfqpoint{2.240599in}{1.673764in}}%
\pgfpathlineto{\pgfqpoint{2.240981in}{1.677838in}}%
\pgfpathlineto{\pgfqpoint{2.241364in}{1.661541in}}%
\pgfpathlineto{\pgfqpoint{2.241364in}{1.661541in}}%
\pgfpathlineto{\pgfqpoint{2.241364in}{1.661541in}}%
\pgfpathlineto{\pgfqpoint{2.242520in}{1.681913in}}%
\pgfpathlineto{\pgfqpoint{2.243287in}{1.669690in}}%
\pgfpathlineto{\pgfqpoint{2.243670in}{1.677838in}}%
\pgfpathlineto{\pgfqpoint{2.244053in}{1.677838in}}%
\pgfpathlineto{\pgfqpoint{2.244437in}{1.665615in}}%
\pgfpathlineto{\pgfqpoint{2.244819in}{1.669690in}}%
\pgfpathlineto{\pgfqpoint{2.245203in}{1.685987in}}%
\pgfpathlineto{\pgfqpoint{2.245586in}{1.673764in}}%
\pgfpathlineto{\pgfqpoint{2.246055in}{1.673764in}}%
\pgfpathlineto{\pgfqpoint{2.246438in}{1.681913in}}%
\pgfpathlineto{\pgfqpoint{2.247203in}{1.677838in}}%
\pgfpathlineto{\pgfqpoint{2.247588in}{1.677838in}}%
\pgfpathlineto{\pgfqpoint{2.247971in}{1.690062in}}%
\pgfpathlineto{\pgfqpoint{2.248354in}{1.685987in}}%
\pgfpathlineto{\pgfqpoint{2.249500in}{1.673764in}}%
\pgfpathlineto{\pgfqpoint{2.249885in}{1.681913in}}%
\pgfpathlineto{\pgfqpoint{2.249885in}{1.681913in}}%
\pgfpathlineto{\pgfqpoint{2.249885in}{1.681913in}}%
\pgfpathlineto{\pgfqpoint{2.250269in}{1.669690in}}%
\pgfpathlineto{\pgfqpoint{2.250653in}{1.673764in}}%
\pgfpathlineto{\pgfqpoint{2.251419in}{1.694136in}}%
\pgfpathlineto{\pgfqpoint{2.251802in}{1.681913in}}%
\pgfpathlineto{\pgfqpoint{2.252185in}{1.685987in}}%
\pgfpathlineto{\pgfqpoint{2.253719in}{1.657467in}}%
\pgfpathlineto{\pgfqpoint{2.255252in}{1.694136in}}%
\pgfpathlineto{\pgfqpoint{2.256403in}{1.657467in}}%
\pgfpathlineto{\pgfqpoint{2.256787in}{1.661541in}}%
\pgfpathlineto{\pgfqpoint{2.257936in}{1.677838in}}%
\pgfpathlineto{\pgfqpoint{2.259086in}{1.665615in}}%
\pgfpathlineto{\pgfqpoint{2.259469in}{1.677838in}}%
\pgfpathlineto{\pgfqpoint{2.260234in}{1.673764in}}%
\pgfpathlineto{\pgfqpoint{2.260617in}{1.661541in}}%
\pgfpathlineto{\pgfqpoint{2.260617in}{1.661541in}}%
\pgfpathlineto{\pgfqpoint{2.260617in}{1.661541in}}%
\pgfpathlineto{\pgfqpoint{2.261385in}{1.681913in}}%
\pgfpathlineto{\pgfqpoint{2.261769in}{1.673764in}}%
\pgfpathlineto{\pgfqpoint{2.262535in}{1.665615in}}%
\pgfpathlineto{\pgfqpoint{2.263301in}{1.690062in}}%
\pgfpathlineto{\pgfqpoint{2.263684in}{1.685987in}}%
\pgfpathlineto{\pgfqpoint{2.264067in}{1.653392in}}%
\pgfpathlineto{\pgfqpoint{2.264834in}{1.673764in}}%
\pgfpathlineto{\pgfqpoint{2.265218in}{1.685987in}}%
\pgfpathlineto{\pgfqpoint{2.265984in}{1.677838in}}%
\pgfpathlineto{\pgfqpoint{2.266367in}{1.681913in}}%
\pgfpathlineto{\pgfqpoint{2.266751in}{1.661541in}}%
\pgfpathlineto{\pgfqpoint{2.267518in}{1.677838in}}%
\pgfpathlineto{\pgfqpoint{2.267901in}{1.673764in}}%
\pgfpathlineto{\pgfqpoint{2.268284in}{1.685987in}}%
\pgfpathlineto{\pgfqpoint{2.269050in}{1.681913in}}%
\pgfpathlineto{\pgfqpoint{2.269434in}{1.669690in}}%
\pgfpathlineto{\pgfqpoint{2.269818in}{1.677838in}}%
\pgfpathlineto{\pgfqpoint{2.270201in}{1.681913in}}%
\pgfpathlineto{\pgfqpoint{2.271351in}{1.661541in}}%
\pgfpathlineto{\pgfqpoint{2.272119in}{1.685987in}}%
\pgfpathlineto{\pgfqpoint{2.272885in}{1.681913in}}%
\pgfpathlineto{\pgfqpoint{2.274417in}{1.665615in}}%
\pgfpathlineto{\pgfqpoint{2.274801in}{1.673764in}}%
\pgfpathlineto{\pgfqpoint{2.275185in}{1.669690in}}%
\pgfpathlineto{\pgfqpoint{2.276333in}{1.661541in}}%
\pgfpathlineto{\pgfqpoint{2.277099in}{1.681913in}}%
\pgfpathlineto{\pgfqpoint{2.277866in}{1.673764in}}%
\pgfpathlineto{\pgfqpoint{2.278631in}{1.685987in}}%
\pgfpathlineto{\pgfqpoint{2.279397in}{1.665615in}}%
\pgfpathlineto{\pgfqpoint{2.280546in}{1.690062in}}%
\pgfpathlineto{\pgfqpoint{2.281781in}{1.665615in}}%
\pgfpathlineto{\pgfqpoint{2.282164in}{1.669690in}}%
\pgfpathlineto{\pgfqpoint{2.282553in}{1.669690in}}%
\pgfpathlineto{\pgfqpoint{2.282937in}{1.694136in}}%
\pgfpathlineto{\pgfqpoint{2.283320in}{1.681913in}}%
\pgfpathlineto{\pgfqpoint{2.284853in}{1.661541in}}%
\pgfpathlineto{\pgfqpoint{2.286386in}{1.677838in}}%
\pgfpathlineto{\pgfqpoint{2.286769in}{1.677838in}}%
\pgfpathlineto{\pgfqpoint{2.288682in}{1.665615in}}%
\pgfpathlineto{\pgfqpoint{2.289066in}{1.677838in}}%
\pgfpathlineto{\pgfqpoint{2.289449in}{1.653392in}}%
\pgfpathlineto{\pgfqpoint{2.289832in}{1.669690in}}%
\pgfpathlineto{\pgfqpoint{2.290598in}{1.681913in}}%
\pgfpathlineto{\pgfqpoint{2.290981in}{1.661541in}}%
\pgfpathlineto{\pgfqpoint{2.291747in}{1.673764in}}%
\pgfpathlineto{\pgfqpoint{2.292514in}{1.657467in}}%
\pgfpathlineto{\pgfqpoint{2.293281in}{1.661541in}}%
\pgfpathlineto{\pgfqpoint{2.294812in}{1.681913in}}%
\pgfpathlineto{\pgfqpoint{2.295196in}{1.665615in}}%
\pgfpathlineto{\pgfqpoint{2.295579in}{1.669690in}}%
\pgfpathlineto{\pgfqpoint{2.295962in}{1.681913in}}%
\pgfpathlineto{\pgfqpoint{2.296729in}{1.673764in}}%
\pgfpathlineto{\pgfqpoint{2.297112in}{1.661541in}}%
\pgfpathlineto{\pgfqpoint{2.297879in}{1.669690in}}%
\pgfpathlineto{\pgfqpoint{2.298262in}{1.677838in}}%
\pgfpathlineto{\pgfqpoint{2.298645in}{1.673764in}}%
\pgfpathlineto{\pgfqpoint{2.299029in}{1.661541in}}%
\pgfpathlineto{\pgfqpoint{2.299795in}{1.685987in}}%
\pgfpathlineto{\pgfqpoint{2.300179in}{1.673764in}}%
\pgfpathlineto{\pgfqpoint{2.301328in}{1.677838in}}%
\pgfpathlineto{\pgfqpoint{2.301710in}{1.677838in}}%
\pgfpathlineto{\pgfqpoint{2.302094in}{1.694136in}}%
\pgfpathlineto{\pgfqpoint{2.302478in}{1.649318in}}%
\pgfpathlineto{\pgfqpoint{2.303245in}{1.669690in}}%
\pgfpathlineto{\pgfqpoint{2.304776in}{1.685987in}}%
\pgfpathlineto{\pgfqpoint{2.305159in}{1.685987in}}%
\pgfpathlineto{\pgfqpoint{2.305925in}{1.657467in}}%
\pgfpathlineto{\pgfqpoint{2.306308in}{1.685987in}}%
\pgfpathlineto{\pgfqpoint{2.307074in}{1.673764in}}%
\pgfpathlineto{\pgfqpoint{2.307457in}{1.669690in}}%
\pgfpathlineto{\pgfqpoint{2.307839in}{1.685987in}}%
\pgfpathlineto{\pgfqpoint{2.307839in}{1.685987in}}%
\pgfpathlineto{\pgfqpoint{2.307839in}{1.685987in}}%
\pgfpathlineto{\pgfqpoint{2.308989in}{1.661541in}}%
\pgfpathlineto{\pgfqpoint{2.309372in}{1.681913in}}%
\pgfpathlineto{\pgfqpoint{2.310138in}{1.673764in}}%
\pgfpathlineto{\pgfqpoint{2.311286in}{1.681913in}}%
\pgfpathlineto{\pgfqpoint{2.311669in}{1.661541in}}%
\pgfpathlineto{\pgfqpoint{2.312435in}{1.673764in}}%
\pgfpathlineto{\pgfqpoint{2.312818in}{1.669690in}}%
\pgfpathlineto{\pgfqpoint{2.313201in}{1.681913in}}%
\pgfpathlineto{\pgfqpoint{2.313585in}{1.673764in}}%
\pgfpathlineto{\pgfqpoint{2.313969in}{1.661541in}}%
\pgfpathlineto{\pgfqpoint{2.313969in}{1.661541in}}%
\pgfpathlineto{\pgfqpoint{2.313969in}{1.661541in}}%
\pgfpathlineto{\pgfqpoint{2.315502in}{1.694136in}}%
\pgfpathlineto{\pgfqpoint{2.316269in}{1.673764in}}%
\pgfpathlineto{\pgfqpoint{2.316653in}{1.681913in}}%
\pgfpathlineto{\pgfqpoint{2.317504in}{1.665615in}}%
\pgfpathlineto{\pgfqpoint{2.317888in}{1.673764in}}%
\pgfpathlineto{\pgfqpoint{2.318271in}{1.690062in}}%
\pgfpathlineto{\pgfqpoint{2.318271in}{1.690062in}}%
\pgfpathlineto{\pgfqpoint{2.318271in}{1.690062in}}%
\pgfpathlineto{\pgfqpoint{2.319802in}{1.665615in}}%
\pgfpathlineto{\pgfqpoint{2.320185in}{1.657467in}}%
\pgfpathlineto{\pgfqpoint{2.320569in}{1.677838in}}%
\pgfpathlineto{\pgfqpoint{2.320569in}{1.677838in}}%
\pgfpathlineto{\pgfqpoint{2.320569in}{1.677838in}}%
\pgfpathlineto{\pgfqpoint{2.320953in}{1.653392in}}%
\pgfpathlineto{\pgfqpoint{2.321336in}{1.657467in}}%
\pgfpathlineto{\pgfqpoint{2.322878in}{1.694136in}}%
\pgfpathlineto{\pgfqpoint{2.323262in}{1.665615in}}%
\pgfpathlineto{\pgfqpoint{2.324027in}{1.681913in}}%
\pgfpathlineto{\pgfqpoint{2.324794in}{1.669690in}}%
\pgfpathlineto{\pgfqpoint{2.325177in}{1.698210in}}%
\pgfpathlineto{\pgfqpoint{2.325177in}{1.698210in}}%
\pgfpathlineto{\pgfqpoint{2.325177in}{1.698210in}}%
\pgfpathlineto{\pgfqpoint{2.325561in}{1.661541in}}%
\pgfpathlineto{\pgfqpoint{2.326327in}{1.681913in}}%
\pgfpathlineto{\pgfqpoint{2.326709in}{1.694136in}}%
\pgfpathlineto{\pgfqpoint{2.327093in}{1.661541in}}%
\pgfpathlineto{\pgfqpoint{2.327859in}{1.677838in}}%
\pgfpathlineto{\pgfqpoint{2.328242in}{1.661541in}}%
\pgfpathlineto{\pgfqpoint{2.329009in}{1.673764in}}%
\pgfpathlineto{\pgfqpoint{2.329775in}{1.673764in}}%
\pgfpathlineto{\pgfqpoint{2.330540in}{1.649318in}}%
\pgfpathlineto{\pgfqpoint{2.330924in}{1.653392in}}%
\pgfpathlineto{\pgfqpoint{2.332075in}{1.685987in}}%
\pgfpathlineto{\pgfqpoint{2.332458in}{1.673764in}}%
\pgfpathlineto{\pgfqpoint{2.333224in}{1.681913in}}%
\pgfpathlineto{\pgfqpoint{2.334757in}{1.673764in}}%
\pgfpathlineto{\pgfqpoint{2.336290in}{1.690062in}}%
\pgfpathlineto{\pgfqpoint{2.337440in}{1.673764in}}%
\pgfpathlineto{\pgfqpoint{2.337822in}{1.653392in}}%
\pgfpathlineto{\pgfqpoint{2.338206in}{1.657467in}}%
\pgfpathlineto{\pgfqpoint{2.338588in}{1.685987in}}%
\pgfpathlineto{\pgfqpoint{2.339355in}{1.665615in}}%
\pgfpathlineto{\pgfqpoint{2.339738in}{1.661541in}}%
\pgfpathlineto{\pgfqpoint{2.340122in}{1.685987in}}%
\pgfpathlineto{\pgfqpoint{2.340122in}{1.685987in}}%
\pgfpathlineto{\pgfqpoint{2.340122in}{1.685987in}}%
\pgfpathlineto{\pgfqpoint{2.340506in}{1.657467in}}%
\pgfpathlineto{\pgfqpoint{2.340889in}{1.665615in}}%
\pgfpathlineto{\pgfqpoint{2.341272in}{1.685987in}}%
\pgfpathlineto{\pgfqpoint{2.341272in}{1.685987in}}%
\pgfpathlineto{\pgfqpoint{2.341272in}{1.685987in}}%
\pgfpathlineto{\pgfqpoint{2.341653in}{1.661541in}}%
\pgfpathlineto{\pgfqpoint{2.342421in}{1.681913in}}%
\pgfpathlineto{\pgfqpoint{2.343188in}{1.673764in}}%
\pgfpathlineto{\pgfqpoint{2.343571in}{1.677838in}}%
\pgfpathlineto{\pgfqpoint{2.343954in}{1.677838in}}%
\pgfpathlineto{\pgfqpoint{2.345105in}{1.690062in}}%
\pgfpathlineto{\pgfqpoint{2.345872in}{1.661541in}}%
\pgfpathlineto{\pgfqpoint{2.346255in}{1.673764in}}%
\pgfpathlineto{\pgfqpoint{2.347021in}{1.669690in}}%
\pgfpathlineto{\pgfqpoint{2.347406in}{1.677838in}}%
\pgfpathlineto{\pgfqpoint{2.347789in}{1.669690in}}%
\pgfpathlineto{\pgfqpoint{2.348173in}{1.657467in}}%
\pgfpathlineto{\pgfqpoint{2.348556in}{1.661541in}}%
\pgfpathlineto{\pgfqpoint{2.348939in}{1.690062in}}%
\pgfpathlineto{\pgfqpoint{2.349705in}{1.685987in}}%
\pgfpathlineto{\pgfqpoint{2.350089in}{1.669690in}}%
\pgfpathlineto{\pgfqpoint{2.350473in}{1.685987in}}%
\pgfpathlineto{\pgfqpoint{2.350856in}{1.685987in}}%
\pgfpathlineto{\pgfqpoint{2.351239in}{1.665615in}}%
\pgfpathlineto{\pgfqpoint{2.351623in}{1.685987in}}%
\pgfpathlineto{\pgfqpoint{2.352006in}{1.685987in}}%
\pgfpathlineto{\pgfqpoint{2.352390in}{1.669690in}}%
\pgfpathlineto{\pgfqpoint{2.353156in}{1.673764in}}%
\pgfpathlineto{\pgfqpoint{2.353540in}{1.681913in}}%
\pgfpathlineto{\pgfqpoint{2.353924in}{1.677838in}}%
\pgfpathlineto{\pgfqpoint{2.354307in}{1.665615in}}%
\pgfpathlineto{\pgfqpoint{2.354691in}{1.669690in}}%
\pgfpathlineto{\pgfqpoint{2.355075in}{1.685987in}}%
\pgfpathlineto{\pgfqpoint{2.355075in}{1.685987in}}%
\pgfpathlineto{\pgfqpoint{2.355075in}{1.685987in}}%
\pgfpathlineto{\pgfqpoint{2.355459in}{1.661541in}}%
\pgfpathlineto{\pgfqpoint{2.356225in}{1.669690in}}%
\pgfpathlineto{\pgfqpoint{2.356608in}{1.690062in}}%
\pgfpathlineto{\pgfqpoint{2.356608in}{1.690062in}}%
\pgfpathlineto{\pgfqpoint{2.356608in}{1.690062in}}%
\pgfpathlineto{\pgfqpoint{2.356991in}{1.657467in}}%
\pgfpathlineto{\pgfqpoint{2.357758in}{1.669690in}}%
\pgfpathlineto{\pgfqpoint{2.358142in}{1.665615in}}%
\pgfpathlineto{\pgfqpoint{2.358525in}{1.677838in}}%
\pgfpathlineto{\pgfqpoint{2.359377in}{1.673764in}}%
\pgfpathlineto{\pgfqpoint{2.360526in}{1.657467in}}%
\pgfpathlineto{\pgfqpoint{2.360909in}{1.690062in}}%
\pgfpathlineto{\pgfqpoint{2.361676in}{1.661541in}}%
\pgfpathlineto{\pgfqpoint{2.362059in}{1.685987in}}%
\pgfpathlineto{\pgfqpoint{2.362833in}{1.669690in}}%
\pgfpathlineto{\pgfqpoint{2.363984in}{1.661541in}}%
\pgfpathlineto{\pgfqpoint{2.364367in}{1.690062in}}%
\pgfpathlineto{\pgfqpoint{2.365132in}{1.677838in}}%
\pgfpathlineto{\pgfqpoint{2.366284in}{1.661541in}}%
\pgfpathlineto{\pgfqpoint{2.366667in}{1.677838in}}%
\pgfpathlineto{\pgfqpoint{2.367050in}{1.661541in}}%
\pgfpathlineto{\pgfqpoint{2.367433in}{1.645243in}}%
\pgfpathlineto{\pgfqpoint{2.367815in}{1.698210in}}%
\pgfpathlineto{\pgfqpoint{2.368582in}{1.673764in}}%
\pgfpathlineto{\pgfqpoint{2.368966in}{1.677838in}}%
\pgfpathlineto{\pgfqpoint{2.369348in}{1.673764in}}%
\pgfpathlineto{\pgfqpoint{2.369731in}{1.673764in}}%
\pgfpathlineto{\pgfqpoint{2.370114in}{1.661541in}}%
\pgfpathlineto{\pgfqpoint{2.371262in}{1.694136in}}%
\pgfpathlineto{\pgfqpoint{2.372412in}{1.673764in}}%
\pgfpathlineto{\pgfqpoint{2.373178in}{1.681913in}}%
\pgfpathlineto{\pgfqpoint{2.373561in}{1.677838in}}%
\pgfpathlineto{\pgfqpoint{2.373944in}{1.669690in}}%
\pgfpathlineto{\pgfqpoint{2.374711in}{1.673764in}}%
\pgfpathlineto{\pgfqpoint{2.375478in}{1.698210in}}%
\pgfpathlineto{\pgfqpoint{2.376244in}{1.661541in}}%
\pgfpathlineto{\pgfqpoint{2.376626in}{1.673764in}}%
\pgfpathlineto{\pgfqpoint{2.377775in}{1.677838in}}%
\pgfpathlineto{\pgfqpoint{2.378541in}{1.661541in}}%
\pgfpathlineto{\pgfqpoint{2.380072in}{1.694136in}}%
\pgfpathlineto{\pgfqpoint{2.380839in}{1.673764in}}%
\pgfpathlineto{\pgfqpoint{2.381605in}{1.677838in}}%
\pgfpathlineto{\pgfqpoint{2.382754in}{1.690062in}}%
\pgfpathlineto{\pgfqpoint{2.383138in}{1.685987in}}%
\pgfpathlineto{\pgfqpoint{2.384670in}{1.661541in}}%
\pgfpathlineto{\pgfqpoint{2.385053in}{1.685987in}}%
\pgfpathlineto{\pgfqpoint{2.385819in}{1.677838in}}%
\pgfpathlineto{\pgfqpoint{2.386203in}{1.669690in}}%
\pgfpathlineto{\pgfqpoint{2.386586in}{1.677838in}}%
\pgfpathlineto{\pgfqpoint{2.386969in}{1.690062in}}%
\pgfpathlineto{\pgfqpoint{2.386969in}{1.690062in}}%
\pgfpathlineto{\pgfqpoint{2.386969in}{1.690062in}}%
\pgfpathlineto{\pgfqpoint{2.388117in}{1.661541in}}%
\pgfpathlineto{\pgfqpoint{2.388501in}{1.665615in}}%
\pgfpathlineto{\pgfqpoint{2.389653in}{1.681913in}}%
\pgfpathlineto{\pgfqpoint{2.390035in}{1.669690in}}%
\pgfpathlineto{\pgfqpoint{2.390035in}{1.669690in}}%
\pgfpathlineto{\pgfqpoint{2.390035in}{1.669690in}}%
\pgfpathlineto{\pgfqpoint{2.390418in}{1.685987in}}%
\pgfpathlineto{\pgfqpoint{2.390802in}{1.673764in}}%
\pgfpathlineto{\pgfqpoint{2.391185in}{1.661541in}}%
\pgfpathlineto{\pgfqpoint{2.391952in}{1.669690in}}%
\pgfpathlineto{\pgfqpoint{2.393486in}{1.685987in}}%
\pgfpathlineto{\pgfqpoint{2.395021in}{1.657467in}}%
\pgfpathlineto{\pgfqpoint{2.395403in}{1.657467in}}%
\pgfpathlineto{\pgfqpoint{2.396170in}{1.694136in}}%
\pgfpathlineto{\pgfqpoint{2.396638in}{1.661541in}}%
\pgfpathlineto{\pgfqpoint{2.397406in}{1.661541in}}%
\pgfpathlineto{\pgfqpoint{2.397790in}{1.673764in}}%
\pgfpathlineto{\pgfqpoint{2.398173in}{1.665615in}}%
\pgfpathlineto{\pgfqpoint{2.398556in}{1.657467in}}%
\pgfpathlineto{\pgfqpoint{2.398939in}{1.677838in}}%
\pgfpathlineto{\pgfqpoint{2.399705in}{1.673764in}}%
\pgfpathlineto{\pgfqpoint{2.400088in}{1.645243in}}%
\pgfpathlineto{\pgfqpoint{2.400088in}{1.645243in}}%
\pgfpathlineto{\pgfqpoint{2.400088in}{1.645243in}}%
\pgfpathlineto{\pgfqpoint{2.401236in}{1.681913in}}%
\pgfpathlineto{\pgfqpoint{2.401619in}{1.677838in}}%
\pgfpathlineto{\pgfqpoint{2.402777in}{1.661541in}}%
\pgfpathlineto{\pgfqpoint{2.404308in}{1.694136in}}%
\pgfpathlineto{\pgfqpoint{2.405075in}{1.665615in}}%
\pgfpathlineto{\pgfqpoint{2.405458in}{1.677838in}}%
\pgfpathlineto{\pgfqpoint{2.406991in}{1.653392in}}%
\pgfpathlineto{\pgfqpoint{2.407374in}{1.685987in}}%
\pgfpathlineto{\pgfqpoint{2.408142in}{1.673764in}}%
\pgfpathlineto{\pgfqpoint{2.408525in}{1.669690in}}%
\pgfpathlineto{\pgfqpoint{2.409291in}{1.685987in}}%
\pgfpathlineto{\pgfqpoint{2.409674in}{1.669690in}}%
\pgfpathlineto{\pgfqpoint{2.409674in}{1.669690in}}%
\pgfpathlineto{\pgfqpoint{2.409674in}{1.669690in}}%
\pgfpathlineto{\pgfqpoint{2.410058in}{1.690062in}}%
\pgfpathlineto{\pgfqpoint{2.410825in}{1.673764in}}%
\pgfpathlineto{\pgfqpoint{2.411207in}{1.673764in}}%
\pgfpathlineto{\pgfqpoint{2.411591in}{1.681913in}}%
\pgfpathlineto{\pgfqpoint{2.411974in}{1.673764in}}%
\pgfpathlineto{\pgfqpoint{2.412741in}{1.665615in}}%
\pgfpathlineto{\pgfqpoint{2.413509in}{1.685987in}}%
\pgfpathlineto{\pgfqpoint{2.414658in}{1.661541in}}%
\pgfpathlineto{\pgfqpoint{2.415808in}{1.685987in}}%
\pgfpathlineto{\pgfqpoint{2.416191in}{1.661541in}}%
\pgfpathlineto{\pgfqpoint{2.416958in}{1.677838in}}%
\pgfpathlineto{\pgfqpoint{2.417340in}{1.685987in}}%
\pgfpathlineto{\pgfqpoint{2.417724in}{1.665615in}}%
\pgfpathlineto{\pgfqpoint{2.418492in}{1.669690in}}%
\pgfpathlineto{\pgfqpoint{2.418875in}{1.673764in}}%
\pgfpathlineto{\pgfqpoint{2.419258in}{1.661541in}}%
\pgfpathlineto{\pgfqpoint{2.419258in}{1.661541in}}%
\pgfpathlineto{\pgfqpoint{2.419258in}{1.661541in}}%
\pgfpathlineto{\pgfqpoint{2.420407in}{1.690062in}}%
\pgfpathlineto{\pgfqpoint{2.421558in}{1.665615in}}%
\pgfpathlineto{\pgfqpoint{2.421940in}{1.669690in}}%
\pgfpathlineto{\pgfqpoint{2.422324in}{1.685987in}}%
\pgfpathlineto{\pgfqpoint{2.422708in}{1.669690in}}%
\pgfpathlineto{\pgfqpoint{2.423091in}{1.661541in}}%
\pgfpathlineto{\pgfqpoint{2.423474in}{1.690062in}}%
\pgfpathlineto{\pgfqpoint{2.424241in}{1.685987in}}%
\pgfpathlineto{\pgfqpoint{2.425007in}{1.677838in}}%
\pgfpathlineto{\pgfqpoint{2.425390in}{1.690062in}}%
\pgfpathlineto{\pgfqpoint{2.425773in}{1.681913in}}%
\pgfpathlineto{\pgfqpoint{2.426157in}{1.645243in}}%
\pgfpathlineto{\pgfqpoint{2.426925in}{1.677838in}}%
\pgfpathlineto{\pgfqpoint{2.427308in}{1.673764in}}%
\pgfpathlineto{\pgfqpoint{2.427691in}{1.649318in}}%
\pgfpathlineto{\pgfqpoint{2.427691in}{1.649318in}}%
\pgfpathlineto{\pgfqpoint{2.427691in}{1.649318in}}%
\pgfpathlineto{\pgfqpoint{2.429225in}{1.702285in}}%
\pgfpathlineto{\pgfqpoint{2.430374in}{1.677838in}}%
\pgfpathlineto{\pgfqpoint{2.430757in}{1.694136in}}%
\pgfpathlineto{\pgfqpoint{2.431140in}{1.690062in}}%
\pgfpathlineto{\pgfqpoint{2.432673in}{1.661541in}}%
\pgfpathlineto{\pgfqpoint{2.433822in}{1.677838in}}%
\pgfpathlineto{\pgfqpoint{2.434204in}{1.673764in}}%
\pgfpathlineto{\pgfqpoint{2.434587in}{1.690062in}}%
\pgfpathlineto{\pgfqpoint{2.434970in}{1.677838in}}%
\pgfpathlineto{\pgfqpoint{2.435737in}{1.665615in}}%
\pgfpathlineto{\pgfqpoint{2.436205in}{1.677838in}}%
\pgfpathlineto{\pgfqpoint{2.436588in}{1.665615in}}%
\pgfpathlineto{\pgfqpoint{2.436971in}{1.665615in}}%
\pgfpathlineto{\pgfqpoint{2.437354in}{1.677838in}}%
\pgfpathlineto{\pgfqpoint{2.437737in}{1.653392in}}%
\pgfpathlineto{\pgfqpoint{2.438504in}{1.673764in}}%
\pgfpathlineto{\pgfqpoint{2.439271in}{1.690062in}}%
\pgfpathlineto{\pgfqpoint{2.439653in}{1.661541in}}%
\pgfpathlineto{\pgfqpoint{2.440036in}{1.673764in}}%
\pgfpathlineto{\pgfqpoint{2.440419in}{1.690062in}}%
\pgfpathlineto{\pgfqpoint{2.440419in}{1.690062in}}%
\pgfpathlineto{\pgfqpoint{2.440419in}{1.690062in}}%
\pgfpathlineto{\pgfqpoint{2.441570in}{1.669690in}}%
\pgfpathlineto{\pgfqpoint{2.442343in}{1.685987in}}%
\pgfpathlineto{\pgfqpoint{2.442728in}{1.657467in}}%
\pgfpathlineto{\pgfqpoint{2.443495in}{1.673764in}}%
\pgfpathlineto{\pgfqpoint{2.444644in}{1.669690in}}%
\pgfpathlineto{\pgfqpoint{2.445795in}{1.685987in}}%
\pgfpathlineto{\pgfqpoint{2.446178in}{1.677838in}}%
\pgfpathlineto{\pgfqpoint{2.446561in}{1.677838in}}%
\pgfpathlineto{\pgfqpoint{2.447711in}{1.694136in}}%
\pgfpathlineto{\pgfqpoint{2.448094in}{1.685987in}}%
\pgfpathlineto{\pgfqpoint{2.448477in}{1.653392in}}%
\pgfpathlineto{\pgfqpoint{2.449246in}{1.681913in}}%
\pgfpathlineto{\pgfqpoint{2.450396in}{1.669690in}}%
\pgfpathlineto{\pgfqpoint{2.450780in}{1.673764in}}%
\pgfpathlineto{\pgfqpoint{2.451163in}{1.685987in}}%
\pgfpathlineto{\pgfqpoint{2.451545in}{1.677838in}}%
\pgfpathlineto{\pgfqpoint{2.451929in}{1.657467in}}%
\pgfpathlineto{\pgfqpoint{2.452312in}{1.677838in}}%
\pgfpathlineto{\pgfqpoint{2.452695in}{1.677838in}}%
\pgfpathlineto{\pgfqpoint{2.453461in}{1.657467in}}%
\pgfpathlineto{\pgfqpoint{2.454228in}{1.677838in}}%
\pgfpathlineto{\pgfqpoint{2.454612in}{1.669690in}}%
\pgfpathlineto{\pgfqpoint{2.454994in}{1.661541in}}%
\pgfpathlineto{\pgfqpoint{2.455761in}{1.694136in}}%
\pgfpathlineto{\pgfqpoint{2.456145in}{1.673764in}}%
\pgfpathlineto{\pgfqpoint{2.456529in}{1.673764in}}%
\pgfpathlineto{\pgfqpoint{2.456911in}{1.669690in}}%
\pgfpathlineto{\pgfqpoint{2.457677in}{1.677838in}}%
\pgfpathlineto{\pgfqpoint{2.458060in}{1.661541in}}%
\pgfpathlineto{\pgfqpoint{2.458060in}{1.661541in}}%
\pgfpathlineto{\pgfqpoint{2.458060in}{1.661541in}}%
\pgfpathlineto{\pgfqpoint{2.458443in}{1.690062in}}%
\pgfpathlineto{\pgfqpoint{2.459210in}{1.673764in}}%
\pgfpathlineto{\pgfqpoint{2.459594in}{1.673764in}}%
\pgfpathlineto{\pgfqpoint{2.459978in}{1.685987in}}%
\pgfpathlineto{\pgfqpoint{2.460361in}{1.677838in}}%
\pgfpathlineto{\pgfqpoint{2.460745in}{1.673764in}}%
\pgfpathlineto{\pgfqpoint{2.461128in}{1.677838in}}%
\pgfpathlineto{\pgfqpoint{2.461513in}{1.690062in}}%
\pgfpathlineto{\pgfqpoint{2.461896in}{1.681913in}}%
\pgfpathlineto{\pgfqpoint{2.462279in}{1.665615in}}%
\pgfpathlineto{\pgfqpoint{2.462662in}{1.673764in}}%
\pgfpathlineto{\pgfqpoint{2.463045in}{1.681913in}}%
\pgfpathlineto{\pgfqpoint{2.463045in}{1.681913in}}%
\pgfpathlineto{\pgfqpoint{2.463045in}{1.681913in}}%
\pgfpathlineto{\pgfqpoint{2.463428in}{1.669690in}}%
\pgfpathlineto{\pgfqpoint{2.463812in}{1.681913in}}%
\pgfpathlineto{\pgfqpoint{2.464196in}{1.681913in}}%
\pgfpathlineto{\pgfqpoint{2.464962in}{1.665615in}}%
\pgfpathlineto{\pgfqpoint{2.465345in}{1.681913in}}%
\pgfpathlineto{\pgfqpoint{2.466111in}{1.677838in}}%
\pgfpathlineto{\pgfqpoint{2.466493in}{1.677838in}}%
\pgfpathlineto{\pgfqpoint{2.466878in}{1.665615in}}%
\pgfpathlineto{\pgfqpoint{2.467644in}{1.673764in}}%
\pgfpathlineto{\pgfqpoint{2.468028in}{1.673764in}}%
\pgfpathlineto{\pgfqpoint{2.468794in}{1.694136in}}%
\pgfpathlineto{\pgfqpoint{2.469177in}{1.690062in}}%
\pgfpathlineto{\pgfqpoint{2.470711in}{1.661541in}}%
\pgfpathlineto{\pgfqpoint{2.471859in}{1.677838in}}%
\pgfpathlineto{\pgfqpoint{2.472242in}{1.669690in}}%
\pgfpathlineto{\pgfqpoint{2.472242in}{1.669690in}}%
\pgfpathlineto{\pgfqpoint{2.472242in}{1.669690in}}%
\pgfpathlineto{\pgfqpoint{2.473394in}{1.694136in}}%
\pgfpathlineto{\pgfqpoint{2.475310in}{1.657467in}}%
\pgfpathlineto{\pgfqpoint{2.476078in}{1.685987in}}%
\pgfpathlineto{\pgfqpoint{2.476843in}{1.681913in}}%
\pgfpathlineto{\pgfqpoint{2.477226in}{1.673764in}}%
\pgfpathlineto{\pgfqpoint{2.477610in}{1.681913in}}%
\pgfpathlineto{\pgfqpoint{2.477993in}{1.694136in}}%
\pgfpathlineto{\pgfqpoint{2.478375in}{1.661541in}}%
\pgfpathlineto{\pgfqpoint{2.478375in}{1.661541in}}%
\pgfpathlineto{\pgfqpoint{2.478375in}{1.661541in}}%
\pgfpathlineto{\pgfqpoint{2.478844in}{1.698210in}}%
\pgfpathlineto{\pgfqpoint{2.479610in}{1.673764in}}%
\pgfpathlineto{\pgfqpoint{2.480376in}{1.661541in}}%
\pgfpathlineto{\pgfqpoint{2.481141in}{1.706359in}}%
\pgfpathlineto{\pgfqpoint{2.481526in}{1.690062in}}%
\pgfpathlineto{\pgfqpoint{2.482684in}{1.669690in}}%
\pgfpathlineto{\pgfqpoint{2.483068in}{1.681913in}}%
\pgfpathlineto{\pgfqpoint{2.483452in}{1.690062in}}%
\pgfpathlineto{\pgfqpoint{2.483834in}{1.665615in}}%
\pgfpathlineto{\pgfqpoint{2.483834in}{1.665615in}}%
\pgfpathlineto{\pgfqpoint{2.483834in}{1.665615in}}%
\pgfpathlineto{\pgfqpoint{2.484218in}{1.698210in}}%
\pgfpathlineto{\pgfqpoint{2.484984in}{1.673764in}}%
\pgfpathlineto{\pgfqpoint{2.485367in}{1.690062in}}%
\pgfpathlineto{\pgfqpoint{2.486134in}{1.685987in}}%
\pgfpathlineto{\pgfqpoint{2.486900in}{1.673764in}}%
\pgfpathlineto{\pgfqpoint{2.487283in}{1.681913in}}%
\pgfpathlineto{\pgfqpoint{2.487666in}{1.690062in}}%
\pgfpathlineto{\pgfqpoint{2.487666in}{1.690062in}}%
\pgfpathlineto{\pgfqpoint{2.487666in}{1.690062in}}%
\pgfpathlineto{\pgfqpoint{2.488432in}{1.661541in}}%
\pgfpathlineto{\pgfqpoint{2.488815in}{1.681913in}}%
\pgfpathlineto{\pgfqpoint{2.489964in}{1.673764in}}%
\pgfpathlineto{\pgfqpoint{2.490347in}{1.677838in}}%
\pgfpathlineto{\pgfqpoint{2.491114in}{1.669690in}}%
\pgfpathlineto{\pgfqpoint{2.491881in}{1.690062in}}%
\pgfpathlineto{\pgfqpoint{2.492648in}{1.685987in}}%
\pgfpathlineto{\pgfqpoint{2.493415in}{1.685987in}}%
\pgfpathlineto{\pgfqpoint{2.494181in}{1.665615in}}%
\pgfpathlineto{\pgfqpoint{2.494564in}{1.673764in}}%
\pgfpathlineto{\pgfqpoint{2.495331in}{1.694136in}}%
\pgfpathlineto{\pgfqpoint{2.495714in}{1.677838in}}%
\pgfpathlineto{\pgfqpoint{2.496097in}{1.665615in}}%
\pgfpathlineto{\pgfqpoint{2.496481in}{1.677838in}}%
\pgfpathlineto{\pgfqpoint{2.496865in}{1.677838in}}%
\pgfpathlineto{\pgfqpoint{2.498015in}{1.661541in}}%
\pgfpathlineto{\pgfqpoint{2.498781in}{1.665615in}}%
\pgfpathlineto{\pgfqpoint{2.499165in}{1.661541in}}%
\pgfpathlineto{\pgfqpoint{2.500698in}{1.681913in}}%
\pgfpathlineto{\pgfqpoint{2.501464in}{1.673764in}}%
\pgfpathlineto{\pgfqpoint{2.501848in}{1.677838in}}%
\pgfpathlineto{\pgfqpoint{2.502232in}{1.681913in}}%
\pgfpathlineto{\pgfqpoint{2.502617in}{1.677838in}}%
\pgfpathlineto{\pgfqpoint{2.503000in}{1.677838in}}%
\pgfpathlineto{\pgfqpoint{2.503383in}{1.702285in}}%
\pgfpathlineto{\pgfqpoint{2.503765in}{1.685987in}}%
\pgfpathlineto{\pgfqpoint{2.504915in}{1.665615in}}%
\pgfpathlineto{\pgfqpoint{2.505299in}{1.677838in}}%
\pgfpathlineto{\pgfqpoint{2.506065in}{1.669690in}}%
\pgfpathlineto{\pgfqpoint{2.506448in}{1.673764in}}%
\pgfpathlineto{\pgfqpoint{2.506832in}{1.657467in}}%
\pgfpathlineto{\pgfqpoint{2.507216in}{1.690062in}}%
\pgfpathlineto{\pgfqpoint{2.507983in}{1.661541in}}%
\pgfpathlineto{\pgfqpoint{2.508365in}{1.690062in}}%
\pgfpathlineto{\pgfqpoint{2.509132in}{1.685987in}}%
\pgfpathlineto{\pgfqpoint{2.509515in}{1.685987in}}%
\pgfpathlineto{\pgfqpoint{2.510281in}{1.681913in}}%
\pgfpathlineto{\pgfqpoint{2.511815in}{1.653392in}}%
\pgfpathlineto{\pgfqpoint{2.512198in}{1.669690in}}%
\pgfpathlineto{\pgfqpoint{2.512964in}{1.657467in}}%
\pgfpathlineto{\pgfqpoint{2.514114in}{1.681913in}}%
\pgfpathlineto{\pgfqpoint{2.514497in}{1.653392in}}%
\pgfpathlineto{\pgfqpoint{2.515264in}{1.665615in}}%
\pgfpathlineto{\pgfqpoint{2.515647in}{1.673764in}}%
\pgfpathlineto{\pgfqpoint{2.516414in}{1.669690in}}%
\pgfpathlineto{\pgfqpoint{2.517948in}{1.681913in}}%
\pgfpathlineto{\pgfqpoint{2.518715in}{1.665615in}}%
\pgfpathlineto{\pgfqpoint{2.519098in}{1.673764in}}%
\pgfpathlineto{\pgfqpoint{2.519481in}{1.673764in}}%
\pgfpathlineto{\pgfqpoint{2.519864in}{1.681913in}}%
\pgfpathlineto{\pgfqpoint{2.520247in}{1.673764in}}%
\pgfpathlineto{\pgfqpoint{2.520716in}{1.673764in}}%
\pgfpathlineto{\pgfqpoint{2.521100in}{1.649318in}}%
\pgfpathlineto{\pgfqpoint{2.521484in}{1.669690in}}%
\pgfpathlineto{\pgfqpoint{2.522640in}{1.690062in}}%
\pgfpathlineto{\pgfqpoint{2.523025in}{1.681913in}}%
\pgfpathlineto{\pgfqpoint{2.523793in}{1.657467in}}%
\pgfpathlineto{\pgfqpoint{2.524176in}{1.685987in}}%
\pgfpathlineto{\pgfqpoint{2.524559in}{1.677838in}}%
\pgfpathlineto{\pgfqpoint{2.524942in}{1.657467in}}%
\pgfpathlineto{\pgfqpoint{2.525326in}{1.698210in}}%
\pgfpathlineto{\pgfqpoint{2.526093in}{1.681913in}}%
\pgfpathlineto{\pgfqpoint{2.528391in}{1.661541in}}%
\pgfpathlineto{\pgfqpoint{2.529926in}{1.677838in}}%
\pgfpathlineto{\pgfqpoint{2.530310in}{1.677838in}}%
\pgfpathlineto{\pgfqpoint{2.530693in}{1.661541in}}%
\pgfpathlineto{\pgfqpoint{2.531077in}{1.677838in}}%
\pgfpathlineto{\pgfqpoint{2.531461in}{1.681913in}}%
\pgfpathlineto{\pgfqpoint{2.531845in}{1.677838in}}%
\pgfpathlineto{\pgfqpoint{2.532993in}{1.669690in}}%
\pgfpathlineto{\pgfqpoint{2.533377in}{1.677838in}}%
\pgfpathlineto{\pgfqpoint{2.533377in}{1.677838in}}%
\pgfpathlineto{\pgfqpoint{2.533377in}{1.677838in}}%
\pgfpathlineto{\pgfqpoint{2.534527in}{1.665615in}}%
\pgfpathlineto{\pgfqpoint{2.536060in}{1.690062in}}%
\pgfpathlineto{\pgfqpoint{2.537593in}{1.665615in}}%
\pgfpathlineto{\pgfqpoint{2.538360in}{1.677838in}}%
\pgfpathlineto{\pgfqpoint{2.538744in}{1.669690in}}%
\pgfpathlineto{\pgfqpoint{2.539128in}{1.665615in}}%
\pgfpathlineto{\pgfqpoint{2.539513in}{1.681913in}}%
\pgfpathlineto{\pgfqpoint{2.540279in}{1.677838in}}%
\pgfpathlineto{\pgfqpoint{2.541429in}{1.673764in}}%
\pgfpathlineto{\pgfqpoint{2.542195in}{1.690062in}}%
\pgfpathlineto{\pgfqpoint{2.542579in}{1.657467in}}%
\pgfpathlineto{\pgfqpoint{2.543345in}{1.677838in}}%
\pgfpathlineto{\pgfqpoint{2.544113in}{1.665615in}}%
\pgfpathlineto{\pgfqpoint{2.544495in}{1.690062in}}%
\pgfpathlineto{\pgfqpoint{2.545261in}{1.673764in}}%
\pgfpathlineto{\pgfqpoint{2.546411in}{1.661541in}}%
\pgfpathlineto{\pgfqpoint{2.547177in}{1.690062in}}%
\pgfpathlineto{\pgfqpoint{2.547944in}{1.681913in}}%
\pgfpathlineto{\pgfqpoint{2.548709in}{1.665615in}}%
\pgfpathlineto{\pgfqpoint{2.549860in}{1.669690in}}%
\pgfpathlineto{\pgfqpoint{2.550245in}{1.685987in}}%
\pgfpathlineto{\pgfqpoint{2.551010in}{1.677838in}}%
\pgfpathlineto{\pgfqpoint{2.551393in}{1.661541in}}%
\pgfpathlineto{\pgfqpoint{2.551393in}{1.661541in}}%
\pgfpathlineto{\pgfqpoint{2.551393in}{1.661541in}}%
\pgfpathlineto{\pgfqpoint{2.551776in}{1.685987in}}%
\pgfpathlineto{\pgfqpoint{2.552543in}{1.681913in}}%
\pgfpathlineto{\pgfqpoint{2.553310in}{1.677838in}}%
\pgfpathlineto{\pgfqpoint{2.554460in}{1.685987in}}%
\pgfpathlineto{\pgfqpoint{2.554843in}{1.661541in}}%
\pgfpathlineto{\pgfqpoint{2.555226in}{1.677838in}}%
\pgfpathlineto{\pgfqpoint{2.555610in}{1.685987in}}%
\pgfpathlineto{\pgfqpoint{2.557143in}{1.665615in}}%
\pgfpathlineto{\pgfqpoint{2.557910in}{1.681913in}}%
\pgfpathlineto{\pgfqpoint{2.558294in}{1.677838in}}%
\pgfpathlineto{\pgfqpoint{2.558677in}{1.673764in}}%
\pgfpathlineto{\pgfqpoint{2.559060in}{1.677838in}}%
\pgfpathlineto{\pgfqpoint{2.559443in}{1.698210in}}%
\pgfpathlineto{\pgfqpoint{2.559826in}{1.685987in}}%
\pgfpathlineto{\pgfqpoint{2.560593in}{1.665615in}}%
\pgfpathlineto{\pgfqpoint{2.560977in}{1.673764in}}%
\pgfpathlineto{\pgfqpoint{2.562135in}{1.694136in}}%
\pgfpathlineto{\pgfqpoint{2.563285in}{1.665615in}}%
\pgfpathlineto{\pgfqpoint{2.564434in}{1.681913in}}%
\pgfpathlineto{\pgfqpoint{2.565201in}{1.669690in}}%
\pgfpathlineto{\pgfqpoint{2.565584in}{1.677838in}}%
\pgfpathlineto{\pgfqpoint{2.565967in}{1.685987in}}%
\pgfpathlineto{\pgfqpoint{2.567116in}{1.665615in}}%
\pgfpathlineto{\pgfqpoint{2.567500in}{1.669690in}}%
\pgfpathlineto{\pgfqpoint{2.568650in}{1.661541in}}%
\pgfpathlineto{\pgfqpoint{2.569501in}{1.685987in}}%
\pgfpathlineto{\pgfqpoint{2.569884in}{1.669690in}}%
\pgfpathlineto{\pgfqpoint{2.570651in}{1.677838in}}%
\pgfpathlineto{\pgfqpoint{2.571035in}{1.673764in}}%
\pgfpathlineto{\pgfqpoint{2.571418in}{1.665615in}}%
\pgfpathlineto{\pgfqpoint{2.571800in}{1.673764in}}%
\pgfpathlineto{\pgfqpoint{2.572566in}{1.681913in}}%
\pgfpathlineto{\pgfqpoint{2.572949in}{1.677838in}}%
\pgfpathlineto{\pgfqpoint{2.574099in}{1.669690in}}%
\pgfpathlineto{\pgfqpoint{2.575247in}{1.685987in}}%
\pgfpathlineto{\pgfqpoint{2.575630in}{1.665615in}}%
\pgfpathlineto{\pgfqpoint{2.576013in}{1.673764in}}%
\pgfpathlineto{\pgfqpoint{2.576397in}{1.698210in}}%
\pgfpathlineto{\pgfqpoint{2.576397in}{1.698210in}}%
\pgfpathlineto{\pgfqpoint{2.576397in}{1.698210in}}%
\pgfpathlineto{\pgfqpoint{2.576781in}{1.657467in}}%
\pgfpathlineto{\pgfqpoint{2.577547in}{1.665615in}}%
\pgfpathlineto{\pgfqpoint{2.578697in}{1.681913in}}%
\pgfpathlineto{\pgfqpoint{2.580613in}{1.657467in}}%
\pgfpathlineto{\pgfqpoint{2.581378in}{1.677838in}}%
\pgfpathlineto{\pgfqpoint{2.581761in}{1.669690in}}%
\pgfpathlineto{\pgfqpoint{2.582530in}{1.669690in}}%
\pgfpathlineto{\pgfqpoint{2.583296in}{1.657467in}}%
\pgfpathlineto{\pgfqpoint{2.584445in}{1.694136in}}%
\pgfpathlineto{\pgfqpoint{2.584830in}{1.685987in}}%
\pgfpathlineto{\pgfqpoint{2.585214in}{1.669690in}}%
\pgfpathlineto{\pgfqpoint{2.585214in}{1.669690in}}%
\pgfpathlineto{\pgfqpoint{2.585214in}{1.669690in}}%
\pgfpathlineto{\pgfqpoint{2.585596in}{1.690062in}}%
\pgfpathlineto{\pgfqpoint{2.586362in}{1.677838in}}%
\pgfpathlineto{\pgfqpoint{2.586746in}{1.677838in}}%
\pgfpathlineto{\pgfqpoint{2.587130in}{1.690062in}}%
\pgfpathlineto{\pgfqpoint{2.588279in}{1.653392in}}%
\pgfpathlineto{\pgfqpoint{2.589429in}{1.685987in}}%
\pgfpathlineto{\pgfqpoint{2.589811in}{1.677838in}}%
\pgfpathlineto{\pgfqpoint{2.590195in}{1.661541in}}%
\pgfpathlineto{\pgfqpoint{2.590962in}{1.673764in}}%
\pgfpathlineto{\pgfqpoint{2.591346in}{1.677838in}}%
\pgfpathlineto{\pgfqpoint{2.591729in}{1.673764in}}%
\pgfpathlineto{\pgfqpoint{2.592495in}{1.669690in}}%
\pgfpathlineto{\pgfqpoint{2.593646in}{1.673764in}}%
\pgfpathlineto{\pgfqpoint{2.594412in}{1.669690in}}%
\pgfpathlineto{\pgfqpoint{2.594795in}{1.685987in}}%
\pgfpathlineto{\pgfqpoint{2.594795in}{1.685987in}}%
\pgfpathlineto{\pgfqpoint{2.594795in}{1.685987in}}%
\pgfpathlineto{\pgfqpoint{2.595562in}{1.653392in}}%
\pgfpathlineto{\pgfqpoint{2.595947in}{1.669690in}}%
\pgfpathlineto{\pgfqpoint{2.596330in}{1.673764in}}%
\pgfpathlineto{\pgfqpoint{2.596713in}{1.657467in}}%
\pgfpathlineto{\pgfqpoint{2.596713in}{1.657467in}}%
\pgfpathlineto{\pgfqpoint{2.596713in}{1.657467in}}%
\pgfpathlineto{\pgfqpoint{2.597861in}{1.681913in}}%
\pgfpathlineto{\pgfqpoint{2.598629in}{1.665615in}}%
\pgfpathlineto{\pgfqpoint{2.599395in}{1.681913in}}%
\pgfpathlineto{\pgfqpoint{2.599779in}{1.657467in}}%
\pgfpathlineto{\pgfqpoint{2.600545in}{1.669690in}}%
\pgfpathlineto{\pgfqpoint{2.600930in}{1.677838in}}%
\pgfpathlineto{\pgfqpoint{2.601696in}{1.673764in}}%
\pgfpathlineto{\pgfqpoint{2.602086in}{1.665615in}}%
\pgfpathlineto{\pgfqpoint{2.602471in}{1.669690in}}%
\pgfpathlineto{\pgfqpoint{2.602855in}{1.673764in}}%
\pgfpathlineto{\pgfqpoint{2.603238in}{1.653392in}}%
\pgfpathlineto{\pgfqpoint{2.603621in}{1.665615in}}%
\pgfpathlineto{\pgfqpoint{2.604004in}{1.685987in}}%
\pgfpathlineto{\pgfqpoint{2.604771in}{1.673764in}}%
\pgfpathlineto{\pgfqpoint{2.605155in}{1.685987in}}%
\pgfpathlineto{\pgfqpoint{2.605155in}{1.685987in}}%
\pgfpathlineto{\pgfqpoint{2.605155in}{1.685987in}}%
\pgfpathlineto{\pgfqpoint{2.606688in}{1.661541in}}%
\pgfpathlineto{\pgfqpoint{2.608221in}{1.685987in}}%
\pgfpathlineto{\pgfqpoint{2.609371in}{1.661541in}}%
\pgfpathlineto{\pgfqpoint{2.610905in}{1.677838in}}%
\pgfpathlineto{\pgfqpoint{2.611288in}{1.677838in}}%
\pgfpathlineto{\pgfqpoint{2.612141in}{1.681913in}}%
\pgfpathlineto{\pgfqpoint{2.612907in}{1.657467in}}%
\pgfpathlineto{\pgfqpoint{2.613290in}{1.681913in}}%
\pgfpathlineto{\pgfqpoint{2.614056in}{1.661541in}}%
\pgfpathlineto{\pgfqpoint{2.615590in}{1.685987in}}%
\pgfpathlineto{\pgfqpoint{2.615974in}{1.657467in}}%
\pgfpathlineto{\pgfqpoint{2.616741in}{1.669690in}}%
\pgfpathlineto{\pgfqpoint{2.617124in}{1.677838in}}%
\pgfpathlineto{\pgfqpoint{2.617507in}{1.661541in}}%
\pgfpathlineto{\pgfqpoint{2.618273in}{1.669690in}}%
\pgfpathlineto{\pgfqpoint{2.618656in}{1.685987in}}%
\pgfpathlineto{\pgfqpoint{2.618656in}{1.685987in}}%
\pgfpathlineto{\pgfqpoint{2.618656in}{1.685987in}}%
\pgfpathlineto{\pgfqpoint{2.620190in}{1.657467in}}%
\pgfpathlineto{\pgfqpoint{2.622105in}{1.694136in}}%
\pgfpathlineto{\pgfqpoint{2.622489in}{1.694136in}}%
\pgfpathlineto{\pgfqpoint{2.623254in}{1.661541in}}%
\pgfpathlineto{\pgfqpoint{2.623637in}{1.685987in}}%
\pgfpathlineto{\pgfqpoint{2.624020in}{1.685987in}}%
\pgfpathlineto{\pgfqpoint{2.625169in}{1.661541in}}%
\pgfpathlineto{\pgfqpoint{2.626715in}{1.681913in}}%
\pgfpathlineto{\pgfqpoint{2.627100in}{1.661541in}}%
\pgfpathlineto{\pgfqpoint{2.627867in}{1.669690in}}%
\pgfpathlineto{\pgfqpoint{2.628250in}{1.690062in}}%
\pgfpathlineto{\pgfqpoint{2.629017in}{1.685987in}}%
\pgfpathlineto{\pgfqpoint{2.630551in}{1.661541in}}%
\pgfpathlineto{\pgfqpoint{2.631318in}{1.694136in}}%
\pgfpathlineto{\pgfqpoint{2.632084in}{1.677838in}}%
\pgfpathlineto{\pgfqpoint{2.632468in}{1.673764in}}%
\pgfpathlineto{\pgfqpoint{2.632851in}{1.685987in}}%
\pgfpathlineto{\pgfqpoint{2.632851in}{1.685987in}}%
\pgfpathlineto{\pgfqpoint{2.632851in}{1.685987in}}%
\pgfpathlineto{\pgfqpoint{2.633234in}{1.669690in}}%
\pgfpathlineto{\pgfqpoint{2.633234in}{1.669690in}}%
\pgfpathlineto{\pgfqpoint{2.633234in}{1.669690in}}%
\pgfpathlineto{\pgfqpoint{2.633618in}{1.694136in}}%
\pgfpathlineto{\pgfqpoint{2.634002in}{1.669690in}}%
\pgfpathlineto{\pgfqpoint{2.634768in}{1.661541in}}%
\pgfpathlineto{\pgfqpoint{2.635151in}{1.685987in}}%
\pgfpathlineto{\pgfqpoint{2.635917in}{1.677838in}}%
\pgfpathlineto{\pgfqpoint{2.636301in}{1.673764in}}%
\pgfpathlineto{\pgfqpoint{2.636685in}{1.677838in}}%
\pgfpathlineto{\pgfqpoint{2.637069in}{1.677838in}}%
\pgfpathlineto{\pgfqpoint{2.637452in}{1.694136in}}%
\pgfpathlineto{\pgfqpoint{2.637452in}{1.694136in}}%
\pgfpathlineto{\pgfqpoint{2.637452in}{1.694136in}}%
\pgfpathlineto{\pgfqpoint{2.638601in}{1.665615in}}%
\pgfpathlineto{\pgfqpoint{2.639368in}{1.694136in}}%
\pgfpathlineto{\pgfqpoint{2.639751in}{1.669690in}}%
\pgfpathlineto{\pgfqpoint{2.640134in}{1.669690in}}%
\pgfpathlineto{\pgfqpoint{2.640900in}{1.677838in}}%
\pgfpathlineto{\pgfqpoint{2.641666in}{1.665615in}}%
\pgfpathlineto{\pgfqpoint{2.642057in}{1.669690in}}%
\pgfpathlineto{\pgfqpoint{2.642441in}{1.685987in}}%
\pgfpathlineto{\pgfqpoint{2.642824in}{1.677838in}}%
\pgfpathlineto{\pgfqpoint{2.643206in}{1.665615in}}%
\pgfpathlineto{\pgfqpoint{2.643975in}{1.673764in}}%
\pgfpathlineto{\pgfqpoint{2.644358in}{1.677838in}}%
\pgfpathlineto{\pgfqpoint{2.644740in}{1.669690in}}%
\pgfpathlineto{\pgfqpoint{2.644740in}{1.669690in}}%
\pgfpathlineto{\pgfqpoint{2.644740in}{1.669690in}}%
\pgfpathlineto{\pgfqpoint{2.645123in}{1.681913in}}%
\pgfpathlineto{\pgfqpoint{2.645890in}{1.677838in}}%
\pgfpathlineto{\pgfqpoint{2.646273in}{1.661541in}}%
\pgfpathlineto{\pgfqpoint{2.647040in}{1.673764in}}%
\pgfpathlineto{\pgfqpoint{2.647423in}{1.690062in}}%
\pgfpathlineto{\pgfqpoint{2.648190in}{1.681913in}}%
\pgfpathlineto{\pgfqpoint{2.648573in}{1.685987in}}%
\pgfpathlineto{\pgfqpoint{2.648957in}{1.681913in}}%
\pgfpathlineto{\pgfqpoint{2.650491in}{1.661541in}}%
\pgfpathlineto{\pgfqpoint{2.650874in}{1.669690in}}%
\pgfpathlineto{\pgfqpoint{2.652791in}{1.690062in}}%
\pgfpathlineto{\pgfqpoint{2.653174in}{1.685987in}}%
\pgfpathlineto{\pgfqpoint{2.653557in}{1.665615in}}%
\pgfpathlineto{\pgfqpoint{2.654324in}{1.673764in}}%
\pgfpathlineto{\pgfqpoint{2.654708in}{1.669690in}}%
\pgfpathlineto{\pgfqpoint{2.655092in}{1.681913in}}%
\pgfpathlineto{\pgfqpoint{2.655475in}{1.673764in}}%
\pgfpathlineto{\pgfqpoint{2.655857in}{1.661541in}}%
\pgfpathlineto{\pgfqpoint{2.656623in}{1.690062in}}%
\pgfpathlineto{\pgfqpoint{2.657006in}{1.681913in}}%
\pgfpathlineto{\pgfqpoint{2.657390in}{1.681913in}}%
\pgfpathlineto{\pgfqpoint{2.658157in}{1.677838in}}%
\pgfpathlineto{\pgfqpoint{2.658540in}{1.690062in}}%
\pgfpathlineto{\pgfqpoint{2.658540in}{1.690062in}}%
\pgfpathlineto{\pgfqpoint{2.658540in}{1.690062in}}%
\pgfpathlineto{\pgfqpoint{2.660157in}{1.657467in}}%
\pgfpathlineto{\pgfqpoint{2.660924in}{1.665615in}}%
\pgfpathlineto{\pgfqpoint{2.661307in}{1.681913in}}%
\pgfpathlineto{\pgfqpoint{2.662073in}{1.669690in}}%
\pgfpathlineto{\pgfqpoint{2.662839in}{1.661541in}}%
\pgfpathlineto{\pgfqpoint{2.663990in}{1.677838in}}%
\pgfpathlineto{\pgfqpoint{2.665521in}{1.665615in}}%
\pgfpathlineto{\pgfqpoint{2.667055in}{1.677838in}}%
\pgfpathlineto{\pgfqpoint{2.667437in}{1.677838in}}%
\pgfpathlineto{\pgfqpoint{2.667819in}{1.665615in}}%
\pgfpathlineto{\pgfqpoint{2.668585in}{1.673764in}}%
\pgfpathlineto{\pgfqpoint{2.669352in}{1.685987in}}%
\pgfpathlineto{\pgfqpoint{2.669736in}{1.681913in}}%
\pgfpathlineto{\pgfqpoint{2.670502in}{1.665615in}}%
\pgfpathlineto{\pgfqpoint{2.670885in}{1.685987in}}%
\pgfpathlineto{\pgfqpoint{2.671268in}{1.665615in}}%
\pgfpathlineto{\pgfqpoint{2.671651in}{1.665615in}}%
\pgfpathlineto{\pgfqpoint{2.672803in}{1.677838in}}%
\pgfpathlineto{\pgfqpoint{2.673186in}{1.677838in}}%
\pgfpathlineto{\pgfqpoint{2.673568in}{1.690062in}}%
\pgfpathlineto{\pgfqpoint{2.673951in}{1.677838in}}%
\pgfpathlineto{\pgfqpoint{2.674335in}{1.665615in}}%
\pgfpathlineto{\pgfqpoint{2.674335in}{1.665615in}}%
\pgfpathlineto{\pgfqpoint{2.674335in}{1.665615in}}%
\pgfpathlineto{\pgfqpoint{2.674718in}{1.685987in}}%
\pgfpathlineto{\pgfqpoint{2.674718in}{1.685987in}}%
\pgfpathlineto{\pgfqpoint{2.674718in}{1.685987in}}%
\pgfpathlineto{\pgfqpoint{2.675103in}{1.661541in}}%
\pgfpathlineto{\pgfqpoint{2.675869in}{1.673764in}}%
\pgfpathlineto{\pgfqpoint{2.676635in}{1.657467in}}%
\pgfpathlineto{\pgfqpoint{2.677402in}{1.698210in}}%
\pgfpathlineto{\pgfqpoint{2.678169in}{1.685987in}}%
\pgfpathlineto{\pgfqpoint{2.678551in}{1.681913in}}%
\pgfpathlineto{\pgfqpoint{2.678934in}{1.649318in}}%
\pgfpathlineto{\pgfqpoint{2.679701in}{1.673764in}}%
\pgfpathlineto{\pgfqpoint{2.680084in}{1.681913in}}%
\pgfpathlineto{\pgfqpoint{2.680468in}{1.665615in}}%
\pgfpathlineto{\pgfqpoint{2.681234in}{1.673764in}}%
\pgfpathlineto{\pgfqpoint{2.681616in}{1.669690in}}%
\pgfpathlineto{\pgfqpoint{2.682392in}{1.694136in}}%
\pgfpathlineto{\pgfqpoint{2.682775in}{1.685987in}}%
\pgfpathlineto{\pgfqpoint{2.683923in}{1.673764in}}%
\pgfpathlineto{\pgfqpoint{2.684307in}{1.677838in}}%
\pgfpathlineto{\pgfqpoint{2.684690in}{1.669690in}}%
\pgfpathlineto{\pgfqpoint{2.685073in}{1.685987in}}%
\pgfpathlineto{\pgfqpoint{2.685839in}{1.677838in}}%
\pgfpathlineto{\pgfqpoint{2.686222in}{1.685987in}}%
\pgfpathlineto{\pgfqpoint{2.686606in}{1.661541in}}%
\pgfpathlineto{\pgfqpoint{2.686988in}{1.669690in}}%
\pgfpathlineto{\pgfqpoint{2.687372in}{1.690062in}}%
\pgfpathlineto{\pgfqpoint{2.687372in}{1.690062in}}%
\pgfpathlineto{\pgfqpoint{2.687372in}{1.690062in}}%
\pgfpathlineto{\pgfqpoint{2.688905in}{1.661541in}}%
\pgfpathlineto{\pgfqpoint{2.690053in}{1.681913in}}%
\pgfpathlineto{\pgfqpoint{2.691204in}{1.665615in}}%
\pgfpathlineto{\pgfqpoint{2.691587in}{1.681913in}}%
\pgfpathlineto{\pgfqpoint{2.691587in}{1.681913in}}%
\pgfpathlineto{\pgfqpoint{2.691587in}{1.681913in}}%
\pgfpathlineto{\pgfqpoint{2.691969in}{1.661541in}}%
\pgfpathlineto{\pgfqpoint{2.691969in}{1.661541in}}%
\pgfpathlineto{\pgfqpoint{2.691969in}{1.661541in}}%
\pgfpathlineto{\pgfqpoint{2.692352in}{1.690062in}}%
\pgfpathlineto{\pgfqpoint{2.693119in}{1.677838in}}%
\pgfpathlineto{\pgfqpoint{2.693504in}{1.669690in}}%
\pgfpathlineto{\pgfqpoint{2.693504in}{1.669690in}}%
\pgfpathlineto{\pgfqpoint{2.693504in}{1.669690in}}%
\pgfpathlineto{\pgfqpoint{2.693887in}{1.681913in}}%
\pgfpathlineto{\pgfqpoint{2.694269in}{1.673764in}}%
\pgfpathlineto{\pgfqpoint{2.694652in}{1.665615in}}%
\pgfpathlineto{\pgfqpoint{2.695035in}{1.669690in}}%
\pgfpathlineto{\pgfqpoint{2.695801in}{1.685987in}}%
\pgfpathlineto{\pgfqpoint{2.696952in}{1.665615in}}%
\pgfpathlineto{\pgfqpoint{2.697805in}{1.694136in}}%
\pgfpathlineto{\pgfqpoint{2.698187in}{1.665615in}}%
\pgfpathlineto{\pgfqpoint{2.698953in}{1.681913in}}%
\pgfpathlineto{\pgfqpoint{2.700102in}{1.669690in}}%
\pgfpathlineto{\pgfqpoint{2.700485in}{1.690062in}}%
\pgfpathlineto{\pgfqpoint{2.700485in}{1.690062in}}%
\pgfpathlineto{\pgfqpoint{2.700485in}{1.690062in}}%
\pgfpathlineto{\pgfqpoint{2.700868in}{1.665615in}}%
\pgfpathlineto{\pgfqpoint{2.701633in}{1.681913in}}%
\pgfpathlineto{\pgfqpoint{2.702016in}{1.673764in}}%
\pgfpathlineto{\pgfqpoint{2.702016in}{1.673764in}}%
\pgfpathlineto{\pgfqpoint{2.702016in}{1.673764in}}%
\pgfpathlineto{\pgfqpoint{2.702400in}{1.685987in}}%
\pgfpathlineto{\pgfqpoint{2.702783in}{1.677838in}}%
\pgfpathlineto{\pgfqpoint{2.703166in}{1.665615in}}%
\pgfpathlineto{\pgfqpoint{2.703932in}{1.673764in}}%
\pgfpathlineto{\pgfqpoint{2.704315in}{1.690062in}}%
\pgfpathlineto{\pgfqpoint{2.704699in}{1.681913in}}%
\pgfpathlineto{\pgfqpoint{2.705082in}{1.669690in}}%
\pgfpathlineto{\pgfqpoint{2.705466in}{1.706359in}}%
\pgfpathlineto{\pgfqpoint{2.706232in}{1.685987in}}%
\pgfpathlineto{\pgfqpoint{2.707380in}{1.661541in}}%
\pgfpathlineto{\pgfqpoint{2.708913in}{1.681913in}}%
\pgfpathlineto{\pgfqpoint{2.710063in}{1.657467in}}%
\pgfpathlineto{\pgfqpoint{2.710446in}{1.665615in}}%
\pgfpathlineto{\pgfqpoint{2.710829in}{1.677838in}}%
\pgfpathlineto{\pgfqpoint{2.711213in}{1.669690in}}%
\pgfpathlineto{\pgfqpoint{2.711597in}{1.665615in}}%
\pgfpathlineto{\pgfqpoint{2.712364in}{1.685987in}}%
\pgfpathlineto{\pgfqpoint{2.712746in}{1.657467in}}%
\pgfpathlineto{\pgfqpoint{2.712746in}{1.657467in}}%
\pgfpathlineto{\pgfqpoint{2.712746in}{1.657467in}}%
\pgfpathlineto{\pgfqpoint{2.713129in}{1.698210in}}%
\pgfpathlineto{\pgfqpoint{2.713894in}{1.669690in}}%
\pgfpathlineto{\pgfqpoint{2.714278in}{1.657467in}}%
\pgfpathlineto{\pgfqpoint{2.714278in}{1.657467in}}%
\pgfpathlineto{\pgfqpoint{2.714278in}{1.657467in}}%
\pgfpathlineto{\pgfqpoint{2.715044in}{1.681913in}}%
\pgfpathlineto{\pgfqpoint{2.715427in}{1.665615in}}%
\pgfpathlineto{\pgfqpoint{2.716193in}{1.681913in}}%
\pgfpathlineto{\pgfqpoint{2.716576in}{1.665615in}}%
\pgfpathlineto{\pgfqpoint{2.716576in}{1.665615in}}%
\pgfpathlineto{\pgfqpoint{2.716576in}{1.665615in}}%
\pgfpathlineto{\pgfqpoint{2.716960in}{1.690062in}}%
\pgfpathlineto{\pgfqpoint{2.717344in}{1.669690in}}%
\pgfpathlineto{\pgfqpoint{2.717727in}{1.665615in}}%
\pgfpathlineto{\pgfqpoint{2.718111in}{1.669690in}}%
\pgfpathlineto{\pgfqpoint{2.719643in}{1.681913in}}%
\pgfpathlineto{\pgfqpoint{2.720026in}{1.661541in}}%
\pgfpathlineto{\pgfqpoint{2.720026in}{1.661541in}}%
\pgfpathlineto{\pgfqpoint{2.720026in}{1.661541in}}%
\pgfpathlineto{\pgfqpoint{2.720410in}{1.690062in}}%
\pgfpathlineto{\pgfqpoint{2.721176in}{1.685987in}}%
\pgfpathlineto{\pgfqpoint{2.721559in}{1.665615in}}%
\pgfpathlineto{\pgfqpoint{2.722332in}{1.669690in}}%
\pgfpathlineto{\pgfqpoint{2.723098in}{1.685987in}}%
\pgfpathlineto{\pgfqpoint{2.723481in}{1.673764in}}%
\pgfpathlineto{\pgfqpoint{2.724249in}{1.657467in}}%
\pgfpathlineto{\pgfqpoint{2.724631in}{1.690062in}}%
\pgfpathlineto{\pgfqpoint{2.725398in}{1.661541in}}%
\pgfpathlineto{\pgfqpoint{2.726163in}{1.702285in}}%
\pgfpathlineto{\pgfqpoint{2.726930in}{1.677838in}}%
\pgfpathlineto{\pgfqpoint{2.727313in}{1.665615in}}%
\pgfpathlineto{\pgfqpoint{2.727697in}{1.673764in}}%
\pgfpathlineto{\pgfqpoint{2.728080in}{1.694136in}}%
\pgfpathlineto{\pgfqpoint{2.728846in}{1.677838in}}%
\pgfpathlineto{\pgfqpoint{2.729996in}{1.673764in}}%
\pgfpathlineto{\pgfqpoint{2.731146in}{1.690062in}}%
\pgfpathlineto{\pgfqpoint{2.731529in}{1.661541in}}%
\pgfpathlineto{\pgfqpoint{2.732295in}{1.665615in}}%
\pgfpathlineto{\pgfqpoint{2.733062in}{1.694136in}}%
\pgfpathlineto{\pgfqpoint{2.733445in}{1.669690in}}%
\pgfpathlineto{\pgfqpoint{2.734297in}{1.681913in}}%
\pgfpathlineto{\pgfqpoint{2.734680in}{1.677838in}}%
\pgfpathlineto{\pgfqpoint{2.735447in}{1.673764in}}%
\pgfpathlineto{\pgfqpoint{2.736598in}{1.694136in}}%
\pgfpathlineto{\pgfqpoint{2.737363in}{1.645243in}}%
\pgfpathlineto{\pgfqpoint{2.737746in}{1.661541in}}%
\pgfpathlineto{\pgfqpoint{2.738129in}{1.681913in}}%
\pgfpathlineto{\pgfqpoint{2.738894in}{1.673764in}}%
\pgfpathlineto{\pgfqpoint{2.739660in}{1.681913in}}%
\pgfpathlineto{\pgfqpoint{2.740809in}{1.665615in}}%
\pgfpathlineto{\pgfqpoint{2.741192in}{1.673764in}}%
\pgfpathlineto{\pgfqpoint{2.741575in}{1.665615in}}%
\pgfpathlineto{\pgfqpoint{2.741959in}{1.661541in}}%
\pgfpathlineto{\pgfqpoint{2.742725in}{1.677838in}}%
\pgfpathlineto{\pgfqpoint{2.743107in}{1.669690in}}%
\pgfpathlineto{\pgfqpoint{2.743491in}{1.669690in}}%
\pgfpathlineto{\pgfqpoint{2.743874in}{1.649318in}}%
\pgfpathlineto{\pgfqpoint{2.744257in}{1.669690in}}%
\pgfpathlineto{\pgfqpoint{2.744641in}{2.712733in}}%
\pgfpathlineto{\pgfqpoint{2.745024in}{1.673764in}}%
\pgfpathlineto{\pgfqpoint{2.746173in}{1.673764in}}%
\pgfpathlineto{\pgfqpoint{2.746556in}{1.669690in}}%
\pgfpathlineto{\pgfqpoint{2.746939in}{1.673764in}}%
\pgfpathlineto{\pgfqpoint{2.747706in}{1.677838in}}%
\pgfpathlineto{\pgfqpoint{2.748089in}{1.657467in}}%
\pgfpathlineto{\pgfqpoint{2.748855in}{1.673764in}}%
\pgfpathlineto{\pgfqpoint{2.749237in}{1.665615in}}%
\pgfpathlineto{\pgfqpoint{2.750388in}{1.681913in}}%
\pgfpathlineto{\pgfqpoint{2.750771in}{1.661541in}}%
\pgfpathlineto{\pgfqpoint{2.750771in}{1.661541in}}%
\pgfpathlineto{\pgfqpoint{2.750771in}{1.661541in}}%
\pgfpathlineto{\pgfqpoint{2.751154in}{1.694136in}}%
\pgfpathlineto{\pgfqpoint{2.751537in}{1.685987in}}%
\pgfpathlineto{\pgfqpoint{2.751919in}{1.661541in}}%
\pgfpathlineto{\pgfqpoint{2.752686in}{1.669690in}}%
\pgfpathlineto{\pgfqpoint{2.754219in}{1.677838in}}%
\pgfpathlineto{\pgfqpoint{2.754603in}{1.669690in}}%
\pgfpathlineto{\pgfqpoint{2.754986in}{1.690062in}}%
\pgfpathlineto{\pgfqpoint{2.755368in}{1.677838in}}%
\pgfpathlineto{\pgfqpoint{2.755752in}{1.657467in}}%
\pgfpathlineto{\pgfqpoint{2.755752in}{1.657467in}}%
\pgfpathlineto{\pgfqpoint{2.755752in}{1.657467in}}%
\pgfpathlineto{\pgfqpoint{2.756519in}{1.685987in}}%
\pgfpathlineto{\pgfqpoint{2.756903in}{1.669690in}}%
\pgfpathlineto{\pgfqpoint{2.757286in}{1.661541in}}%
\pgfpathlineto{\pgfqpoint{2.758051in}{1.685987in}}%
\pgfpathlineto{\pgfqpoint{2.758434in}{1.681913in}}%
\pgfpathlineto{\pgfqpoint{2.759202in}{1.673764in}}%
\pgfpathlineto{\pgfqpoint{2.759968in}{1.690062in}}%
\pgfpathlineto{\pgfqpoint{2.761115in}{1.661541in}}%
\pgfpathlineto{\pgfqpoint{2.761499in}{1.669690in}}%
\pgfpathlineto{\pgfqpoint{2.761889in}{1.673764in}}%
\pgfpathlineto{\pgfqpoint{2.763037in}{1.657467in}}%
\pgfpathlineto{\pgfqpoint{2.764188in}{1.677838in}}%
\pgfpathlineto{\pgfqpoint{2.764571in}{1.698210in}}%
\pgfpathlineto{\pgfqpoint{2.764953in}{1.677838in}}%
\pgfpathlineto{\pgfqpoint{2.765336in}{1.669690in}}%
\pgfpathlineto{\pgfqpoint{2.765719in}{1.673764in}}%
\pgfpathlineto{\pgfqpoint{2.766103in}{1.690062in}}%
\pgfpathlineto{\pgfqpoint{2.766870in}{1.677838in}}%
\pgfpathlineto{\pgfqpoint{2.767253in}{1.685987in}}%
\pgfpathlineto{\pgfqpoint{2.768487in}{1.665615in}}%
\pgfpathlineto{\pgfqpoint{2.768870in}{1.681913in}}%
\pgfpathlineto{\pgfqpoint{2.769253in}{1.649318in}}%
\pgfpathlineto{\pgfqpoint{2.769253in}{1.649318in}}%
\pgfpathlineto{\pgfqpoint{2.769253in}{1.649318in}}%
\pgfpathlineto{\pgfqpoint{2.770019in}{1.685987in}}%
\pgfpathlineto{\pgfqpoint{2.770403in}{1.669690in}}%
\pgfpathlineto{\pgfqpoint{2.771170in}{1.685987in}}%
\pgfpathlineto{\pgfqpoint{2.772318in}{1.669690in}}%
\pgfpathlineto{\pgfqpoint{2.773085in}{1.685987in}}%
\pgfpathlineto{\pgfqpoint{2.773467in}{1.677838in}}%
\pgfpathlineto{\pgfqpoint{2.773851in}{1.661541in}}%
\pgfpathlineto{\pgfqpoint{2.773851in}{1.661541in}}%
\pgfpathlineto{\pgfqpoint{2.773851in}{1.661541in}}%
\pgfpathlineto{\pgfqpoint{2.774234in}{1.685987in}}%
\pgfpathlineto{\pgfqpoint{2.775000in}{1.673764in}}%
\pgfpathlineto{\pgfqpoint{2.776534in}{1.690062in}}%
\pgfpathlineto{\pgfqpoint{2.777301in}{1.681913in}}%
\pgfpathlineto{\pgfqpoint{2.777684in}{1.698210in}}%
\pgfpathlineto{\pgfqpoint{2.778067in}{1.661541in}}%
\pgfpathlineto{\pgfqpoint{2.778834in}{1.673764in}}%
\pgfpathlineto{\pgfqpoint{2.779217in}{1.673764in}}%
\pgfpathlineto{\pgfqpoint{2.779600in}{1.669690in}}%
\pgfpathlineto{\pgfqpoint{2.779984in}{1.685987in}}%
\pgfpathlineto{\pgfqpoint{2.780750in}{1.677838in}}%
\pgfpathlineto{\pgfqpoint{2.781133in}{1.673764in}}%
\pgfpathlineto{\pgfqpoint{2.781517in}{1.698210in}}%
\pgfpathlineto{\pgfqpoint{2.782284in}{1.681913in}}%
\pgfpathlineto{\pgfqpoint{2.782667in}{1.690062in}}%
\pgfpathlineto{\pgfqpoint{2.784200in}{1.657467in}}%
\pgfpathlineto{\pgfqpoint{2.785733in}{1.685987in}}%
\pgfpathlineto{\pgfqpoint{2.787267in}{1.661541in}}%
\pgfpathlineto{\pgfqpoint{2.788033in}{1.685987in}}%
\pgfpathlineto{\pgfqpoint{2.788416in}{1.673764in}}%
\pgfpathlineto{\pgfqpoint{2.788800in}{1.665615in}}%
\pgfpathlineto{\pgfqpoint{2.789184in}{1.669690in}}%
\pgfpathlineto{\pgfqpoint{2.789950in}{1.685987in}}%
\pgfpathlineto{\pgfqpoint{2.790333in}{1.673764in}}%
\pgfpathlineto{\pgfqpoint{2.790715in}{1.665615in}}%
\pgfpathlineto{\pgfqpoint{2.791866in}{1.685987in}}%
\pgfpathlineto{\pgfqpoint{2.792250in}{1.665615in}}%
\pgfpathlineto{\pgfqpoint{2.793016in}{1.677838in}}%
\pgfpathlineto{\pgfqpoint{2.793783in}{1.669690in}}%
\pgfpathlineto{\pgfqpoint{2.794167in}{1.681913in}}%
\pgfpathlineto{\pgfqpoint{2.794933in}{1.673764in}}%
\pgfpathlineto{\pgfqpoint{2.795317in}{1.657467in}}%
\pgfpathlineto{\pgfqpoint{2.795700in}{1.673764in}}%
\pgfpathlineto{\pgfqpoint{2.796084in}{1.681913in}}%
\pgfpathlineto{\pgfqpoint{2.796084in}{1.681913in}}%
\pgfpathlineto{\pgfqpoint{2.796084in}{1.681913in}}%
\pgfpathlineto{\pgfqpoint{2.797234in}{1.669690in}}%
\pgfpathlineto{\pgfqpoint{2.797616in}{1.690062in}}%
\pgfpathlineto{\pgfqpoint{2.798000in}{1.669690in}}%
\pgfpathlineto{\pgfqpoint{2.798383in}{1.669690in}}%
\pgfpathlineto{\pgfqpoint{2.798766in}{1.677838in}}%
\pgfpathlineto{\pgfqpoint{2.799149in}{1.653392in}}%
\pgfpathlineto{\pgfqpoint{2.799916in}{1.665615in}}%
\pgfpathlineto{\pgfqpoint{2.800301in}{1.690062in}}%
\pgfpathlineto{\pgfqpoint{2.801067in}{1.685987in}}%
\pgfpathlineto{\pgfqpoint{2.802991in}{1.653392in}}%
\pgfpathlineto{\pgfqpoint{2.803756in}{1.685987in}}%
\pgfpathlineto{\pgfqpoint{2.804139in}{1.677838in}}%
\pgfpathlineto{\pgfqpoint{2.804523in}{1.669690in}}%
\pgfpathlineto{\pgfqpoint{2.804906in}{1.673764in}}%
\pgfpathlineto{\pgfqpoint{2.805289in}{1.677838in}}%
\pgfpathlineto{\pgfqpoint{2.805672in}{1.669690in}}%
\pgfpathlineto{\pgfqpoint{2.806056in}{1.677838in}}%
\pgfpathlineto{\pgfqpoint{2.806438in}{1.698210in}}%
\pgfpathlineto{\pgfqpoint{2.806819in}{1.681913in}}%
\pgfpathlineto{\pgfqpoint{2.807969in}{1.665615in}}%
\pgfpathlineto{\pgfqpoint{2.808437in}{1.677838in}}%
\pgfpathlineto{\pgfqpoint{2.808821in}{1.657467in}}%
\pgfpathlineto{\pgfqpoint{2.809587in}{1.673764in}}%
\pgfpathlineto{\pgfqpoint{2.809969in}{1.681913in}}%
\pgfpathlineto{\pgfqpoint{2.810736in}{1.677838in}}%
\pgfpathlineto{\pgfqpoint{2.811503in}{1.681913in}}%
\pgfpathlineto{\pgfqpoint{2.811886in}{1.657467in}}%
\pgfpathlineto{\pgfqpoint{2.812652in}{1.669690in}}%
\pgfpathlineto{\pgfqpoint{2.814185in}{1.690062in}}%
\pgfpathlineto{\pgfqpoint{2.815716in}{1.661541in}}%
\pgfpathlineto{\pgfqpoint{2.817632in}{1.677838in}}%
\pgfpathlineto{\pgfqpoint{2.818015in}{1.645243in}}%
\pgfpathlineto{\pgfqpoint{2.818015in}{1.645243in}}%
\pgfpathlineto{\pgfqpoint{2.818015in}{1.645243in}}%
\pgfpathlineto{\pgfqpoint{2.818397in}{1.685987in}}%
\pgfpathlineto{\pgfqpoint{2.819162in}{1.681913in}}%
\pgfpathlineto{\pgfqpoint{2.819929in}{1.677838in}}%
\pgfpathlineto{\pgfqpoint{2.820696in}{1.690062in}}%
\pgfpathlineto{\pgfqpoint{2.822228in}{1.665615in}}%
\pgfpathlineto{\pgfqpoint{2.822611in}{1.657467in}}%
\pgfpathlineto{\pgfqpoint{2.823379in}{1.661541in}}%
\pgfpathlineto{\pgfqpoint{2.823761in}{1.661541in}}%
\pgfpathlineto{\pgfqpoint{2.824910in}{1.681913in}}%
\pgfpathlineto{\pgfqpoint{2.826061in}{1.665615in}}%
\pgfpathlineto{\pgfqpoint{2.827210in}{1.694136in}}%
\pgfpathlineto{\pgfqpoint{2.827977in}{1.685987in}}%
\pgfpathlineto{\pgfqpoint{2.828360in}{1.673764in}}%
\pgfpathlineto{\pgfqpoint{2.829126in}{1.681913in}}%
\pgfpathlineto{\pgfqpoint{2.829509in}{1.677838in}}%
\pgfpathlineto{\pgfqpoint{2.829893in}{1.681913in}}%
\pgfpathlineto{\pgfqpoint{2.830276in}{1.657467in}}%
\pgfpathlineto{\pgfqpoint{2.831043in}{1.669690in}}%
\pgfpathlineto{\pgfqpoint{2.831426in}{1.690062in}}%
\pgfpathlineto{\pgfqpoint{2.832193in}{1.673764in}}%
\pgfpathlineto{\pgfqpoint{2.832575in}{1.657467in}}%
\pgfpathlineto{\pgfqpoint{2.832575in}{1.657467in}}%
\pgfpathlineto{\pgfqpoint{2.832575in}{1.657467in}}%
\pgfpathlineto{\pgfqpoint{2.833342in}{1.685987in}}%
\pgfpathlineto{\pgfqpoint{2.833725in}{1.677838in}}%
\pgfpathlineto{\pgfqpoint{2.834109in}{1.681913in}}%
\pgfpathlineto{\pgfqpoint{2.834493in}{1.677838in}}%
\pgfpathlineto{\pgfqpoint{2.834876in}{1.661541in}}%
\pgfpathlineto{\pgfqpoint{2.835259in}{1.673764in}}%
\pgfpathlineto{\pgfqpoint{2.835641in}{1.681913in}}%
\pgfpathlineto{\pgfqpoint{2.836024in}{1.673764in}}%
\pgfpathlineto{\pgfqpoint{2.836408in}{1.673764in}}%
\pgfpathlineto{\pgfqpoint{2.836793in}{1.677838in}}%
\pgfpathlineto{\pgfqpoint{2.837175in}{1.673764in}}%
\pgfpathlineto{\pgfqpoint{2.837558in}{1.673764in}}%
\pgfpathlineto{\pgfqpoint{2.837941in}{1.698210in}}%
\pgfpathlineto{\pgfqpoint{2.838323in}{1.681913in}}%
\pgfpathlineto{\pgfqpoint{2.839090in}{1.661541in}}%
\pgfpathlineto{\pgfqpoint{2.839473in}{1.685987in}}%
\pgfpathlineto{\pgfqpoint{2.839473in}{1.685987in}}%
\pgfpathlineto{\pgfqpoint{2.839473in}{1.685987in}}%
\pgfpathlineto{\pgfqpoint{2.839856in}{1.657467in}}%
\pgfpathlineto{\pgfqpoint{2.840621in}{1.661541in}}%
\pgfpathlineto{\pgfqpoint{2.841864in}{1.673764in}}%
\pgfpathlineto{\pgfqpoint{2.842247in}{1.649318in}}%
\pgfpathlineto{\pgfqpoint{2.842247in}{1.649318in}}%
\pgfpathlineto{\pgfqpoint{2.842247in}{1.649318in}}%
\pgfpathlineto{\pgfqpoint{2.843780in}{1.685987in}}%
\pgfpathlineto{\pgfqpoint{2.844547in}{1.657467in}}%
\pgfpathlineto{\pgfqpoint{2.844930in}{1.677838in}}%
\pgfpathlineto{\pgfqpoint{2.845313in}{1.657467in}}%
\pgfpathlineto{\pgfqpoint{2.845313in}{1.657467in}}%
\pgfpathlineto{\pgfqpoint{2.845313in}{1.657467in}}%
\pgfpathlineto{\pgfqpoint{2.845697in}{1.681913in}}%
\pgfpathlineto{\pgfqpoint{2.846463in}{1.669690in}}%
\pgfpathlineto{\pgfqpoint{2.846846in}{1.665615in}}%
\pgfpathlineto{\pgfqpoint{2.848380in}{1.685987in}}%
\pgfpathlineto{\pgfqpoint{2.848763in}{1.669690in}}%
\pgfpathlineto{\pgfqpoint{2.849532in}{1.677838in}}%
\pgfpathlineto{\pgfqpoint{2.849915in}{1.681913in}}%
\pgfpathlineto{\pgfqpoint{2.850298in}{1.677838in}}%
\pgfpathlineto{\pgfqpoint{2.850681in}{1.669690in}}%
\pgfpathlineto{\pgfqpoint{2.851064in}{1.673764in}}%
\pgfpathlineto{\pgfqpoint{2.851829in}{1.681913in}}%
\pgfpathlineto{\pgfqpoint{2.852211in}{1.677838in}}%
\pgfpathlineto{\pgfqpoint{2.852978in}{1.669690in}}%
\pgfpathlineto{\pgfqpoint{2.853359in}{1.653392in}}%
\pgfpathlineto{\pgfqpoint{2.853743in}{1.665615in}}%
\pgfpathlineto{\pgfqpoint{2.854126in}{1.694136in}}%
\pgfpathlineto{\pgfqpoint{2.854892in}{1.690062in}}%
\pgfpathlineto{\pgfqpoint{2.855658in}{1.657467in}}%
\pgfpathlineto{\pgfqpoint{2.856807in}{1.665615in}}%
\pgfpathlineto{\pgfqpoint{2.858723in}{1.690062in}}%
\pgfpathlineto{\pgfqpoint{2.859874in}{1.657467in}}%
\pgfpathlineto{\pgfqpoint{2.860256in}{1.669690in}}%
\pgfpathlineto{\pgfqpoint{2.860639in}{1.677838in}}%
\pgfpathlineto{\pgfqpoint{2.860639in}{1.677838in}}%
\pgfpathlineto{\pgfqpoint{2.860639in}{1.677838in}}%
\pgfpathlineto{\pgfqpoint{2.861022in}{1.665615in}}%
\pgfpathlineto{\pgfqpoint{2.861405in}{1.690062in}}%
\pgfpathlineto{\pgfqpoint{2.862173in}{1.673764in}}%
\pgfpathlineto{\pgfqpoint{2.862556in}{1.673764in}}%
\pgfpathlineto{\pgfqpoint{2.863322in}{1.685987in}}%
\pgfpathlineto{\pgfqpoint{2.864855in}{1.657467in}}%
\pgfpathlineto{\pgfqpoint{2.865238in}{1.677838in}}%
\pgfpathlineto{\pgfqpoint{2.866003in}{1.661541in}}%
\pgfpathlineto{\pgfqpoint{2.866387in}{1.665615in}}%
\pgfpathlineto{\pgfqpoint{2.866770in}{1.653392in}}%
\pgfpathlineto{\pgfqpoint{2.867920in}{1.685987in}}%
\pgfpathlineto{\pgfqpoint{2.868303in}{1.653392in}}%
\pgfpathlineto{\pgfqpoint{2.869069in}{1.665615in}}%
\pgfpathlineto{\pgfqpoint{2.869452in}{1.685987in}}%
\pgfpathlineto{\pgfqpoint{2.870220in}{1.669690in}}%
\pgfpathlineto{\pgfqpoint{2.870603in}{1.665615in}}%
\pgfpathlineto{\pgfqpoint{2.870987in}{1.677838in}}%
\pgfpathlineto{\pgfqpoint{2.871369in}{1.669690in}}%
\pgfpathlineto{\pgfqpoint{2.871752in}{1.665615in}}%
\pgfpathlineto{\pgfqpoint{2.872134in}{1.694136in}}%
\pgfpathlineto{\pgfqpoint{2.872518in}{1.669690in}}%
\pgfpathlineto{\pgfqpoint{2.872902in}{1.653392in}}%
\pgfpathlineto{\pgfqpoint{2.873285in}{1.661541in}}%
\pgfpathlineto{\pgfqpoint{2.873668in}{1.677838in}}%
\pgfpathlineto{\pgfqpoint{2.874051in}{1.661541in}}%
\pgfpathlineto{\pgfqpoint{2.874434in}{1.661541in}}%
\pgfpathlineto{\pgfqpoint{2.874817in}{1.685987in}}%
\pgfpathlineto{\pgfqpoint{2.875586in}{1.673764in}}%
\pgfpathlineto{\pgfqpoint{2.876735in}{1.685987in}}%
\pgfpathlineto{\pgfqpoint{2.878268in}{1.657467in}}%
\pgfpathlineto{\pgfqpoint{2.879503in}{1.677838in}}%
\pgfpathlineto{\pgfqpoint{2.879886in}{1.669690in}}%
\pgfpathlineto{\pgfqpoint{2.880269in}{1.669690in}}%
\pgfpathlineto{\pgfqpoint{2.880652in}{1.665615in}}%
\pgfpathlineto{\pgfqpoint{2.881035in}{1.669690in}}%
\pgfpathlineto{\pgfqpoint{2.881808in}{1.690062in}}%
\pgfpathlineto{\pgfqpoint{2.882958in}{1.657467in}}%
\pgfpathlineto{\pgfqpoint{2.883342in}{1.681913in}}%
\pgfpathlineto{\pgfqpoint{2.884109in}{1.661541in}}%
\pgfpathlineto{\pgfqpoint{2.885259in}{1.685987in}}%
\pgfpathlineto{\pgfqpoint{2.886792in}{1.657467in}}%
\pgfpathlineto{\pgfqpoint{2.888707in}{1.694136in}}%
\pgfpathlineto{\pgfqpoint{2.890239in}{1.657467in}}%
\pgfpathlineto{\pgfqpoint{2.891772in}{1.690062in}}%
\pgfpathlineto{\pgfqpoint{2.892538in}{1.665615in}}%
\pgfpathlineto{\pgfqpoint{2.892921in}{1.681913in}}%
\pgfpathlineto{\pgfqpoint{2.894070in}{1.661541in}}%
\pgfpathlineto{\pgfqpoint{2.894453in}{1.698210in}}%
\pgfpathlineto{\pgfqpoint{2.895220in}{1.673764in}}%
\pgfpathlineto{\pgfqpoint{2.895602in}{1.677838in}}%
\pgfpathlineto{\pgfqpoint{2.895986in}{1.669690in}}%
\pgfpathlineto{\pgfqpoint{2.896368in}{1.685987in}}%
\pgfpathlineto{\pgfqpoint{2.897134in}{1.673764in}}%
\pgfpathlineto{\pgfqpoint{2.897517in}{1.661541in}}%
\pgfpathlineto{\pgfqpoint{2.897901in}{1.665615in}}%
\pgfpathlineto{\pgfqpoint{2.898284in}{1.673764in}}%
\pgfpathlineto{\pgfqpoint{2.898667in}{1.669690in}}%
\pgfpathlineto{\pgfqpoint{2.899049in}{1.665615in}}%
\pgfpathlineto{\pgfqpoint{2.900196in}{1.681913in}}%
\pgfpathlineto{\pgfqpoint{2.900581in}{1.677838in}}%
\pgfpathlineto{\pgfqpoint{2.901347in}{1.673764in}}%
\pgfpathlineto{\pgfqpoint{2.901730in}{1.677838in}}%
\pgfpathlineto{\pgfqpoint{2.902113in}{1.673764in}}%
\pgfpathlineto{\pgfqpoint{2.902495in}{1.669690in}}%
\pgfpathlineto{\pgfqpoint{2.902878in}{1.681913in}}%
\pgfpathlineto{\pgfqpoint{2.903261in}{1.657467in}}%
\pgfpathlineto{\pgfqpoint{2.904028in}{1.665615in}}%
\pgfpathlineto{\pgfqpoint{2.904410in}{1.661541in}}%
\pgfpathlineto{\pgfqpoint{2.905560in}{1.685987in}}%
\pgfpathlineto{\pgfqpoint{2.905943in}{1.649318in}}%
\pgfpathlineto{\pgfqpoint{2.906326in}{1.665615in}}%
\pgfpathlineto{\pgfqpoint{2.907093in}{1.690062in}}%
\pgfpathlineto{\pgfqpoint{2.907477in}{1.669690in}}%
\pgfpathlineto{\pgfqpoint{2.907860in}{1.669690in}}%
\pgfpathlineto{\pgfqpoint{2.908243in}{1.673764in}}%
\pgfpathlineto{\pgfqpoint{2.908626in}{1.665615in}}%
\pgfpathlineto{\pgfqpoint{2.908626in}{1.665615in}}%
\pgfpathlineto{\pgfqpoint{2.908626in}{1.665615in}}%
\pgfpathlineto{\pgfqpoint{2.909010in}{1.677838in}}%
\pgfpathlineto{\pgfqpoint{2.909778in}{1.673764in}}%
\pgfpathlineto{\pgfqpoint{2.910161in}{1.669690in}}%
\pgfpathlineto{\pgfqpoint{2.910544in}{1.677838in}}%
\pgfpathlineto{\pgfqpoint{2.911693in}{1.661541in}}%
\pgfpathlineto{\pgfqpoint{2.912078in}{1.677838in}}%
\pgfpathlineto{\pgfqpoint{2.912460in}{1.669690in}}%
\pgfpathlineto{\pgfqpoint{2.912844in}{1.657467in}}%
\pgfpathlineto{\pgfqpoint{2.913227in}{1.669690in}}%
\pgfpathlineto{\pgfqpoint{2.914378in}{1.685987in}}%
\pgfpathlineto{\pgfqpoint{2.915996in}{1.661541in}}%
\pgfpathlineto{\pgfqpoint{2.916379in}{1.677838in}}%
\pgfpathlineto{\pgfqpoint{2.917146in}{1.669690in}}%
\pgfpathlineto{\pgfqpoint{2.917530in}{1.669690in}}%
\pgfpathlineto{\pgfqpoint{2.917913in}{1.685987in}}%
\pgfpathlineto{\pgfqpoint{2.918296in}{1.677838in}}%
\pgfpathlineto{\pgfqpoint{2.918679in}{1.669690in}}%
\pgfpathlineto{\pgfqpoint{2.919446in}{1.673764in}}%
\pgfpathlineto{\pgfqpoint{2.919829in}{1.673764in}}%
\pgfpathlineto{\pgfqpoint{2.920213in}{1.685987in}}%
\pgfpathlineto{\pgfqpoint{2.920597in}{1.661541in}}%
\pgfpathlineto{\pgfqpoint{2.921364in}{1.669690in}}%
\pgfpathlineto{\pgfqpoint{2.921753in}{1.669690in}}%
\pgfpathlineto{\pgfqpoint{2.922137in}{1.677838in}}%
\pgfpathlineto{\pgfqpoint{2.922520in}{1.673764in}}%
\pgfpathlineto{\pgfqpoint{2.922903in}{1.669690in}}%
\pgfpathlineto{\pgfqpoint{2.923286in}{1.673764in}}%
\pgfpathlineto{\pgfqpoint{2.924436in}{1.690062in}}%
\pgfpathlineto{\pgfqpoint{2.925970in}{1.661541in}}%
\pgfpathlineto{\pgfqpoint{2.927502in}{1.681913in}}%
\pgfpathlineto{\pgfqpoint{2.927887in}{1.685987in}}%
\pgfpathlineto{\pgfqpoint{2.928270in}{1.681913in}}%
\pgfpathlineto{\pgfqpoint{2.929420in}{1.669690in}}%
\pgfpathlineto{\pgfqpoint{2.930570in}{1.681913in}}%
\pgfpathlineto{\pgfqpoint{2.931720in}{1.669690in}}%
\pgfpathlineto{\pgfqpoint{2.932870in}{1.677838in}}%
\pgfpathlineto{\pgfqpoint{2.933636in}{1.665615in}}%
\pgfpathlineto{\pgfqpoint{2.934019in}{1.673764in}}%
\pgfpathlineto{\pgfqpoint{2.934403in}{1.694136in}}%
\pgfpathlineto{\pgfqpoint{2.934786in}{1.653392in}}%
\pgfpathlineto{\pgfqpoint{2.935552in}{1.665615in}}%
\pgfpathlineto{\pgfqpoint{2.935936in}{1.669690in}}%
\pgfpathlineto{\pgfqpoint{2.936319in}{1.685987in}}%
\pgfpathlineto{\pgfqpoint{2.936702in}{1.673764in}}%
\pgfpathlineto{\pgfqpoint{2.937085in}{1.657467in}}%
\pgfpathlineto{\pgfqpoint{2.937468in}{1.690062in}}%
\pgfpathlineto{\pgfqpoint{2.938234in}{1.685987in}}%
\pgfpathlineto{\pgfqpoint{2.938618in}{1.681913in}}%
\pgfpathlineto{\pgfqpoint{2.939002in}{1.665615in}}%
\pgfpathlineto{\pgfqpoint{2.939769in}{1.677838in}}%
\pgfpathlineto{\pgfqpoint{2.940535in}{1.677838in}}%
\pgfpathlineto{\pgfqpoint{2.940919in}{1.665615in}}%
\pgfpathlineto{\pgfqpoint{2.941303in}{1.677838in}}%
\pgfpathlineto{\pgfqpoint{2.941686in}{1.677838in}}%
\pgfpathlineto{\pgfqpoint{2.942453in}{1.673764in}}%
\pgfpathlineto{\pgfqpoint{2.942836in}{1.694136in}}%
\pgfpathlineto{\pgfqpoint{2.943219in}{1.677838in}}%
\pgfpathlineto{\pgfqpoint{2.943602in}{1.673764in}}%
\pgfpathlineto{\pgfqpoint{2.943986in}{1.677838in}}%
\pgfpathlineto{\pgfqpoint{2.945136in}{1.694136in}}%
\pgfpathlineto{\pgfqpoint{2.946285in}{1.673764in}}%
\pgfpathlineto{\pgfqpoint{2.947052in}{1.690062in}}%
\pgfpathlineto{\pgfqpoint{2.947818in}{1.653392in}}%
\pgfpathlineto{\pgfqpoint{2.948202in}{1.677838in}}%
\pgfpathlineto{\pgfqpoint{2.948585in}{1.669690in}}%
\pgfpathlineto{\pgfqpoint{2.948585in}{1.669690in}}%
\pgfpathlineto{\pgfqpoint{2.948585in}{1.669690in}}%
\pgfpathlineto{\pgfqpoint{2.949735in}{1.694136in}}%
\pgfpathlineto{\pgfqpoint{2.950502in}{1.661541in}}%
\pgfpathlineto{\pgfqpoint{2.950885in}{1.669690in}}%
\pgfpathlineto{\pgfqpoint{2.951269in}{1.677838in}}%
\pgfpathlineto{\pgfqpoint{2.951652in}{1.657467in}}%
\pgfpathlineto{\pgfqpoint{2.951652in}{1.657467in}}%
\pgfpathlineto{\pgfqpoint{2.951652in}{1.657467in}}%
\pgfpathlineto{\pgfqpoint{2.952036in}{1.681913in}}%
\pgfpathlineto{\pgfqpoint{2.952802in}{1.665615in}}%
\pgfpathlineto{\pgfqpoint{2.953186in}{1.669690in}}%
\pgfpathlineto{\pgfqpoint{2.953570in}{1.657467in}}%
\pgfpathlineto{\pgfqpoint{2.953570in}{1.657467in}}%
\pgfpathlineto{\pgfqpoint{2.953570in}{1.657467in}}%
\pgfpathlineto{\pgfqpoint{2.954720in}{1.677838in}}%
\pgfpathlineto{\pgfqpoint{2.955871in}{1.669690in}}%
\pgfpathlineto{\pgfqpoint{2.956638in}{1.669690in}}%
\pgfpathlineto{\pgfqpoint{2.957022in}{1.694136in}}%
\pgfpathlineto{\pgfqpoint{2.957788in}{1.685987in}}%
\pgfpathlineto{\pgfqpoint{2.958172in}{1.677838in}}%
\pgfpathlineto{\pgfqpoint{2.958555in}{1.681913in}}%
\pgfpathlineto{\pgfqpoint{2.958937in}{1.685987in}}%
\pgfpathlineto{\pgfqpoint{2.959321in}{1.702285in}}%
\pgfpathlineto{\pgfqpoint{2.959706in}{1.685987in}}%
\pgfpathlineto{\pgfqpoint{2.961323in}{1.669690in}}%
\pgfpathlineto{\pgfqpoint{2.961713in}{1.661541in}}%
\pgfpathlineto{\pgfqpoint{2.963247in}{1.690062in}}%
\pgfpathlineto{\pgfqpoint{2.963631in}{1.694136in}}%
\pgfpathlineto{\pgfqpoint{2.964780in}{1.657467in}}%
\pgfpathlineto{\pgfqpoint{2.965930in}{1.685987in}}%
\pgfpathlineto{\pgfqpoint{2.966314in}{1.677838in}}%
\pgfpathlineto{\pgfqpoint{2.967080in}{1.681913in}}%
\pgfpathlineto{\pgfqpoint{2.967464in}{1.681913in}}%
\pgfpathlineto{\pgfqpoint{2.967847in}{1.673764in}}%
\pgfpathlineto{\pgfqpoint{2.968614in}{1.677838in}}%
\pgfpathlineto{\pgfqpoint{2.968997in}{1.681913in}}%
\pgfpathlineto{\pgfqpoint{2.970530in}{1.669690in}}%
\pgfpathlineto{\pgfqpoint{2.970914in}{1.665615in}}%
\pgfpathlineto{\pgfqpoint{2.972063in}{1.685987in}}%
\pgfpathlineto{\pgfqpoint{2.972446in}{1.677838in}}%
\pgfpathlineto{\pgfqpoint{2.973597in}{1.657467in}}%
\pgfpathlineto{\pgfqpoint{2.973980in}{1.681913in}}%
\pgfpathlineto{\pgfqpoint{2.974746in}{1.661541in}}%
\pgfpathlineto{\pgfqpoint{2.975129in}{1.669690in}}%
\pgfpathlineto{\pgfqpoint{2.975513in}{1.645243in}}%
\pgfpathlineto{\pgfqpoint{2.975513in}{1.645243in}}%
\pgfpathlineto{\pgfqpoint{2.975513in}{1.645243in}}%
\pgfpathlineto{\pgfqpoint{2.976663in}{1.685987in}}%
\pgfpathlineto{\pgfqpoint{2.977811in}{1.653392in}}%
\pgfpathlineto{\pgfqpoint{2.978961in}{1.681913in}}%
\pgfpathlineto{\pgfqpoint{2.980110in}{1.669690in}}%
\pgfpathlineto{\pgfqpoint{2.981261in}{1.685987in}}%
\pgfpathlineto{\pgfqpoint{2.981643in}{1.665615in}}%
\pgfpathlineto{\pgfqpoint{2.982027in}{1.669690in}}%
\pgfpathlineto{\pgfqpoint{2.982793in}{1.694136in}}%
\pgfpathlineto{\pgfqpoint{2.983942in}{1.657467in}}%
\pgfpathlineto{\pgfqpoint{2.984710in}{1.694136in}}%
\pgfpathlineto{\pgfqpoint{2.985093in}{1.690062in}}%
\pgfpathlineto{\pgfqpoint{2.985859in}{1.690062in}}%
\pgfpathlineto{\pgfqpoint{2.987394in}{1.657467in}}%
\pgfpathlineto{\pgfqpoint{2.987777in}{1.669690in}}%
\pgfpathlineto{\pgfqpoint{2.987777in}{1.669690in}}%
\pgfpathlineto{\pgfqpoint{2.987777in}{1.669690in}}%
\pgfpathlineto{\pgfqpoint{2.988160in}{1.653392in}}%
\pgfpathlineto{\pgfqpoint{2.988160in}{1.653392in}}%
\pgfpathlineto{\pgfqpoint{2.988160in}{1.653392in}}%
\pgfpathlineto{\pgfqpoint{2.988543in}{1.677838in}}%
\pgfpathlineto{\pgfqpoint{2.989309in}{1.669690in}}%
\pgfpathlineto{\pgfqpoint{2.990842in}{1.690062in}}%
\pgfpathlineto{\pgfqpoint{2.991225in}{1.665615in}}%
\pgfpathlineto{\pgfqpoint{2.991992in}{1.673764in}}%
\pgfpathlineto{\pgfqpoint{2.992375in}{1.665615in}}%
\pgfpathlineto{\pgfqpoint{2.992375in}{1.665615in}}%
\pgfpathlineto{\pgfqpoint{2.992375in}{1.665615in}}%
\pgfpathlineto{\pgfqpoint{2.993138in}{1.677838in}}%
\pgfpathlineto{\pgfqpoint{2.994286in}{1.665615in}}%
\pgfpathlineto{\pgfqpoint{2.995817in}{1.681913in}}%
\pgfpathlineto{\pgfqpoint{2.996583in}{1.669690in}}%
\pgfpathlineto{\pgfqpoint{2.996965in}{1.698210in}}%
\pgfpathlineto{\pgfqpoint{2.997433in}{1.669690in}}%
\pgfpathlineto{\pgfqpoint{2.997815in}{1.657467in}}%
\pgfpathlineto{\pgfqpoint{2.998962in}{1.681913in}}%
\pgfpathlineto{\pgfqpoint{2.999345in}{1.681913in}}%
\pgfpathlineto{\pgfqpoint{3.000493in}{1.657467in}}%
\pgfpathlineto{\pgfqpoint{3.002033in}{1.694136in}}%
\pgfpathlineto{\pgfqpoint{3.002415in}{1.661541in}}%
\pgfpathlineto{\pgfqpoint{3.003180in}{1.677838in}}%
\pgfpathlineto{\pgfqpoint{3.003563in}{1.669690in}}%
\pgfpathlineto{\pgfqpoint{3.003945in}{1.690062in}}%
\pgfpathlineto{\pgfqpoint{3.003945in}{1.690062in}}%
\pgfpathlineto{\pgfqpoint{3.003945in}{1.690062in}}%
\pgfpathlineto{\pgfqpoint{3.004327in}{1.665615in}}%
\pgfpathlineto{\pgfqpoint{3.005092in}{1.677838in}}%
\pgfpathlineto{\pgfqpoint{3.005474in}{1.698210in}}%
\pgfpathlineto{\pgfqpoint{3.005474in}{1.698210in}}%
\pgfpathlineto{\pgfqpoint{3.005474in}{1.698210in}}%
\pgfpathlineto{\pgfqpoint{3.005858in}{1.661541in}}%
\pgfpathlineto{\pgfqpoint{3.006623in}{1.681913in}}%
\pgfpathlineto{\pgfqpoint{3.007388in}{1.653392in}}%
\pgfpathlineto{\pgfqpoint{3.007770in}{1.677838in}}%
\pgfpathlineto{\pgfqpoint{3.008154in}{1.677838in}}%
\pgfpathlineto{\pgfqpoint{3.008537in}{1.657467in}}%
\pgfpathlineto{\pgfqpoint{3.008537in}{1.657467in}}%
\pgfpathlineto{\pgfqpoint{3.008537in}{1.657467in}}%
\pgfpathlineto{\pgfqpoint{3.009302in}{1.694136in}}%
\pgfpathlineto{\pgfqpoint{3.009684in}{1.645243in}}%
\pgfpathlineto{\pgfqpoint{3.010450in}{1.661541in}}%
\pgfpathlineto{\pgfqpoint{3.010833in}{1.685987in}}%
\pgfpathlineto{\pgfqpoint{3.011216in}{1.673764in}}%
\pgfpathlineto{\pgfqpoint{3.011600in}{1.657467in}}%
\pgfpathlineto{\pgfqpoint{3.011984in}{1.694136in}}%
\pgfpathlineto{\pgfqpoint{3.012750in}{1.661541in}}%
\pgfpathlineto{\pgfqpoint{3.013515in}{1.665615in}}%
\pgfpathlineto{\pgfqpoint{3.013898in}{1.661541in}}%
\pgfpathlineto{\pgfqpoint{3.014281in}{1.685987in}}%
\pgfpathlineto{\pgfqpoint{3.015049in}{1.681913in}}%
\pgfpathlineto{\pgfqpoint{3.015432in}{1.665615in}}%
\pgfpathlineto{\pgfqpoint{3.016197in}{1.673764in}}%
\pgfpathlineto{\pgfqpoint{3.016580in}{1.677838in}}%
\pgfpathlineto{\pgfqpoint{3.016963in}{1.657467in}}%
\pgfpathlineto{\pgfqpoint{3.016963in}{1.657467in}}%
\pgfpathlineto{\pgfqpoint{3.016963in}{1.657467in}}%
\pgfpathlineto{\pgfqpoint{3.017347in}{1.690062in}}%
\pgfpathlineto{\pgfqpoint{3.018113in}{1.661541in}}%
\pgfpathlineto{\pgfqpoint{3.019262in}{1.681913in}}%
\pgfpathlineto{\pgfqpoint{3.019645in}{1.673764in}}%
\pgfpathlineto{\pgfqpoint{3.020796in}{1.665615in}}%
\pgfpathlineto{\pgfqpoint{3.021945in}{1.681913in}}%
\pgfpathlineto{\pgfqpoint{3.023477in}{1.657467in}}%
\pgfpathlineto{\pgfqpoint{3.024626in}{1.673764in}}%
\pgfpathlineto{\pgfqpoint{3.025009in}{1.669690in}}%
\pgfpathlineto{\pgfqpoint{3.026627in}{1.681913in}}%
\pgfpathlineto{\pgfqpoint{3.027010in}{1.661541in}}%
\pgfpathlineto{\pgfqpoint{3.027776in}{1.669690in}}%
\pgfpathlineto{\pgfqpoint{3.028543in}{1.657467in}}%
\pgfpathlineto{\pgfqpoint{3.029310in}{1.685987in}}%
\pgfpathlineto{\pgfqpoint{3.029694in}{1.653392in}}%
\pgfpathlineto{\pgfqpoint{3.030459in}{1.665615in}}%
\pgfpathlineto{\pgfqpoint{3.030843in}{1.685987in}}%
\pgfpathlineto{\pgfqpoint{3.030843in}{1.685987in}}%
\pgfpathlineto{\pgfqpoint{3.030843in}{1.685987in}}%
\pgfpathlineto{\pgfqpoint{3.031226in}{1.653392in}}%
\pgfpathlineto{\pgfqpoint{3.031993in}{1.673764in}}%
\pgfpathlineto{\pgfqpoint{3.032377in}{1.657467in}}%
\pgfpathlineto{\pgfqpoint{3.032377in}{1.657467in}}%
\pgfpathlineto{\pgfqpoint{3.032377in}{1.657467in}}%
\pgfpathlineto{\pgfqpoint{3.033909in}{1.685987in}}%
\pgfpathlineto{\pgfqpoint{3.034293in}{1.665615in}}%
\pgfpathlineto{\pgfqpoint{3.035060in}{1.673764in}}%
\pgfpathlineto{\pgfqpoint{3.035443in}{1.685987in}}%
\pgfpathlineto{\pgfqpoint{3.035826in}{1.677838in}}%
\pgfpathlineto{\pgfqpoint{3.036593in}{1.673764in}}%
\pgfpathlineto{\pgfqpoint{3.036976in}{1.685987in}}%
\pgfpathlineto{\pgfqpoint{3.036976in}{1.685987in}}%
\pgfpathlineto{\pgfqpoint{3.036976in}{1.685987in}}%
\pgfpathlineto{\pgfqpoint{3.038127in}{1.669690in}}%
\pgfpathlineto{\pgfqpoint{3.039658in}{1.698210in}}%
\pgfpathlineto{\pgfqpoint{3.040041in}{1.665615in}}%
\pgfpathlineto{\pgfqpoint{3.040808in}{1.681913in}}%
\pgfpathlineto{\pgfqpoint{3.041965in}{1.673764in}}%
\pgfpathlineto{\pgfqpoint{3.042350in}{1.685987in}}%
\pgfpathlineto{\pgfqpoint{3.042350in}{1.685987in}}%
\pgfpathlineto{\pgfqpoint{3.042350in}{1.685987in}}%
\pgfpathlineto{\pgfqpoint{3.042733in}{1.669690in}}%
\pgfpathlineto{\pgfqpoint{3.042733in}{1.669690in}}%
\pgfpathlineto{\pgfqpoint{3.042733in}{1.669690in}}%
\pgfpathlineto{\pgfqpoint{3.043499in}{1.690062in}}%
\pgfpathlineto{\pgfqpoint{3.044266in}{1.669690in}}%
\pgfpathlineto{\pgfqpoint{3.044649in}{1.685987in}}%
\pgfpathlineto{\pgfqpoint{3.045033in}{1.685987in}}%
\pgfpathlineto{\pgfqpoint{3.045417in}{1.653392in}}%
\pgfpathlineto{\pgfqpoint{3.046183in}{1.669690in}}%
\pgfpathlineto{\pgfqpoint{3.047332in}{1.665615in}}%
\pgfpathlineto{\pgfqpoint{3.047717in}{1.685987in}}%
\pgfpathlineto{\pgfqpoint{3.048482in}{1.673764in}}%
\pgfpathlineto{\pgfqpoint{3.048866in}{1.673764in}}%
\pgfpathlineto{\pgfqpoint{3.049631in}{1.677838in}}%
\pgfpathlineto{\pgfqpoint{3.050398in}{1.661541in}}%
\pgfpathlineto{\pgfqpoint{3.050781in}{1.669690in}}%
\pgfpathlineto{\pgfqpoint{3.051165in}{1.681913in}}%
\pgfpathlineto{\pgfqpoint{3.051165in}{1.681913in}}%
\pgfpathlineto{\pgfqpoint{3.051165in}{1.681913in}}%
\pgfpathlineto{\pgfqpoint{3.051549in}{1.665615in}}%
\pgfpathlineto{\pgfqpoint{3.052315in}{1.677838in}}%
\pgfpathlineto{\pgfqpoint{3.052698in}{1.690062in}}%
\pgfpathlineto{\pgfqpoint{3.053081in}{1.685987in}}%
\pgfpathlineto{\pgfqpoint{3.053464in}{1.669690in}}%
\pgfpathlineto{\pgfqpoint{3.054231in}{1.681913in}}%
\pgfpathlineto{\pgfqpoint{3.056529in}{1.665615in}}%
\pgfpathlineto{\pgfqpoint{3.056911in}{1.669690in}}%
\pgfpathlineto{\pgfqpoint{3.057294in}{1.657467in}}%
\pgfpathlineto{\pgfqpoint{3.057677in}{1.669690in}}%
\pgfpathlineto{\pgfqpoint{3.058443in}{1.685987in}}%
\pgfpathlineto{\pgfqpoint{3.058827in}{1.677838in}}%
\pgfpathlineto{\pgfqpoint{3.059209in}{1.661541in}}%
\pgfpathlineto{\pgfqpoint{3.059592in}{1.669690in}}%
\pgfpathlineto{\pgfqpoint{3.059975in}{1.681913in}}%
\pgfpathlineto{\pgfqpoint{3.059975in}{1.681913in}}%
\pgfpathlineto{\pgfqpoint{3.059975in}{1.681913in}}%
\pgfpathlineto{\pgfqpoint{3.060742in}{1.649318in}}%
\pgfpathlineto{\pgfqpoint{3.061509in}{1.681913in}}%
\pgfpathlineto{\pgfqpoint{3.061892in}{1.673764in}}%
\pgfpathlineto{\pgfqpoint{3.063042in}{1.665615in}}%
\pgfpathlineto{\pgfqpoint{3.064192in}{1.685987in}}%
\pgfpathlineto{\pgfqpoint{3.065811in}{1.669690in}}%
\pgfpathlineto{\pgfqpoint{3.066960in}{1.685987in}}%
\pgfpathlineto{\pgfqpoint{3.067726in}{1.661541in}}%
\pgfpathlineto{\pgfqpoint{3.068110in}{1.685987in}}%
\pgfpathlineto{\pgfqpoint{3.068875in}{1.677838in}}%
\pgfpathlineto{\pgfqpoint{3.069642in}{1.673764in}}%
\pgfpathlineto{\pgfqpoint{3.070407in}{1.685987in}}%
\pgfpathlineto{\pgfqpoint{3.070791in}{1.677838in}}%
\pgfpathlineto{\pgfqpoint{3.071940in}{1.657467in}}%
\pgfpathlineto{\pgfqpoint{3.073473in}{1.698210in}}%
\pgfpathlineto{\pgfqpoint{3.074623in}{1.665615in}}%
\pgfpathlineto{\pgfqpoint{3.076156in}{1.690062in}}%
\pgfpathlineto{\pgfqpoint{3.076540in}{1.685987in}}%
\pgfpathlineto{\pgfqpoint{3.078070in}{1.657467in}}%
\pgfpathlineto{\pgfqpoint{3.078453in}{1.681913in}}%
\pgfpathlineto{\pgfqpoint{3.079219in}{1.677838in}}%
\pgfpathlineto{\pgfqpoint{3.079602in}{1.657467in}}%
\pgfpathlineto{\pgfqpoint{3.079602in}{1.657467in}}%
\pgfpathlineto{\pgfqpoint{3.079602in}{1.657467in}}%
\pgfpathlineto{\pgfqpoint{3.080751in}{1.690062in}}%
\pgfpathlineto{\pgfqpoint{3.081133in}{1.661541in}}%
\pgfpathlineto{\pgfqpoint{3.081524in}{1.669690in}}%
\pgfpathlineto{\pgfqpoint{3.082290in}{1.694136in}}%
\pgfpathlineto{\pgfqpoint{3.082673in}{1.677838in}}%
\pgfpathlineto{\pgfqpoint{3.083440in}{1.665615in}}%
\pgfpathlineto{\pgfqpoint{3.084972in}{1.685987in}}%
\pgfpathlineto{\pgfqpoint{3.086503in}{1.665615in}}%
\pgfpathlineto{\pgfqpoint{3.086886in}{1.677838in}}%
\pgfpathlineto{\pgfqpoint{3.087270in}{1.669690in}}%
\pgfpathlineto{\pgfqpoint{3.088037in}{1.653392in}}%
\pgfpathlineto{\pgfqpoint{3.088420in}{1.665615in}}%
\pgfpathlineto{\pgfqpoint{3.089953in}{1.685987in}}%
\pgfpathlineto{\pgfqpoint{3.090336in}{1.681913in}}%
\pgfpathlineto{\pgfqpoint{3.091103in}{1.653392in}}%
\pgfpathlineto{\pgfqpoint{3.091869in}{1.669690in}}%
\pgfpathlineto{\pgfqpoint{3.092636in}{1.673764in}}%
\pgfpathlineto{\pgfqpoint{3.093019in}{1.665615in}}%
\pgfpathlineto{\pgfqpoint{3.093784in}{1.669690in}}%
\pgfpathlineto{\pgfqpoint{3.094167in}{1.669690in}}%
\pgfpathlineto{\pgfqpoint{3.095318in}{1.685987in}}%
\pgfpathlineto{\pgfqpoint{3.096085in}{1.669690in}}%
\pgfpathlineto{\pgfqpoint{3.096468in}{1.673764in}}%
\pgfpathlineto{\pgfqpoint{3.097616in}{1.685987in}}%
\pgfpathlineto{\pgfqpoint{3.099148in}{1.653392in}}%
\pgfpathlineto{\pgfqpoint{3.099617in}{1.681913in}}%
\pgfpathlineto{\pgfqpoint{3.100384in}{1.669690in}}%
\pgfpathlineto{\pgfqpoint{3.101152in}{1.677838in}}%
\pgfpathlineto{\pgfqpoint{3.101535in}{1.653392in}}%
\pgfpathlineto{\pgfqpoint{3.101535in}{1.653392in}}%
\pgfpathlineto{\pgfqpoint{3.101535in}{1.653392in}}%
\pgfpathlineto{\pgfqpoint{3.102302in}{1.685987in}}%
\pgfpathlineto{\pgfqpoint{3.102685in}{1.673764in}}%
\pgfpathlineto{\pgfqpoint{3.103068in}{1.673764in}}%
\pgfpathlineto{\pgfqpoint{3.103451in}{1.677838in}}%
\pgfpathlineto{\pgfqpoint{3.103834in}{1.669690in}}%
\pgfpathlineto{\pgfqpoint{3.104218in}{1.694136in}}%
\pgfpathlineto{\pgfqpoint{3.104218in}{1.694136in}}%
\pgfpathlineto{\pgfqpoint{3.104218in}{1.694136in}}%
\pgfpathlineto{\pgfqpoint{3.104602in}{1.657467in}}%
\pgfpathlineto{\pgfqpoint{3.105367in}{1.673764in}}%
\pgfpathlineto{\pgfqpoint{3.105751in}{1.669690in}}%
\pgfpathlineto{\pgfqpoint{3.106134in}{1.685987in}}%
\pgfpathlineto{\pgfqpoint{3.106134in}{1.685987in}}%
\pgfpathlineto{\pgfqpoint{3.106134in}{1.685987in}}%
\pgfpathlineto{\pgfqpoint{3.106901in}{1.661541in}}%
\pgfpathlineto{\pgfqpoint{3.108052in}{1.690062in}}%
\pgfpathlineto{\pgfqpoint{3.109584in}{1.673764in}}%
\pgfpathlineto{\pgfqpoint{3.109968in}{1.673764in}}%
\pgfpathlineto{\pgfqpoint{3.110352in}{1.694136in}}%
\pgfpathlineto{\pgfqpoint{3.110735in}{1.690062in}}%
\pgfpathlineto{\pgfqpoint{3.111119in}{1.657467in}}%
\pgfpathlineto{\pgfqpoint{3.111884in}{1.669690in}}%
\pgfpathlineto{\pgfqpoint{3.112268in}{1.690062in}}%
\pgfpathlineto{\pgfqpoint{3.112268in}{1.690062in}}%
\pgfpathlineto{\pgfqpoint{3.112268in}{1.690062in}}%
\pgfpathlineto{\pgfqpoint{3.112652in}{1.665615in}}%
\pgfpathlineto{\pgfqpoint{3.113417in}{1.673764in}}%
\pgfpathlineto{\pgfqpoint{3.113801in}{1.694136in}}%
\pgfpathlineto{\pgfqpoint{3.114568in}{1.681913in}}%
\pgfpathlineto{\pgfqpoint{3.116102in}{1.665615in}}%
\pgfpathlineto{\pgfqpoint{3.116485in}{1.694136in}}%
\pgfpathlineto{\pgfqpoint{3.117252in}{1.673764in}}%
\pgfpathlineto{\pgfqpoint{3.117636in}{1.677838in}}%
\pgfpathlineto{\pgfqpoint{3.118402in}{1.661541in}}%
\pgfpathlineto{\pgfqpoint{3.118785in}{1.673764in}}%
\pgfpathlineto{\pgfqpoint{3.119551in}{1.685987in}}%
\pgfpathlineto{\pgfqpoint{3.119934in}{1.657467in}}%
\pgfpathlineto{\pgfqpoint{3.119934in}{1.657467in}}%
\pgfpathlineto{\pgfqpoint{3.119934in}{1.657467in}}%
\pgfpathlineto{\pgfqpoint{3.120317in}{1.690062in}}%
\pgfpathlineto{\pgfqpoint{3.120701in}{1.673764in}}%
\pgfpathlineto{\pgfqpoint{3.121084in}{1.649318in}}%
\pgfpathlineto{\pgfqpoint{3.121475in}{1.665615in}}%
\pgfpathlineto{\pgfqpoint{3.121858in}{1.690062in}}%
\pgfpathlineto{\pgfqpoint{3.122624in}{1.669690in}}%
\pgfpathlineto{\pgfqpoint{3.123391in}{1.657467in}}%
\pgfpathlineto{\pgfqpoint{3.124924in}{1.685987in}}%
\pgfpathlineto{\pgfqpoint{3.125690in}{1.661541in}}%
\pgfpathlineto{\pgfqpoint{3.126074in}{1.665615in}}%
\pgfpathlineto{\pgfqpoint{3.126457in}{1.681913in}}%
\pgfpathlineto{\pgfqpoint{3.126840in}{1.673764in}}%
\pgfpathlineto{\pgfqpoint{3.127223in}{1.657467in}}%
\pgfpathlineto{\pgfqpoint{3.127606in}{1.694136in}}%
\pgfpathlineto{\pgfqpoint{3.128374in}{1.685987in}}%
\pgfpathlineto{\pgfqpoint{3.129140in}{1.669690in}}%
\pgfpathlineto{\pgfqpoint{3.129523in}{1.685987in}}%
\pgfpathlineto{\pgfqpoint{3.130289in}{1.673764in}}%
\pgfpathlineto{\pgfqpoint{3.130673in}{1.665615in}}%
\pgfpathlineto{\pgfqpoint{3.131056in}{1.681913in}}%
\pgfpathlineto{\pgfqpoint{3.131056in}{1.681913in}}%
\pgfpathlineto{\pgfqpoint{3.131056in}{1.681913in}}%
\pgfpathlineto{\pgfqpoint{3.131439in}{1.653392in}}%
\pgfpathlineto{\pgfqpoint{3.132205in}{1.673764in}}%
\pgfpathlineto{\pgfqpoint{3.132585in}{1.677838in}}%
\pgfpathlineto{\pgfqpoint{3.132969in}{1.645243in}}%
\pgfpathlineto{\pgfqpoint{3.133735in}{1.673764in}}%
\pgfpathlineto{\pgfqpoint{3.134119in}{1.661541in}}%
\pgfpathlineto{\pgfqpoint{3.134119in}{1.661541in}}%
\pgfpathlineto{\pgfqpoint{3.134119in}{1.661541in}}%
\pgfpathlineto{\pgfqpoint{3.134502in}{1.681913in}}%
\pgfpathlineto{\pgfqpoint{3.135269in}{1.673764in}}%
\pgfpathlineto{\pgfqpoint{3.135652in}{1.665615in}}%
\pgfpathlineto{\pgfqpoint{3.136803in}{1.690062in}}%
\pgfpathlineto{\pgfqpoint{3.137569in}{1.677838in}}%
\pgfpathlineto{\pgfqpoint{3.137951in}{1.694136in}}%
\pgfpathlineto{\pgfqpoint{3.137951in}{1.694136in}}%
\pgfpathlineto{\pgfqpoint{3.137951in}{1.694136in}}%
\pgfpathlineto{\pgfqpoint{3.139484in}{1.661541in}}%
\pgfpathlineto{\pgfqpoint{3.140719in}{1.685987in}}%
\pgfpathlineto{\pgfqpoint{3.141102in}{1.677838in}}%
\pgfpathlineto{\pgfqpoint{3.141868in}{1.681913in}}%
\pgfpathlineto{\pgfqpoint{3.142635in}{1.669690in}}%
\pgfpathlineto{\pgfqpoint{3.143019in}{1.681913in}}%
\pgfpathlineto{\pgfqpoint{3.143402in}{1.677838in}}%
\pgfpathlineto{\pgfqpoint{3.143785in}{1.657467in}}%
\pgfpathlineto{\pgfqpoint{3.144551in}{1.669690in}}%
\pgfpathlineto{\pgfqpoint{3.144934in}{1.706359in}}%
\pgfpathlineto{\pgfqpoint{3.144934in}{1.706359in}}%
\pgfpathlineto{\pgfqpoint{3.144934in}{1.706359in}}%
\pgfpathlineto{\pgfqpoint{3.146083in}{1.657467in}}%
\pgfpathlineto{\pgfqpoint{3.147615in}{1.685987in}}%
\pgfpathlineto{\pgfqpoint{3.147999in}{1.653392in}}%
\pgfpathlineto{\pgfqpoint{3.148766in}{1.657467in}}%
\pgfpathlineto{\pgfqpoint{3.149916in}{1.677838in}}%
\pgfpathlineto{\pgfqpoint{3.150299in}{1.669690in}}%
\pgfpathlineto{\pgfqpoint{3.150683in}{1.661541in}}%
\pgfpathlineto{\pgfqpoint{3.151834in}{1.677838in}}%
\pgfpathlineto{\pgfqpoint{3.152217in}{1.653392in}}%
\pgfpathlineto{\pgfqpoint{3.152217in}{1.653392in}}%
\pgfpathlineto{\pgfqpoint{3.152217in}{1.653392in}}%
\pgfpathlineto{\pgfqpoint{3.153750in}{1.690062in}}%
\pgfpathlineto{\pgfqpoint{3.154518in}{1.669690in}}%
\pgfpathlineto{\pgfqpoint{3.154901in}{1.694136in}}%
\pgfpathlineto{\pgfqpoint{3.155668in}{1.673764in}}%
\pgfpathlineto{\pgfqpoint{3.156051in}{1.661541in}}%
\pgfpathlineto{\pgfqpoint{3.156818in}{1.669690in}}%
\pgfpathlineto{\pgfqpoint{3.157201in}{1.677838in}}%
\pgfpathlineto{\pgfqpoint{3.157584in}{1.661541in}}%
\pgfpathlineto{\pgfqpoint{3.157584in}{1.661541in}}%
\pgfpathlineto{\pgfqpoint{3.157584in}{1.661541in}}%
\pgfpathlineto{\pgfqpoint{3.157967in}{1.690062in}}%
\pgfpathlineto{\pgfqpoint{3.158350in}{1.677838in}}%
\pgfpathlineto{\pgfqpoint{3.158733in}{1.657467in}}%
\pgfpathlineto{\pgfqpoint{3.159500in}{1.673764in}}%
\pgfpathlineto{\pgfqpoint{3.159883in}{1.677838in}}%
\pgfpathlineto{\pgfqpoint{3.160649in}{1.665615in}}%
\pgfpathlineto{\pgfqpoint{3.161032in}{1.685987in}}%
\pgfpathlineto{\pgfqpoint{3.161807in}{1.673764in}}%
\pgfpathlineto{\pgfqpoint{3.162190in}{1.665615in}}%
\pgfpathlineto{\pgfqpoint{3.162574in}{1.685987in}}%
\pgfpathlineto{\pgfqpoint{3.163341in}{1.677838in}}%
\pgfpathlineto{\pgfqpoint{3.163724in}{1.673764in}}%
\pgfpathlineto{\pgfqpoint{3.164108in}{1.685987in}}%
\pgfpathlineto{\pgfqpoint{3.164491in}{1.673764in}}%
\pgfpathlineto{\pgfqpoint{3.164874in}{1.665615in}}%
\pgfpathlineto{\pgfqpoint{3.164874in}{1.665615in}}%
\pgfpathlineto{\pgfqpoint{3.164874in}{1.665615in}}%
\pgfpathlineto{\pgfqpoint{3.166023in}{1.685987in}}%
\pgfpathlineto{\pgfqpoint{3.167174in}{1.661541in}}%
\pgfpathlineto{\pgfqpoint{3.167939in}{1.681913in}}%
\pgfpathlineto{\pgfqpoint{3.168322in}{1.661541in}}%
\pgfpathlineto{\pgfqpoint{3.168322in}{1.661541in}}%
\pgfpathlineto{\pgfqpoint{3.168322in}{1.661541in}}%
\pgfpathlineto{\pgfqpoint{3.169473in}{1.685987in}}%
\pgfpathlineto{\pgfqpoint{3.170623in}{1.665615in}}%
\pgfpathlineto{\pgfqpoint{3.171005in}{1.685987in}}%
\pgfpathlineto{\pgfqpoint{3.171389in}{1.673764in}}%
\pgfpathlineto{\pgfqpoint{3.171773in}{1.657467in}}%
\pgfpathlineto{\pgfqpoint{3.171773in}{1.657467in}}%
\pgfpathlineto{\pgfqpoint{3.171773in}{1.657467in}}%
\pgfpathlineto{\pgfqpoint{3.172923in}{1.677838in}}%
\pgfpathlineto{\pgfqpoint{3.173306in}{1.661541in}}%
\pgfpathlineto{\pgfqpoint{3.174073in}{1.665615in}}%
\pgfpathlineto{\pgfqpoint{3.175223in}{1.685987in}}%
\pgfpathlineto{\pgfqpoint{3.175988in}{1.665615in}}%
\pgfpathlineto{\pgfqpoint{3.176755in}{1.669690in}}%
\pgfpathlineto{\pgfqpoint{3.177139in}{1.681913in}}%
\pgfpathlineto{\pgfqpoint{3.177522in}{1.649318in}}%
\pgfpathlineto{\pgfqpoint{3.177905in}{1.653392in}}%
\pgfpathlineto{\pgfqpoint{3.178289in}{1.694136in}}%
\pgfpathlineto{\pgfqpoint{3.178672in}{1.681913in}}%
\pgfpathlineto{\pgfqpoint{3.179055in}{1.657467in}}%
\pgfpathlineto{\pgfqpoint{3.179055in}{1.657467in}}%
\pgfpathlineto{\pgfqpoint{3.179055in}{1.657467in}}%
\pgfpathlineto{\pgfqpoint{3.179438in}{1.694136in}}%
\pgfpathlineto{\pgfqpoint{3.180205in}{1.669690in}}%
\pgfpathlineto{\pgfqpoint{3.180588in}{1.681913in}}%
\pgfpathlineto{\pgfqpoint{3.180588in}{1.681913in}}%
\pgfpathlineto{\pgfqpoint{3.180588in}{1.681913in}}%
\pgfpathlineto{\pgfqpoint{3.180972in}{1.661541in}}%
\pgfpathlineto{\pgfqpoint{3.180972in}{1.661541in}}%
\pgfpathlineto{\pgfqpoint{3.180972in}{1.661541in}}%
\pgfpathlineto{\pgfqpoint{3.181355in}{1.690062in}}%
\pgfpathlineto{\pgfqpoint{3.181823in}{1.677838in}}%
\pgfpathlineto{\pgfqpoint{3.182589in}{1.665615in}}%
\pgfpathlineto{\pgfqpoint{3.182973in}{1.673764in}}%
\pgfpathlineto{\pgfqpoint{3.183741in}{1.665615in}}%
\pgfpathlineto{\pgfqpoint{3.184124in}{1.681913in}}%
\pgfpathlineto{\pgfqpoint{3.184890in}{1.677838in}}%
\pgfpathlineto{\pgfqpoint{3.186039in}{1.661541in}}%
\pgfpathlineto{\pgfqpoint{3.186805in}{1.681913in}}%
\pgfpathlineto{\pgfqpoint{3.187188in}{1.661541in}}%
\pgfpathlineto{\pgfqpoint{3.187571in}{1.665615in}}%
\pgfpathlineto{\pgfqpoint{3.187955in}{1.681913in}}%
\pgfpathlineto{\pgfqpoint{3.188339in}{1.677838in}}%
\pgfpathlineto{\pgfqpoint{3.188722in}{1.665615in}}%
\pgfpathlineto{\pgfqpoint{3.188722in}{1.665615in}}%
\pgfpathlineto{\pgfqpoint{3.188722in}{1.665615in}}%
\pgfpathlineto{\pgfqpoint{3.189879in}{1.690062in}}%
\pgfpathlineto{\pgfqpoint{3.190265in}{1.661541in}}%
\pgfpathlineto{\pgfqpoint{3.191038in}{1.673764in}}%
\pgfpathlineto{\pgfqpoint{3.192188in}{1.685987in}}%
\pgfpathlineto{\pgfqpoint{3.192956in}{1.653392in}}%
\pgfpathlineto{\pgfqpoint{3.193339in}{1.669690in}}%
\pgfpathlineto{\pgfqpoint{3.193722in}{1.673764in}}%
\pgfpathlineto{\pgfqpoint{3.194873in}{1.665615in}}%
\pgfpathlineto{\pgfqpoint{3.195256in}{1.681913in}}%
\pgfpathlineto{\pgfqpoint{3.196023in}{1.677838in}}%
\pgfpathlineto{\pgfqpoint{3.196791in}{1.653392in}}%
\pgfpathlineto{\pgfqpoint{3.197174in}{1.673764in}}%
\pgfpathlineto{\pgfqpoint{3.197558in}{1.681913in}}%
\pgfpathlineto{\pgfqpoint{3.197558in}{1.681913in}}%
\pgfpathlineto{\pgfqpoint{3.197558in}{1.681913in}}%
\pgfpathlineto{\pgfqpoint{3.197941in}{1.669690in}}%
\pgfpathlineto{\pgfqpoint{3.197941in}{1.669690in}}%
\pgfpathlineto{\pgfqpoint{3.197941in}{1.669690in}}%
\pgfpathlineto{\pgfqpoint{3.198324in}{1.685987in}}%
\pgfpathlineto{\pgfqpoint{3.199090in}{1.673764in}}%
\pgfpathlineto{\pgfqpoint{3.199473in}{1.669690in}}%
\pgfpathlineto{\pgfqpoint{3.199857in}{1.681913in}}%
\pgfpathlineto{\pgfqpoint{3.200240in}{1.669690in}}%
\pgfpathlineto{\pgfqpoint{3.200623in}{1.669690in}}%
\pgfpathlineto{\pgfqpoint{3.201401in}{1.694136in}}%
\pgfpathlineto{\pgfqpoint{3.201785in}{1.653392in}}%
\pgfpathlineto{\pgfqpoint{3.202551in}{1.665615in}}%
\pgfpathlineto{\pgfqpoint{3.203316in}{1.677838in}}%
\pgfpathlineto{\pgfqpoint{3.203700in}{1.673764in}}%
\pgfpathlineto{\pgfqpoint{3.204083in}{1.669690in}}%
\pgfpathlineto{\pgfqpoint{3.204466in}{1.673764in}}%
\pgfpathlineto{\pgfqpoint{3.205616in}{1.677838in}}%
\pgfpathlineto{\pgfqpoint{3.205999in}{1.669690in}}%
\pgfpathlineto{\pgfqpoint{3.206383in}{1.673764in}}%
\pgfpathlineto{\pgfqpoint{3.206767in}{1.690062in}}%
\pgfpathlineto{\pgfqpoint{3.207149in}{1.681913in}}%
\pgfpathlineto{\pgfqpoint{3.207915in}{1.657467in}}%
\pgfpathlineto{\pgfqpoint{3.208298in}{1.673764in}}%
\pgfpathlineto{\pgfqpoint{3.209832in}{1.694136in}}%
\pgfpathlineto{\pgfqpoint{3.210981in}{1.661541in}}%
\pgfpathlineto{\pgfqpoint{3.212131in}{1.681913in}}%
\pgfpathlineto{\pgfqpoint{3.212515in}{1.677838in}}%
\pgfpathlineto{\pgfqpoint{3.212898in}{1.677838in}}%
\pgfpathlineto{\pgfqpoint{3.213281in}{1.669690in}}%
\pgfpathlineto{\pgfqpoint{3.214048in}{1.673764in}}%
\pgfpathlineto{\pgfqpoint{3.214815in}{1.661541in}}%
\pgfpathlineto{\pgfqpoint{3.215199in}{1.649318in}}%
\pgfpathlineto{\pgfqpoint{3.215965in}{1.685987in}}%
\pgfpathlineto{\pgfqpoint{3.216731in}{1.677838in}}%
\pgfpathlineto{\pgfqpoint{3.217114in}{1.657467in}}%
\pgfpathlineto{\pgfqpoint{3.217114in}{1.657467in}}%
\pgfpathlineto{\pgfqpoint{3.217114in}{1.657467in}}%
\pgfpathlineto{\pgfqpoint{3.218265in}{1.694136in}}%
\pgfpathlineto{\pgfqpoint{3.218648in}{1.665615in}}%
\pgfpathlineto{\pgfqpoint{3.219416in}{1.669690in}}%
\pgfpathlineto{\pgfqpoint{3.219800in}{1.673764in}}%
\pgfpathlineto{\pgfqpoint{3.220184in}{1.661541in}}%
\pgfpathlineto{\pgfqpoint{3.220184in}{1.661541in}}%
\pgfpathlineto{\pgfqpoint{3.220184in}{1.661541in}}%
\pgfpathlineto{\pgfqpoint{3.221334in}{1.681913in}}%
\pgfpathlineto{\pgfqpoint{3.222100in}{1.673764in}}%
\pgfpathlineto{\pgfqpoint{3.222484in}{1.681913in}}%
\pgfpathlineto{\pgfqpoint{3.222484in}{1.681913in}}%
\pgfpathlineto{\pgfqpoint{3.222484in}{1.681913in}}%
\pgfpathlineto{\pgfqpoint{3.222867in}{1.669690in}}%
\pgfpathlineto{\pgfqpoint{3.222867in}{1.669690in}}%
\pgfpathlineto{\pgfqpoint{3.222867in}{1.669690in}}%
\pgfpathlineto{\pgfqpoint{3.223251in}{1.685987in}}%
\pgfpathlineto{\pgfqpoint{3.224017in}{1.681913in}}%
\pgfpathlineto{\pgfqpoint{3.224400in}{1.669690in}}%
\pgfpathlineto{\pgfqpoint{3.225168in}{1.677838in}}%
\pgfpathlineto{\pgfqpoint{3.225550in}{1.681913in}}%
\pgfpathlineto{\pgfqpoint{3.225933in}{1.669690in}}%
\pgfpathlineto{\pgfqpoint{3.226316in}{1.673764in}}%
\pgfpathlineto{\pgfqpoint{3.226699in}{1.681913in}}%
\pgfpathlineto{\pgfqpoint{3.227082in}{1.665615in}}%
\pgfpathlineto{\pgfqpoint{3.227849in}{1.673764in}}%
\pgfpathlineto{\pgfqpoint{3.228999in}{1.694136in}}%
\pgfpathlineto{\pgfqpoint{3.230148in}{1.665615in}}%
\pgfpathlineto{\pgfqpoint{3.230915in}{1.685987in}}%
\pgfpathlineto{\pgfqpoint{3.231299in}{1.649318in}}%
\pgfpathlineto{\pgfqpoint{3.231299in}{1.649318in}}%
\pgfpathlineto{\pgfqpoint{3.231299in}{1.649318in}}%
\pgfpathlineto{\pgfqpoint{3.231682in}{1.690062in}}%
\pgfpathlineto{\pgfqpoint{3.232449in}{1.677838in}}%
\pgfpathlineto{\pgfqpoint{3.233216in}{1.665615in}}%
\pgfpathlineto{\pgfqpoint{3.233600in}{1.673764in}}%
\pgfpathlineto{\pgfqpoint{3.233982in}{1.685987in}}%
\pgfpathlineto{\pgfqpoint{3.234365in}{1.677838in}}%
\pgfpathlineto{\pgfqpoint{3.234834in}{1.665615in}}%
\pgfpathlineto{\pgfqpoint{3.235217in}{1.690062in}}%
\pgfpathlineto{\pgfqpoint{3.235983in}{1.685987in}}%
\pgfpathlineto{\pgfqpoint{3.236366in}{1.661541in}}%
\pgfpathlineto{\pgfqpoint{3.237134in}{1.673764in}}%
\pgfpathlineto{\pgfqpoint{3.237517in}{1.661541in}}%
\pgfpathlineto{\pgfqpoint{3.237517in}{1.661541in}}%
\pgfpathlineto{\pgfqpoint{3.237517in}{1.661541in}}%
\pgfpathlineto{\pgfqpoint{3.237901in}{1.681913in}}%
\pgfpathlineto{\pgfqpoint{3.238667in}{1.673764in}}%
\pgfpathlineto{\pgfqpoint{3.239816in}{1.694136in}}%
\pgfpathlineto{\pgfqpoint{3.240585in}{1.669690in}}%
\pgfpathlineto{\pgfqpoint{3.240968in}{1.673764in}}%
\pgfpathlineto{\pgfqpoint{3.241359in}{1.673764in}}%
\pgfpathlineto{\pgfqpoint{3.241743in}{1.661541in}}%
\pgfpathlineto{\pgfqpoint{3.242126in}{1.665615in}}%
\pgfpathlineto{\pgfqpoint{3.242893in}{1.681913in}}%
\pgfpathlineto{\pgfqpoint{3.243276in}{1.677838in}}%
\pgfpathlineto{\pgfqpoint{3.244044in}{1.661541in}}%
\pgfpathlineto{\pgfqpoint{3.244427in}{1.653392in}}%
\pgfpathlineto{\pgfqpoint{3.244810in}{1.657467in}}%
\pgfpathlineto{\pgfqpoint{3.245193in}{1.661541in}}%
\pgfpathlineto{\pgfqpoint{3.245576in}{1.681913in}}%
\pgfpathlineto{\pgfqpoint{3.246342in}{1.673764in}}%
\pgfpathlineto{\pgfqpoint{3.246725in}{1.673764in}}%
\pgfpathlineto{\pgfqpoint{3.247109in}{1.657467in}}%
\pgfpathlineto{\pgfqpoint{3.247492in}{1.665615in}}%
\pgfpathlineto{\pgfqpoint{3.249024in}{1.685987in}}%
\pgfpathlineto{\pgfqpoint{3.249409in}{1.669690in}}%
\pgfpathlineto{\pgfqpoint{3.250175in}{1.673764in}}%
\pgfpathlineto{\pgfqpoint{3.250558in}{1.669690in}}%
\pgfpathlineto{\pgfqpoint{3.252092in}{1.694136in}}%
\pgfpathlineto{\pgfqpoint{3.252475in}{1.673764in}}%
\pgfpathlineto{\pgfqpoint{3.253240in}{1.685987in}}%
\pgfpathlineto{\pgfqpoint{3.254007in}{1.665615in}}%
\pgfpathlineto{\pgfqpoint{3.254391in}{1.685987in}}%
\pgfpathlineto{\pgfqpoint{3.255158in}{1.681913in}}%
\pgfpathlineto{\pgfqpoint{3.257074in}{1.665615in}}%
\pgfpathlineto{\pgfqpoint{3.257458in}{1.673764in}}%
\pgfpathlineto{\pgfqpoint{3.257841in}{1.653392in}}%
\pgfpathlineto{\pgfqpoint{3.258224in}{1.673764in}}%
\pgfpathlineto{\pgfqpoint{3.259373in}{1.685987in}}%
\pgfpathlineto{\pgfqpoint{3.259757in}{1.681913in}}%
\pgfpathlineto{\pgfqpoint{3.260141in}{1.665615in}}%
\pgfpathlineto{\pgfqpoint{3.260908in}{1.673764in}}%
\pgfpathlineto{\pgfqpoint{3.261291in}{1.669690in}}%
\pgfpathlineto{\pgfqpoint{3.261674in}{1.681913in}}%
\pgfpathlineto{\pgfqpoint{3.262056in}{1.653392in}}%
\pgfpathlineto{\pgfqpoint{3.262824in}{1.677838in}}%
\pgfpathlineto{\pgfqpoint{3.263590in}{1.665615in}}%
\pgfpathlineto{\pgfqpoint{3.263973in}{1.685987in}}%
\pgfpathlineto{\pgfqpoint{3.264740in}{1.677838in}}%
\pgfpathlineto{\pgfqpoint{3.265507in}{1.677838in}}%
\pgfpathlineto{\pgfqpoint{3.266656in}{1.665615in}}%
\pgfpathlineto{\pgfqpoint{3.267421in}{1.677838in}}%
\pgfpathlineto{\pgfqpoint{3.267805in}{1.669690in}}%
\pgfpathlineto{\pgfqpoint{3.268571in}{1.681913in}}%
\pgfpathlineto{\pgfqpoint{3.269720in}{1.669690in}}%
\pgfpathlineto{\pgfqpoint{3.270871in}{1.690062in}}%
\pgfpathlineto{\pgfqpoint{3.272019in}{1.657467in}}%
\pgfpathlineto{\pgfqpoint{3.273169in}{1.685987in}}%
\pgfpathlineto{\pgfqpoint{3.274319in}{1.657467in}}%
\pgfpathlineto{\pgfqpoint{3.274703in}{1.665615in}}%
\pgfpathlineto{\pgfqpoint{3.275085in}{1.669690in}}%
\pgfpathlineto{\pgfqpoint{3.275553in}{1.694136in}}%
\pgfpathlineto{\pgfqpoint{3.275553in}{1.694136in}}%
\pgfpathlineto{\pgfqpoint{3.275553in}{1.694136in}}%
\pgfpathlineto{\pgfqpoint{3.277086in}{1.661541in}}%
\pgfpathlineto{\pgfqpoint{3.277853in}{1.677838in}}%
\pgfpathlineto{\pgfqpoint{3.278236in}{1.653392in}}%
\pgfpathlineto{\pgfqpoint{3.278236in}{1.653392in}}%
\pgfpathlineto{\pgfqpoint{3.278236in}{1.653392in}}%
\pgfpathlineto{\pgfqpoint{3.279385in}{1.690062in}}%
\pgfpathlineto{\pgfqpoint{3.280917in}{1.657467in}}%
\pgfpathlineto{\pgfqpoint{3.281307in}{1.657467in}}%
\pgfpathlineto{\pgfqpoint{3.282460in}{1.677838in}}%
\pgfpathlineto{\pgfqpoint{3.282845in}{1.673764in}}%
\pgfpathlineto{\pgfqpoint{3.283228in}{1.649318in}}%
\pgfpathlineto{\pgfqpoint{3.283228in}{1.649318in}}%
\pgfpathlineto{\pgfqpoint{3.283228in}{1.649318in}}%
\pgfpathlineto{\pgfqpoint{3.283611in}{1.685987in}}%
\pgfpathlineto{\pgfqpoint{3.284379in}{1.661541in}}%
\pgfpathlineto{\pgfqpoint{3.284763in}{1.685987in}}%
\pgfpathlineto{\pgfqpoint{3.285529in}{1.681913in}}%
\pgfpathlineto{\pgfqpoint{3.286296in}{1.649318in}}%
\pgfpathlineto{\pgfqpoint{3.287064in}{1.694136in}}%
\pgfpathlineto{\pgfqpoint{3.287447in}{1.681913in}}%
\pgfpathlineto{\pgfqpoint{3.287829in}{1.685987in}}%
\pgfpathlineto{\pgfqpoint{3.288213in}{1.681913in}}%
\pgfpathlineto{\pgfqpoint{3.290131in}{1.669690in}}%
\pgfpathlineto{\pgfqpoint{3.290514in}{1.669690in}}%
\pgfpathlineto{\pgfqpoint{3.290897in}{1.657467in}}%
\pgfpathlineto{\pgfqpoint{3.291280in}{1.669690in}}%
\pgfpathlineto{\pgfqpoint{3.292430in}{1.681913in}}%
\pgfpathlineto{\pgfqpoint{3.293580in}{1.665615in}}%
\pgfpathlineto{\pgfqpoint{3.293963in}{1.669690in}}%
\pgfpathlineto{\pgfqpoint{3.294346in}{1.669690in}}%
\pgfpathlineto{\pgfqpoint{3.294729in}{1.649318in}}%
\pgfpathlineto{\pgfqpoint{3.295114in}{1.661541in}}%
\pgfpathlineto{\pgfqpoint{3.296264in}{1.685987in}}%
\pgfpathlineto{\pgfqpoint{3.296647in}{1.669690in}}%
\pgfpathlineto{\pgfqpoint{3.297029in}{1.673764in}}%
\pgfpathlineto{\pgfqpoint{3.297413in}{1.690062in}}%
\pgfpathlineto{\pgfqpoint{3.297797in}{1.673764in}}%
\pgfpathlineto{\pgfqpoint{3.298181in}{1.673764in}}%
\pgfpathlineto{\pgfqpoint{3.298564in}{1.669690in}}%
\pgfpathlineto{\pgfqpoint{3.298947in}{1.645243in}}%
\pgfpathlineto{\pgfqpoint{3.298947in}{1.645243in}}%
\pgfpathlineto{\pgfqpoint{3.298947in}{1.645243in}}%
\pgfpathlineto{\pgfqpoint{3.299713in}{1.681913in}}%
\pgfpathlineto{\pgfqpoint{3.300097in}{1.669690in}}%
\pgfpathlineto{\pgfqpoint{3.300481in}{1.677838in}}%
\pgfpathlineto{\pgfqpoint{3.300864in}{1.657467in}}%
\pgfpathlineto{\pgfqpoint{3.301247in}{1.677838in}}%
\pgfpathlineto{\pgfqpoint{3.301630in}{1.681913in}}%
\pgfpathlineto{\pgfqpoint{3.302396in}{1.661541in}}%
\pgfpathlineto{\pgfqpoint{3.302780in}{1.665615in}}%
\pgfpathlineto{\pgfqpoint{3.304314in}{1.681913in}}%
\pgfpathlineto{\pgfqpoint{3.305080in}{1.661541in}}%
\pgfpathlineto{\pgfqpoint{3.305464in}{1.677838in}}%
\pgfpathlineto{\pgfqpoint{3.305848in}{1.673764in}}%
\pgfpathlineto{\pgfqpoint{3.306231in}{1.677838in}}%
\pgfpathlineto{\pgfqpoint{3.306614in}{1.681913in}}%
\pgfpathlineto{\pgfqpoint{3.306997in}{1.677838in}}%
\pgfpathlineto{\pgfqpoint{3.308145in}{1.669690in}}%
\pgfpathlineto{\pgfqpoint{3.309680in}{1.685987in}}%
\pgfpathlineto{\pgfqpoint{3.310829in}{1.661541in}}%
\pgfpathlineto{\pgfqpoint{3.311214in}{1.665615in}}%
\pgfpathlineto{\pgfqpoint{3.311980in}{1.677838in}}%
\pgfpathlineto{\pgfqpoint{3.312364in}{1.657467in}}%
\pgfpathlineto{\pgfqpoint{3.312746in}{1.669690in}}%
\pgfpathlineto{\pgfqpoint{3.313130in}{1.681913in}}%
\pgfpathlineto{\pgfqpoint{3.313515in}{1.669690in}}%
\pgfpathlineto{\pgfqpoint{3.313898in}{1.669690in}}%
\pgfpathlineto{\pgfqpoint{3.314281in}{1.657467in}}%
\pgfpathlineto{\pgfqpoint{3.314281in}{1.657467in}}%
\pgfpathlineto{\pgfqpoint{3.314281in}{1.657467in}}%
\pgfpathlineto{\pgfqpoint{3.315813in}{1.685987in}}%
\pgfpathlineto{\pgfqpoint{3.316196in}{1.657467in}}%
\pgfpathlineto{\pgfqpoint{3.316964in}{1.673764in}}%
\pgfpathlineto{\pgfqpoint{3.317347in}{1.669690in}}%
\pgfpathlineto{\pgfqpoint{3.318113in}{1.681913in}}%
\pgfpathlineto{\pgfqpoint{3.319647in}{1.657467in}}%
\pgfpathlineto{\pgfqpoint{3.320499in}{1.685987in}}%
\pgfpathlineto{\pgfqpoint{3.320882in}{1.665615in}}%
\pgfpathlineto{\pgfqpoint{3.321273in}{1.657467in}}%
\pgfpathlineto{\pgfqpoint{3.322040in}{1.685987in}}%
\pgfpathlineto{\pgfqpoint{3.322424in}{1.649318in}}%
\pgfpathlineto{\pgfqpoint{3.323191in}{1.677838in}}%
\pgfpathlineto{\pgfqpoint{3.323575in}{1.657467in}}%
\pgfpathlineto{\pgfqpoint{3.324341in}{1.673764in}}%
\pgfpathlineto{\pgfqpoint{3.324724in}{1.657467in}}%
\pgfpathlineto{\pgfqpoint{3.325108in}{1.669690in}}%
\pgfpathlineto{\pgfqpoint{3.325492in}{1.673764in}}%
\pgfpathlineto{\pgfqpoint{3.326643in}{1.661541in}}%
\pgfpathlineto{\pgfqpoint{3.328176in}{1.681913in}}%
\pgfpathlineto{\pgfqpoint{3.328560in}{1.685987in}}%
\pgfpathlineto{\pgfqpoint{3.328944in}{1.669690in}}%
\pgfpathlineto{\pgfqpoint{3.329327in}{1.677838in}}%
\pgfpathlineto{\pgfqpoint{3.330476in}{1.690062in}}%
\pgfpathlineto{\pgfqpoint{3.332010in}{1.637095in}}%
\pgfpathlineto{\pgfqpoint{3.333159in}{1.685987in}}%
\pgfpathlineto{\pgfqpoint{3.333542in}{1.669690in}}%
\pgfpathlineto{\pgfqpoint{3.334308in}{1.677838in}}%
\pgfpathlineto{\pgfqpoint{3.335074in}{1.665615in}}%
\pgfpathlineto{\pgfqpoint{3.335841in}{1.681913in}}%
\pgfpathlineto{\pgfqpoint{3.336224in}{1.673764in}}%
\pgfpathlineto{\pgfqpoint{3.336607in}{1.673764in}}%
\pgfpathlineto{\pgfqpoint{3.337374in}{1.685987in}}%
\pgfpathlineto{\pgfqpoint{3.338141in}{1.661541in}}%
\pgfpathlineto{\pgfqpoint{3.338524in}{1.673764in}}%
\pgfpathlineto{\pgfqpoint{3.338907in}{1.690062in}}%
\pgfpathlineto{\pgfqpoint{3.339290in}{1.681913in}}%
\pgfpathlineto{\pgfqpoint{3.339674in}{1.673764in}}%
\pgfpathlineto{\pgfqpoint{3.339674in}{1.673764in}}%
\pgfpathlineto{\pgfqpoint{3.339674in}{1.673764in}}%
\pgfpathlineto{\pgfqpoint{3.340058in}{1.685987in}}%
\pgfpathlineto{\pgfqpoint{3.340441in}{1.681913in}}%
\pgfpathlineto{\pgfqpoint{3.340824in}{1.657467in}}%
\pgfpathlineto{\pgfqpoint{3.341208in}{1.665615in}}%
\pgfpathlineto{\pgfqpoint{3.341591in}{1.694136in}}%
\pgfpathlineto{\pgfqpoint{3.342357in}{1.677838in}}%
\pgfpathlineto{\pgfqpoint{3.342741in}{1.661541in}}%
\pgfpathlineto{\pgfqpoint{3.342741in}{1.661541in}}%
\pgfpathlineto{\pgfqpoint{3.342741in}{1.661541in}}%
\pgfpathlineto{\pgfqpoint{3.343125in}{1.681913in}}%
\pgfpathlineto{\pgfqpoint{3.343892in}{1.673764in}}%
\pgfpathlineto{\pgfqpoint{3.345041in}{1.665615in}}%
\pgfpathlineto{\pgfqpoint{3.345425in}{1.665615in}}%
\pgfpathlineto{\pgfqpoint{3.346191in}{1.673764in}}%
\pgfpathlineto{\pgfqpoint{3.346575in}{1.665615in}}%
\pgfpathlineto{\pgfqpoint{3.347342in}{1.685987in}}%
\pgfpathlineto{\pgfqpoint{3.347726in}{1.677838in}}%
\pgfpathlineto{\pgfqpoint{3.348875in}{1.673764in}}%
\pgfpathlineto{\pgfqpoint{3.349258in}{1.677838in}}%
\pgfpathlineto{\pgfqpoint{3.349641in}{1.669690in}}%
\pgfpathlineto{\pgfqpoint{3.349641in}{1.669690in}}%
\pgfpathlineto{\pgfqpoint{3.349641in}{1.669690in}}%
\pgfpathlineto{\pgfqpoint{3.350026in}{1.681913in}}%
\pgfpathlineto{\pgfqpoint{3.350026in}{1.681913in}}%
\pgfpathlineto{\pgfqpoint{3.350026in}{1.681913in}}%
\pgfpathlineto{\pgfqpoint{3.351176in}{1.661541in}}%
\pgfpathlineto{\pgfqpoint{3.351559in}{1.673764in}}%
\pgfpathlineto{\pgfqpoint{3.351941in}{1.669690in}}%
\pgfpathlineto{\pgfqpoint{3.353092in}{1.657467in}}%
\pgfpathlineto{\pgfqpoint{3.353476in}{1.673764in}}%
\pgfpathlineto{\pgfqpoint{3.353476in}{1.673764in}}%
\pgfpathlineto{\pgfqpoint{3.353476in}{1.673764in}}%
\pgfpathlineto{\pgfqpoint{3.353858in}{1.649318in}}%
\pgfpathlineto{\pgfqpoint{3.354242in}{1.669690in}}%
\pgfpathlineto{\pgfqpoint{3.354625in}{1.694136in}}%
\pgfpathlineto{\pgfqpoint{3.355391in}{1.673764in}}%
\pgfpathlineto{\pgfqpoint{3.355775in}{1.677838in}}%
\pgfpathlineto{\pgfqpoint{3.356159in}{1.694136in}}%
\pgfpathlineto{\pgfqpoint{3.356542in}{1.690062in}}%
\pgfpathlineto{\pgfqpoint{3.358074in}{1.661541in}}%
\pgfpathlineto{\pgfqpoint{3.358458in}{1.657467in}}%
\pgfpathlineto{\pgfqpoint{3.359607in}{1.698210in}}%
\pgfpathlineto{\pgfqpoint{3.360375in}{1.665615in}}%
\pgfpathlineto{\pgfqpoint{3.360758in}{1.681913in}}%
\pgfpathlineto{\pgfqpoint{3.361149in}{1.673764in}}%
\pgfpathlineto{\pgfqpoint{3.361149in}{1.673764in}}%
\pgfpathlineto{\pgfqpoint{3.361149in}{1.673764in}}%
\pgfpathlineto{\pgfqpoint{3.361533in}{1.685987in}}%
\pgfpathlineto{\pgfqpoint{3.361533in}{1.685987in}}%
\pgfpathlineto{\pgfqpoint{3.361533in}{1.685987in}}%
\pgfpathlineto{\pgfqpoint{3.361916in}{1.669690in}}%
\pgfpathlineto{\pgfqpoint{3.362299in}{1.685987in}}%
\pgfpathlineto{\pgfqpoint{3.362683in}{1.698210in}}%
\pgfpathlineto{\pgfqpoint{3.362683in}{1.698210in}}%
\pgfpathlineto{\pgfqpoint{3.362683in}{1.698210in}}%
\pgfpathlineto{\pgfqpoint{3.364216in}{1.665615in}}%
\pgfpathlineto{\pgfqpoint{3.365366in}{1.681913in}}%
\pgfpathlineto{\pgfqpoint{3.366135in}{1.661541in}}%
\pgfpathlineto{\pgfqpoint{3.366518in}{1.673764in}}%
\pgfpathlineto{\pgfqpoint{3.366901in}{1.673764in}}%
\pgfpathlineto{\pgfqpoint{3.368049in}{1.677838in}}%
\pgfpathlineto{\pgfqpoint{3.368817in}{1.665615in}}%
\pgfpathlineto{\pgfqpoint{3.370051in}{1.681913in}}%
\pgfpathlineto{\pgfqpoint{3.370434in}{1.665615in}}%
\pgfpathlineto{\pgfqpoint{3.370817in}{1.677838in}}%
\pgfpathlineto{\pgfqpoint{3.371200in}{1.685987in}}%
\pgfpathlineto{\pgfqpoint{3.371583in}{1.661541in}}%
\pgfpathlineto{\pgfqpoint{3.372349in}{1.681913in}}%
\pgfpathlineto{\pgfqpoint{3.373117in}{1.681913in}}%
\pgfpathlineto{\pgfqpoint{3.374649in}{1.665615in}}%
\pgfpathlineto{\pgfqpoint{3.375033in}{1.690062in}}%
\pgfpathlineto{\pgfqpoint{3.375033in}{1.690062in}}%
\pgfpathlineto{\pgfqpoint{3.375033in}{1.690062in}}%
\pgfpathlineto{\pgfqpoint{3.375417in}{1.657467in}}%
\pgfpathlineto{\pgfqpoint{3.376183in}{1.673764in}}%
\pgfpathlineto{\pgfqpoint{3.377333in}{1.694136in}}%
\pgfpathlineto{\pgfqpoint{3.378483in}{1.657467in}}%
\pgfpathlineto{\pgfqpoint{3.378866in}{1.669690in}}%
\pgfpathlineto{\pgfqpoint{3.379249in}{1.681913in}}%
\pgfpathlineto{\pgfqpoint{3.379633in}{1.669690in}}%
\pgfpathlineto{\pgfqpoint{3.380017in}{1.669690in}}%
\pgfpathlineto{\pgfqpoint{3.380400in}{1.681913in}}%
\pgfpathlineto{\pgfqpoint{3.380400in}{1.681913in}}%
\pgfpathlineto{\pgfqpoint{3.380400in}{1.681913in}}%
\pgfpathlineto{\pgfqpoint{3.380784in}{1.665615in}}%
\pgfpathlineto{\pgfqpoint{3.381167in}{1.677838in}}%
\pgfpathlineto{\pgfqpoint{3.381550in}{1.685987in}}%
\pgfpathlineto{\pgfqpoint{3.381933in}{1.665615in}}%
\pgfpathlineto{\pgfqpoint{3.382316in}{1.677838in}}%
\pgfpathlineto{\pgfqpoint{3.382700in}{1.685987in}}%
\pgfpathlineto{\pgfqpoint{3.383466in}{1.665615in}}%
\pgfpathlineto{\pgfqpoint{3.383849in}{1.706359in}}%
\pgfpathlineto{\pgfqpoint{3.383849in}{1.706359in}}%
\pgfpathlineto{\pgfqpoint{3.383849in}{1.706359in}}%
\pgfpathlineto{\pgfqpoint{3.384232in}{1.661541in}}%
\pgfpathlineto{\pgfqpoint{3.384999in}{1.685987in}}%
\pgfpathlineto{\pgfqpoint{3.385383in}{1.681913in}}%
\pgfpathlineto{\pgfqpoint{3.385766in}{1.685987in}}%
\pgfpathlineto{\pgfqpoint{3.386149in}{1.698210in}}%
\pgfpathlineto{\pgfqpoint{3.386149in}{1.698210in}}%
\pgfpathlineto{\pgfqpoint{3.386149in}{1.698210in}}%
\pgfpathlineto{\pgfqpoint{3.386917in}{1.669690in}}%
\pgfpathlineto{\pgfqpoint{3.387684in}{1.673764in}}%
\pgfpathlineto{\pgfqpoint{3.388067in}{1.665615in}}%
\pgfpathlineto{\pgfqpoint{3.388451in}{1.669690in}}%
\pgfpathlineto{\pgfqpoint{3.388835in}{1.677838in}}%
\pgfpathlineto{\pgfqpoint{3.389601in}{1.673764in}}%
\pgfpathlineto{\pgfqpoint{3.390368in}{1.665615in}}%
\pgfpathlineto{\pgfqpoint{3.390751in}{1.669690in}}%
\pgfpathlineto{\pgfqpoint{3.391135in}{1.673764in}}%
\pgfpathlineto{\pgfqpoint{3.391518in}{1.657467in}}%
\pgfpathlineto{\pgfqpoint{3.391518in}{1.657467in}}%
\pgfpathlineto{\pgfqpoint{3.391518in}{1.657467in}}%
\pgfpathlineto{\pgfqpoint{3.393051in}{1.690062in}}%
\pgfpathlineto{\pgfqpoint{3.393435in}{1.698210in}}%
\pgfpathlineto{\pgfqpoint{3.394202in}{1.641169in}}%
\pgfpathlineto{\pgfqpoint{3.394585in}{1.661541in}}%
\pgfpathlineto{\pgfqpoint{3.394968in}{1.681913in}}%
\pgfpathlineto{\pgfqpoint{3.394968in}{1.681913in}}%
\pgfpathlineto{\pgfqpoint{3.394968in}{1.681913in}}%
\pgfpathlineto{\pgfqpoint{3.395352in}{1.657467in}}%
\pgfpathlineto{\pgfqpoint{3.395736in}{1.661541in}}%
\pgfpathlineto{\pgfqpoint{3.396120in}{1.681913in}}%
\pgfpathlineto{\pgfqpoint{3.396120in}{1.681913in}}%
\pgfpathlineto{\pgfqpoint{3.396120in}{1.681913in}}%
\pgfpathlineto{\pgfqpoint{3.396503in}{1.657467in}}%
\pgfpathlineto{\pgfqpoint{3.396886in}{1.673764in}}%
\pgfpathlineto{\pgfqpoint{3.398419in}{1.690062in}}%
\pgfpathlineto{\pgfqpoint{3.399186in}{1.665615in}}%
\pgfpathlineto{\pgfqpoint{3.399569in}{1.673764in}}%
\pgfpathlineto{\pgfqpoint{3.400335in}{1.665615in}}%
\pgfpathlineto{\pgfqpoint{3.401877in}{1.694136in}}%
\pgfpathlineto{\pgfqpoint{3.402260in}{1.657467in}}%
\pgfpathlineto{\pgfqpoint{3.403028in}{1.669690in}}%
\pgfpathlineto{\pgfqpoint{3.403794in}{1.677838in}}%
\pgfpathlineto{\pgfqpoint{3.404177in}{1.673764in}}%
\pgfpathlineto{\pgfqpoint{3.404560in}{1.661541in}}%
\pgfpathlineto{\pgfqpoint{3.404560in}{1.661541in}}%
\pgfpathlineto{\pgfqpoint{3.404560in}{1.661541in}}%
\pgfpathlineto{\pgfqpoint{3.405711in}{1.685987in}}%
\pgfpathlineto{\pgfqpoint{3.406477in}{1.657467in}}%
\pgfpathlineto{\pgfqpoint{3.406860in}{1.665615in}}%
\pgfpathlineto{\pgfqpoint{3.407244in}{1.690062in}}%
\pgfpathlineto{\pgfqpoint{3.408011in}{1.685987in}}%
\pgfpathlineto{\pgfqpoint{3.408394in}{1.661541in}}%
\pgfpathlineto{\pgfqpoint{3.408394in}{1.661541in}}%
\pgfpathlineto{\pgfqpoint{3.408394in}{1.661541in}}%
\pgfpathlineto{\pgfqpoint{3.408777in}{1.690062in}}%
\pgfpathlineto{\pgfqpoint{3.409544in}{1.669690in}}%
\pgfpathlineto{\pgfqpoint{3.411078in}{1.685987in}}%
\pgfpathlineto{\pgfqpoint{3.411458in}{1.669690in}}%
\pgfpathlineto{\pgfqpoint{3.411842in}{1.677838in}}%
\pgfpathlineto{\pgfqpoint{3.412225in}{1.685987in}}%
\pgfpathlineto{\pgfqpoint{3.412609in}{1.677838in}}%
\pgfpathlineto{\pgfqpoint{3.413460in}{1.669690in}}%
\pgfpathlineto{\pgfqpoint{3.414612in}{1.681913in}}%
\pgfpathlineto{\pgfqpoint{3.415764in}{1.673764in}}%
\pgfpathlineto{\pgfqpoint{3.416147in}{1.673764in}}%
\pgfpathlineto{\pgfqpoint{3.416531in}{1.669690in}}%
\pgfpathlineto{\pgfqpoint{3.416914in}{1.673764in}}%
\pgfpathlineto{\pgfqpoint{3.417297in}{1.673764in}}%
\pgfpathlineto{\pgfqpoint{3.417680in}{1.690062in}}%
\pgfpathlineto{\pgfqpoint{3.418063in}{1.673764in}}%
\pgfpathlineto{\pgfqpoint{3.418446in}{1.669690in}}%
\pgfpathlineto{\pgfqpoint{3.418831in}{1.673764in}}%
\pgfpathlineto{\pgfqpoint{3.419214in}{1.690062in}}%
\pgfpathlineto{\pgfqpoint{3.419214in}{1.690062in}}%
\pgfpathlineto{\pgfqpoint{3.419214in}{1.690062in}}%
\pgfpathlineto{\pgfqpoint{3.419980in}{1.669690in}}%
\pgfpathlineto{\pgfqpoint{3.420364in}{1.677838in}}%
\pgfpathlineto{\pgfqpoint{3.421513in}{1.669690in}}%
\pgfpathlineto{\pgfqpoint{3.422663in}{1.694136in}}%
\pgfpathlineto{\pgfqpoint{3.423046in}{1.681913in}}%
\pgfpathlineto{\pgfqpoint{3.423429in}{1.657467in}}%
\pgfpathlineto{\pgfqpoint{3.423429in}{1.657467in}}%
\pgfpathlineto{\pgfqpoint{3.423429in}{1.657467in}}%
\pgfpathlineto{\pgfqpoint{3.423812in}{1.685987in}}%
\pgfpathlineto{\pgfqpoint{3.424578in}{1.673764in}}%
\pgfpathlineto{\pgfqpoint{3.424961in}{1.677838in}}%
\pgfpathlineto{\pgfqpoint{3.425345in}{1.694136in}}%
\pgfpathlineto{\pgfqpoint{3.425345in}{1.694136in}}%
\pgfpathlineto{\pgfqpoint{3.425345in}{1.694136in}}%
\pgfpathlineto{\pgfqpoint{3.426494in}{1.669690in}}%
\pgfpathlineto{\pgfqpoint{3.427644in}{1.677838in}}%
\pgfpathlineto{\pgfqpoint{3.428027in}{1.673764in}}%
\pgfpathlineto{\pgfqpoint{3.428411in}{1.690062in}}%
\pgfpathlineto{\pgfqpoint{3.428411in}{1.690062in}}%
\pgfpathlineto{\pgfqpoint{3.428411in}{1.690062in}}%
\pgfpathlineto{\pgfqpoint{3.428794in}{1.661541in}}%
\pgfpathlineto{\pgfqpoint{3.429559in}{1.669690in}}%
\pgfpathlineto{\pgfqpoint{3.431478in}{1.690062in}}%
\pgfpathlineto{\pgfqpoint{3.431861in}{1.685987in}}%
\pgfpathlineto{\pgfqpoint{3.433395in}{1.661541in}}%
\pgfpathlineto{\pgfqpoint{3.434162in}{1.685987in}}%
\pgfpathlineto{\pgfqpoint{3.434545in}{1.681913in}}%
\pgfpathlineto{\pgfqpoint{3.435313in}{1.665615in}}%
\pgfpathlineto{\pgfqpoint{3.436079in}{1.669690in}}%
\pgfpathlineto{\pgfqpoint{3.436845in}{1.677838in}}%
\pgfpathlineto{\pgfqpoint{3.437228in}{1.669690in}}%
\pgfpathlineto{\pgfqpoint{3.437611in}{1.694136in}}%
\pgfpathlineto{\pgfqpoint{3.437994in}{1.685987in}}%
\pgfpathlineto{\pgfqpoint{3.438762in}{1.661541in}}%
\pgfpathlineto{\pgfqpoint{3.439146in}{1.669690in}}%
\pgfpathlineto{\pgfqpoint{3.440678in}{1.677838in}}%
\pgfpathlineto{\pgfqpoint{3.441068in}{1.677838in}}%
\pgfpathlineto{\pgfqpoint{3.441451in}{1.661541in}}%
\pgfpathlineto{\pgfqpoint{3.442217in}{1.673764in}}%
\pgfpathlineto{\pgfqpoint{3.442601in}{1.669690in}}%
\pgfpathlineto{\pgfqpoint{3.442985in}{1.673764in}}%
\pgfpathlineto{\pgfqpoint{3.444134in}{1.681913in}}%
\pgfpathlineto{\pgfqpoint{3.444517in}{1.681913in}}%
\pgfpathlineto{\pgfqpoint{3.444899in}{1.685987in}}%
\pgfpathlineto{\pgfqpoint{3.445666in}{1.661541in}}%
\pgfpathlineto{\pgfqpoint{3.446050in}{1.673764in}}%
\pgfpathlineto{\pgfqpoint{3.446432in}{1.681913in}}%
\pgfpathlineto{\pgfqpoint{3.446432in}{1.681913in}}%
\pgfpathlineto{\pgfqpoint{3.446432in}{1.681913in}}%
\pgfpathlineto{\pgfqpoint{3.447198in}{1.653392in}}%
\pgfpathlineto{\pgfqpoint{3.447581in}{1.690062in}}%
\pgfpathlineto{\pgfqpoint{3.448347in}{1.673764in}}%
\pgfpathlineto{\pgfqpoint{3.448730in}{1.665615in}}%
\pgfpathlineto{\pgfqpoint{3.448730in}{1.665615in}}%
\pgfpathlineto{\pgfqpoint{3.448730in}{1.665615in}}%
\pgfpathlineto{\pgfqpoint{3.449113in}{1.677838in}}%
\pgfpathlineto{\pgfqpoint{3.449879in}{1.673764in}}%
\pgfpathlineto{\pgfqpoint{3.450262in}{1.661541in}}%
\pgfpathlineto{\pgfqpoint{3.450262in}{1.661541in}}%
\pgfpathlineto{\pgfqpoint{3.450262in}{1.661541in}}%
\pgfpathlineto{\pgfqpoint{3.451413in}{1.694136in}}%
\pgfpathlineto{\pgfqpoint{3.451796in}{1.690062in}}%
\pgfpathlineto{\pgfqpoint{3.453711in}{1.665615in}}%
\pgfpathlineto{\pgfqpoint{3.454179in}{1.665615in}}%
\pgfpathlineto{\pgfqpoint{3.454562in}{1.685987in}}%
\pgfpathlineto{\pgfqpoint{3.454562in}{1.685987in}}%
\pgfpathlineto{\pgfqpoint{3.454562in}{1.685987in}}%
\pgfpathlineto{\pgfqpoint{3.454946in}{1.653392in}}%
\pgfpathlineto{\pgfqpoint{3.455712in}{1.669690in}}%
\pgfpathlineto{\pgfqpoint{3.456477in}{1.685987in}}%
\pgfpathlineto{\pgfqpoint{3.456860in}{1.669690in}}%
\pgfpathlineto{\pgfqpoint{3.457629in}{1.681913in}}%
\pgfpathlineto{\pgfqpoint{3.458010in}{1.681913in}}%
\pgfpathlineto{\pgfqpoint{3.458394in}{1.685987in}}%
\pgfpathlineto{\pgfqpoint{3.459543in}{1.657467in}}%
\pgfpathlineto{\pgfqpoint{3.460693in}{1.681913in}}%
\pgfpathlineto{\pgfqpoint{3.461458in}{1.657467in}}%
\pgfpathlineto{\pgfqpoint{3.461842in}{1.669690in}}%
\pgfpathlineto{\pgfqpoint{3.462608in}{1.681913in}}%
\pgfpathlineto{\pgfqpoint{3.462991in}{1.673764in}}%
\pgfpathlineto{\pgfqpoint{3.463759in}{1.653392in}}%
\pgfpathlineto{\pgfqpoint{3.464526in}{1.685987in}}%
\pgfpathlineto{\pgfqpoint{3.464909in}{1.673764in}}%
\pgfpathlineto{\pgfqpoint{3.465291in}{1.677838in}}%
\pgfpathlineto{\pgfqpoint{3.465675in}{1.673764in}}%
\pgfpathlineto{\pgfqpoint{3.466059in}{1.669690in}}%
\pgfpathlineto{\pgfqpoint{3.466443in}{1.694136in}}%
\pgfpathlineto{\pgfqpoint{3.467209in}{1.677838in}}%
\pgfpathlineto{\pgfqpoint{3.468359in}{1.665615in}}%
\pgfpathlineto{\pgfqpoint{3.469510in}{1.690062in}}%
\pgfpathlineto{\pgfqpoint{3.469893in}{1.661541in}}%
\pgfpathlineto{\pgfqpoint{3.470659in}{1.673764in}}%
\pgfpathlineto{\pgfqpoint{3.471043in}{1.677838in}}%
\pgfpathlineto{\pgfqpoint{3.471810in}{1.653392in}}%
\pgfpathlineto{\pgfqpoint{3.472193in}{1.690062in}}%
\pgfpathlineto{\pgfqpoint{3.472959in}{1.677838in}}%
\pgfpathlineto{\pgfqpoint{3.473342in}{1.685987in}}%
\pgfpathlineto{\pgfqpoint{3.473342in}{1.685987in}}%
\pgfpathlineto{\pgfqpoint{3.473342in}{1.685987in}}%
\pgfpathlineto{\pgfqpoint{3.474494in}{1.673764in}}%
\pgfpathlineto{\pgfqpoint{3.475260in}{1.685987in}}%
\pgfpathlineto{\pgfqpoint{3.475643in}{1.641169in}}%
\pgfpathlineto{\pgfqpoint{3.476410in}{1.661541in}}%
\pgfpathlineto{\pgfqpoint{3.476794in}{1.681913in}}%
\pgfpathlineto{\pgfqpoint{3.477560in}{1.673764in}}%
\pgfpathlineto{\pgfqpoint{3.477944in}{1.673764in}}%
\pgfpathlineto{\pgfqpoint{3.478326in}{1.677838in}}%
\pgfpathlineto{\pgfqpoint{3.478709in}{1.673764in}}%
\pgfpathlineto{\pgfqpoint{3.479477in}{1.665615in}}%
\pgfpathlineto{\pgfqpoint{3.480244in}{1.694136in}}%
\pgfpathlineto{\pgfqpoint{3.480626in}{1.673764in}}%
\pgfpathlineto{\pgfqpoint{3.481017in}{1.673764in}}%
\pgfpathlineto{\pgfqpoint{3.481400in}{1.681913in}}%
\pgfpathlineto{\pgfqpoint{3.481400in}{1.681913in}}%
\pgfpathlineto{\pgfqpoint{3.481400in}{1.681913in}}%
\pgfpathlineto{\pgfqpoint{3.482168in}{1.665615in}}%
\pgfpathlineto{\pgfqpoint{3.482552in}{1.685987in}}%
\pgfpathlineto{\pgfqpoint{3.483318in}{1.669690in}}%
\pgfpathlineto{\pgfqpoint{3.483702in}{1.661541in}}%
\pgfpathlineto{\pgfqpoint{3.484468in}{1.690062in}}%
\pgfpathlineto{\pgfqpoint{3.484851in}{1.657467in}}%
\pgfpathlineto{\pgfqpoint{3.485234in}{1.681913in}}%
\pgfpathlineto{\pgfqpoint{3.485617in}{1.694136in}}%
\pgfpathlineto{\pgfqpoint{3.486001in}{1.690062in}}%
\pgfpathlineto{\pgfqpoint{3.486385in}{1.657467in}}%
\pgfpathlineto{\pgfqpoint{3.487151in}{1.669690in}}%
\pgfpathlineto{\pgfqpoint{3.488300in}{1.673764in}}%
\pgfpathlineto{\pgfqpoint{3.488684in}{1.669690in}}%
\pgfpathlineto{\pgfqpoint{3.489835in}{1.685987in}}%
\pgfpathlineto{\pgfqpoint{3.490218in}{1.669690in}}%
\pgfpathlineto{\pgfqpoint{3.490984in}{1.677838in}}%
\pgfpathlineto{\pgfqpoint{3.491368in}{1.681913in}}%
\pgfpathlineto{\pgfqpoint{3.491752in}{1.665615in}}%
\pgfpathlineto{\pgfqpoint{3.492135in}{1.673764in}}%
\pgfpathlineto{\pgfqpoint{3.492518in}{1.685987in}}%
\pgfpathlineto{\pgfqpoint{3.492901in}{1.681913in}}%
\pgfpathlineto{\pgfqpoint{3.493284in}{1.653392in}}%
\pgfpathlineto{\pgfqpoint{3.494051in}{1.665615in}}%
\pgfpathlineto{\pgfqpoint{3.494435in}{1.677838in}}%
\pgfpathlineto{\pgfqpoint{3.495201in}{1.669690in}}%
\pgfpathlineto{\pgfqpoint{3.496051in}{1.685987in}}%
\pgfpathlineto{\pgfqpoint{3.496435in}{1.677838in}}%
\pgfpathlineto{\pgfqpoint{3.497200in}{1.661541in}}%
\pgfpathlineto{\pgfqpoint{3.498734in}{1.681913in}}%
\pgfpathlineto{\pgfqpoint{3.499117in}{1.657467in}}%
\pgfpathlineto{\pgfqpoint{3.499117in}{1.657467in}}%
\pgfpathlineto{\pgfqpoint{3.499117in}{1.657467in}}%
\pgfpathlineto{\pgfqpoint{3.499500in}{1.690062in}}%
\pgfpathlineto{\pgfqpoint{3.500266in}{1.669690in}}%
\pgfpathlineto{\pgfqpoint{3.501416in}{1.673764in}}%
\pgfpathlineto{\pgfqpoint{3.501799in}{1.673764in}}%
\pgfpathlineto{\pgfqpoint{3.502182in}{1.681913in}}%
\pgfpathlineto{\pgfqpoint{3.502564in}{1.665615in}}%
\pgfpathlineto{\pgfqpoint{3.502947in}{1.677838in}}%
\pgfpathlineto{\pgfqpoint{3.503331in}{1.685987in}}%
\pgfpathlineto{\pgfqpoint{3.503714in}{1.677838in}}%
\pgfpathlineto{\pgfqpoint{3.504097in}{1.665615in}}%
\pgfpathlineto{\pgfqpoint{3.504478in}{1.669690in}}%
\pgfpathlineto{\pgfqpoint{3.504860in}{1.685987in}}%
\pgfpathlineto{\pgfqpoint{3.506011in}{1.653392in}}%
\pgfpathlineto{\pgfqpoint{3.506395in}{1.657467in}}%
\pgfpathlineto{\pgfqpoint{3.507545in}{1.681913in}}%
\pgfpathlineto{\pgfqpoint{3.509079in}{1.657467in}}%
\pgfpathlineto{\pgfqpoint{3.510613in}{1.681913in}}%
\pgfpathlineto{\pgfqpoint{3.510996in}{1.665615in}}%
\pgfpathlineto{\pgfqpoint{3.511380in}{1.673764in}}%
\pgfpathlineto{\pgfqpoint{3.511763in}{1.690062in}}%
\pgfpathlineto{\pgfqpoint{3.511763in}{1.690062in}}%
\pgfpathlineto{\pgfqpoint{3.511763in}{1.690062in}}%
\pgfpathlineto{\pgfqpoint{3.513296in}{1.661541in}}%
\pgfpathlineto{\pgfqpoint{3.514063in}{1.694136in}}%
\pgfpathlineto{\pgfqpoint{3.514447in}{1.673764in}}%
\pgfpathlineto{\pgfqpoint{3.514830in}{1.673764in}}%
\pgfpathlineto{\pgfqpoint{3.515213in}{1.685987in}}%
\pgfpathlineto{\pgfqpoint{3.515978in}{1.681913in}}%
\pgfpathlineto{\pgfqpoint{3.516361in}{1.665615in}}%
\pgfpathlineto{\pgfqpoint{3.517128in}{1.673764in}}%
\pgfpathlineto{\pgfqpoint{3.517511in}{1.677838in}}%
\pgfpathlineto{\pgfqpoint{3.517896in}{1.653392in}}%
\pgfpathlineto{\pgfqpoint{3.518661in}{1.669690in}}%
\pgfpathlineto{\pgfqpoint{3.519429in}{1.677838in}}%
\pgfpathlineto{\pgfqpoint{3.519813in}{1.673764in}}%
\pgfpathlineto{\pgfqpoint{3.520197in}{1.657467in}}%
\pgfpathlineto{\pgfqpoint{3.520197in}{1.657467in}}%
\pgfpathlineto{\pgfqpoint{3.520197in}{1.657467in}}%
\pgfpathlineto{\pgfqpoint{3.521737in}{1.690062in}}%
\pgfpathlineto{\pgfqpoint{3.522120in}{1.673764in}}%
\pgfpathlineto{\pgfqpoint{3.522886in}{1.677838in}}%
\pgfpathlineto{\pgfqpoint{3.524804in}{1.661541in}}%
\pgfpathlineto{\pgfqpoint{3.525570in}{1.677838in}}%
\pgfpathlineto{\pgfqpoint{3.525953in}{1.661541in}}%
\pgfpathlineto{\pgfqpoint{3.526336in}{1.677838in}}%
\pgfpathlineto{\pgfqpoint{3.526720in}{1.677838in}}%
\pgfpathlineto{\pgfqpoint{3.527104in}{1.665615in}}%
\pgfpathlineto{\pgfqpoint{3.527487in}{1.669690in}}%
\pgfpathlineto{\pgfqpoint{3.527869in}{1.690062in}}%
\pgfpathlineto{\pgfqpoint{3.528253in}{1.681913in}}%
\pgfpathlineto{\pgfqpoint{3.528636in}{1.657467in}}%
\pgfpathlineto{\pgfqpoint{3.529019in}{1.681913in}}%
\pgfpathlineto{\pgfqpoint{3.529402in}{1.681913in}}%
\pgfpathlineto{\pgfqpoint{3.529786in}{1.661541in}}%
\pgfpathlineto{\pgfqpoint{3.530170in}{1.673764in}}%
\pgfpathlineto{\pgfqpoint{3.530554in}{1.685987in}}%
\pgfpathlineto{\pgfqpoint{3.530937in}{1.657467in}}%
\pgfpathlineto{\pgfqpoint{3.531703in}{1.673764in}}%
\pgfpathlineto{\pgfqpoint{3.532853in}{1.669690in}}%
\pgfpathlineto{\pgfqpoint{3.534001in}{1.677838in}}%
\pgfpathlineto{\pgfqpoint{3.535150in}{1.657467in}}%
\pgfpathlineto{\pgfqpoint{3.535534in}{1.685987in}}%
\pgfpathlineto{\pgfqpoint{3.536386in}{1.681913in}}%
\pgfpathlineto{\pgfqpoint{3.537152in}{1.653392in}}%
\pgfpathlineto{\pgfqpoint{3.537918in}{1.661541in}}%
\pgfpathlineto{\pgfqpoint{3.539068in}{1.681913in}}%
\pgfpathlineto{\pgfqpoint{3.539451in}{1.645243in}}%
\pgfpathlineto{\pgfqpoint{3.540217in}{1.677838in}}%
\pgfpathlineto{\pgfqpoint{3.540600in}{1.657467in}}%
\pgfpathlineto{\pgfqpoint{3.540600in}{1.657467in}}%
\pgfpathlineto{\pgfqpoint{3.540600in}{1.657467in}}%
\pgfpathlineto{\pgfqpoint{3.540983in}{1.690062in}}%
\pgfpathlineto{\pgfqpoint{3.541750in}{1.677838in}}%
\pgfpathlineto{\pgfqpoint{3.542900in}{1.661541in}}%
\pgfpathlineto{\pgfqpoint{3.543283in}{1.665615in}}%
\pgfpathlineto{\pgfqpoint{3.543667in}{1.677838in}}%
\pgfpathlineto{\pgfqpoint{3.544433in}{1.673764in}}%
\pgfpathlineto{\pgfqpoint{3.544816in}{1.669690in}}%
\pgfpathlineto{\pgfqpoint{3.545200in}{1.677838in}}%
\pgfpathlineto{\pgfqpoint{3.545200in}{1.677838in}}%
\pgfpathlineto{\pgfqpoint{3.545200in}{1.677838in}}%
\pgfpathlineto{\pgfqpoint{3.545584in}{1.665615in}}%
\pgfpathlineto{\pgfqpoint{3.545966in}{1.669690in}}%
\pgfpathlineto{\pgfqpoint{3.546351in}{1.681913in}}%
\pgfpathlineto{\pgfqpoint{3.546351in}{1.681913in}}%
\pgfpathlineto{\pgfqpoint{3.546351in}{1.681913in}}%
\pgfpathlineto{\pgfqpoint{3.546734in}{1.661541in}}%
\pgfpathlineto{\pgfqpoint{3.546734in}{1.661541in}}%
\pgfpathlineto{\pgfqpoint{3.546734in}{1.661541in}}%
\pgfpathlineto{\pgfqpoint{3.547118in}{1.690062in}}%
\pgfpathlineto{\pgfqpoint{3.547884in}{1.669690in}}%
\pgfpathlineto{\pgfqpoint{3.548267in}{1.669690in}}%
\pgfpathlineto{\pgfqpoint{3.548650in}{1.694136in}}%
\pgfpathlineto{\pgfqpoint{3.549416in}{1.677838in}}%
\pgfpathlineto{\pgfqpoint{3.549799in}{1.681913in}}%
\pgfpathlineto{\pgfqpoint{3.550182in}{1.677838in}}%
\pgfpathlineto{\pgfqpoint{3.551332in}{1.657467in}}%
\pgfpathlineto{\pgfqpoint{3.551715in}{1.673764in}}%
\pgfpathlineto{\pgfqpoint{3.551715in}{1.673764in}}%
\pgfpathlineto{\pgfqpoint{3.551715in}{1.673764in}}%
\pgfpathlineto{\pgfqpoint{3.552098in}{1.653392in}}%
\pgfpathlineto{\pgfqpoint{3.552098in}{1.653392in}}%
\pgfpathlineto{\pgfqpoint{3.552098in}{1.653392in}}%
\pgfpathlineto{\pgfqpoint{3.552482in}{1.681913in}}%
\pgfpathlineto{\pgfqpoint{3.553249in}{1.665615in}}%
\pgfpathlineto{\pgfqpoint{3.553633in}{1.677838in}}%
\pgfpathlineto{\pgfqpoint{3.553633in}{1.677838in}}%
\pgfpathlineto{\pgfqpoint{3.553633in}{1.677838in}}%
\pgfpathlineto{\pgfqpoint{3.554015in}{1.661541in}}%
\pgfpathlineto{\pgfqpoint{3.554781in}{1.665615in}}%
\pgfpathlineto{\pgfqpoint{3.555549in}{1.685987in}}%
\pgfpathlineto{\pgfqpoint{3.555933in}{1.661541in}}%
\pgfpathlineto{\pgfqpoint{3.556699in}{1.677838in}}%
\pgfpathlineto{\pgfqpoint{3.557464in}{1.685987in}}%
\pgfpathlineto{\pgfqpoint{3.558999in}{1.665615in}}%
\pgfpathlineto{\pgfqpoint{3.559382in}{1.681913in}}%
\pgfpathlineto{\pgfqpoint{3.559382in}{1.681913in}}%
\pgfpathlineto{\pgfqpoint{3.559382in}{1.681913in}}%
\pgfpathlineto{\pgfqpoint{3.559764in}{1.657467in}}%
\pgfpathlineto{\pgfqpoint{3.559764in}{1.657467in}}%
\pgfpathlineto{\pgfqpoint{3.559764in}{1.657467in}}%
\pgfpathlineto{\pgfqpoint{3.560149in}{1.690062in}}%
\pgfpathlineto{\pgfqpoint{3.560922in}{1.661541in}}%
\pgfpathlineto{\pgfqpoint{3.562072in}{1.681913in}}%
\pgfpathlineto{\pgfqpoint{3.563223in}{1.665615in}}%
\pgfpathlineto{\pgfqpoint{3.564372in}{1.681913in}}%
\pgfpathlineto{\pgfqpoint{3.564756in}{1.669690in}}%
\pgfpathlineto{\pgfqpoint{3.565139in}{1.681913in}}%
\pgfpathlineto{\pgfqpoint{3.565522in}{1.685987in}}%
\pgfpathlineto{\pgfqpoint{3.566670in}{1.665615in}}%
\pgfpathlineto{\pgfqpoint{3.567054in}{1.685987in}}%
\pgfpathlineto{\pgfqpoint{3.567821in}{1.669690in}}%
\pgfpathlineto{\pgfqpoint{3.568970in}{1.677838in}}%
\pgfpathlineto{\pgfqpoint{3.570119in}{1.657467in}}%
\pgfpathlineto{\pgfqpoint{3.570886in}{1.673764in}}%
\pgfpathlineto{\pgfqpoint{3.571269in}{1.665615in}}%
\pgfpathlineto{\pgfqpoint{3.571652in}{1.661541in}}%
\pgfpathlineto{\pgfqpoint{3.573187in}{1.685987in}}%
\pgfpathlineto{\pgfqpoint{3.574719in}{1.653392in}}%
\pgfpathlineto{\pgfqpoint{3.576338in}{1.677838in}}%
\pgfpathlineto{\pgfqpoint{3.576721in}{1.677838in}}%
\pgfpathlineto{\pgfqpoint{3.577104in}{1.669690in}}%
\pgfpathlineto{\pgfqpoint{3.577489in}{1.673764in}}%
\pgfpathlineto{\pgfqpoint{3.577872in}{1.677838in}}%
\pgfpathlineto{\pgfqpoint{3.578255in}{1.669690in}}%
\pgfpathlineto{\pgfqpoint{3.578638in}{1.673764in}}%
\pgfpathlineto{\pgfqpoint{3.579023in}{1.681913in}}%
\pgfpathlineto{\pgfqpoint{3.579406in}{1.673764in}}%
\pgfpathlineto{\pgfqpoint{3.579790in}{1.661541in}}%
\pgfpathlineto{\pgfqpoint{3.579790in}{1.661541in}}%
\pgfpathlineto{\pgfqpoint{3.579790in}{1.661541in}}%
\pgfpathlineto{\pgfqpoint{3.580556in}{1.681913in}}%
\pgfpathlineto{\pgfqpoint{3.580940in}{1.669690in}}%
\pgfpathlineto{\pgfqpoint{3.581707in}{1.677838in}}%
\pgfpathlineto{\pgfqpoint{3.582857in}{1.653392in}}%
\pgfpathlineto{\pgfqpoint{3.583241in}{1.681913in}}%
\pgfpathlineto{\pgfqpoint{3.584008in}{1.661541in}}%
\pgfpathlineto{\pgfqpoint{3.584391in}{1.661541in}}%
\pgfpathlineto{\pgfqpoint{3.584774in}{1.681913in}}%
\pgfpathlineto{\pgfqpoint{3.584774in}{1.681913in}}%
\pgfpathlineto{\pgfqpoint{3.584774in}{1.681913in}}%
\pgfpathlineto{\pgfqpoint{3.585158in}{1.653392in}}%
\pgfpathlineto{\pgfqpoint{3.585925in}{1.665615in}}%
\pgfpathlineto{\pgfqpoint{3.586308in}{1.669690in}}%
\pgfpathlineto{\pgfqpoint{3.586692in}{1.685987in}}%
\pgfpathlineto{\pgfqpoint{3.587075in}{1.669690in}}%
\pgfpathlineto{\pgfqpoint{3.587843in}{1.665615in}}%
\pgfpathlineto{\pgfqpoint{3.589378in}{1.685987in}}%
\pgfpathlineto{\pgfqpoint{3.590529in}{1.649318in}}%
\pgfpathlineto{\pgfqpoint{3.590911in}{1.677838in}}%
\pgfpathlineto{\pgfqpoint{3.591678in}{1.657467in}}%
\pgfpathlineto{\pgfqpoint{3.592829in}{1.677838in}}%
\pgfpathlineto{\pgfqpoint{3.593212in}{1.665615in}}%
\pgfpathlineto{\pgfqpoint{3.593978in}{1.669690in}}%
\pgfpathlineto{\pgfqpoint{3.595129in}{1.685987in}}%
\pgfpathlineto{\pgfqpoint{3.596278in}{1.653392in}}%
\pgfpathlineto{\pgfqpoint{3.597813in}{1.681913in}}%
\pgfpathlineto{\pgfqpoint{3.598196in}{1.665615in}}%
\pgfpathlineto{\pgfqpoint{3.598962in}{1.669690in}}%
\pgfpathlineto{\pgfqpoint{3.599346in}{1.690062in}}%
\pgfpathlineto{\pgfqpoint{3.599731in}{1.669690in}}%
\pgfpathlineto{\pgfqpoint{3.600496in}{1.669690in}}%
\pgfpathlineto{\pgfqpoint{3.601271in}{1.681913in}}%
\pgfpathlineto{\pgfqpoint{3.602804in}{1.653392in}}%
\pgfpathlineto{\pgfqpoint{3.603187in}{1.681913in}}%
\pgfpathlineto{\pgfqpoint{3.603954in}{1.665615in}}%
\pgfpathlineto{\pgfqpoint{3.604337in}{1.677838in}}%
\pgfpathlineto{\pgfqpoint{3.604721in}{1.653392in}}%
\pgfpathlineto{\pgfqpoint{3.605487in}{1.665615in}}%
\pgfpathlineto{\pgfqpoint{3.607022in}{1.694136in}}%
\pgfpathlineto{\pgfqpoint{3.608554in}{1.657467in}}%
\pgfpathlineto{\pgfqpoint{3.610086in}{1.681913in}}%
\pgfpathlineto{\pgfqpoint{3.610853in}{1.677838in}}%
\pgfpathlineto{\pgfqpoint{3.611619in}{1.661541in}}%
\pgfpathlineto{\pgfqpoint{3.612003in}{1.665615in}}%
\pgfpathlineto{\pgfqpoint{3.612771in}{1.690062in}}%
\pgfpathlineto{\pgfqpoint{3.613154in}{1.681913in}}%
\pgfpathlineto{\pgfqpoint{3.614303in}{1.669690in}}%
\pgfpathlineto{\pgfqpoint{3.614686in}{1.690062in}}%
\pgfpathlineto{\pgfqpoint{3.615454in}{1.685987in}}%
\pgfpathlineto{\pgfqpoint{3.615837in}{1.669690in}}%
\pgfpathlineto{\pgfqpoint{3.616602in}{1.677838in}}%
\pgfpathlineto{\pgfqpoint{3.617370in}{1.661541in}}%
\pgfpathlineto{\pgfqpoint{3.617753in}{1.698210in}}%
\pgfpathlineto{\pgfqpoint{3.618519in}{1.681913in}}%
\pgfpathlineto{\pgfqpoint{3.618902in}{1.690062in}}%
\pgfpathlineto{\pgfqpoint{3.619285in}{1.669690in}}%
\pgfpathlineto{\pgfqpoint{3.620052in}{1.681913in}}%
\pgfpathlineto{\pgfqpoint{3.621586in}{1.665615in}}%
\pgfpathlineto{\pgfqpoint{3.622352in}{1.685987in}}%
\pgfpathlineto{\pgfqpoint{3.623587in}{1.653392in}}%
\pgfpathlineto{\pgfqpoint{3.624739in}{1.669690in}}%
\pgfpathlineto{\pgfqpoint{3.625122in}{1.665615in}}%
\pgfpathlineto{\pgfqpoint{3.625506in}{1.681913in}}%
\pgfpathlineto{\pgfqpoint{3.625890in}{1.649318in}}%
\pgfpathlineto{\pgfqpoint{3.626657in}{1.673764in}}%
\pgfpathlineto{\pgfqpoint{3.627040in}{1.685987in}}%
\pgfpathlineto{\pgfqpoint{3.627423in}{1.673764in}}%
\pgfpathlineto{\pgfqpoint{3.627806in}{1.661541in}}%
\pgfpathlineto{\pgfqpoint{3.628189in}{1.669690in}}%
\pgfpathlineto{\pgfqpoint{3.628573in}{1.673764in}}%
\pgfpathlineto{\pgfqpoint{3.628956in}{1.665615in}}%
\pgfpathlineto{\pgfqpoint{3.630106in}{1.681913in}}%
\pgfpathlineto{\pgfqpoint{3.630873in}{1.661541in}}%
\pgfpathlineto{\pgfqpoint{3.631257in}{1.685987in}}%
\pgfpathlineto{\pgfqpoint{3.632023in}{1.677838in}}%
\pgfpathlineto{\pgfqpoint{3.632789in}{1.657467in}}%
\pgfpathlineto{\pgfqpoint{3.633173in}{1.690062in}}%
\pgfpathlineto{\pgfqpoint{3.633939in}{1.665615in}}%
\pgfpathlineto{\pgfqpoint{3.634322in}{1.685987in}}%
\pgfpathlineto{\pgfqpoint{3.634322in}{1.685987in}}%
\pgfpathlineto{\pgfqpoint{3.634322in}{1.685987in}}%
\pgfpathlineto{\pgfqpoint{3.635088in}{1.653392in}}%
\pgfpathlineto{\pgfqpoint{3.635471in}{1.669690in}}%
\pgfpathlineto{\pgfqpoint{3.635855in}{1.690062in}}%
\pgfpathlineto{\pgfqpoint{3.636238in}{1.673764in}}%
\pgfpathlineto{\pgfqpoint{3.636622in}{1.661541in}}%
\pgfpathlineto{\pgfqpoint{3.637005in}{1.673764in}}%
\pgfpathlineto{\pgfqpoint{3.638155in}{1.694136in}}%
\pgfpathlineto{\pgfqpoint{3.638921in}{1.661541in}}%
\pgfpathlineto{\pgfqpoint{3.639305in}{1.673764in}}%
\pgfpathlineto{\pgfqpoint{3.639689in}{1.681913in}}%
\pgfpathlineto{\pgfqpoint{3.640073in}{1.661541in}}%
\pgfpathlineto{\pgfqpoint{3.640847in}{1.673764in}}%
\pgfpathlineto{\pgfqpoint{3.641231in}{1.669690in}}%
\pgfpathlineto{\pgfqpoint{3.641615in}{1.685987in}}%
\pgfpathlineto{\pgfqpoint{3.641615in}{1.685987in}}%
\pgfpathlineto{\pgfqpoint{3.641615in}{1.685987in}}%
\pgfpathlineto{\pgfqpoint{3.643147in}{1.657467in}}%
\pgfpathlineto{\pgfqpoint{3.644296in}{1.685987in}}%
\pgfpathlineto{\pgfqpoint{3.645829in}{1.661541in}}%
\pgfpathlineto{\pgfqpoint{3.646595in}{1.665615in}}%
\pgfpathlineto{\pgfqpoint{3.646979in}{1.669690in}}%
\pgfpathlineto{\pgfqpoint{3.647363in}{1.690062in}}%
\pgfpathlineto{\pgfqpoint{3.647746in}{1.677838in}}%
\pgfpathlineto{\pgfqpoint{3.648129in}{1.661541in}}%
\pgfpathlineto{\pgfqpoint{3.648895in}{1.665615in}}%
\pgfpathlineto{\pgfqpoint{3.649279in}{1.677838in}}%
\pgfpathlineto{\pgfqpoint{3.649279in}{1.677838in}}%
\pgfpathlineto{\pgfqpoint{3.649279in}{1.677838in}}%
\pgfpathlineto{\pgfqpoint{3.649664in}{1.657467in}}%
\pgfpathlineto{\pgfqpoint{3.650046in}{1.665615in}}%
\pgfpathlineto{\pgfqpoint{3.650429in}{1.677838in}}%
\pgfpathlineto{\pgfqpoint{3.651196in}{1.673764in}}%
\pgfpathlineto{\pgfqpoint{3.651579in}{1.661541in}}%
\pgfpathlineto{\pgfqpoint{3.651963in}{1.685987in}}%
\pgfpathlineto{\pgfqpoint{3.652730in}{1.677838in}}%
\pgfpathlineto{\pgfqpoint{3.653113in}{1.669690in}}%
\pgfpathlineto{\pgfqpoint{3.653880in}{1.685987in}}%
\pgfpathlineto{\pgfqpoint{3.654263in}{1.677838in}}%
\pgfpathlineto{\pgfqpoint{3.654647in}{1.665615in}}%
\pgfpathlineto{\pgfqpoint{3.654647in}{1.665615in}}%
\pgfpathlineto{\pgfqpoint{3.654647in}{1.665615in}}%
\pgfpathlineto{\pgfqpoint{3.655031in}{1.685987in}}%
\pgfpathlineto{\pgfqpoint{3.655798in}{1.681913in}}%
\pgfpathlineto{\pgfqpoint{3.656180in}{1.665615in}}%
\pgfpathlineto{\pgfqpoint{3.656947in}{1.673764in}}%
\pgfpathlineto{\pgfqpoint{3.658098in}{1.677838in}}%
\pgfpathlineto{\pgfqpoint{3.658481in}{1.669690in}}%
\pgfpathlineto{\pgfqpoint{3.658864in}{1.673764in}}%
\pgfpathlineto{\pgfqpoint{3.659247in}{1.677838in}}%
\pgfpathlineto{\pgfqpoint{3.659630in}{1.657467in}}%
\pgfpathlineto{\pgfqpoint{3.660398in}{1.661541in}}%
\pgfpathlineto{\pgfqpoint{3.661929in}{1.690062in}}%
\pgfpathlineto{\pgfqpoint{3.662697in}{1.661541in}}%
\pgfpathlineto{\pgfqpoint{3.663081in}{1.677838in}}%
\pgfpathlineto{\pgfqpoint{3.663846in}{1.665615in}}%
\pgfpathlineto{\pgfqpoint{3.664229in}{1.673764in}}%
\pgfpathlineto{\pgfqpoint{3.664613in}{1.673764in}}%
\pgfpathlineto{\pgfqpoint{3.665764in}{1.661541in}}%
\pgfpathlineto{\pgfqpoint{3.666147in}{1.681913in}}%
\pgfpathlineto{\pgfqpoint{3.666147in}{1.681913in}}%
\pgfpathlineto{\pgfqpoint{3.666147in}{1.681913in}}%
\pgfpathlineto{\pgfqpoint{3.666530in}{1.657467in}}%
\pgfpathlineto{\pgfqpoint{3.666530in}{1.657467in}}%
\pgfpathlineto{\pgfqpoint{3.666530in}{1.657467in}}%
\pgfpathlineto{\pgfqpoint{3.666913in}{1.685987in}}%
\pgfpathlineto{\pgfqpoint{3.667679in}{1.665615in}}%
\pgfpathlineto{\pgfqpoint{3.668914in}{1.685987in}}%
\pgfpathlineto{\pgfqpoint{3.669297in}{1.653392in}}%
\pgfpathlineto{\pgfqpoint{3.670064in}{1.665615in}}%
\pgfpathlineto{\pgfqpoint{3.670446in}{1.661541in}}%
\pgfpathlineto{\pgfqpoint{3.670829in}{1.685987in}}%
\pgfpathlineto{\pgfqpoint{3.670829in}{1.685987in}}%
\pgfpathlineto{\pgfqpoint{3.670829in}{1.685987in}}%
\pgfpathlineto{\pgfqpoint{3.671212in}{1.657467in}}%
\pgfpathlineto{\pgfqpoint{3.671980in}{1.673764in}}%
\pgfpathlineto{\pgfqpoint{3.672364in}{1.665615in}}%
\pgfpathlineto{\pgfqpoint{3.672747in}{1.681913in}}%
\pgfpathlineto{\pgfqpoint{3.673513in}{1.673764in}}%
\pgfpathlineto{\pgfqpoint{3.673897in}{1.677838in}}%
\pgfpathlineto{\pgfqpoint{3.675047in}{1.669690in}}%
\pgfpathlineto{\pgfqpoint{3.676196in}{1.685987in}}%
\pgfpathlineto{\pgfqpoint{3.677346in}{1.665615in}}%
\pgfpathlineto{\pgfqpoint{3.677730in}{1.673764in}}%
\pgfpathlineto{\pgfqpoint{3.678496in}{1.669690in}}%
\pgfpathlineto{\pgfqpoint{3.679262in}{1.677838in}}%
\pgfpathlineto{\pgfqpoint{3.679645in}{1.665615in}}%
\pgfpathlineto{\pgfqpoint{3.679645in}{1.665615in}}%
\pgfpathlineto{\pgfqpoint{3.679645in}{1.665615in}}%
\pgfpathlineto{\pgfqpoint{3.680028in}{1.681913in}}%
\pgfpathlineto{\pgfqpoint{3.680803in}{1.673764in}}%
\pgfpathlineto{\pgfqpoint{3.681187in}{1.681913in}}%
\pgfpathlineto{\pgfqpoint{3.682336in}{1.661541in}}%
\pgfpathlineto{\pgfqpoint{3.683101in}{1.685987in}}%
\pgfpathlineto{\pgfqpoint{3.683484in}{1.677838in}}%
\pgfpathlineto{\pgfqpoint{3.683867in}{1.673764in}}%
\pgfpathlineto{\pgfqpoint{3.684250in}{1.685987in}}%
\pgfpathlineto{\pgfqpoint{3.684634in}{1.657467in}}%
\pgfpathlineto{\pgfqpoint{3.685401in}{1.681913in}}%
\pgfpathlineto{\pgfqpoint{3.686550in}{1.665615in}}%
\pgfpathlineto{\pgfqpoint{3.686933in}{1.677838in}}%
\pgfpathlineto{\pgfqpoint{3.687317in}{1.669690in}}%
\pgfpathlineto{\pgfqpoint{3.687701in}{1.661541in}}%
\pgfpathlineto{\pgfqpoint{3.688467in}{1.681913in}}%
\pgfpathlineto{\pgfqpoint{3.688850in}{1.649318in}}%
\pgfpathlineto{\pgfqpoint{3.689617in}{1.665615in}}%
\pgfpathlineto{\pgfqpoint{3.691149in}{1.677838in}}%
\pgfpathlineto{\pgfqpoint{3.691916in}{1.657467in}}%
\pgfpathlineto{\pgfqpoint{3.692300in}{1.669690in}}%
\pgfpathlineto{\pgfqpoint{3.693066in}{1.681913in}}%
\pgfpathlineto{\pgfqpoint{3.693448in}{1.677838in}}%
\pgfpathlineto{\pgfqpoint{3.694598in}{1.661541in}}%
\pgfpathlineto{\pgfqpoint{3.694981in}{1.677838in}}%
\pgfpathlineto{\pgfqpoint{3.694981in}{1.677838in}}%
\pgfpathlineto{\pgfqpoint{3.694981in}{1.677838in}}%
\pgfpathlineto{\pgfqpoint{3.695365in}{1.657467in}}%
\pgfpathlineto{\pgfqpoint{3.696131in}{1.669690in}}%
\pgfpathlineto{\pgfqpoint{3.696515in}{1.665615in}}%
\pgfpathlineto{\pgfqpoint{3.696899in}{1.669690in}}%
\pgfpathlineto{\pgfqpoint{3.697282in}{1.669690in}}%
\pgfpathlineto{\pgfqpoint{3.698817in}{1.685987in}}%
\pgfpathlineto{\pgfqpoint{3.699966in}{1.661541in}}%
\pgfpathlineto{\pgfqpoint{3.700734in}{1.690062in}}%
\pgfpathlineto{\pgfqpoint{3.701117in}{1.677838in}}%
\pgfpathlineto{\pgfqpoint{3.701501in}{1.677838in}}%
\pgfpathlineto{\pgfqpoint{3.701884in}{1.665615in}}%
\pgfpathlineto{\pgfqpoint{3.702649in}{1.669690in}}%
\pgfpathlineto{\pgfqpoint{3.703034in}{1.681913in}}%
\pgfpathlineto{\pgfqpoint{3.703801in}{1.677838in}}%
\pgfpathlineto{\pgfqpoint{3.704184in}{1.673764in}}%
\pgfpathlineto{\pgfqpoint{3.704567in}{1.677838in}}%
\pgfpathlineto{\pgfqpoint{3.704950in}{1.653392in}}%
\pgfpathlineto{\pgfqpoint{3.705717in}{1.669690in}}%
\pgfpathlineto{\pgfqpoint{3.706100in}{1.690062in}}%
\pgfpathlineto{\pgfqpoint{3.706866in}{1.685987in}}%
\pgfpathlineto{\pgfqpoint{3.707250in}{1.685987in}}%
\pgfpathlineto{\pgfqpoint{3.707634in}{1.657467in}}%
\pgfpathlineto{\pgfqpoint{3.708400in}{1.669690in}}%
\pgfpathlineto{\pgfqpoint{3.709167in}{1.681913in}}%
\pgfpathlineto{\pgfqpoint{3.710020in}{1.657467in}}%
\pgfpathlineto{\pgfqpoint{3.710404in}{1.673764in}}%
\pgfpathlineto{\pgfqpoint{3.711170in}{1.690062in}}%
\pgfpathlineto{\pgfqpoint{3.711553in}{1.681913in}}%
\pgfpathlineto{\pgfqpoint{3.711936in}{1.681913in}}%
\pgfpathlineto{\pgfqpoint{3.713087in}{1.669690in}}%
\pgfpathlineto{\pgfqpoint{3.713852in}{1.677838in}}%
\pgfpathlineto{\pgfqpoint{3.714236in}{1.657467in}}%
\pgfpathlineto{\pgfqpoint{3.714619in}{1.661541in}}%
\pgfpathlineto{\pgfqpoint{3.716150in}{1.690062in}}%
\pgfpathlineto{\pgfqpoint{3.716916in}{1.657467in}}%
\pgfpathlineto{\pgfqpoint{3.717299in}{1.673764in}}%
\pgfpathlineto{\pgfqpoint{3.717683in}{1.681913in}}%
\pgfpathlineto{\pgfqpoint{3.718448in}{1.677838in}}%
\pgfpathlineto{\pgfqpoint{3.719980in}{1.661541in}}%
\pgfpathlineto{\pgfqpoint{3.720364in}{1.673764in}}%
\pgfpathlineto{\pgfqpoint{3.720364in}{1.673764in}}%
\pgfpathlineto{\pgfqpoint{3.720364in}{1.673764in}}%
\pgfpathlineto{\pgfqpoint{3.720755in}{1.653392in}}%
\pgfpathlineto{\pgfqpoint{3.721521in}{1.665615in}}%
\pgfpathlineto{\pgfqpoint{3.721905in}{1.661541in}}%
\pgfpathlineto{\pgfqpoint{3.722289in}{1.673764in}}%
\pgfpathlineto{\pgfqpoint{3.722289in}{1.673764in}}%
\pgfpathlineto{\pgfqpoint{3.722289in}{1.673764in}}%
\pgfpathlineto{\pgfqpoint{3.722672in}{1.653392in}}%
\pgfpathlineto{\pgfqpoint{3.722672in}{1.653392in}}%
\pgfpathlineto{\pgfqpoint{3.722672in}{1.653392in}}%
\pgfpathlineto{\pgfqpoint{3.723055in}{1.685987in}}%
\pgfpathlineto{\pgfqpoint{3.723439in}{1.677838in}}%
\pgfpathlineto{\pgfqpoint{3.723822in}{1.653392in}}%
\pgfpathlineto{\pgfqpoint{3.724588in}{1.673764in}}%
\pgfpathlineto{\pgfqpoint{3.724972in}{1.677838in}}%
\pgfpathlineto{\pgfqpoint{3.725355in}{1.665615in}}%
\pgfpathlineto{\pgfqpoint{3.725740in}{1.673764in}}%
\pgfpathlineto{\pgfqpoint{3.726888in}{1.694136in}}%
\pgfpathlineto{\pgfqpoint{3.727271in}{1.685987in}}%
\pgfpathlineto{\pgfqpoint{3.728423in}{1.665615in}}%
\pgfpathlineto{\pgfqpoint{3.728806in}{1.657467in}}%
\pgfpathlineto{\pgfqpoint{3.729571in}{1.690062in}}%
\pgfpathlineto{\pgfqpoint{3.729954in}{1.657467in}}%
\pgfpathlineto{\pgfqpoint{3.730723in}{1.673764in}}%
\pgfpathlineto{\pgfqpoint{3.731106in}{1.694136in}}%
\pgfpathlineto{\pgfqpoint{3.731489in}{1.677838in}}%
\pgfpathlineto{\pgfqpoint{3.731872in}{1.669690in}}%
\pgfpathlineto{\pgfqpoint{3.732254in}{1.677838in}}%
\pgfpathlineto{\pgfqpoint{3.733405in}{1.698210in}}%
\pgfpathlineto{\pgfqpoint{3.733788in}{1.661541in}}%
\pgfpathlineto{\pgfqpoint{3.734554in}{1.665615in}}%
\pgfpathlineto{\pgfqpoint{3.736088in}{1.706359in}}%
\pgfpathlineto{\pgfqpoint{3.736472in}{1.677838in}}%
\pgfpathlineto{\pgfqpoint{3.737236in}{1.681913in}}%
\pgfpathlineto{\pgfqpoint{3.737619in}{1.677838in}}%
\pgfpathlineto{\pgfqpoint{3.738002in}{1.690062in}}%
\pgfpathlineto{\pgfqpoint{3.738002in}{1.690062in}}%
\pgfpathlineto{\pgfqpoint{3.738002in}{1.690062in}}%
\pgfpathlineto{\pgfqpoint{3.738385in}{1.673764in}}%
\pgfpathlineto{\pgfqpoint{3.739152in}{1.681913in}}%
\pgfpathlineto{\pgfqpoint{3.739536in}{1.681913in}}%
\pgfpathlineto{\pgfqpoint{3.739919in}{1.690062in}}%
\pgfpathlineto{\pgfqpoint{3.739919in}{1.690062in}}%
\pgfpathlineto{\pgfqpoint{3.739919in}{1.690062in}}%
\pgfpathlineto{\pgfqpoint{3.741069in}{1.661541in}}%
\pgfpathlineto{\pgfqpoint{3.741453in}{1.685987in}}%
\pgfpathlineto{\pgfqpoint{3.741837in}{1.661541in}}%
\pgfpathlineto{\pgfqpoint{3.742220in}{1.657467in}}%
\pgfpathlineto{\pgfqpoint{3.743753in}{1.677838in}}%
\pgfpathlineto{\pgfqpoint{3.745284in}{1.669690in}}%
\pgfpathlineto{\pgfqpoint{3.746052in}{1.669690in}}%
\pgfpathlineto{\pgfqpoint{3.746436in}{1.677838in}}%
\pgfpathlineto{\pgfqpoint{3.747203in}{1.673764in}}%
\pgfpathlineto{\pgfqpoint{3.747586in}{1.677838in}}%
\pgfpathlineto{\pgfqpoint{3.747968in}{1.673764in}}%
\pgfpathlineto{\pgfqpoint{3.748351in}{1.665615in}}%
\pgfpathlineto{\pgfqpoint{3.748734in}{1.681913in}}%
\pgfpathlineto{\pgfqpoint{3.748734in}{1.681913in}}%
\pgfpathlineto{\pgfqpoint{3.748734in}{1.681913in}}%
\pgfpathlineto{\pgfqpoint{3.749117in}{1.661541in}}%
\pgfpathlineto{\pgfqpoint{3.749501in}{1.681913in}}%
\pgfpathlineto{\pgfqpoint{3.750351in}{1.665615in}}%
\pgfpathlineto{\pgfqpoint{3.750740in}{1.690062in}}%
\pgfpathlineto{\pgfqpoint{3.750740in}{1.690062in}}%
\pgfpathlineto{\pgfqpoint{3.750740in}{1.690062in}}%
\pgfpathlineto{\pgfqpoint{3.751122in}{1.657467in}}%
\pgfpathlineto{\pgfqpoint{3.751504in}{1.677838in}}%
\pgfpathlineto{\pgfqpoint{3.751887in}{1.694136in}}%
\pgfpathlineto{\pgfqpoint{3.752270in}{1.685987in}}%
\pgfpathlineto{\pgfqpoint{3.752654in}{1.657467in}}%
\pgfpathlineto{\pgfqpoint{3.753421in}{1.661541in}}%
\pgfpathlineto{\pgfqpoint{3.754953in}{1.685987in}}%
\pgfpathlineto{\pgfqpoint{3.755336in}{1.657467in}}%
\pgfpathlineto{\pgfqpoint{3.756103in}{1.661541in}}%
\pgfpathlineto{\pgfqpoint{3.757251in}{1.677838in}}%
\pgfpathlineto{\pgfqpoint{3.757634in}{1.677838in}}%
\pgfpathlineto{\pgfqpoint{3.758017in}{1.681913in}}%
\pgfpathlineto{\pgfqpoint{3.758401in}{1.677838in}}%
\pgfpathlineto{\pgfqpoint{3.758784in}{1.669690in}}%
\pgfpathlineto{\pgfqpoint{3.759167in}{1.685987in}}%
\pgfpathlineto{\pgfqpoint{3.759549in}{1.669690in}}%
\pgfpathlineto{\pgfqpoint{3.759932in}{1.653392in}}%
\pgfpathlineto{\pgfqpoint{3.760316in}{1.661541in}}%
\pgfpathlineto{\pgfqpoint{3.760707in}{1.669690in}}%
\pgfpathlineto{\pgfqpoint{3.761090in}{1.661541in}}%
\pgfpathlineto{\pgfqpoint{3.761856in}{1.653392in}}%
\pgfpathlineto{\pgfqpoint{3.763009in}{1.685987in}}%
\pgfpathlineto{\pgfqpoint{3.763392in}{1.673764in}}%
\pgfpathlineto{\pgfqpoint{3.763774in}{1.661541in}}%
\pgfpathlineto{\pgfqpoint{3.764158in}{1.673764in}}%
\pgfpathlineto{\pgfqpoint{3.764541in}{1.673764in}}%
\pgfpathlineto{\pgfqpoint{3.764924in}{1.681913in}}%
\pgfpathlineto{\pgfqpoint{3.765307in}{1.677838in}}%
\pgfpathlineto{\pgfqpoint{3.765690in}{1.673764in}}%
\pgfpathlineto{\pgfqpoint{3.766073in}{1.694136in}}%
\pgfpathlineto{\pgfqpoint{3.766455in}{1.677838in}}%
\pgfpathlineto{\pgfqpoint{3.767221in}{1.657467in}}%
\pgfpathlineto{\pgfqpoint{3.767604in}{1.661541in}}%
\pgfpathlineto{\pgfqpoint{3.767988in}{1.685987in}}%
\pgfpathlineto{\pgfqpoint{3.768754in}{1.681913in}}%
\pgfpathlineto{\pgfqpoint{3.770285in}{1.649318in}}%
\pgfpathlineto{\pgfqpoint{3.770668in}{1.690062in}}%
\pgfpathlineto{\pgfqpoint{3.771435in}{1.669690in}}%
\pgfpathlineto{\pgfqpoint{3.771818in}{1.661541in}}%
\pgfpathlineto{\pgfqpoint{3.772201in}{1.665615in}}%
\pgfpathlineto{\pgfqpoint{3.772583in}{1.681913in}}%
\pgfpathlineto{\pgfqpoint{3.773348in}{1.669690in}}%
\pgfpathlineto{\pgfqpoint{3.773732in}{1.669690in}}%
\pgfpathlineto{\pgfqpoint{3.774498in}{1.690062in}}%
\pgfpathlineto{\pgfqpoint{3.776032in}{1.661541in}}%
\pgfpathlineto{\pgfqpoint{3.776798in}{1.669690in}}%
\pgfpathlineto{\pgfqpoint{3.777182in}{1.685987in}}%
\pgfpathlineto{\pgfqpoint{3.777182in}{1.685987in}}%
\pgfpathlineto{\pgfqpoint{3.777182in}{1.685987in}}%
\pgfpathlineto{\pgfqpoint{3.778332in}{1.657467in}}%
\pgfpathlineto{\pgfqpoint{3.779097in}{1.685987in}}%
\pgfpathlineto{\pgfqpoint{3.779480in}{1.669690in}}%
\pgfpathlineto{\pgfqpoint{3.779863in}{1.673764in}}%
\pgfpathlineto{\pgfqpoint{3.780247in}{1.669690in}}%
\pgfpathlineto{\pgfqpoint{3.780631in}{1.698210in}}%
\pgfpathlineto{\pgfqpoint{3.781014in}{1.673764in}}%
\pgfpathlineto{\pgfqpoint{3.781397in}{1.665615in}}%
\pgfpathlineto{\pgfqpoint{3.781397in}{1.665615in}}%
\pgfpathlineto{\pgfqpoint{3.781397in}{1.665615in}}%
\pgfpathlineto{\pgfqpoint{3.781780in}{1.677838in}}%
\pgfpathlineto{\pgfqpoint{3.782546in}{1.669690in}}%
\pgfpathlineto{\pgfqpoint{3.783315in}{1.657467in}}%
\pgfpathlineto{\pgfqpoint{3.784461in}{1.677838in}}%
\pgfpathlineto{\pgfqpoint{3.784844in}{1.669690in}}%
\pgfpathlineto{\pgfqpoint{3.785227in}{1.677838in}}%
\pgfpathlineto{\pgfqpoint{3.785994in}{1.690062in}}%
\pgfpathlineto{\pgfqpoint{3.786463in}{1.657467in}}%
\pgfpathlineto{\pgfqpoint{3.787230in}{1.661541in}}%
\pgfpathlineto{\pgfqpoint{3.787997in}{1.681913in}}%
\pgfpathlineto{\pgfqpoint{3.788380in}{1.665615in}}%
\pgfpathlineto{\pgfqpoint{3.788763in}{1.661541in}}%
\pgfpathlineto{\pgfqpoint{3.789147in}{1.690062in}}%
\pgfpathlineto{\pgfqpoint{3.789914in}{1.665615in}}%
\pgfpathlineto{\pgfqpoint{3.790297in}{1.669690in}}%
\pgfpathlineto{\pgfqpoint{3.790680in}{1.661541in}}%
\pgfpathlineto{\pgfqpoint{3.790680in}{1.661541in}}%
\pgfpathlineto{\pgfqpoint{3.790680in}{1.661541in}}%
\pgfpathlineto{\pgfqpoint{3.791063in}{1.673764in}}%
\pgfpathlineto{\pgfqpoint{3.791830in}{1.669690in}}%
\pgfpathlineto{\pgfqpoint{3.792213in}{1.661541in}}%
\pgfpathlineto{\pgfqpoint{3.792596in}{1.669690in}}%
\pgfpathlineto{\pgfqpoint{3.793745in}{1.681913in}}%
\pgfpathlineto{\pgfqpoint{3.795279in}{1.661541in}}%
\pgfpathlineto{\pgfqpoint{3.796044in}{1.690062in}}%
\pgfpathlineto{\pgfqpoint{3.796810in}{1.677838in}}%
\pgfpathlineto{\pgfqpoint{3.797193in}{1.673764in}}%
\pgfpathlineto{\pgfqpoint{3.797577in}{1.690062in}}%
\pgfpathlineto{\pgfqpoint{3.798342in}{1.677838in}}%
\pgfpathlineto{\pgfqpoint{3.798726in}{1.690062in}}%
\pgfpathlineto{\pgfqpoint{3.799109in}{1.677838in}}%
\pgfpathlineto{\pgfqpoint{3.799875in}{1.661541in}}%
\pgfpathlineto{\pgfqpoint{3.800258in}{1.673764in}}%
\pgfpathlineto{\pgfqpoint{3.800649in}{1.690062in}}%
\pgfpathlineto{\pgfqpoint{3.800649in}{1.690062in}}%
\pgfpathlineto{\pgfqpoint{3.800649in}{1.690062in}}%
\pgfpathlineto{\pgfqpoint{3.801033in}{1.669690in}}%
\pgfpathlineto{\pgfqpoint{3.801416in}{1.673764in}}%
\pgfpathlineto{\pgfqpoint{3.801799in}{1.694136in}}%
\pgfpathlineto{\pgfqpoint{3.802183in}{1.673764in}}%
\pgfpathlineto{\pgfqpoint{3.802566in}{1.669690in}}%
\pgfpathlineto{\pgfqpoint{3.802949in}{1.673764in}}%
\pgfpathlineto{\pgfqpoint{3.803716in}{1.685987in}}%
\pgfpathlineto{\pgfqpoint{3.804099in}{1.657467in}}%
\pgfpathlineto{\pgfqpoint{3.804866in}{1.673764in}}%
\pgfpathlineto{\pgfqpoint{3.805633in}{1.661541in}}%
\pgfpathlineto{\pgfqpoint{3.806017in}{1.669690in}}%
\pgfpathlineto{\pgfqpoint{3.806782in}{1.661541in}}%
\pgfpathlineto{\pgfqpoint{3.807549in}{1.685987in}}%
\pgfpathlineto{\pgfqpoint{3.807933in}{1.661541in}}%
\pgfpathlineto{\pgfqpoint{3.808699in}{1.669690in}}%
\pgfpathlineto{\pgfqpoint{3.809082in}{1.661541in}}%
\pgfpathlineto{\pgfqpoint{3.809465in}{1.677838in}}%
\pgfpathlineto{\pgfqpoint{3.810232in}{1.669690in}}%
\pgfpathlineto{\pgfqpoint{3.810615in}{1.677838in}}%
\pgfpathlineto{\pgfqpoint{3.811382in}{1.673764in}}%
\pgfpathlineto{\pgfqpoint{3.811765in}{1.665615in}}%
\pgfpathlineto{\pgfqpoint{3.812533in}{1.681913in}}%
\pgfpathlineto{\pgfqpoint{3.812916in}{1.665615in}}%
\pgfpathlineto{\pgfqpoint{3.813683in}{1.677838in}}%
\pgfpathlineto{\pgfqpoint{3.814065in}{1.681913in}}%
\pgfpathlineto{\pgfqpoint{3.814448in}{1.677838in}}%
\pgfpathlineto{\pgfqpoint{3.814832in}{1.677838in}}%
\pgfpathlineto{\pgfqpoint{3.815216in}{1.653392in}}%
\pgfpathlineto{\pgfqpoint{3.815983in}{1.673764in}}%
\pgfpathlineto{\pgfqpoint{3.817518in}{1.661541in}}%
\pgfpathlineto{\pgfqpoint{3.817902in}{1.694136in}}%
\pgfpathlineto{\pgfqpoint{3.818668in}{1.673764in}}%
\pgfpathlineto{\pgfqpoint{3.819052in}{1.673764in}}%
\pgfpathlineto{\pgfqpoint{3.819818in}{1.665615in}}%
\pgfpathlineto{\pgfqpoint{3.820969in}{1.673764in}}%
\pgfpathlineto{\pgfqpoint{3.821735in}{1.661541in}}%
\pgfpathlineto{\pgfqpoint{3.822501in}{1.685987in}}%
\pgfpathlineto{\pgfqpoint{3.822884in}{1.681913in}}%
\pgfpathlineto{\pgfqpoint{3.824035in}{1.665615in}}%
\pgfpathlineto{\pgfqpoint{3.825185in}{1.685987in}}%
\pgfpathlineto{\pgfqpoint{3.826719in}{1.665615in}}%
\pgfpathlineto{\pgfqpoint{3.827102in}{1.677838in}}%
\pgfpathlineto{\pgfqpoint{3.827102in}{1.677838in}}%
\pgfpathlineto{\pgfqpoint{3.827102in}{1.677838in}}%
\pgfpathlineto{\pgfqpoint{3.827485in}{1.661541in}}%
\pgfpathlineto{\pgfqpoint{3.827485in}{1.661541in}}%
\pgfpathlineto{\pgfqpoint{3.827485in}{1.661541in}}%
\pgfpathlineto{\pgfqpoint{3.828636in}{1.685987in}}%
\pgfpathlineto{\pgfqpoint{3.829403in}{1.657467in}}%
\pgfpathlineto{\pgfqpoint{3.829871in}{1.677838in}}%
\pgfpathlineto{\pgfqpoint{3.830253in}{1.673764in}}%
\pgfpathlineto{\pgfqpoint{3.830636in}{1.681913in}}%
\pgfpathlineto{\pgfqpoint{3.830636in}{1.681913in}}%
\pgfpathlineto{\pgfqpoint{3.830636in}{1.681913in}}%
\pgfpathlineto{\pgfqpoint{3.831020in}{1.669690in}}%
\pgfpathlineto{\pgfqpoint{3.831403in}{1.681913in}}%
\pgfpathlineto{\pgfqpoint{3.831787in}{1.685987in}}%
\pgfpathlineto{\pgfqpoint{3.832171in}{1.649318in}}%
\pgfpathlineto{\pgfqpoint{3.832553in}{1.677838in}}%
\pgfpathlineto{\pgfqpoint{3.832936in}{1.685987in}}%
\pgfpathlineto{\pgfqpoint{3.834088in}{1.669690in}}%
\pgfpathlineto{\pgfqpoint{3.834471in}{1.669690in}}%
\pgfpathlineto{\pgfqpoint{3.835237in}{1.677838in}}%
\pgfpathlineto{\pgfqpoint{3.836388in}{1.661541in}}%
\pgfpathlineto{\pgfqpoint{3.837538in}{1.677838in}}%
\pgfpathlineto{\pgfqpoint{3.837922in}{1.657467in}}%
\pgfpathlineto{\pgfqpoint{3.838687in}{1.669690in}}%
\pgfpathlineto{\pgfqpoint{3.839070in}{1.657467in}}%
\pgfpathlineto{\pgfqpoint{3.839070in}{1.657467in}}%
\pgfpathlineto{\pgfqpoint{3.839070in}{1.657467in}}%
\pgfpathlineto{\pgfqpoint{3.840994in}{1.698210in}}%
\pgfpathlineto{\pgfqpoint{3.842529in}{1.653392in}}%
\pgfpathlineto{\pgfqpoint{3.843678in}{1.694136in}}%
\pgfpathlineto{\pgfqpoint{3.844829in}{1.653392in}}%
\pgfpathlineto{\pgfqpoint{3.846361in}{1.685987in}}%
\pgfpathlineto{\pgfqpoint{3.846745in}{1.681913in}}%
\pgfpathlineto{\pgfqpoint{3.847896in}{1.669690in}}%
\pgfpathlineto{\pgfqpoint{3.848278in}{1.681913in}}%
\pgfpathlineto{\pgfqpoint{3.849045in}{1.677838in}}%
\pgfpathlineto{\pgfqpoint{3.849812in}{1.661541in}}%
\pgfpathlineto{\pgfqpoint{3.850579in}{1.690062in}}%
\pgfpathlineto{\pgfqpoint{3.850962in}{1.677838in}}%
\pgfpathlineto{\pgfqpoint{3.851730in}{1.661541in}}%
\pgfpathlineto{\pgfqpoint{3.852114in}{1.669690in}}%
\pgfpathlineto{\pgfqpoint{3.852497in}{1.673764in}}%
\pgfpathlineto{\pgfqpoint{3.853263in}{1.665615in}}%
\pgfpathlineto{\pgfqpoint{3.854797in}{1.690062in}}%
\pgfpathlineto{\pgfqpoint{3.856330in}{1.657467in}}%
\pgfpathlineto{\pgfqpoint{3.856713in}{1.665615in}}%
\pgfpathlineto{\pgfqpoint{3.857097in}{1.673764in}}%
\pgfpathlineto{\pgfqpoint{3.857480in}{1.649318in}}%
\pgfpathlineto{\pgfqpoint{3.858246in}{1.665615in}}%
\pgfpathlineto{\pgfqpoint{3.858631in}{1.665615in}}%
\pgfpathlineto{\pgfqpoint{3.859016in}{1.681913in}}%
\pgfpathlineto{\pgfqpoint{3.859783in}{1.673764in}}%
\pgfpathlineto{\pgfqpoint{3.860166in}{1.673764in}}%
\pgfpathlineto{\pgfqpoint{3.860550in}{1.669690in}}%
\pgfpathlineto{\pgfqpoint{3.860933in}{1.685987in}}%
\pgfpathlineto{\pgfqpoint{3.861699in}{1.673764in}}%
\pgfpathlineto{\pgfqpoint{3.862850in}{1.665615in}}%
\pgfpathlineto{\pgfqpoint{3.863999in}{1.673764in}}%
\pgfpathlineto{\pgfqpoint{3.864382in}{1.657467in}}%
\pgfpathlineto{\pgfqpoint{3.864766in}{1.698210in}}%
\pgfpathlineto{\pgfqpoint{3.865534in}{1.669690in}}%
\pgfpathlineto{\pgfqpoint{3.865918in}{1.661541in}}%
\pgfpathlineto{\pgfqpoint{3.865918in}{1.661541in}}%
\pgfpathlineto{\pgfqpoint{3.865918in}{1.661541in}}%
\pgfpathlineto{\pgfqpoint{3.867067in}{1.673764in}}%
\pgfpathlineto{\pgfqpoint{3.867452in}{1.661541in}}%
\pgfpathlineto{\pgfqpoint{3.867836in}{1.665615in}}%
\pgfpathlineto{\pgfqpoint{3.868602in}{1.681913in}}%
\pgfpathlineto{\pgfqpoint{3.868984in}{1.673764in}}%
\pgfpathlineto{\pgfqpoint{3.869367in}{1.673764in}}%
\pgfpathlineto{\pgfqpoint{3.869751in}{1.681913in}}%
\pgfpathlineto{\pgfqpoint{3.870136in}{1.677838in}}%
\pgfpathlineto{\pgfqpoint{3.870519in}{1.661541in}}%
\pgfpathlineto{\pgfqpoint{3.871285in}{1.673764in}}%
\pgfpathlineto{\pgfqpoint{3.871668in}{1.677838in}}%
\pgfpathlineto{\pgfqpoint{3.872435in}{1.669690in}}%
\pgfpathlineto{\pgfqpoint{3.872819in}{1.685987in}}%
\pgfpathlineto{\pgfqpoint{3.873202in}{1.681913in}}%
\pgfpathlineto{\pgfqpoint{3.873968in}{1.661541in}}%
\pgfpathlineto{\pgfqpoint{3.874352in}{1.669690in}}%
\pgfpathlineto{\pgfqpoint{3.874735in}{1.665615in}}%
\pgfpathlineto{\pgfqpoint{3.875502in}{1.685987in}}%
\pgfpathlineto{\pgfqpoint{3.875886in}{1.681913in}}%
\pgfpathlineto{\pgfqpoint{3.876650in}{1.653392in}}%
\pgfpathlineto{\pgfqpoint{3.877119in}{1.681913in}}%
\pgfpathlineto{\pgfqpoint{3.877885in}{1.669690in}}%
\pgfpathlineto{\pgfqpoint{3.878268in}{1.673764in}}%
\pgfpathlineto{\pgfqpoint{3.878652in}{1.665615in}}%
\pgfpathlineto{\pgfqpoint{3.878652in}{1.665615in}}%
\pgfpathlineto{\pgfqpoint{3.878652in}{1.665615in}}%
\pgfpathlineto{\pgfqpoint{3.879036in}{1.677838in}}%
\pgfpathlineto{\pgfqpoint{3.879419in}{1.673764in}}%
\pgfpathlineto{\pgfqpoint{3.879803in}{1.649318in}}%
\pgfpathlineto{\pgfqpoint{3.879803in}{1.649318in}}%
\pgfpathlineto{\pgfqpoint{3.879803in}{1.649318in}}%
\pgfpathlineto{\pgfqpoint{3.881344in}{1.681913in}}%
\pgfpathlineto{\pgfqpoint{3.881727in}{1.669690in}}%
\pgfpathlineto{\pgfqpoint{3.882493in}{1.677838in}}%
\pgfpathlineto{\pgfqpoint{3.882876in}{1.677838in}}%
\pgfpathlineto{\pgfqpoint{3.883260in}{1.669690in}}%
\pgfpathlineto{\pgfqpoint{3.883644in}{1.677838in}}%
\pgfpathlineto{\pgfqpoint{3.884027in}{1.677838in}}%
\pgfpathlineto{\pgfqpoint{3.884410in}{1.653392in}}%
\pgfpathlineto{\pgfqpoint{3.884795in}{1.677838in}}%
\pgfpathlineto{\pgfqpoint{3.885178in}{1.677838in}}%
\pgfpathlineto{\pgfqpoint{3.885561in}{1.685987in}}%
\pgfpathlineto{\pgfqpoint{3.885946in}{1.669690in}}%
\pgfpathlineto{\pgfqpoint{3.886713in}{1.681913in}}%
\pgfpathlineto{\pgfqpoint{3.887096in}{1.681913in}}%
\pgfpathlineto{\pgfqpoint{3.887479in}{1.669690in}}%
\pgfpathlineto{\pgfqpoint{3.887479in}{1.669690in}}%
\pgfpathlineto{\pgfqpoint{3.887479in}{1.669690in}}%
\pgfpathlineto{\pgfqpoint{3.887861in}{1.685987in}}%
\pgfpathlineto{\pgfqpoint{3.888629in}{1.673764in}}%
\pgfpathlineto{\pgfqpoint{3.889013in}{1.673764in}}%
\pgfpathlineto{\pgfqpoint{3.889396in}{1.669690in}}%
\pgfpathlineto{\pgfqpoint{3.890929in}{1.685987in}}%
\pgfpathlineto{\pgfqpoint{3.891312in}{1.685987in}}%
\pgfpathlineto{\pgfqpoint{3.891695in}{1.661541in}}%
\pgfpathlineto{\pgfqpoint{3.892078in}{1.677838in}}%
\pgfpathlineto{\pgfqpoint{3.892844in}{1.685987in}}%
\pgfpathlineto{\pgfqpoint{3.893227in}{1.681913in}}%
\pgfpathlineto{\pgfqpoint{3.893611in}{1.665615in}}%
\pgfpathlineto{\pgfqpoint{3.894378in}{1.673764in}}%
\pgfpathlineto{\pgfqpoint{3.895528in}{1.665615in}}%
\pgfpathlineto{\pgfqpoint{3.896294in}{1.685987in}}%
\pgfpathlineto{\pgfqpoint{3.896678in}{1.669690in}}%
\pgfpathlineto{\pgfqpoint{3.897062in}{1.665615in}}%
\pgfpathlineto{\pgfqpoint{3.897445in}{1.677838in}}%
\pgfpathlineto{\pgfqpoint{3.897828in}{1.653392in}}%
\pgfpathlineto{\pgfqpoint{3.898596in}{1.673764in}}%
\pgfpathlineto{\pgfqpoint{3.898979in}{1.657467in}}%
\pgfpathlineto{\pgfqpoint{3.899746in}{1.661541in}}%
\pgfpathlineto{\pgfqpoint{3.900895in}{1.685987in}}%
\pgfpathlineto{\pgfqpoint{3.901278in}{1.661541in}}%
\pgfpathlineto{\pgfqpoint{3.902045in}{1.681913in}}%
\pgfpathlineto{\pgfqpoint{3.903194in}{1.673764in}}%
\pgfpathlineto{\pgfqpoint{3.903577in}{1.677838in}}%
\pgfpathlineto{\pgfqpoint{3.903961in}{1.661541in}}%
\pgfpathlineto{\pgfqpoint{3.904344in}{1.669690in}}%
\pgfpathlineto{\pgfqpoint{3.905111in}{1.685987in}}%
\pgfpathlineto{\pgfqpoint{3.905494in}{1.673764in}}%
\pgfpathlineto{\pgfqpoint{3.905877in}{1.641169in}}%
\pgfpathlineto{\pgfqpoint{3.906260in}{1.665615in}}%
\pgfpathlineto{\pgfqpoint{3.907411in}{1.694136in}}%
\pgfpathlineto{\pgfqpoint{3.907794in}{1.681913in}}%
\pgfpathlineto{\pgfqpoint{3.908561in}{1.669690in}}%
\pgfpathlineto{\pgfqpoint{3.908944in}{1.673764in}}%
\pgfpathlineto{\pgfqpoint{3.909327in}{1.690062in}}%
\pgfpathlineto{\pgfqpoint{3.909327in}{1.690062in}}%
\pgfpathlineto{\pgfqpoint{3.909327in}{1.690062in}}%
\pgfpathlineto{\pgfqpoint{3.910095in}{1.661541in}}%
\pgfpathlineto{\pgfqpoint{3.910478in}{1.665615in}}%
\pgfpathlineto{\pgfqpoint{3.910861in}{1.681913in}}%
\pgfpathlineto{\pgfqpoint{3.911244in}{1.677838in}}%
\pgfpathlineto{\pgfqpoint{3.912010in}{1.665615in}}%
\pgfpathlineto{\pgfqpoint{3.912394in}{1.673764in}}%
\pgfpathlineto{\pgfqpoint{3.913545in}{1.685987in}}%
\pgfpathlineto{\pgfqpoint{3.915077in}{1.665615in}}%
\pgfpathlineto{\pgfqpoint{3.915846in}{1.669690in}}%
\pgfpathlineto{\pgfqpoint{3.916229in}{1.694136in}}%
\pgfpathlineto{\pgfqpoint{3.916612in}{1.681913in}}%
\pgfpathlineto{\pgfqpoint{3.917761in}{1.665615in}}%
\pgfpathlineto{\pgfqpoint{3.918145in}{1.681913in}}%
\pgfpathlineto{\pgfqpoint{3.918145in}{1.681913in}}%
\pgfpathlineto{\pgfqpoint{3.918145in}{1.681913in}}%
\pgfpathlineto{\pgfqpoint{3.918529in}{1.661541in}}%
\pgfpathlineto{\pgfqpoint{3.918912in}{1.681913in}}%
\pgfpathlineto{\pgfqpoint{3.919295in}{1.681913in}}%
\pgfpathlineto{\pgfqpoint{3.919678in}{1.665615in}}%
\pgfpathlineto{\pgfqpoint{3.919678in}{1.665615in}}%
\pgfpathlineto{\pgfqpoint{3.919678in}{1.665615in}}%
\pgfpathlineto{\pgfqpoint{3.920452in}{1.685987in}}%
\pgfpathlineto{\pgfqpoint{3.920835in}{1.681913in}}%
\pgfpathlineto{\pgfqpoint{3.922368in}{1.665615in}}%
\pgfpathlineto{\pgfqpoint{3.923134in}{1.657467in}}%
\pgfpathlineto{\pgfqpoint{3.923602in}{1.661541in}}%
\pgfpathlineto{\pgfqpoint{3.923985in}{1.694136in}}%
\pgfpathlineto{\pgfqpoint{3.924752in}{1.681913in}}%
\pgfpathlineto{\pgfqpoint{3.925136in}{1.661541in}}%
\pgfpathlineto{\pgfqpoint{3.925136in}{1.661541in}}%
\pgfpathlineto{\pgfqpoint{3.925136in}{1.661541in}}%
\pgfpathlineto{\pgfqpoint{3.925520in}{1.685987in}}%
\pgfpathlineto{\pgfqpoint{3.926286in}{1.673764in}}%
\pgfpathlineto{\pgfqpoint{3.927055in}{1.665615in}}%
\pgfpathlineto{\pgfqpoint{3.927821in}{1.690062in}}%
\pgfpathlineto{\pgfqpoint{3.929355in}{1.657467in}}%
\pgfpathlineto{\pgfqpoint{3.930504in}{1.681913in}}%
\pgfpathlineto{\pgfqpoint{3.930887in}{1.669690in}}%
\pgfpathlineto{\pgfqpoint{3.931271in}{1.681913in}}%
\pgfpathlineto{\pgfqpoint{3.931656in}{1.685987in}}%
\pgfpathlineto{\pgfqpoint{3.932805in}{1.669690in}}%
\pgfpathlineto{\pgfqpoint{3.933189in}{1.673764in}}%
\pgfpathlineto{\pgfqpoint{3.933572in}{1.665615in}}%
\pgfpathlineto{\pgfqpoint{3.933956in}{1.698210in}}%
\pgfpathlineto{\pgfqpoint{3.934722in}{1.685987in}}%
\pgfpathlineto{\pgfqpoint{3.936256in}{1.657467in}}%
\pgfpathlineto{\pgfqpoint{3.936639in}{1.685987in}}%
\pgfpathlineto{\pgfqpoint{3.937405in}{1.677838in}}%
\pgfpathlineto{\pgfqpoint{3.937788in}{1.681913in}}%
\pgfpathlineto{\pgfqpoint{3.938172in}{1.665615in}}%
\pgfpathlineto{\pgfqpoint{3.938940in}{1.673764in}}%
\pgfpathlineto{\pgfqpoint{3.939322in}{1.665615in}}%
\pgfpathlineto{\pgfqpoint{3.939706in}{1.685987in}}%
\pgfpathlineto{\pgfqpoint{3.940472in}{1.669690in}}%
\pgfpathlineto{\pgfqpoint{3.940855in}{1.657467in}}%
\pgfpathlineto{\pgfqpoint{3.941239in}{1.669690in}}%
\pgfpathlineto{\pgfqpoint{3.941623in}{1.669690in}}%
\pgfpathlineto{\pgfqpoint{3.942006in}{1.685987in}}%
\pgfpathlineto{\pgfqpoint{3.942389in}{1.653392in}}%
\pgfpathlineto{\pgfqpoint{3.943156in}{1.665615in}}%
\pgfpathlineto{\pgfqpoint{3.944306in}{1.694136in}}%
\pgfpathlineto{\pgfqpoint{3.945458in}{1.665615in}}%
\pgfpathlineto{\pgfqpoint{3.946608in}{1.685987in}}%
\pgfpathlineto{\pgfqpoint{3.947375in}{1.665615in}}%
\pgfpathlineto{\pgfqpoint{3.947758in}{1.685987in}}%
\pgfpathlineto{\pgfqpoint{3.947758in}{1.685987in}}%
\pgfpathlineto{\pgfqpoint{3.947758in}{1.685987in}}%
\pgfpathlineto{\pgfqpoint{3.948141in}{1.649318in}}%
\pgfpathlineto{\pgfqpoint{3.948907in}{1.677838in}}%
\pgfpathlineto{\pgfqpoint{3.949290in}{1.673764in}}%
\pgfpathlineto{\pgfqpoint{3.949674in}{1.677838in}}%
\pgfpathlineto{\pgfqpoint{3.950057in}{1.681913in}}%
\pgfpathlineto{\pgfqpoint{3.950441in}{1.673764in}}%
\pgfpathlineto{\pgfqpoint{3.950823in}{1.690062in}}%
\pgfpathlineto{\pgfqpoint{3.950823in}{1.690062in}}%
\pgfpathlineto{\pgfqpoint{3.950823in}{1.690062in}}%
\pgfpathlineto{\pgfqpoint{3.951974in}{1.657467in}}%
\pgfpathlineto{\pgfqpoint{3.952358in}{1.694136in}}%
\pgfpathlineto{\pgfqpoint{3.953124in}{1.673764in}}%
\pgfpathlineto{\pgfqpoint{3.953507in}{1.669690in}}%
\pgfpathlineto{\pgfqpoint{3.953890in}{1.649318in}}%
\pgfpathlineto{\pgfqpoint{3.953890in}{1.649318in}}%
\pgfpathlineto{\pgfqpoint{3.953890in}{1.649318in}}%
\pgfpathlineto{\pgfqpoint{3.955042in}{1.685987in}}%
\pgfpathlineto{\pgfqpoint{3.955425in}{1.681913in}}%
\pgfpathlineto{\pgfqpoint{3.956957in}{1.653392in}}%
\pgfpathlineto{\pgfqpoint{3.957725in}{1.681913in}}%
\pgfpathlineto{\pgfqpoint{3.958108in}{1.677838in}}%
\pgfpathlineto{\pgfqpoint{3.958491in}{1.661541in}}%
\pgfpathlineto{\pgfqpoint{3.959258in}{1.665615in}}%
\pgfpathlineto{\pgfqpoint{3.960416in}{1.694136in}}%
\pgfpathlineto{\pgfqpoint{3.961183in}{1.665615in}}%
\pgfpathlineto{\pgfqpoint{3.961567in}{1.681913in}}%
\pgfpathlineto{\pgfqpoint{3.962334in}{1.653392in}}%
\pgfpathlineto{\pgfqpoint{3.962716in}{1.669690in}}%
\pgfpathlineto{\pgfqpoint{3.963100in}{1.681913in}}%
\pgfpathlineto{\pgfqpoint{3.963100in}{1.681913in}}%
\pgfpathlineto{\pgfqpoint{3.963100in}{1.681913in}}%
\pgfpathlineto{\pgfqpoint{3.963484in}{1.661541in}}%
\pgfpathlineto{\pgfqpoint{3.964251in}{1.677838in}}%
\pgfpathlineto{\pgfqpoint{3.964634in}{1.677838in}}%
\pgfpathlineto{\pgfqpoint{3.965019in}{1.681913in}}%
\pgfpathlineto{\pgfqpoint{3.965402in}{1.677838in}}%
\pgfpathlineto{\pgfqpoint{3.965786in}{1.665615in}}%
\pgfpathlineto{\pgfqpoint{3.966169in}{1.669690in}}%
\pgfpathlineto{\pgfqpoint{3.966552in}{1.677838in}}%
\pgfpathlineto{\pgfqpoint{3.966552in}{1.677838in}}%
\pgfpathlineto{\pgfqpoint{3.966552in}{1.677838in}}%
\pgfpathlineto{\pgfqpoint{3.967703in}{1.665615in}}%
\pgfpathlineto{\pgfqpoint{3.968087in}{1.665615in}}%
\pgfpathlineto{\pgfqpoint{3.968853in}{1.677838in}}%
\pgfpathlineto{\pgfqpoint{3.969237in}{1.673764in}}%
\pgfpathlineto{\pgfqpoint{3.969619in}{1.673764in}}%
\pgfpathlineto{\pgfqpoint{3.970003in}{1.677838in}}%
\pgfpathlineto{\pgfqpoint{3.970387in}{1.669690in}}%
\pgfpathlineto{\pgfqpoint{3.970387in}{1.669690in}}%
\pgfpathlineto{\pgfqpoint{3.970387in}{1.669690in}}%
\pgfpathlineto{\pgfqpoint{3.970770in}{1.681913in}}%
\pgfpathlineto{\pgfqpoint{3.971153in}{1.677838in}}%
\pgfpathlineto{\pgfqpoint{3.972389in}{1.669690in}}%
\pgfpathlineto{\pgfqpoint{3.972773in}{1.681913in}}%
\pgfpathlineto{\pgfqpoint{3.973154in}{1.669690in}}%
\pgfpathlineto{\pgfqpoint{3.973537in}{1.661541in}}%
\pgfpathlineto{\pgfqpoint{3.974304in}{1.665615in}}%
\pgfpathlineto{\pgfqpoint{3.974688in}{1.677838in}}%
\pgfpathlineto{\pgfqpoint{3.974688in}{1.677838in}}%
\pgfpathlineto{\pgfqpoint{3.974688in}{1.677838in}}%
\pgfpathlineto{\pgfqpoint{3.975072in}{1.661541in}}%
\pgfpathlineto{\pgfqpoint{3.975838in}{1.673764in}}%
\pgfpathlineto{\pgfqpoint{3.976603in}{1.673764in}}%
\pgfpathlineto{\pgfqpoint{3.976986in}{1.681913in}}%
\pgfpathlineto{\pgfqpoint{3.976986in}{1.681913in}}%
\pgfpathlineto{\pgfqpoint{3.976986in}{1.681913in}}%
\pgfpathlineto{\pgfqpoint{3.977754in}{1.661541in}}%
\pgfpathlineto{\pgfqpoint{3.978138in}{1.677838in}}%
\pgfpathlineto{\pgfqpoint{3.978521in}{1.685987in}}%
\pgfpathlineto{\pgfqpoint{3.978521in}{1.685987in}}%
\pgfpathlineto{\pgfqpoint{3.978521in}{1.685987in}}%
\pgfpathlineto{\pgfqpoint{3.979673in}{1.661541in}}%
\pgfpathlineto{\pgfqpoint{3.980441in}{1.690062in}}%
\pgfpathlineto{\pgfqpoint{3.980824in}{1.657467in}}%
\pgfpathlineto{\pgfqpoint{3.981590in}{1.665615in}}%
\pgfpathlineto{\pgfqpoint{3.981973in}{1.665615in}}%
\pgfpathlineto{\pgfqpoint{3.982740in}{1.677838in}}%
\pgfpathlineto{\pgfqpoint{3.983124in}{1.661541in}}%
\pgfpathlineto{\pgfqpoint{3.983891in}{1.665615in}}%
\pgfpathlineto{\pgfqpoint{3.984657in}{1.690062in}}%
\pgfpathlineto{\pgfqpoint{3.985040in}{1.681913in}}%
\pgfpathlineto{\pgfqpoint{3.986191in}{1.665615in}}%
\pgfpathlineto{\pgfqpoint{3.986958in}{1.681913in}}%
\pgfpathlineto{\pgfqpoint{3.987341in}{1.665615in}}%
\pgfpathlineto{\pgfqpoint{3.988107in}{1.677838in}}%
\pgfpathlineto{\pgfqpoint{3.988875in}{1.665615in}}%
\pgfpathlineto{\pgfqpoint{3.989258in}{1.685987in}}%
\pgfpathlineto{\pgfqpoint{3.990025in}{1.669690in}}%
\pgfpathlineto{\pgfqpoint{3.990408in}{1.677838in}}%
\pgfpathlineto{\pgfqpoint{3.990792in}{1.669690in}}%
\pgfpathlineto{\pgfqpoint{3.991176in}{1.669690in}}%
\pgfpathlineto{\pgfqpoint{3.991559in}{1.661541in}}%
\pgfpathlineto{\pgfqpoint{3.991942in}{1.685987in}}%
\pgfpathlineto{\pgfqpoint{3.992707in}{1.681913in}}%
\pgfpathlineto{\pgfqpoint{3.993092in}{1.681913in}}%
\pgfpathlineto{\pgfqpoint{3.994625in}{1.665615in}}%
\pgfpathlineto{\pgfqpoint{3.995008in}{1.649318in}}%
\pgfpathlineto{\pgfqpoint{3.995008in}{1.649318in}}%
\pgfpathlineto{\pgfqpoint{3.995008in}{1.649318in}}%
\pgfpathlineto{\pgfqpoint{3.996158in}{1.677838in}}%
\pgfpathlineto{\pgfqpoint{3.996541in}{1.677838in}}%
\pgfpathlineto{\pgfqpoint{3.997691in}{1.661541in}}%
\pgfpathlineto{\pgfqpoint{3.998457in}{1.677838in}}%
\pgfpathlineto{\pgfqpoint{3.998840in}{1.657467in}}%
\pgfpathlineto{\pgfqpoint{3.999606in}{1.673764in}}%
\pgfpathlineto{\pgfqpoint{4.000764in}{1.685987in}}%
\pgfpathlineto{\pgfqpoint{4.001147in}{1.657467in}}%
\pgfpathlineto{\pgfqpoint{4.001914in}{1.677838in}}%
\pgfpathlineto{\pgfqpoint{4.003064in}{1.665615in}}%
\pgfpathlineto{\pgfqpoint{4.003447in}{1.673764in}}%
\pgfpathlineto{\pgfqpoint{4.003831in}{1.669690in}}%
\pgfpathlineto{\pgfqpoint{4.004215in}{1.657467in}}%
\pgfpathlineto{\pgfqpoint{4.004598in}{1.669690in}}%
\pgfpathlineto{\pgfqpoint{4.005363in}{1.677838in}}%
\pgfpathlineto{\pgfqpoint{4.005746in}{1.673764in}}%
\pgfpathlineto{\pgfqpoint{4.006129in}{1.669690in}}%
\pgfpathlineto{\pgfqpoint{4.007663in}{1.685987in}}%
\pgfpathlineto{\pgfqpoint{4.008046in}{1.685987in}}%
\pgfpathlineto{\pgfqpoint{4.008429in}{1.665615in}}%
\pgfpathlineto{\pgfqpoint{4.009199in}{1.677838in}}%
\pgfpathlineto{\pgfqpoint{4.009582in}{1.669690in}}%
\pgfpathlineto{\pgfqpoint{4.009582in}{1.669690in}}%
\pgfpathlineto{\pgfqpoint{4.009582in}{1.669690in}}%
\pgfpathlineto{\pgfqpoint{4.010731in}{1.681913in}}%
\pgfpathlineto{\pgfqpoint{4.011113in}{1.681913in}}%
\pgfpathlineto{\pgfqpoint{4.011498in}{1.685987in}}%
\pgfpathlineto{\pgfqpoint{4.011882in}{1.681913in}}%
\pgfpathlineto{\pgfqpoint{4.012647in}{1.653392in}}%
\pgfpathlineto{\pgfqpoint{4.013030in}{1.690062in}}%
\pgfpathlineto{\pgfqpoint{4.013797in}{1.669690in}}%
\pgfpathlineto{\pgfqpoint{4.014948in}{1.694136in}}%
\pgfpathlineto{\pgfqpoint{4.015714in}{1.657467in}}%
\pgfpathlineto{\pgfqpoint{4.016095in}{1.665615in}}%
\pgfpathlineto{\pgfqpoint{4.016563in}{1.657467in}}%
\pgfpathlineto{\pgfqpoint{4.018096in}{1.677838in}}%
\pgfpathlineto{\pgfqpoint{4.018480in}{1.665615in}}%
\pgfpathlineto{\pgfqpoint{4.018864in}{1.690062in}}%
\pgfpathlineto{\pgfqpoint{4.019630in}{1.673764in}}%
\pgfpathlineto{\pgfqpoint{4.020012in}{1.677838in}}%
\pgfpathlineto{\pgfqpoint{4.021162in}{1.665615in}}%
\pgfpathlineto{\pgfqpoint{4.021545in}{1.673764in}}%
\pgfpathlineto{\pgfqpoint{4.021928in}{1.669690in}}%
\pgfpathlineto{\pgfqpoint{4.022694in}{1.661541in}}%
\pgfpathlineto{\pgfqpoint{4.023078in}{1.685987in}}%
\pgfpathlineto{\pgfqpoint{4.023462in}{1.677838in}}%
\pgfpathlineto{\pgfqpoint{4.023845in}{1.645243in}}%
\pgfpathlineto{\pgfqpoint{4.023845in}{1.645243in}}%
\pgfpathlineto{\pgfqpoint{4.023845in}{1.645243in}}%
\pgfpathlineto{\pgfqpoint{4.024228in}{1.694136in}}%
\pgfpathlineto{\pgfqpoint{4.024994in}{1.685987in}}%
\pgfpathlineto{\pgfqpoint{4.025762in}{1.669690in}}%
\pgfpathlineto{\pgfqpoint{4.026146in}{1.694136in}}%
\pgfpathlineto{\pgfqpoint{4.026146in}{1.694136in}}%
\pgfpathlineto{\pgfqpoint{4.026146in}{1.694136in}}%
\pgfpathlineto{\pgfqpoint{4.026529in}{1.665615in}}%
\pgfpathlineto{\pgfqpoint{4.027295in}{1.669690in}}%
\pgfpathlineto{\pgfqpoint{4.027678in}{1.669690in}}%
\pgfpathlineto{\pgfqpoint{4.028062in}{1.690062in}}%
\pgfpathlineto{\pgfqpoint{4.028062in}{1.690062in}}%
\pgfpathlineto{\pgfqpoint{4.028062in}{1.690062in}}%
\pgfpathlineto{\pgfqpoint{4.028445in}{1.665615in}}%
\pgfpathlineto{\pgfqpoint{4.029212in}{1.681913in}}%
\pgfpathlineto{\pgfqpoint{4.029979in}{1.669690in}}%
\pgfpathlineto{\pgfqpoint{4.030362in}{1.677838in}}%
\pgfpathlineto{\pgfqpoint{4.030745in}{1.677838in}}%
\pgfpathlineto{\pgfqpoint{4.031128in}{1.665615in}}%
\pgfpathlineto{\pgfqpoint{4.031512in}{1.669690in}}%
\pgfpathlineto{\pgfqpoint{4.031896in}{1.681913in}}%
\pgfpathlineto{\pgfqpoint{4.032663in}{1.673764in}}%
\pgfpathlineto{\pgfqpoint{4.033046in}{1.669690in}}%
\pgfpathlineto{\pgfqpoint{4.033428in}{1.677838in}}%
\pgfpathlineto{\pgfqpoint{4.033812in}{1.661541in}}%
\pgfpathlineto{\pgfqpoint{4.033812in}{1.661541in}}%
\pgfpathlineto{\pgfqpoint{4.033812in}{1.661541in}}%
\pgfpathlineto{\pgfqpoint{4.035346in}{1.694136in}}%
\pgfpathlineto{\pgfqpoint{4.035729in}{1.661541in}}%
\pgfpathlineto{\pgfqpoint{4.036496in}{1.669690in}}%
\pgfpathlineto{\pgfqpoint{4.036880in}{1.669690in}}%
\pgfpathlineto{\pgfqpoint{4.037263in}{1.677838in}}%
\pgfpathlineto{\pgfqpoint{4.037263in}{1.677838in}}%
\pgfpathlineto{\pgfqpoint{4.037263in}{1.677838in}}%
\pgfpathlineto{\pgfqpoint{4.038798in}{1.661541in}}%
\pgfpathlineto{\pgfqpoint{4.039947in}{1.685987in}}%
\pgfpathlineto{\pgfqpoint{4.040336in}{1.681913in}}%
\pgfpathlineto{\pgfqpoint{4.041488in}{1.657467in}}%
\pgfpathlineto{\pgfqpoint{4.041871in}{1.681913in}}%
\pgfpathlineto{\pgfqpoint{4.042637in}{1.661541in}}%
\pgfpathlineto{\pgfqpoint{4.043405in}{1.677838in}}%
\pgfpathlineto{\pgfqpoint{4.043789in}{1.661541in}}%
\pgfpathlineto{\pgfqpoint{4.044172in}{1.665615in}}%
\pgfpathlineto{\pgfqpoint{4.044555in}{1.685987in}}%
\pgfpathlineto{\pgfqpoint{4.045322in}{1.681913in}}%
\pgfpathlineto{\pgfqpoint{4.045706in}{1.665615in}}%
\pgfpathlineto{\pgfqpoint{4.046473in}{1.673764in}}%
\pgfpathlineto{\pgfqpoint{4.047239in}{1.685987in}}%
\pgfpathlineto{\pgfqpoint{4.047622in}{1.677838in}}%
\pgfpathlineto{\pgfqpoint{4.048006in}{1.645243in}}%
\pgfpathlineto{\pgfqpoint{4.048773in}{1.669690in}}%
\pgfpathlineto{\pgfqpoint{4.049539in}{1.649318in}}%
\pgfpathlineto{\pgfqpoint{4.049922in}{1.657467in}}%
\pgfpathlineto{\pgfqpoint{4.050689in}{1.673764in}}%
\pgfpathlineto{\pgfqpoint{4.051072in}{1.669690in}}%
\pgfpathlineto{\pgfqpoint{4.051456in}{1.669690in}}%
\pgfpathlineto{\pgfqpoint{4.052223in}{1.673764in}}%
\pgfpathlineto{\pgfqpoint{4.052606in}{1.661541in}}%
\pgfpathlineto{\pgfqpoint{4.053372in}{1.698210in}}%
\pgfpathlineto{\pgfqpoint{4.054523in}{1.661541in}}%
\pgfpathlineto{\pgfqpoint{4.056442in}{1.681913in}}%
\pgfpathlineto{\pgfqpoint{4.057592in}{1.673764in}}%
\pgfpathlineto{\pgfqpoint{4.057975in}{1.681913in}}%
\pgfpathlineto{\pgfqpoint{4.058358in}{1.677838in}}%
\pgfpathlineto{\pgfqpoint{4.058741in}{1.669690in}}%
\pgfpathlineto{\pgfqpoint{4.059125in}{1.690062in}}%
\pgfpathlineto{\pgfqpoint{4.059510in}{1.673764in}}%
\pgfpathlineto{\pgfqpoint{4.060276in}{1.645243in}}%
\pgfpathlineto{\pgfqpoint{4.061042in}{1.677838in}}%
\pgfpathlineto{\pgfqpoint{4.061425in}{1.661541in}}%
\pgfpathlineto{\pgfqpoint{4.062278in}{1.690062in}}%
\pgfpathlineto{\pgfqpoint{4.062658in}{1.665615in}}%
\pgfpathlineto{\pgfqpoint{4.063042in}{1.665615in}}%
\pgfpathlineto{\pgfqpoint{4.064192in}{1.673764in}}%
\pgfpathlineto{\pgfqpoint{4.064957in}{1.661541in}}%
\pgfpathlineto{\pgfqpoint{4.065341in}{1.690062in}}%
\pgfpathlineto{\pgfqpoint{4.065341in}{1.690062in}}%
\pgfpathlineto{\pgfqpoint{4.065341in}{1.690062in}}%
\pgfpathlineto{\pgfqpoint{4.065724in}{1.657467in}}%
\pgfpathlineto{\pgfqpoint{4.066492in}{1.665615in}}%
\pgfpathlineto{\pgfqpoint{4.067257in}{1.673764in}}%
\pgfpathlineto{\pgfqpoint{4.067640in}{1.661541in}}%
\pgfpathlineto{\pgfqpoint{4.068023in}{1.673764in}}%
\pgfpathlineto{\pgfqpoint{4.069557in}{1.690062in}}%
\pgfpathlineto{\pgfqpoint{4.070323in}{1.661541in}}%
\pgfpathlineto{\pgfqpoint{4.071091in}{1.669690in}}%
\pgfpathlineto{\pgfqpoint{4.071475in}{1.681913in}}%
\pgfpathlineto{\pgfqpoint{4.072240in}{1.677838in}}%
\pgfpathlineto{\pgfqpoint{4.072623in}{1.657467in}}%
\pgfpathlineto{\pgfqpoint{4.073390in}{1.673764in}}%
\pgfpathlineto{\pgfqpoint{4.073774in}{1.673764in}}%
\pgfpathlineto{\pgfqpoint{4.074157in}{1.661541in}}%
\pgfpathlineto{\pgfqpoint{4.074540in}{1.665615in}}%
\pgfpathlineto{\pgfqpoint{4.074922in}{1.677838in}}%
\pgfpathlineto{\pgfqpoint{4.074922in}{1.677838in}}%
\pgfpathlineto{\pgfqpoint{4.074922in}{1.677838in}}%
\pgfpathlineto{\pgfqpoint{4.075689in}{1.657467in}}%
\pgfpathlineto{\pgfqpoint{4.076073in}{1.669690in}}%
\pgfpathlineto{\pgfqpoint{4.076457in}{1.677838in}}%
\pgfpathlineto{\pgfqpoint{4.077223in}{1.657467in}}%
\pgfpathlineto{\pgfqpoint{4.077606in}{1.669690in}}%
\pgfpathlineto{\pgfqpoint{4.077989in}{1.690062in}}%
\pgfpathlineto{\pgfqpoint{4.077989in}{1.690062in}}%
\pgfpathlineto{\pgfqpoint{4.077989in}{1.690062in}}%
\pgfpathlineto{\pgfqpoint{4.078372in}{1.661541in}}%
\pgfpathlineto{\pgfqpoint{4.079140in}{1.677838in}}%
\pgfpathlineto{\pgfqpoint{4.080296in}{1.661541in}}%
\pgfpathlineto{\pgfqpoint{4.080680in}{1.677838in}}%
\pgfpathlineto{\pgfqpoint{4.081448in}{1.665615in}}%
\pgfpathlineto{\pgfqpoint{4.081831in}{1.673764in}}%
\pgfpathlineto{\pgfqpoint{4.082215in}{1.653392in}}%
\pgfpathlineto{\pgfqpoint{4.082598in}{1.669690in}}%
\pgfpathlineto{\pgfqpoint{4.083365in}{1.681913in}}%
\pgfpathlineto{\pgfqpoint{4.083748in}{1.677838in}}%
\pgfpathlineto{\pgfqpoint{4.084515in}{1.665615in}}%
\pgfpathlineto{\pgfqpoint{4.084898in}{1.669690in}}%
\pgfpathlineto{\pgfqpoint{4.085281in}{1.669690in}}%
\pgfpathlineto{\pgfqpoint{4.085663in}{1.661541in}}%
\pgfpathlineto{\pgfqpoint{4.086430in}{1.677838in}}%
\pgfpathlineto{\pgfqpoint{4.086813in}{1.665615in}}%
\pgfpathlineto{\pgfqpoint{4.087579in}{1.661541in}}%
\pgfpathlineto{\pgfqpoint{4.087961in}{1.681913in}}%
\pgfpathlineto{\pgfqpoint{4.088727in}{1.673764in}}%
\pgfpathlineto{\pgfqpoint{4.089111in}{1.665615in}}%
\pgfpathlineto{\pgfqpoint{4.089495in}{1.685987in}}%
\pgfpathlineto{\pgfqpoint{4.089495in}{1.685987in}}%
\pgfpathlineto{\pgfqpoint{4.089495in}{1.685987in}}%
\pgfpathlineto{\pgfqpoint{4.089878in}{1.653392in}}%
\pgfpathlineto{\pgfqpoint{4.090646in}{1.677838in}}%
\pgfpathlineto{\pgfqpoint{4.091029in}{1.690062in}}%
\pgfpathlineto{\pgfqpoint{4.091029in}{1.690062in}}%
\pgfpathlineto{\pgfqpoint{4.091029in}{1.690062in}}%
\pgfpathlineto{\pgfqpoint{4.092181in}{1.669690in}}%
\pgfpathlineto{\pgfqpoint{4.092564in}{1.677838in}}%
\pgfpathlineto{\pgfqpoint{4.092947in}{1.661541in}}%
\pgfpathlineto{\pgfqpoint{4.093330in}{1.694136in}}%
\pgfpathlineto{\pgfqpoint{4.094098in}{1.669690in}}%
\pgfpathlineto{\pgfqpoint{4.094481in}{1.681913in}}%
\pgfpathlineto{\pgfqpoint{4.094865in}{1.673764in}}%
\pgfpathlineto{\pgfqpoint{4.095631in}{1.665615in}}%
\pgfpathlineto{\pgfqpoint{4.096014in}{1.690062in}}%
\pgfpathlineto{\pgfqpoint{4.096396in}{1.673764in}}%
\pgfpathlineto{\pgfqpoint{4.096780in}{1.665615in}}%
\pgfpathlineto{\pgfqpoint{4.097163in}{1.669690in}}%
\pgfpathlineto{\pgfqpoint{4.097546in}{1.685987in}}%
\pgfpathlineto{\pgfqpoint{4.097929in}{1.669690in}}%
\pgfpathlineto{\pgfqpoint{4.098312in}{1.665615in}}%
\pgfpathlineto{\pgfqpoint{4.098695in}{1.677838in}}%
\pgfpathlineto{\pgfqpoint{4.099079in}{1.673764in}}%
\pgfpathlineto{\pgfqpoint{4.099462in}{1.649318in}}%
\pgfpathlineto{\pgfqpoint{4.099462in}{1.649318in}}%
\pgfpathlineto{\pgfqpoint{4.099462in}{1.649318in}}%
\pgfpathlineto{\pgfqpoint{4.100230in}{1.677838in}}%
\pgfpathlineto{\pgfqpoint{4.100613in}{1.673764in}}%
\pgfpathlineto{\pgfqpoint{4.100996in}{1.673764in}}%
\pgfpathlineto{\pgfqpoint{4.101379in}{1.685987in}}%
\pgfpathlineto{\pgfqpoint{4.101762in}{1.681913in}}%
\pgfpathlineto{\pgfqpoint{4.102914in}{1.669690in}}%
\pgfpathlineto{\pgfqpoint{4.103680in}{1.710434in}}%
\pgfpathlineto{\pgfqpoint{4.104149in}{1.669690in}}%
\pgfpathlineto{\pgfqpoint{4.104916in}{1.677838in}}%
\pgfpathlineto{\pgfqpoint{4.106450in}{1.657467in}}%
\pgfpathlineto{\pgfqpoint{4.107601in}{1.685987in}}%
\pgfpathlineto{\pgfqpoint{4.109132in}{1.661541in}}%
\pgfpathlineto{\pgfqpoint{4.110282in}{1.681913in}}%
\pgfpathlineto{\pgfqpoint{4.110667in}{1.665615in}}%
\pgfpathlineto{\pgfqpoint{4.110667in}{1.665615in}}%
\pgfpathlineto{\pgfqpoint{4.110667in}{1.665615in}}%
\pgfpathlineto{\pgfqpoint{4.111049in}{1.690062in}}%
\pgfpathlineto{\pgfqpoint{4.111817in}{1.673764in}}%
\pgfpathlineto{\pgfqpoint{4.112201in}{1.677838in}}%
\pgfpathlineto{\pgfqpoint{4.112584in}{1.665615in}}%
\pgfpathlineto{\pgfqpoint{4.112967in}{1.690062in}}%
\pgfpathlineto{\pgfqpoint{4.113734in}{1.673764in}}%
\pgfpathlineto{\pgfqpoint{4.114118in}{1.681913in}}%
\pgfpathlineto{\pgfqpoint{4.114884in}{1.677838in}}%
\pgfpathlineto{\pgfqpoint{4.116036in}{1.694136in}}%
\pgfpathlineto{\pgfqpoint{4.117567in}{1.657467in}}%
\pgfpathlineto{\pgfqpoint{4.117950in}{1.694136in}}%
\pgfpathlineto{\pgfqpoint{4.118718in}{1.677838in}}%
\pgfpathlineto{\pgfqpoint{4.119101in}{1.677838in}}%
\pgfpathlineto{\pgfqpoint{4.119485in}{1.661541in}}%
\pgfpathlineto{\pgfqpoint{4.120260in}{1.673764in}}%
\pgfpathlineto{\pgfqpoint{4.120644in}{1.685987in}}%
\pgfpathlineto{\pgfqpoint{4.121411in}{1.681913in}}%
\pgfpathlineto{\pgfqpoint{4.122561in}{1.665615in}}%
\pgfpathlineto{\pgfqpoint{4.123327in}{1.677838in}}%
\pgfpathlineto{\pgfqpoint{4.123711in}{1.673764in}}%
\pgfpathlineto{\pgfqpoint{4.125244in}{1.657467in}}%
\pgfpathlineto{\pgfqpoint{4.125628in}{1.661541in}}%
\pgfpathlineto{\pgfqpoint{4.126011in}{1.694136in}}%
\pgfpathlineto{\pgfqpoint{4.126777in}{1.685987in}}%
\pgfpathlineto{\pgfqpoint{4.127928in}{1.665615in}}%
\pgfpathlineto{\pgfqpoint{4.128311in}{1.669690in}}%
\pgfpathlineto{\pgfqpoint{4.129077in}{1.677838in}}%
\pgfpathlineto{\pgfqpoint{4.129460in}{1.673764in}}%
\pgfpathlineto{\pgfqpoint{4.130227in}{1.673764in}}%
\pgfpathlineto{\pgfqpoint{4.130611in}{1.685987in}}%
\pgfpathlineto{\pgfqpoint{4.131379in}{1.677838in}}%
\pgfpathlineto{\pgfqpoint{4.132146in}{1.665615in}}%
\pgfpathlineto{\pgfqpoint{4.133678in}{1.681913in}}%
\pgfpathlineto{\pgfqpoint{4.134061in}{1.649318in}}%
\pgfpathlineto{\pgfqpoint{4.134827in}{1.669690in}}%
\pgfpathlineto{\pgfqpoint{4.135211in}{1.665615in}}%
\pgfpathlineto{\pgfqpoint{4.135594in}{1.673764in}}%
\pgfpathlineto{\pgfqpoint{4.135594in}{1.673764in}}%
\pgfpathlineto{\pgfqpoint{4.135594in}{1.673764in}}%
\pgfpathlineto{\pgfqpoint{4.135977in}{1.661541in}}%
\pgfpathlineto{\pgfqpoint{4.135977in}{1.661541in}}%
\pgfpathlineto{\pgfqpoint{4.135977in}{1.661541in}}%
\pgfpathlineto{\pgfqpoint{4.137128in}{1.698210in}}%
\pgfpathlineto{\pgfqpoint{4.138661in}{1.665615in}}%
\pgfpathlineto{\pgfqpoint{4.139045in}{1.665615in}}%
\pgfpathlineto{\pgfqpoint{4.139428in}{1.681913in}}%
\pgfpathlineto{\pgfqpoint{4.140195in}{1.673764in}}%
\pgfpathlineto{\pgfqpoint{4.140578in}{1.677838in}}%
\pgfpathlineto{\pgfqpoint{4.140961in}{1.669690in}}%
\pgfpathlineto{\pgfqpoint{4.140961in}{1.669690in}}%
\pgfpathlineto{\pgfqpoint{4.140961in}{1.669690in}}%
\pgfpathlineto{\pgfqpoint{4.141728in}{1.690062in}}%
\pgfpathlineto{\pgfqpoint{4.142110in}{1.677838in}}%
\pgfpathlineto{\pgfqpoint{4.142494in}{1.677838in}}%
\pgfpathlineto{\pgfqpoint{4.142877in}{1.661541in}}%
\pgfpathlineto{\pgfqpoint{4.143261in}{1.673764in}}%
\pgfpathlineto{\pgfqpoint{4.143644in}{1.681913in}}%
\pgfpathlineto{\pgfqpoint{4.145177in}{1.653392in}}%
\pgfpathlineto{\pgfqpoint{4.145561in}{1.681913in}}%
\pgfpathlineto{\pgfqpoint{4.146327in}{1.677838in}}%
\pgfpathlineto{\pgfqpoint{4.147094in}{1.681913in}}%
\pgfpathlineto{\pgfqpoint{4.148627in}{1.649318in}}%
\pgfpathlineto{\pgfqpoint{4.149863in}{1.681913in}}%
\pgfpathlineto{\pgfqpoint{4.150247in}{1.677838in}}%
\pgfpathlineto{\pgfqpoint{4.151395in}{1.669690in}}%
\pgfpathlineto{\pgfqpoint{4.151778in}{1.681913in}}%
\pgfpathlineto{\pgfqpoint{4.152546in}{1.677838in}}%
\pgfpathlineto{\pgfqpoint{4.153696in}{1.665615in}}%
\pgfpathlineto{\pgfqpoint{4.154462in}{1.685987in}}%
\pgfpathlineto{\pgfqpoint{4.155228in}{1.661541in}}%
\pgfpathlineto{\pgfqpoint{4.155610in}{1.665615in}}%
\pgfpathlineto{\pgfqpoint{4.155994in}{1.673764in}}%
\pgfpathlineto{\pgfqpoint{4.156377in}{1.669690in}}%
\pgfpathlineto{\pgfqpoint{4.157527in}{1.661541in}}%
\pgfpathlineto{\pgfqpoint{4.159061in}{1.685987in}}%
\pgfpathlineto{\pgfqpoint{4.159445in}{1.653392in}}%
\pgfpathlineto{\pgfqpoint{4.159828in}{1.665615in}}%
\pgfpathlineto{\pgfqpoint{4.160220in}{1.685987in}}%
\pgfpathlineto{\pgfqpoint{4.160603in}{1.673764in}}%
\pgfpathlineto{\pgfqpoint{4.160986in}{1.653392in}}%
\pgfpathlineto{\pgfqpoint{4.160986in}{1.653392in}}%
\pgfpathlineto{\pgfqpoint{4.160986in}{1.653392in}}%
\pgfpathlineto{\pgfqpoint{4.162136in}{1.685987in}}%
\pgfpathlineto{\pgfqpoint{4.163286in}{1.669690in}}%
\pgfpathlineto{\pgfqpoint{4.164819in}{1.690062in}}%
\pgfpathlineto{\pgfqpoint{4.165203in}{1.685987in}}%
\pgfpathlineto{\pgfqpoint{4.166353in}{1.661541in}}%
\pgfpathlineto{\pgfqpoint{4.166736in}{1.669690in}}%
\pgfpathlineto{\pgfqpoint{4.167118in}{1.690062in}}%
\pgfpathlineto{\pgfqpoint{4.167118in}{1.690062in}}%
\pgfpathlineto{\pgfqpoint{4.167118in}{1.690062in}}%
\pgfpathlineto{\pgfqpoint{4.167503in}{1.661541in}}%
\pgfpathlineto{\pgfqpoint{4.168270in}{1.677838in}}%
\pgfpathlineto{\pgfqpoint{4.168654in}{1.665615in}}%
\pgfpathlineto{\pgfqpoint{4.169420in}{1.669690in}}%
\pgfpathlineto{\pgfqpoint{4.169803in}{1.690062in}}%
\pgfpathlineto{\pgfqpoint{4.170570in}{1.677838in}}%
\pgfpathlineto{\pgfqpoint{4.170953in}{1.690062in}}%
\pgfpathlineto{\pgfqpoint{4.171337in}{1.677838in}}%
\pgfpathlineto{\pgfqpoint{4.171720in}{1.677838in}}%
\pgfpathlineto{\pgfqpoint{4.172487in}{1.665615in}}%
\pgfpathlineto{\pgfqpoint{4.172870in}{1.681913in}}%
\pgfpathlineto{\pgfqpoint{4.172870in}{1.681913in}}%
\pgfpathlineto{\pgfqpoint{4.172870in}{1.681913in}}%
\pgfpathlineto{\pgfqpoint{4.174021in}{1.657467in}}%
\pgfpathlineto{\pgfqpoint{4.175938in}{1.685987in}}%
\pgfpathlineto{\pgfqpoint{4.176321in}{1.685987in}}%
\pgfpathlineto{\pgfqpoint{4.177087in}{1.657467in}}%
\pgfpathlineto{\pgfqpoint{4.177471in}{1.677838in}}%
\pgfpathlineto{\pgfqpoint{4.178621in}{1.661541in}}%
\pgfpathlineto{\pgfqpoint{4.179772in}{1.681913in}}%
\pgfpathlineto{\pgfqpoint{4.181306in}{1.653392in}}%
\pgfpathlineto{\pgfqpoint{4.182072in}{1.681913in}}%
\pgfpathlineto{\pgfqpoint{4.182455in}{1.665615in}}%
\pgfpathlineto{\pgfqpoint{4.182838in}{1.653392in}}%
\pgfpathlineto{\pgfqpoint{4.183221in}{1.665615in}}%
\pgfpathlineto{\pgfqpoint{4.183989in}{1.690062in}}%
\pgfpathlineto{\pgfqpoint{4.184372in}{1.673764in}}%
\pgfpathlineto{\pgfqpoint{4.184756in}{1.665615in}}%
\pgfpathlineto{\pgfqpoint{4.184756in}{1.665615in}}%
\pgfpathlineto{\pgfqpoint{4.184756in}{1.665615in}}%
\pgfpathlineto{\pgfqpoint{4.185521in}{1.681913in}}%
\pgfpathlineto{\pgfqpoint{4.185905in}{1.677838in}}%
\pgfpathlineto{\pgfqpoint{4.186673in}{1.665615in}}%
\pgfpathlineto{\pgfqpoint{4.187056in}{1.669690in}}%
\pgfpathlineto{\pgfqpoint{4.187440in}{1.673764in}}%
\pgfpathlineto{\pgfqpoint{4.188206in}{1.645243in}}%
\pgfpathlineto{\pgfqpoint{4.189357in}{1.677838in}}%
\pgfpathlineto{\pgfqpoint{4.189740in}{1.665615in}}%
\pgfpathlineto{\pgfqpoint{4.190123in}{1.673764in}}%
\pgfpathlineto{\pgfqpoint{4.190889in}{1.690062in}}%
\pgfpathlineto{\pgfqpoint{4.192423in}{1.665615in}}%
\pgfpathlineto{\pgfqpoint{4.193956in}{1.690062in}}%
\pgfpathlineto{\pgfqpoint{4.195107in}{1.669690in}}%
\pgfpathlineto{\pgfqpoint{4.196726in}{1.677838in}}%
\pgfpathlineto{\pgfqpoint{4.197108in}{1.669690in}}%
\pgfpathlineto{\pgfqpoint{4.197492in}{1.673764in}}%
\pgfpathlineto{\pgfqpoint{4.198257in}{1.685987in}}%
\pgfpathlineto{\pgfqpoint{4.199024in}{1.657467in}}%
\pgfpathlineto{\pgfqpoint{4.199406in}{1.681913in}}%
\pgfpathlineto{\pgfqpoint{4.200564in}{1.661541in}}%
\pgfpathlineto{\pgfqpoint{4.200947in}{1.690062in}}%
\pgfpathlineto{\pgfqpoint{4.201714in}{1.677838in}}%
\pgfpathlineto{\pgfqpoint{4.202095in}{1.673764in}}%
\pgfpathlineto{\pgfqpoint{4.202480in}{1.681913in}}%
\pgfpathlineto{\pgfqpoint{4.202480in}{1.681913in}}%
\pgfpathlineto{\pgfqpoint{4.202480in}{1.681913in}}%
\pgfpathlineto{\pgfqpoint{4.204014in}{1.665615in}}%
\pgfpathlineto{\pgfqpoint{4.204397in}{1.685987in}}%
\pgfpathlineto{\pgfqpoint{4.204781in}{1.673764in}}%
\pgfpathlineto{\pgfqpoint{4.205932in}{1.657467in}}%
\pgfpathlineto{\pgfqpoint{4.206698in}{1.681913in}}%
\pgfpathlineto{\pgfqpoint{4.207083in}{1.677838in}}%
\pgfpathlineto{\pgfqpoint{4.207467in}{1.677838in}}%
\pgfpathlineto{\pgfqpoint{4.209766in}{1.657467in}}%
\pgfpathlineto{\pgfqpoint{4.211299in}{1.673764in}}%
\pgfpathlineto{\pgfqpoint{4.211683in}{1.661541in}}%
\pgfpathlineto{\pgfqpoint{4.212067in}{1.673764in}}%
\pgfpathlineto{\pgfqpoint{4.212450in}{1.673764in}}%
\pgfpathlineto{\pgfqpoint{4.213217in}{1.669690in}}%
\pgfpathlineto{\pgfqpoint{4.213600in}{1.681913in}}%
\pgfpathlineto{\pgfqpoint{4.213600in}{1.681913in}}%
\pgfpathlineto{\pgfqpoint{4.213600in}{1.681913in}}%
\pgfpathlineto{\pgfqpoint{4.214366in}{1.661541in}}%
\pgfpathlineto{\pgfqpoint{4.214749in}{1.681913in}}%
\pgfpathlineto{\pgfqpoint{4.215517in}{1.665615in}}%
\pgfpathlineto{\pgfqpoint{4.216667in}{1.677838in}}%
\pgfpathlineto{\pgfqpoint{4.217433in}{1.657467in}}%
\pgfpathlineto{\pgfqpoint{4.217816in}{1.669690in}}%
\pgfpathlineto{\pgfqpoint{4.218200in}{1.669690in}}%
\pgfpathlineto{\pgfqpoint{4.218584in}{1.661541in}}%
\pgfpathlineto{\pgfqpoint{4.218967in}{1.681913in}}%
\pgfpathlineto{\pgfqpoint{4.219734in}{1.673764in}}%
\pgfpathlineto{\pgfqpoint{4.220117in}{1.677838in}}%
\pgfpathlineto{\pgfqpoint{4.221651in}{1.661541in}}%
\pgfpathlineto{\pgfqpoint{4.223185in}{1.677838in}}%
\pgfpathlineto{\pgfqpoint{4.223569in}{1.649318in}}%
\pgfpathlineto{\pgfqpoint{4.223952in}{1.673764in}}%
\pgfpathlineto{\pgfqpoint{4.225101in}{1.681913in}}%
\pgfpathlineto{\pgfqpoint{4.225869in}{1.661541in}}%
\pgfpathlineto{\pgfqpoint{4.226253in}{1.681913in}}%
\pgfpathlineto{\pgfqpoint{4.227018in}{1.665615in}}%
\pgfpathlineto{\pgfqpoint{4.227785in}{1.665615in}}%
\pgfpathlineto{\pgfqpoint{4.228936in}{1.681913in}}%
\pgfpathlineto{\pgfqpoint{4.229319in}{1.669690in}}%
\pgfpathlineto{\pgfqpoint{4.230086in}{1.677838in}}%
\pgfpathlineto{\pgfqpoint{4.230469in}{1.677838in}}%
\pgfpathlineto{\pgfqpoint{4.230854in}{1.685987in}}%
\pgfpathlineto{\pgfqpoint{4.232004in}{1.665615in}}%
\pgfpathlineto{\pgfqpoint{4.232771in}{1.681913in}}%
\pgfpathlineto{\pgfqpoint{4.233921in}{1.665615in}}%
\pgfpathlineto{\pgfqpoint{4.235454in}{1.681913in}}%
\pgfpathlineto{\pgfqpoint{4.236605in}{1.665615in}}%
\pgfpathlineto{\pgfqpoint{4.237372in}{1.681913in}}%
\pgfpathlineto{\pgfqpoint{4.237754in}{1.677838in}}%
\pgfpathlineto{\pgfqpoint{4.238905in}{1.657467in}}%
\pgfpathlineto{\pgfqpoint{4.240062in}{1.694136in}}%
\pgfpathlineto{\pgfqpoint{4.240446in}{1.681913in}}%
\pgfpathlineto{\pgfqpoint{4.241598in}{1.669690in}}%
\pgfpathlineto{\pgfqpoint{4.241981in}{1.690062in}}%
\pgfpathlineto{\pgfqpoint{4.242363in}{1.677838in}}%
\pgfpathlineto{\pgfqpoint{4.243513in}{1.661541in}}%
\pgfpathlineto{\pgfqpoint{4.243896in}{1.665615in}}%
\pgfpathlineto{\pgfqpoint{4.244280in}{1.669690in}}%
\pgfpathlineto{\pgfqpoint{4.244663in}{1.661541in}}%
\pgfpathlineto{\pgfqpoint{4.244663in}{1.661541in}}%
\pgfpathlineto{\pgfqpoint{4.244663in}{1.661541in}}%
\pgfpathlineto{\pgfqpoint{4.246282in}{1.681913in}}%
\pgfpathlineto{\pgfqpoint{4.246666in}{1.657467in}}%
\pgfpathlineto{\pgfqpoint{4.247433in}{1.669690in}}%
\pgfpathlineto{\pgfqpoint{4.248581in}{1.673764in}}%
\pgfpathlineto{\pgfqpoint{4.248966in}{1.665615in}}%
\pgfpathlineto{\pgfqpoint{4.249349in}{1.673764in}}%
\pgfpathlineto{\pgfqpoint{4.249733in}{1.673764in}}%
\pgfpathlineto{\pgfqpoint{4.250116in}{1.665615in}}%
\pgfpathlineto{\pgfqpoint{4.251266in}{1.694136in}}%
\pgfpathlineto{\pgfqpoint{4.252418in}{1.661541in}}%
\pgfpathlineto{\pgfqpoint{4.253567in}{1.681913in}}%
\pgfpathlineto{\pgfqpoint{4.253951in}{1.649318in}}%
\pgfpathlineto{\pgfqpoint{4.254718in}{1.677838in}}%
\pgfpathlineto{\pgfqpoint{4.255101in}{1.669690in}}%
\pgfpathlineto{\pgfqpoint{4.255484in}{1.685987in}}%
\pgfpathlineto{\pgfqpoint{4.255484in}{1.685987in}}%
\pgfpathlineto{\pgfqpoint{4.255484in}{1.685987in}}%
\pgfpathlineto{\pgfqpoint{4.256633in}{1.665615in}}%
\pgfpathlineto{\pgfqpoint{4.257401in}{1.685987in}}%
\pgfpathlineto{\pgfqpoint{4.257783in}{1.661541in}}%
\pgfpathlineto{\pgfqpoint{4.258550in}{1.677838in}}%
\pgfpathlineto{\pgfqpoint{4.258932in}{1.661541in}}%
\pgfpathlineto{\pgfqpoint{4.259700in}{1.673764in}}%
\pgfpathlineto{\pgfqpoint{4.260084in}{1.673764in}}%
\pgfpathlineto{\pgfqpoint{4.260467in}{1.677838in}}%
\pgfpathlineto{\pgfqpoint{4.261232in}{1.661541in}}%
\pgfpathlineto{\pgfqpoint{4.261616in}{1.690062in}}%
\pgfpathlineto{\pgfqpoint{4.262384in}{1.665615in}}%
\pgfpathlineto{\pgfqpoint{4.263151in}{1.685987in}}%
\pgfpathlineto{\pgfqpoint{4.263533in}{1.681913in}}%
\pgfpathlineto{\pgfqpoint{4.264300in}{1.657467in}}%
\pgfpathlineto{\pgfqpoint{4.264683in}{1.661541in}}%
\pgfpathlineto{\pgfqpoint{4.265834in}{1.673764in}}%
\pgfpathlineto{\pgfqpoint{4.266217in}{1.669690in}}%
\pgfpathlineto{\pgfqpoint{4.267367in}{1.694136in}}%
\pgfpathlineto{\pgfqpoint{4.267751in}{1.665615in}}%
\pgfpathlineto{\pgfqpoint{4.267751in}{1.665615in}}%
\pgfpathlineto{\pgfqpoint{4.267751in}{1.665615in}}%
\pgfpathlineto{\pgfqpoint{4.268134in}{1.698210in}}%
\pgfpathlineto{\pgfqpoint{4.268134in}{1.698210in}}%
\pgfpathlineto{\pgfqpoint{4.268134in}{1.698210in}}%
\pgfpathlineto{\pgfqpoint{4.268517in}{1.661541in}}%
\pgfpathlineto{\pgfqpoint{4.269284in}{1.673764in}}%
\pgfpathlineto{\pgfqpoint{4.269667in}{1.673764in}}%
\pgfpathlineto{\pgfqpoint{4.270051in}{1.649318in}}%
\pgfpathlineto{\pgfqpoint{4.270818in}{1.669690in}}%
\pgfpathlineto{\pgfqpoint{4.271584in}{1.677838in}}%
\pgfpathlineto{\pgfqpoint{4.271967in}{1.673764in}}%
\pgfpathlineto{\pgfqpoint{4.272350in}{1.673764in}}%
\pgfpathlineto{\pgfqpoint{4.272734in}{1.665615in}}%
\pgfpathlineto{\pgfqpoint{4.273118in}{1.690062in}}%
\pgfpathlineto{\pgfqpoint{4.273501in}{1.677838in}}%
\pgfpathlineto{\pgfqpoint{4.274652in}{1.661541in}}%
\pgfpathlineto{\pgfqpoint{4.275035in}{1.685987in}}%
\pgfpathlineto{\pgfqpoint{4.275803in}{1.669690in}}%
\pgfpathlineto{\pgfqpoint{4.276186in}{1.685987in}}%
\pgfpathlineto{\pgfqpoint{4.276952in}{1.681913in}}%
\pgfpathlineto{\pgfqpoint{4.277335in}{1.669690in}}%
\pgfpathlineto{\pgfqpoint{4.278102in}{1.677838in}}%
\pgfpathlineto{\pgfqpoint{4.278869in}{1.681913in}}%
\pgfpathlineto{\pgfqpoint{4.280120in}{1.661541in}}%
\pgfpathlineto{\pgfqpoint{4.280506in}{1.673764in}}%
\pgfpathlineto{\pgfqpoint{4.281273in}{1.669690in}}%
\pgfpathlineto{\pgfqpoint{4.281656in}{1.661541in}}%
\pgfpathlineto{\pgfqpoint{4.281656in}{1.661541in}}%
\pgfpathlineto{\pgfqpoint{4.281656in}{1.661541in}}%
\pgfpathlineto{\pgfqpoint{4.283191in}{1.677838in}}%
\pgfpathlineto{\pgfqpoint{4.283574in}{1.685987in}}%
\pgfpathlineto{\pgfqpoint{4.284340in}{1.665615in}}%
\pgfpathlineto{\pgfqpoint{4.284724in}{1.677838in}}%
\pgfpathlineto{\pgfqpoint{4.285108in}{1.665615in}}%
\pgfpathlineto{\pgfqpoint{4.285108in}{1.665615in}}%
\pgfpathlineto{\pgfqpoint{4.285108in}{1.665615in}}%
\pgfpathlineto{\pgfqpoint{4.285492in}{1.685987in}}%
\pgfpathlineto{\pgfqpoint{4.285492in}{1.685987in}}%
\pgfpathlineto{\pgfqpoint{4.285492in}{1.685987in}}%
\pgfpathlineto{\pgfqpoint{4.285874in}{1.661541in}}%
\pgfpathlineto{\pgfqpoint{4.286640in}{1.681913in}}%
\pgfpathlineto{\pgfqpoint{4.287790in}{1.669690in}}%
\pgfpathlineto{\pgfqpoint{4.288174in}{1.649318in}}%
\pgfpathlineto{\pgfqpoint{4.288174in}{1.649318in}}%
\pgfpathlineto{\pgfqpoint{4.288174in}{1.649318in}}%
\pgfpathlineto{\pgfqpoint{4.288940in}{1.681913in}}%
\pgfpathlineto{\pgfqpoint{4.289322in}{1.673764in}}%
\pgfpathlineto{\pgfqpoint{4.289707in}{1.665615in}}%
\pgfpathlineto{\pgfqpoint{4.290090in}{1.681913in}}%
\pgfpathlineto{\pgfqpoint{4.290858in}{1.673764in}}%
\pgfpathlineto{\pgfqpoint{4.291624in}{1.677838in}}%
\pgfpathlineto{\pgfqpoint{4.292007in}{1.673764in}}%
\pgfpathlineto{\pgfqpoint{4.292774in}{1.690062in}}%
\pgfpathlineto{\pgfqpoint{4.293541in}{1.661541in}}%
\pgfpathlineto{\pgfqpoint{4.293925in}{1.690062in}}%
\pgfpathlineto{\pgfqpoint{4.294691in}{1.681913in}}%
\pgfpathlineto{\pgfqpoint{4.295074in}{1.690062in}}%
\pgfpathlineto{\pgfqpoint{4.295074in}{1.690062in}}%
\pgfpathlineto{\pgfqpoint{4.295074in}{1.690062in}}%
\pgfpathlineto{\pgfqpoint{4.295840in}{1.669690in}}%
\pgfpathlineto{\pgfqpoint{4.296225in}{1.681913in}}%
\pgfpathlineto{\pgfqpoint{4.296608in}{1.681913in}}%
\pgfpathlineto{\pgfqpoint{4.296991in}{1.653392in}}%
\pgfpathlineto{\pgfqpoint{4.297758in}{1.677838in}}%
\pgfpathlineto{\pgfqpoint{4.298142in}{1.677838in}}%
\pgfpathlineto{\pgfqpoint{4.298526in}{1.690062in}}%
\pgfpathlineto{\pgfqpoint{4.299293in}{1.685987in}}%
\pgfpathlineto{\pgfqpoint{4.299675in}{1.685987in}}%
\pgfpathlineto{\pgfqpoint{4.300826in}{1.673764in}}%
\pgfpathlineto{\pgfqpoint{4.301209in}{1.685987in}}%
\pgfpathlineto{\pgfqpoint{4.301209in}{1.685987in}}%
\pgfpathlineto{\pgfqpoint{4.301209in}{1.685987in}}%
\pgfpathlineto{\pgfqpoint{4.301978in}{1.661541in}}%
\pgfpathlineto{\pgfqpoint{4.302361in}{1.681913in}}%
\pgfpathlineto{\pgfqpoint{4.303511in}{1.665615in}}%
\pgfpathlineto{\pgfqpoint{4.304277in}{1.690062in}}%
\pgfpathlineto{\pgfqpoint{4.304660in}{1.681913in}}%
\pgfpathlineto{\pgfqpoint{4.306577in}{1.669690in}}%
\pgfpathlineto{\pgfqpoint{4.306961in}{1.669690in}}%
\pgfpathlineto{\pgfqpoint{4.307729in}{1.661541in}}%
\pgfpathlineto{\pgfqpoint{4.308496in}{1.677838in}}%
\pgfpathlineto{\pgfqpoint{4.309262in}{1.657467in}}%
\pgfpathlineto{\pgfqpoint{4.310411in}{1.677838in}}%
\pgfpathlineto{\pgfqpoint{4.310795in}{1.669690in}}%
\pgfpathlineto{\pgfqpoint{4.310795in}{1.669690in}}%
\pgfpathlineto{\pgfqpoint{4.310795in}{1.669690in}}%
\pgfpathlineto{\pgfqpoint{4.311178in}{1.681913in}}%
\pgfpathlineto{\pgfqpoint{4.311561in}{1.669690in}}%
\pgfpathlineto{\pgfqpoint{4.311944in}{1.665615in}}%
\pgfpathlineto{\pgfqpoint{4.312711in}{1.681913in}}%
\pgfpathlineto{\pgfqpoint{4.313478in}{1.677838in}}%
\pgfpathlineto{\pgfqpoint{4.314244in}{1.665615in}}%
\pgfpathlineto{\pgfqpoint{4.314628in}{1.673764in}}%
\pgfpathlineto{\pgfqpoint{4.315011in}{1.645243in}}%
\pgfpathlineto{\pgfqpoint{4.315394in}{1.673764in}}%
\pgfpathlineto{\pgfqpoint{4.315778in}{1.681913in}}%
\pgfpathlineto{\pgfqpoint{4.316545in}{1.677838in}}%
\pgfpathlineto{\pgfqpoint{4.316929in}{1.681913in}}%
\pgfpathlineto{\pgfqpoint{4.317311in}{1.677838in}}%
\pgfpathlineto{\pgfqpoint{4.318077in}{1.665615in}}%
\pgfpathlineto{\pgfqpoint{4.318461in}{1.677838in}}%
\pgfpathlineto{\pgfqpoint{4.318844in}{1.669690in}}%
\pgfpathlineto{\pgfqpoint{4.319229in}{1.665615in}}%
\pgfpathlineto{\pgfqpoint{4.319612in}{1.677838in}}%
\pgfpathlineto{\pgfqpoint{4.319612in}{1.677838in}}%
\pgfpathlineto{\pgfqpoint{4.319612in}{1.677838in}}%
\pgfpathlineto{\pgfqpoint{4.320003in}{1.661541in}}%
\pgfpathlineto{\pgfqpoint{4.320387in}{1.669690in}}%
\pgfpathlineto{\pgfqpoint{4.321537in}{1.685987in}}%
\pgfpathlineto{\pgfqpoint{4.322685in}{1.673764in}}%
\pgfpathlineto{\pgfqpoint{4.323069in}{1.685987in}}%
\pgfpathlineto{\pgfqpoint{4.323453in}{1.677838in}}%
\pgfpathlineto{\pgfqpoint{4.323836in}{1.669690in}}%
\pgfpathlineto{\pgfqpoint{4.324220in}{1.677838in}}%
\pgfpathlineto{\pgfqpoint{4.324602in}{1.681913in}}%
\pgfpathlineto{\pgfqpoint{4.325368in}{1.657467in}}%
\pgfpathlineto{\pgfqpoint{4.325752in}{1.673764in}}%
\pgfpathlineto{\pgfqpoint{4.326135in}{1.673764in}}%
\pgfpathlineto{\pgfqpoint{4.327284in}{1.661541in}}%
\pgfpathlineto{\pgfqpoint{4.327666in}{1.681913in}}%
\pgfpathlineto{\pgfqpoint{4.328433in}{1.673764in}}%
\pgfpathlineto{\pgfqpoint{4.329584in}{1.677838in}}%
\pgfpathlineto{\pgfqpoint{4.329967in}{1.673764in}}%
\pgfpathlineto{\pgfqpoint{4.330349in}{1.685987in}}%
\pgfpathlineto{\pgfqpoint{4.330349in}{1.685987in}}%
\pgfpathlineto{\pgfqpoint{4.330349in}{1.685987in}}%
\pgfpathlineto{\pgfqpoint{4.330733in}{1.665615in}}%
\pgfpathlineto{\pgfqpoint{4.331501in}{1.673764in}}%
\pgfpathlineto{\pgfqpoint{4.331885in}{1.673764in}}%
\pgfpathlineto{\pgfqpoint{4.332269in}{1.690062in}}%
\pgfpathlineto{\pgfqpoint{4.332652in}{1.653392in}}%
\pgfpathlineto{\pgfqpoint{4.332652in}{1.653392in}}%
\pgfpathlineto{\pgfqpoint{4.332652in}{1.653392in}}%
\pgfpathlineto{\pgfqpoint{4.333035in}{1.702285in}}%
\pgfpathlineto{\pgfqpoint{4.333803in}{1.673764in}}%
\pgfpathlineto{\pgfqpoint{4.334570in}{1.681913in}}%
\pgfpathlineto{\pgfqpoint{4.334953in}{1.673764in}}%
\pgfpathlineto{\pgfqpoint{4.335336in}{1.694136in}}%
\pgfpathlineto{\pgfqpoint{4.335336in}{1.694136in}}%
\pgfpathlineto{\pgfqpoint{4.335336in}{1.694136in}}%
\pgfpathlineto{\pgfqpoint{4.336485in}{1.665615in}}%
\pgfpathlineto{\pgfqpoint{4.337636in}{1.677838in}}%
\pgfpathlineto{\pgfqpoint{4.338019in}{1.665615in}}%
\pgfpathlineto{\pgfqpoint{4.338785in}{1.673764in}}%
\pgfpathlineto{\pgfqpoint{4.339168in}{1.694136in}}%
\pgfpathlineto{\pgfqpoint{4.339935in}{1.681913in}}%
\pgfpathlineto{\pgfqpoint{4.340318in}{1.681913in}}%
\pgfpathlineto{\pgfqpoint{4.341084in}{1.669690in}}%
\pgfpathlineto{\pgfqpoint{4.341467in}{1.673764in}}%
\pgfpathlineto{\pgfqpoint{4.341849in}{1.673764in}}%
\pgfpathlineto{\pgfqpoint{4.343083in}{1.681913in}}%
\pgfpathlineto{\pgfqpoint{4.343466in}{1.665615in}}%
\pgfpathlineto{\pgfqpoint{4.343849in}{1.677838in}}%
\pgfpathlineto{\pgfqpoint{4.344615in}{1.690062in}}%
\pgfpathlineto{\pgfqpoint{4.345380in}{1.665615in}}%
\pgfpathlineto{\pgfqpoint{4.345763in}{1.690062in}}%
\pgfpathlineto{\pgfqpoint{4.346530in}{1.681913in}}%
\pgfpathlineto{\pgfqpoint{4.346913in}{1.685987in}}%
\pgfpathlineto{\pgfqpoint{4.348827in}{1.657467in}}%
\pgfpathlineto{\pgfqpoint{4.349210in}{1.677838in}}%
\pgfpathlineto{\pgfqpoint{4.349210in}{1.677838in}}%
\pgfpathlineto{\pgfqpoint{4.349210in}{1.677838in}}%
\pgfpathlineto{\pgfqpoint{4.349593in}{1.645243in}}%
\pgfpathlineto{\pgfqpoint{4.350358in}{1.665615in}}%
\pgfpathlineto{\pgfqpoint{4.351124in}{1.677838in}}%
\pgfpathlineto{\pgfqpoint{4.351507in}{1.669690in}}%
\pgfpathlineto{\pgfqpoint{4.351894in}{1.673764in}}%
\pgfpathlineto{\pgfqpoint{4.352277in}{1.661541in}}%
\pgfpathlineto{\pgfqpoint{4.352660in}{1.665615in}}%
\pgfpathlineto{\pgfqpoint{4.353427in}{1.677838in}}%
\pgfpathlineto{\pgfqpoint{4.353810in}{1.661541in}}%
\pgfpathlineto{\pgfqpoint{4.354577in}{1.669690in}}%
\pgfpathlineto{\pgfqpoint{4.354960in}{1.661541in}}%
\pgfpathlineto{\pgfqpoint{4.356110in}{1.681913in}}%
\pgfpathlineto{\pgfqpoint{4.356875in}{1.673764in}}%
\pgfpathlineto{\pgfqpoint{4.358026in}{1.685987in}}%
\pgfpathlineto{\pgfqpoint{4.358409in}{1.665615in}}%
\pgfpathlineto{\pgfqpoint{4.359175in}{1.681913in}}%
\pgfpathlineto{\pgfqpoint{4.360335in}{1.661541in}}%
\pgfpathlineto{\pgfqpoint{4.361101in}{1.681913in}}%
\pgfpathlineto{\pgfqpoint{4.361869in}{1.677838in}}%
\pgfpathlineto{\pgfqpoint{4.362252in}{1.673764in}}%
\pgfpathlineto{\pgfqpoint{4.362636in}{1.690062in}}%
\pgfpathlineto{\pgfqpoint{4.363019in}{1.681913in}}%
\pgfpathlineto{\pgfqpoint{4.364171in}{1.653392in}}%
\pgfpathlineto{\pgfqpoint{4.364555in}{1.677838in}}%
\pgfpathlineto{\pgfqpoint{4.365321in}{1.673764in}}%
\pgfpathlineto{\pgfqpoint{4.366471in}{1.681913in}}%
\pgfpathlineto{\pgfqpoint{4.366855in}{1.677838in}}%
\pgfpathlineto{\pgfqpoint{4.367238in}{1.661541in}}%
\pgfpathlineto{\pgfqpoint{4.367238in}{1.661541in}}%
\pgfpathlineto{\pgfqpoint{4.367238in}{1.661541in}}%
\pgfpathlineto{\pgfqpoint{4.368387in}{1.681913in}}%
\pgfpathlineto{\pgfqpoint{4.368770in}{1.657467in}}%
\pgfpathlineto{\pgfqpoint{4.368770in}{1.657467in}}%
\pgfpathlineto{\pgfqpoint{4.368770in}{1.657467in}}%
\pgfpathlineto{\pgfqpoint{4.369155in}{1.685987in}}%
\pgfpathlineto{\pgfqpoint{4.369921in}{1.673764in}}%
\pgfpathlineto{\pgfqpoint{4.370688in}{1.681913in}}%
\pgfpathlineto{\pgfqpoint{4.371838in}{1.665615in}}%
\pgfpathlineto{\pgfqpoint{4.372987in}{1.690062in}}%
\pgfpathlineto{\pgfqpoint{4.373371in}{1.665615in}}%
\pgfpathlineto{\pgfqpoint{4.374138in}{1.685987in}}%
\pgfpathlineto{\pgfqpoint{4.374905in}{1.669690in}}%
\pgfpathlineto{\pgfqpoint{4.375288in}{1.673764in}}%
\pgfpathlineto{\pgfqpoint{4.375671in}{1.677838in}}%
\pgfpathlineto{\pgfqpoint{4.376821in}{1.665615in}}%
\pgfpathlineto{\pgfqpoint{4.377588in}{1.681913in}}%
\pgfpathlineto{\pgfqpoint{4.377971in}{1.669690in}}%
\pgfpathlineto{\pgfqpoint{4.378355in}{1.669690in}}%
\pgfpathlineto{\pgfqpoint{4.378738in}{1.677838in}}%
\pgfpathlineto{\pgfqpoint{4.379121in}{1.661541in}}%
\pgfpathlineto{\pgfqpoint{4.379121in}{1.661541in}}%
\pgfpathlineto{\pgfqpoint{4.379121in}{1.661541in}}%
\pgfpathlineto{\pgfqpoint{4.379505in}{1.685987in}}%
\pgfpathlineto{\pgfqpoint{4.379505in}{1.685987in}}%
\pgfpathlineto{\pgfqpoint{4.379505in}{1.685987in}}%
\pgfpathlineto{\pgfqpoint{4.379888in}{1.649318in}}%
\pgfpathlineto{\pgfqpoint{4.380655in}{1.673764in}}%
\pgfpathlineto{\pgfqpoint{4.381038in}{1.685987in}}%
\pgfpathlineto{\pgfqpoint{4.381421in}{1.681913in}}%
\pgfpathlineto{\pgfqpoint{4.381804in}{1.665615in}}%
\pgfpathlineto{\pgfqpoint{4.381804in}{1.665615in}}%
\pgfpathlineto{\pgfqpoint{4.381804in}{1.665615in}}%
\pgfpathlineto{\pgfqpoint{4.382956in}{1.698210in}}%
\pgfpathlineto{\pgfqpoint{4.384189in}{1.657467in}}%
\pgfpathlineto{\pgfqpoint{4.384955in}{1.681913in}}%
\pgfpathlineto{\pgfqpoint{4.385338in}{1.677838in}}%
\pgfpathlineto{\pgfqpoint{4.385722in}{1.677838in}}%
\pgfpathlineto{\pgfqpoint{4.386871in}{1.657467in}}%
\pgfpathlineto{\pgfqpoint{4.387254in}{1.685987in}}%
\pgfpathlineto{\pgfqpoint{4.388019in}{1.669690in}}%
\pgfpathlineto{\pgfqpoint{4.389169in}{1.677838in}}%
\pgfpathlineto{\pgfqpoint{4.389552in}{1.653392in}}%
\pgfpathlineto{\pgfqpoint{4.390318in}{1.673764in}}%
\pgfpathlineto{\pgfqpoint{4.391084in}{1.685987in}}%
\pgfpathlineto{\pgfqpoint{4.391466in}{1.681913in}}%
\pgfpathlineto{\pgfqpoint{4.391850in}{1.661541in}}%
\pgfpathlineto{\pgfqpoint{4.392235in}{1.677838in}}%
\pgfpathlineto{\pgfqpoint{4.392617in}{1.690062in}}%
\pgfpathlineto{\pgfqpoint{4.392617in}{1.690062in}}%
\pgfpathlineto{\pgfqpoint{4.392617in}{1.690062in}}%
\pgfpathlineto{\pgfqpoint{4.393766in}{1.649318in}}%
\pgfpathlineto{\pgfqpoint{4.395299in}{1.690062in}}%
\pgfpathlineto{\pgfqpoint{4.395681in}{1.685987in}}%
\pgfpathlineto{\pgfqpoint{4.396831in}{1.665615in}}%
\pgfpathlineto{\pgfqpoint{4.398364in}{1.690062in}}%
\pgfpathlineto{\pgfqpoint{4.399131in}{1.661541in}}%
\pgfpathlineto{\pgfqpoint{4.399513in}{1.694136in}}%
\pgfpathlineto{\pgfqpoint{4.399905in}{1.685987in}}%
\pgfpathlineto{\pgfqpoint{4.400288in}{1.661541in}}%
\pgfpathlineto{\pgfqpoint{4.401054in}{1.677838in}}%
\pgfpathlineto{\pgfqpoint{4.401437in}{1.685987in}}%
\pgfpathlineto{\pgfqpoint{4.401821in}{1.669690in}}%
\pgfpathlineto{\pgfqpoint{4.402205in}{1.673764in}}%
\pgfpathlineto{\pgfqpoint{4.402589in}{1.690062in}}%
\pgfpathlineto{\pgfqpoint{4.402589in}{1.690062in}}%
\pgfpathlineto{\pgfqpoint{4.402589in}{1.690062in}}%
\pgfpathlineto{\pgfqpoint{4.404120in}{1.661541in}}%
\pgfpathlineto{\pgfqpoint{4.405655in}{1.698210in}}%
\pgfpathlineto{\pgfqpoint{4.406038in}{1.694136in}}%
\pgfpathlineto{\pgfqpoint{4.407571in}{1.649318in}}%
\pgfpathlineto{\pgfqpoint{4.408721in}{1.681913in}}%
\pgfpathlineto{\pgfqpoint{4.409104in}{1.669690in}}%
\pgfpathlineto{\pgfqpoint{4.409487in}{1.673764in}}%
\pgfpathlineto{\pgfqpoint{4.409871in}{1.665615in}}%
\pgfpathlineto{\pgfqpoint{4.409871in}{1.665615in}}%
\pgfpathlineto{\pgfqpoint{4.409871in}{1.665615in}}%
\pgfpathlineto{\pgfqpoint{4.410256in}{1.677838in}}%
\pgfpathlineto{\pgfqpoint{4.410639in}{1.669690in}}%
\pgfpathlineto{\pgfqpoint{4.411023in}{1.649318in}}%
\pgfpathlineto{\pgfqpoint{4.411405in}{1.665615in}}%
\pgfpathlineto{\pgfqpoint{4.412171in}{1.690062in}}%
\pgfpathlineto{\pgfqpoint{4.412555in}{1.673764in}}%
\pgfpathlineto{\pgfqpoint{4.413705in}{1.661541in}}%
\pgfpathlineto{\pgfqpoint{4.414470in}{1.694136in}}%
\pgfpathlineto{\pgfqpoint{4.414854in}{1.657467in}}%
\pgfpathlineto{\pgfqpoint{4.415621in}{1.673764in}}%
\pgfpathlineto{\pgfqpoint{4.416387in}{1.685987in}}%
\pgfpathlineto{\pgfqpoint{4.417153in}{1.669690in}}%
\pgfpathlineto{\pgfqpoint{4.417536in}{1.681913in}}%
\pgfpathlineto{\pgfqpoint{4.418685in}{1.665615in}}%
\pgfpathlineto{\pgfqpoint{4.419068in}{1.669690in}}%
\pgfpathlineto{\pgfqpoint{4.419834in}{1.661541in}}%
\pgfpathlineto{\pgfqpoint{4.420217in}{1.677838in}}%
\pgfpathlineto{\pgfqpoint{4.420983in}{1.673764in}}%
\pgfpathlineto{\pgfqpoint{4.421753in}{1.669690in}}%
\pgfpathlineto{\pgfqpoint{4.422136in}{1.685987in}}%
\pgfpathlineto{\pgfqpoint{4.422519in}{1.673764in}}%
\pgfpathlineto{\pgfqpoint{4.422987in}{1.669690in}}%
\pgfpathlineto{\pgfqpoint{4.423370in}{1.673764in}}%
\pgfpathlineto{\pgfqpoint{4.423753in}{1.685987in}}%
\pgfpathlineto{\pgfqpoint{4.424137in}{1.677838in}}%
\pgfpathlineto{\pgfqpoint{4.425670in}{1.661541in}}%
\pgfpathlineto{\pgfqpoint{4.426436in}{1.690062in}}%
\pgfpathlineto{\pgfqpoint{4.426819in}{1.673764in}}%
\pgfpathlineto{\pgfqpoint{4.427203in}{1.669690in}}%
\pgfpathlineto{\pgfqpoint{4.427969in}{1.690062in}}%
\pgfpathlineto{\pgfqpoint{4.428352in}{1.677838in}}%
\pgfpathlineto{\pgfqpoint{4.428736in}{1.673764in}}%
\pgfpathlineto{\pgfqpoint{4.429119in}{1.685987in}}%
\pgfpathlineto{\pgfqpoint{4.429119in}{1.685987in}}%
\pgfpathlineto{\pgfqpoint{4.429119in}{1.685987in}}%
\pgfpathlineto{\pgfqpoint{4.429502in}{1.669690in}}%
\pgfpathlineto{\pgfqpoint{4.429887in}{1.673764in}}%
\pgfpathlineto{\pgfqpoint{4.430270in}{1.690062in}}%
\pgfpathlineto{\pgfqpoint{4.430270in}{1.690062in}}%
\pgfpathlineto{\pgfqpoint{4.430270in}{1.690062in}}%
\pgfpathlineto{\pgfqpoint{4.431037in}{1.661541in}}%
\pgfpathlineto{\pgfqpoint{4.431420in}{1.677838in}}%
\pgfpathlineto{\pgfqpoint{4.431803in}{1.681913in}}%
\pgfpathlineto{\pgfqpoint{4.432569in}{1.669690in}}%
\pgfpathlineto{\pgfqpoint{4.432953in}{1.673764in}}%
\pgfpathlineto{\pgfqpoint{4.433337in}{1.673764in}}%
\pgfpathlineto{\pgfqpoint{4.434102in}{1.661541in}}%
\pgfpathlineto{\pgfqpoint{4.435250in}{1.677838in}}%
\pgfpathlineto{\pgfqpoint{4.436400in}{1.665615in}}%
\pgfpathlineto{\pgfqpoint{4.437548in}{1.685987in}}%
\pgfpathlineto{\pgfqpoint{4.437932in}{1.673764in}}%
\pgfpathlineto{\pgfqpoint{4.437932in}{1.673764in}}%
\pgfpathlineto{\pgfqpoint{4.437932in}{1.673764in}}%
\pgfpathlineto{\pgfqpoint{4.438315in}{1.694136in}}%
\pgfpathlineto{\pgfqpoint{4.438315in}{1.694136in}}%
\pgfpathlineto{\pgfqpoint{4.438315in}{1.694136in}}%
\pgfpathlineto{\pgfqpoint{4.438698in}{1.665615in}}%
\pgfpathlineto{\pgfqpoint{4.439465in}{1.673764in}}%
\pgfpathlineto{\pgfqpoint{4.440241in}{1.673764in}}%
\pgfpathlineto{\pgfqpoint{4.441389in}{1.694136in}}%
\pgfpathlineto{\pgfqpoint{4.442156in}{1.673764in}}%
\pgfpathlineto{\pgfqpoint{4.442156in}{1.673764in}}%
\pgfusepath{stroke}%
\end{pgfscope}%
\begin{pgfscope}%
\pgfsetrectcap%
\pgfsetmiterjoin%
\pgfsetlinewidth{0.803000pt}%
\definecolor{currentstroke}{rgb}{0.000000,0.000000,0.000000}%
\pgfsetstrokecolor{currentstroke}%
\pgfsetdash{}{0pt}%
\pgfpathmoveto{\pgfqpoint{0.812440in}{0.565123in}}%
\pgfpathlineto{\pgfqpoint{0.812440in}{2.815000in}}%
\pgfusepath{stroke}%
\end{pgfscope}%
\begin{pgfscope}%
\pgfsetrectcap%
\pgfsetmiterjoin%
\pgfsetlinewidth{0.803000pt}%
\definecolor{currentstroke}{rgb}{0.000000,0.000000,0.000000}%
\pgfsetstrokecolor{currentstroke}%
\pgfsetdash{}{0pt}%
\pgfpathmoveto{\pgfqpoint{4.615000in}{0.565123in}}%
\pgfpathlineto{\pgfqpoint{4.615000in}{2.815000in}}%
\pgfusepath{stroke}%
\end{pgfscope}%
\begin{pgfscope}%
\pgfsetrectcap%
\pgfsetmiterjoin%
\pgfsetlinewidth{0.803000pt}%
\definecolor{currentstroke}{rgb}{0.000000,0.000000,0.000000}%
\pgfsetstrokecolor{currentstroke}%
\pgfsetdash{}{0pt}%
\pgfpathmoveto{\pgfqpoint{0.812440in}{0.565123in}}%
\pgfpathlineto{\pgfqpoint{4.615000in}{0.565123in}}%
\pgfusepath{stroke}%
\end{pgfscope}%
\begin{pgfscope}%
\pgfsetrectcap%
\pgfsetmiterjoin%
\pgfsetlinewidth{0.803000pt}%
\definecolor{currentstroke}{rgb}{0.000000,0.000000,0.000000}%
\pgfsetstrokecolor{currentstroke}%
\pgfsetdash{}{0pt}%
\pgfpathmoveto{\pgfqpoint{0.812440in}{2.815000in}}%
\pgfpathlineto{\pgfqpoint{4.615000in}{2.815000in}}%
\pgfusepath{stroke}%
\end{pgfscope}%
\begin{pgfscope}%
\pgfsetbuttcap%
\pgfsetmiterjoin%
\definecolor{currentfill}{rgb}{1.000000,1.000000,1.000000}%
\pgfsetfillcolor{currentfill}%
\pgfsetfillopacity{0.800000}%
\pgfsetlinewidth{1.003750pt}%
\definecolor{currentstroke}{rgb}{0.800000,0.800000,0.800000}%
\pgfsetstrokecolor{currentstroke}%
\pgfsetstrokeopacity{0.800000}%
\pgfsetdash{}{0pt}%
\pgfpathmoveto{\pgfqpoint{4.009290in}{2.316543in}}%
\pgfpathlineto{\pgfqpoint{4.517778in}{2.316543in}}%
\pgfpathquadraticcurveto{\pgfqpoint{4.545556in}{2.316543in}}{\pgfqpoint{4.545556in}{2.344321in}}%
\pgfpathlineto{\pgfqpoint{4.545556in}{2.717778in}}%
\pgfpathquadraticcurveto{\pgfqpoint{4.545556in}{2.745556in}}{\pgfqpoint{4.517778in}{2.745556in}}%
\pgfpathlineto{\pgfqpoint{4.009290in}{2.745556in}}%
\pgfpathquadraticcurveto{\pgfqpoint{3.981512in}{2.745556in}}{\pgfqpoint{3.981512in}{2.717778in}}%
\pgfpathlineto{\pgfqpoint{3.981512in}{2.344321in}}%
\pgfpathquadraticcurveto{\pgfqpoint{3.981512in}{2.316543in}}{\pgfqpoint{4.009290in}{2.316543in}}%
\pgfpathclose%
\pgfusepath{stroke,fill}%
\end{pgfscope}%
\begin{pgfscope}%
\pgfsetrectcap%
\pgfsetroundjoin%
\pgfsetlinewidth{1.505625pt}%
\definecolor{currentstroke}{rgb}{0.121569,0.466667,0.705882}%
\pgfsetstrokecolor{currentstroke}%
\pgfsetdash{}{0pt}%
\pgfpathmoveto{\pgfqpoint{4.037068in}{2.641389in}}%
\pgfpathlineto{\pgfqpoint{4.314845in}{2.641389in}}%
\pgfusepath{stroke}%
\end{pgfscope}%
\begin{pgfscope}%
\pgftext[x=4.425957in,y=2.592778in,left,base]{\sffamily\fontsize{10.000000}{12.000000}\selectfont x}%
\end{pgfscope}%
\begin{pgfscope}%
\pgfsetrectcap%
\pgfsetroundjoin%
\pgfsetlinewidth{1.505625pt}%
\definecolor{currentstroke}{rgb}{1.000000,0.498039,0.054902}%
\pgfsetstrokecolor{currentstroke}%
\pgfsetdash{}{0pt}%
\pgfpathmoveto{\pgfqpoint{4.037068in}{2.447716in}}%
\pgfpathlineto{\pgfqpoint{4.314845in}{2.447716in}}%
\pgfusepath{stroke}%
\end{pgfscope}%
\begin{pgfscope}%
\pgftext[x=4.425957in,y=2.399105in,left,base]{\sffamily\fontsize{10.000000}{12.000000}\selectfont y}%
\end{pgfscope}%
\end{pgfpicture}%
\makeatother%
\endgroup%

	\caption{Voorbeeld Python figuur waarbij het font door \LaTeX gezet is.}\label{fig:python}
\end{figure}

\blindtext

\begin{table}[htbp]
	\centering\begin{tabularx}{0.6\textwidth}{SSSS}
{tijd [\si{s}]} & {$a_x$ [\si{\meter\per\second\squared}]} & {$a_y$ [\si{\meter\per\second\squared}]} & {$a_z$ [\si{\meter\per\second\squared}]}\\
\hline
0.000 & -0.214 & -0.193 & 0.959 \\
0.017 & -0.218 & -0.191 & 0.958 \\
0.034 & -0.214 & -0.195 & 0.961 \\
0.051 & -0.214 & -0.193 & 0.959 \\
0.069 & -0.214 & -0.194 & 0.957 \\
0.086 & -0.214 & -0.193 & 0.957 \\
0.103 & -0.217 & -0.191 & 0.957 \\
0.120 & -0.217 & -0.194 & 0.958 \\
0.137 & -0.217 & -0.193 & 0.957 \\
0.154 & -0.217 & -0.195 & 0.960 \\
\hline
\end{tabularx}

	\caption{Voorbeeld tabel met data uit een Python-script.}\label{tab:python}
\end{table}

\blindtext

\begin{equation}
E = mc^2 \label{eq:verplichte vergelijking}
\end{equation} 
\begin{grootheden}
E & energie & \sit{J} \\
m & massa & \sit{kg} \\
c & snelheid (van het licht) & \sit[.]{m/s} \\
\end{grootheden}
\blindtext

En nog wat electronica...
\begin{figure}
\centering\begin{circuitikz} \draw
(0,0) to[battery] (0,4)
  to[ammeter] (4,4) -- (4,0)
  to[lamp] (0,0)
(0.5,0) -- (0.5,-2)
  to[voltmeter] (3.5,-2) -- (3.5,0)
;
\end{circuitikz}
	\caption{Electrisch schema...}
\end{figure}

\blindtext

\bibliographystyle{apalike}
\bibliography{citations}

\end{document}

